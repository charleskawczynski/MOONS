\documentclass[landscape]{article}
\usepackage{graphicx}    % needed for including graphics e.g. EPS, PS
\usepackage{epstopdf}
\usepackage{amsmath}
\usepackage{hyperref}
\usepackage{xspace}
\usepackage{mathtools}
\usepackage{tikz}
\usepackage{epsfig}
\usepackage{float}
%\usepackage{natbib}
\usepackage{subfigure}
\usepackage{setspace}
\usepackage{tabularx,ragged2e,booktabs,caption}


% \setlength{\oddsidemargin}{0.1in}
% \setlength{\textwidth}{7.25in}

\setlength{\topmargin}{-1in}     %\topmargin: gap above header
\setlength{\headheight}{0in}     %\headheight: height of header
\setlength{\topskip}{0in}        %\topskip: between header and text
% \setlength{\headsep}{0in}        
\setlength{\textheight}{525pt}   %\textheight: height of main text
\setlength{\textwidth}{10in}    % \textwidth: width of text
\setlength{\oddsidemargin}{-0.5in}  % \oddsidemargin: odd page left margin
\setlength{\evensidemargin}{0in} %\evensidemargin : even page left margin
\setlength{\parindent}{0.25in}   %\parindent: indentation of paragraphs
\setlength{\parskip}{0pt}        %\parskip: gap between paragraphs
% \setlength{\voffset}{0.5in}

% Useful commands:

\pagenumbering{gobble}
% \hfill		aligns-right everything right of \hfill

\begin{document}

\doublespacing
\title{Magnetohydrodynamic Object-Oriented Numerical Simulation (MOONS)}
\author{C. Kawczynski \\
Department of Mechanical and Aerospace Engineering \\
University of California Los Angeles, USA\\}
\maketitle

\section{Finite Difference Schemes}

The results in this document are generated from a linear system solver, using Python, in stencilMaker.py. The file is run from stencils.py. The linear system is a set of Taylor expansions about a point in space and is of the form of a Vandermonde matrix

\begin{equation}
	f_{i+k}
	=
	\sum_{j=0}^n
	\frac{\alpha_{i+k} f_{i}^{(j)}}{(j)!}
	\qquad \qquad
	k = n_{left} \dots n_{right}
\end{equation}

Where the unknowns are the derivatives $f_i^{(j)}$. The input arguments, $n_{left}$ and $n_{right}$, are the number of points to the left and right of $i$ respectively. $f_{i+k}\forall k$ is assumed to be known. Also, by definition
\begin{equation}
	\alpha_{i+k} 
	=
	h_{i+k} - h_{i}
\end{equation}

Since, in MOONS, $\Delta h_1 = h_2 - h_1$, let

\begin{equation}
	\Delta h_{i}
	=
	h_{i+1} - h_{i}
	=
	\alpha_{i+1}
	\qquad
	\rightarrow
	\alpha_{i+k} = \Delta h_{i+k-1}
\end{equation}

Also note that, for $k>1$,

\begin{equation}
	\Delta h_{i+k-1}
	=
	\Delta h_{i+1}
	+
	\dots
	+
	\Delta h_{i+k-1}
\end{equation}

And for $k<1$,

\begin{equation}
	\Delta h_{i+k-1}
	=
	\Delta h_{i-1}
	+
	\dots
	+
	\Delta h_{i+k-1}
\end{equation}

Again, to note how indexing is defined

\begin{equation}
	\boxed{
	\Delta h_{i}
	=
	h_{i+1} - h_{i}
	}
\end{equation}

\section{Generated Finite Difference Schemes}
The results below are generated from stencils.py in the above directory. The type of stencils (central difference / forward difference / number of points to include in stencil) here can be changed by modifying the stencils.py file. The truncation error of the finite difference scheme is included. Sometimes a higher term is present, other times it may be absent (depending if there is perfect cancellation due to symmetry).

Try including equations solved with and without truncation. The equation without truncation terms may be easier to read.

\subsection{Equations Solved (Without Truncation Error)}
\begin{equation} 
f_{{i-1}} = f_{i} - f^{{(1)}}_{i} {\Delta h} + \frac{f^{{(2)}}_{i}}{2} {\Delta h}^{2}
 \end{equation} 
\begin{equation} 
f_{{i+1}} = f_{i} + f^{{(1)}}_{i} {\Delta h} + \frac{f^{{(2)}}_{i}}{2} {\Delta h}^{2}
 \end{equation} 

\subsection{Equations Solved (With Truncation Error)}
\begin{equation} 
f_{{i-2}} = \frac{\alpha_{{i-2}}^{6} f^{{(6)}}_{{\xi}}}{720} + \frac{\alpha_{{i-2}}^{5} f^{{(5)}}_{{\xi}}}{120} + \frac{\alpha_{{i-2}}^{4} f^{{(4)}}_{i}}{24} + \frac{\alpha_{{i-2}}^{3} f^{{(3)}}_{i}}{6} + \frac{\alpha_{{i-2}}^{2} f^{{(2)}}_{i}}{2} + \alpha_{{i-2}} f^{{(1)}}_{i} + f_{i}
 \end{equation} 
\begin{equation} 
f_{{i-1}} = \frac{\alpha_{{i-1}}^{6} f^{{(6)}}_{{\xi}}}{720} + \frac{\alpha_{{i-1}}^{5} f^{{(5)}}_{{\xi}}}{120} + \frac{\alpha_{{i-1}}^{4} f^{{(4)}}_{i}}{24} + \frac{\alpha_{{i-1}}^{3} f^{{(3)}}_{i}}{6} + \frac{\alpha_{{i-1}}^{2} f^{{(2)}}_{i}}{2} + \alpha_{{i-1}} f^{{(1)}}_{i} + f_{i}
 \end{equation} 
\begin{equation} 
f_{{i+1}} = \frac{\alpha_{{i+1}}^{6} f^{{(6)}}_{{\xi}}}{720} + \frac{\alpha_{{i+1}}^{5} f^{{(5)}}_{{\xi}}}{120} + \frac{\alpha_{{i+1}}^{4} f^{{(4)}}_{i}}{24} + \frac{\alpha_{{i+1}}^{3} f^{{(3)}}_{i}}{6} + \frac{\alpha_{{i+1}}^{2} f^{{(2)}}_{i}}{2} + \alpha_{{i+1}} f^{{(1)}}_{i} + f_{i}
 \end{equation} 
\begin{equation} 
f_{{i+2}} = \frac{\alpha_{{i+2}}^{6} f^{{(6)}}_{{\xi}}}{720} + \frac{\alpha_{{i+2}}^{5} f^{{(5)}}_{{\xi}}}{120} + \frac{\alpha_{{i+2}}^{4} f^{{(4)}}_{i}}{24} + \frac{\alpha_{{i+2}}^{3} f^{{(3)}}_{i}}{6} + \frac{\alpha_{{i+2}}^{2} f^{{(2)}}_{i}}{2} + \alpha_{{i+2}} f^{{(1)}}_{i} + f_{i}
 \end{equation} 

\subsection{Knowns}
\begin{equation} 
- {\Delta h} , {\Delta h}
 \end{equation} 
\begin{equation} 
f_{i} , f_{{i-1}} , f_{{i+1}}
 \end{equation}
\subsection{Unknowns}
\begin{equation} 
f^{{(1)}}_{i} , f^{{(2)}}_{i} , f^{{(3)}}_{i} , f^{{(4)}}_{i} , f^{{(5)}}_{i} , f^{{(6)}}_{i} , f^{{(7)}}_{{\xi}} , f^{{(8)}}_{{\xi}}
 \end{equation}
\subsection{First Derivative}
\begin{equation} 
f^{{(1)}}_{i} = \frac{- \Delta h_{{i-1}}^{2} f_{i} + \Delta h_{{i-1}}^{2} f_{{i+1}} + \Delta h_{{i}}^{2} f_{i} - \Delta h_{{i}}^{2} f_{{i-1}}}{\Delta h_{{i-1}} \Delta h_{{i}} \left(\Delta h_{{i-1}} + \Delta h_{{i}}\right)}
 \end{equation}\begin{equation} 
\text{Truncation} = \frac{\Delta h_{{i-1}}}{24} \Delta h_{{i}} \left(\Delta h_{{i-1}} f^{{(4)}}_{{\xi}} - \Delta h_{{i}} f^{{(4)}}_{{\xi}} - 4 f^{{(3)}}_{{\xi}}\right)
 \end{equation}
\subsection{Second Derivative}
\begin{equation} 
f^{{(2)}}_{i} = \frac{1}{{\Delta h}^{2}} \left(- 2 f_{i} + f_{{i+1}} + f_{{i-1}}\right)
 \end{equation} 
\begin{equation} 
\text{Truncation} = - \frac{f^{{(3)}}_{{\xi}}}{6} {\Delta h}^{2}
 \end{equation}

\end{document}
