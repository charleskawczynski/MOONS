\documentclass[11pt]{article}
\usepackage{graphicx}    % needed for including graphics e.g. EPS, PS
\usepackage{epstopdf}
\usepackage{amsmath}
\usepackage{hyperref}
\usepackage{xspace}
\usepackage{mathtools}
\usepackage{tikz}
\usepackage{epsfig}
\usepackage{float}
%\usepackage{natbib}
\usepackage{subfigure}
\usepackage{setspace}
\usepackage{tabularx,ragged2e,booktabs,caption}


\setlength{\oddsidemargin}{0.1in}
\setlength{\textwidth}{7.25in}

\setlength{\topmargin}{-1in}     %\topmargin: gap above header
\setlength{\headheight}{0in}     %\headheight: height of header
\setlength{\topskip}{0in}        %\topskip: between header and text
\setlength{\headsep}{0in}        
\setlength{\textheight}{692pt}   %\textheight: height of main text
\setlength{\textwidth}{7.5in}    % \textwidth: width of text
\setlength{\oddsidemargin}{-0.5in}  % \oddsidemargin: odd page left margin
\setlength{\evensidemargin}{0in} %\evensidemargin : even page left margin
\setlength{\parindent}{0.25in}   %\parindent: indentation of paragraphs
\setlength{\parskip}{0pt}        %\parskip: gap between paragraphs
\setlength{\voffset}{0.5in}


% Useful commands:

% \hfill		aligns-right everything right of \hfill

\begin{document}
\doublespacing
\title{Magnetohydrodynamic Object-Oriented Numerical Simulation (MOONS)}
\author{C. Kawczynski \\
Department of Mechanical and Aerospace Engineering \\
University of California Los Angeles, USA\\
}
\maketitle

\section{Maxwell's equations to Induction Equation}
\begin{equation}
	\pmb{j} = \nabla \times \frac{\pmb{B}}{\mu}
	\qquad
	\frac{\partial \pmb{B}}{\partial t} = 
	- \nabla \times \pmb{E}
	\qquad
	\pmb{j} = \sigma (\pmb{E} + \pmb{u}\times \pmb{B})
\end{equation}
\begin{equation}
	\nabla \bullet \pmb{j} = \pmb{0}
	\qquad
	\nabla \bullet \pmb{B} = \pmb{0}
	\qquad
	\pmb{H} = \frac{\pmb{B}}{\mu}
\end{equation}
Solving for the electric field of the current in Ohm's law yields
\begin{equation}
	\pmb{E} = \frac{\pmb{j}}{\sigma} - \pmb{u} \times \pmb{B}
\end{equation}
Plugging this into Faraday's law yields
\begin{equation}
	\frac{\partial \pmb{B}}{\partial t} = 
	- \nabla \times \left[ \frac{\pmb{j}}{\sigma}
	- \pmb{u} \times \pmb{B} \right]
\end{equation}
Distributing we have the induction equation

\section{Non-dimensionalizing by \texorpdfstring{$B_c$}{LG}}

\begin{equation}
	\frac{\partial \pmb{B}}{\partial t} = 
	\nabla \times (\pmb{u} \times \pmb{B})
	- \nabla \times 
	\left[ \frac{1}{\sigma}
	\nabla \times \frac{\pmb{B}}{\mu} \right]
\end{equation}

Non-dimensionalizing this by

\begin{equation*}
	x_k^* = x_k/L_c \qquad
	\nabla^* = L_c \nabla \qquad
	t^* = t/(L_c/U_c) \qquad
	u_k^* = u_k/U_c \qquad
	\end{equation*}
	\begin{equation*}
	B_i^* = B_i/B_c  \qquad
	\sigma^* = \sigma/\sigma_c \qquad
	\mu^* = \mu/\mu_c \qquad
\end{equation*}

Yields

\begin{equation}
	\frac{B_c}{t_c} 
	\frac{\partial B_i}{\partial t} 
	= 
	\frac{U_c B_c}{L_c}
	(\nabla \times u \times B)_i
	- 
	\frac{B_c}{\sigma_c \mu_c L_c^2}
	\left\{
	\nabla \times 
	\left[ \frac{1}{\sigma}
	\nabla \times \frac{B}{\mu} \right]
	\right\}_i
\end{equation}

Dividing by $\frac{B_c}{t_c}$ yields

\begin{equation}
	\frac{\partial B_i}{\partial t} 
	= 
	(\nabla \times u \times B)_i
	- 
	\frac{1}{\sigma_c \mu_c L_c U_c}
	\left\{
	\nabla \times 
	\left[ \frac{1}{\sigma}
	\nabla \times \frac{B}{\mu} \right]
	\right\}_i
\end{equation}

And finally

\begin{equation}
	\boxed{
	\frac{\partial \pmb{B}}{\partial t} = 
	\nabla \times (\pmb{u} \times \pmb{B})
	- 
	\frac{1}{Re_m}
	\nabla \times 
	\left[ \frac{1}{\sigma}
	\nabla \times \frac{\pmb{B}}{\mu} \right]
	}
\end{equation}

Where

\begin{equation}
	Re_m = \sigma_c \mu_c L_c U_c
\end{equation}


\section{Converting to Divergence form}

Using the vector identity
\begin{equation}
	\nabla \times (A\times B) = 
	A(\nabla \bullet B) 
	- B(\nabla \bullet A)
	+(B \bullet \nabla)A - (A \bullet \nabla)B
\end{equation}
And noting that (index notation helps here)
\begin{equation}
	\partial_j (u_i B_j) = 
	u_i \underbrace{\partial_j B_j}_\text{= 0} + B_j \partial_j u_i
	= B_j \partial_j u_i
\end{equation}
\begin{equation}
	\partial_j (u_j B_i) = 
	u_j \partial_j B_i + B_i \underbrace{\partial_j u_j}_\text{=0}
	= u_j \partial_j B_i
\end{equation}
We can write our advective term as
\begin{equation}
	\nabla \times (u \times B) 
	= u(\nabla \bullet B) 
	- B(\nabla \bullet u)
	+ (B \bullet \nabla)u
	- (u \bullet \nabla)B
	=
	\partial_j (u_i B_j) - \partial_j (u_j B_i)
	=
	-\partial_j (u_j B_i - u_i B_j)
\end{equation}

Using the kronecker delta identity, the diffusion term may be written as
\begin{equation}
	\nabla \times \left( \frac{1}{\sigma} \nabla \times \frac{B}{\mu} \right)
	 = \varepsilon_{ijk} \partial_j \left( \frac{1}{\sigma} \varepsilon_{kmn} \partial_m \frac{B_n}{\mu} \right)
	  = \varepsilon_{ijk} \varepsilon_{kmn} \partial_j \left( \frac{1}{\sigma} \partial_m \frac{B_n}{\mu} \right)
	  = (\delta_{im}\delta_{jn} - \delta_{in}\delta_{jm}) \partial_j \left( \frac{1}{\sigma} \partial_m \frac{B_n}{\mu} \right)
\end{equation}

\begin{equation}
	  = \partial_j \left( \frac{1}{\sigma} \partial_i \frac{B_j}{\mu} \right) 
	  - \partial_j \left( \frac{1}{\sigma} \partial_j \frac{B_i}{\mu} \right)
	  = \partial_j \left[ \frac{1}{\sigma} \left( \partial_i \frac{B_j}{\mu} - \partial_j \frac{B_i}{\mu} \right) \right]
\end{equation}

Finally, we may write our full induction equation as
\begin{equation}
	\boxed{
	\frac{\partial B_i}{\partial t} 
	+ \frac{\partial}{\partial x_j} (u_j B_i - u_i B_j) 
	+ \frac{1}{Re_m}
	\frac{\partial}{\partial x_j} 
	\left\{ \frac{1}{\sigma} 
	\left[ 
	\frac{\partial}{\partial x_i} 
	\left( \frac{B_j}{\mu} \right) - 
	\frac{\partial}{\partial x_j} 
	\left( \frac{B_i}{\mu} \right)
	\right]
	\right\} = 0
	}
\end{equation}

Where
\begin{equation}
	\boxed{
	Re_m = \frac{U_c L_c}{(\mu_c \sigma_c)^{-1}} = \mu_c \sigma_c U_c L_c
	}
\end{equation}

This is the dimensionless induction equation written in divergence form. This is the form that is implemented in MOONS.

\subsection{Expanded form}
Sometimes it is helpful to look at the induction equation in a semi-expanded form. Expanding the dummy index $j$ yields the following equation for the $i$th component:

\begin{multline}
	\frac{\partial B_i}{\partial t} + 
	\partial_x (u_x B_i - u_i B_x) + 
	\partial_y (u_y B_i - u_i B_y) + 
	\partial_z (u_z B_i - u_i B_z) + \\
	\frac{1}{Re_m}
	\partial_x \left[ \frac{1}{\sigma} \left( \partial_i \frac{B_x}{\mu} - \partial_x \frac{B_i}{\mu} \right) \right] + 
	\frac{1}{Re_m}
	\partial_y \left[ \frac{1}{\sigma} \left( \partial_i \frac{B_y}{\mu} - \partial_y \frac{B_i}{\mu} \right) \right] + 
	\frac{1}{Re_m}
	\partial_z \left[ \frac{1}{\sigma} \left( \partial_i \frac{B_z}{\mu} - \partial_z \frac{B_i}{\mu} \right) \right]  = \text{source terms}
\end{multline}


\section{Combining Applied and Induced magnetic field equations for finite \texorpdfstring{$Re_m$}{LG}}

The applied magnetic field is non-dimensionalized by $B^0$, which means the equation for the applied magnetic field is

\begin{equation}
	\frac{\partial B_i^0}{\partial t} 
	+ \frac{\partial}{\partial x_j} (u_j B_i^0 - u_i B_j^0) 
	+ \frac{1}{Re_m}
	\frac{\partial}{\partial x_j} 
	\left\{ \frac{1}{\sigma} 
	\left[ 
	\frac{\partial}{\partial x_i} 
	\left( \frac{B_j^0}{\mu} \right) - 
	\frac{\partial}{\partial x_j} 
	\left( \frac{B_i^0}{\mu} \right)
	\right]
	\right\} = 0
\end{equation}

The induced magnetic field is non-dimensionalized by $Re_m B^0$, which means the equation for hte induced magnetic field is


\begin{equation}
	\frac{\partial B_i}{\partial t} 
	+ \frac{\partial}{\partial x_j} (u_j B_i - u_i B_j) 
	+
	\frac{\partial}{\partial x_j} 
	\left\{ \frac{1}{\sigma} 
	\left[ 
	\frac{\partial}{\partial x_i} 
	\left( \frac{B_j}{\mu} \right) - 
	\frac{\partial}{\partial x_j} 
	\left( \frac{B_i}{\mu} \right)
	\right]
	\right\} = 0
\end{equation}

Since $B^{induced}$ scales with $Re_m B^0$, we may combine these equations by writing $induction^{induced}=Re_m \times induction^{applied}$

\begin{multline}
	\frac{\partial B_i^0}{\partial t} +
	\frac{\partial B_i}{\partial t} 
	+ \frac{\partial}{\partial x_j} (u_j B_i - u_i B_j) 
	+ \frac{\partial}{\partial x_j} (u_j B_i^0 - u_i B_j^0) 
	+ \\
	\frac{1}{Re_m}
	\frac{\partial}{\partial x_j} 
	\left\{ \frac{1}{\sigma} 
	\left[ 
	\frac{\partial}{\partial x_i} 
	\left( \frac{B_j^0}{\mu} \right) - 
	\frac{\partial}{\partial x_j} 
	\left( \frac{B_i^0}{\mu} \right)
	\right]
	\right\} 
	+
	\frac{\partial}{\partial x_j} 
	\left\{ \frac{1}{\sigma} 
	\left[ 
	\frac{\partial}{\partial x_i} 
	\left( \frac{B_j}{\mu} \right) - 
	\frac{\partial}{\partial x_j} 
	\left( \frac{B_i}{\mu} \right)
	\right]
	\right\}
	= 0
\end{multline}

It is apparent that one of the equations may be multiplied by a scalar factor before adding to the other equation. But since combining the equations essentially creates source terms when evolving one field, the scalar must be chosen correctly.

\section{Sergey's Equations}

\subsection{Appendix A - Dimensional/Full}

\begin{equation*}
	\frac{\partial B_i^0}{\partial t}
	+
	\frac{\partial B_i}{\partial t} 
	+ 
	\frac{\partial}{\partial x_j} (u_j B_i^0 - u_i B_j^0) 
	+
	\frac{\partial}{\partial x_j} (u_j B_i - u_i B_j) 
	+ 
	\frac{\partial}{\partial x_j} 
	\left\{ \frac{1}{\sigma} 
	\left[ 
	\frac{\partial}{\partial x_i} 
	\left( \frac{B_j}{\mu} \right) 
	- 
	\frac{\partial}{\partial x_j} 
	\left( \frac{B_i}{\mu} \right)
	\right]
	\right\}
	= 
	0
\end{equation*}


\subsection{Appendix B - Dimensionless/ \texorpdfstring{$Re_m<1$}{LG}}

\begin{equation*}
	\frac{\partial B_i^0}{\partial t}
	+
 	Re_m
	\frac{\partial B_i}{\partial t} 
	+ 
	\frac{\partial}{\partial x_j} (u_j B_i^0 - u_i B_j^0) 
	+
 	Re_m
	\frac{\partial}{\partial x_j} (u_j B_i - u_i B_j) 
	+ 
	\frac{\partial}{\partial x_j} 
	\left\{ \frac{1}{\sigma} 
	\left[ 
	\frac{\partial}{\partial x_i} 
	\left( \frac{B_j}{\mu} \right) 
	- 
	\frac{\partial}{\partial x_j} 
	\left( \frac{B_i}{\mu} \right)
	\right]
	\right\}
	= 
	0
\end{equation*}


\subsection{Appendix C - Dimensionless/Quasi-static approximation for \texorpdfstring{$Re_m<<1$}{LG}}

\begin{equation*}
	\frac{\partial}{\partial x_j} (u_j B_i^0 - u_i B_j^0) 
	+ 
	\frac{\partial}{\partial x_j} 
	\left\{ \frac{1}{\sigma} 
	\left[ 
	\frac{\partial}{\partial x_i} 
	\left( \frac{B_j}{\mu} \right) 
	- 
	\frac{\partial}{\partial x_j} 
	\left( \frac{B_i}{\mu} \right)
	\right]
	\right\}
	= 
	0
\end{equation*}

\subsection{Appendix D - Dimensionless/Quasi-static approximation in with uniform \texorpdfstring{$\sigma$}{LG} for \texorpdfstring{$Re_m<<1$}{LG}}

\begin{equation*}
	\frac{\partial}{\partial x_j} (u_j B_i^0 - u_i B_j^0) 
	- 
	\frac{\partial^2}{\partial x_j^2} 
	\left( \frac{B_j}{\mu} \right) 
	= 
	0
\end{equation*}


\subsection{Appendix E - Dimensionless/ \texorpdfstring{$Re_m \approx 1$ or $Re_m>1$}{LG}}

\begin{equation*}
	\frac{\partial B_i^0}{\partial t}
	+
	\frac{\partial B_i}{\partial t} 
	+ 
	\frac{\partial}{\partial x_j} (u_j B_i^0 - u_i B_j^0) 
	+
	\frac{\partial}{\partial x_j} (u_j B_i - u_i B_j) 
	+ 
	\frac{1}{Re_m}
	\frac{\partial}{\partial x_j} 
	\left\{ \frac{1}{\sigma} 
	\left[ 
	\frac{\partial}{\partial x_i} 
	\left( \frac{B_j}{\mu} \right) 
	- 
	\frac{\partial}{\partial x_j} 
	\left( \frac{B_i}{\mu} \right)
	\right]
	\right\}
	= 
	0
\end{equation*}

\end{document}