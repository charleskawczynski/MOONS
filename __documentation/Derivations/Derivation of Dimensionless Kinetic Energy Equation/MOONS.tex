\documentclass[11pt]{article}
\usepackage{graphicx}    % needed for including graphics e.g. EPS, PS
\usepackage{epstopdf}
\usepackage{amsmath}
\usepackage{hyperref}
\usepackage{xspace}
\usepackage{mathtools}
\usepackage{tikz}
\usepackage{epsfig}
\usepackage{float}
%\usepackage{natbib}
\usepackage{subfigure}
\usepackage{setspace}
\usepackage{tabularx,ragged2e,booktabs,caption}
\usepackage{esint}


\newcommand{\A}{\mathbf{A}}
\newcommand{\B}{\mathbf{B}}
\newcommand{\C}{\mathbf{C}}
\newcommand{\PD}{\partial}
\newcommand{\BM}{\frac{\mathbf{B}}{\mu}}
\newcommand{\J}{\mathbf{j}}
\newcommand{\E}{\mathbf{E}}
\newcommand{\N}{\mathbf{n}}
\newcommand{\JS}{\frac{\mathbf{j}}{\sigma}}
\newcommand{\JSS}{\frac{\mathbf{j}^2}{\sigma}}
\newcommand{\U}{\mathbf{u}}
\newcommand{\SII}{\sigma^{-1}}
\newcommand{\SI}{\sigma}
\newcommand{\M}{\mu}
\newcommand{\MO}{\overline{\mu}}
\newcommand{\SO}{\overline{\sigma}}
\newcommand{\DOT}{\text{\textbullet}}

\setlength{\oddsidemargin}{0.1in}
\setlength{\textwidth}{7.25in}

\setlength{\topmargin}{-1in}     %\topmargin: gap above header
\setlength{\headheight}{0in}     %\headheight: height of header
\setlength{\topskip}{0in}        %\topskip: between header and text
\setlength{\headsep}{0in}        
\setlength{\textheight}{692pt}   %\textheight: height of main text
\setlength{\textwidth}{7.5in}    % \textwidth: width of text
\setlength{\oddsidemargin}{-0.5in}  % \oddsidemargin: odd page left margin
\setlength{\evensidemargin}{0in} %\evensidemargin : even page left margin
\setlength{\parindent}{0.25in}   %\parindent: indentation of paragraphs
\setlength{\parskip}{0pt}        %\parskip: gap between paragraphs
\setlength{\voffset}{0.5in}


% Useful commands:

% \hfill		aligns-right everything right of \hfill

\begin{document}
\doublespacing
\title{Magnetohydrodynamic Object-Oriented Numerical Solver (MOONS)}
\author{C. Kawczynski \\
Department of Mechanical and Aerospace Engineering \\
University of California Los Angeles, USA\\
}
\maketitle

\section{Derivation of the Kinetic Energy Equation}
The dimensional momentum equation is

\begin{equation}
	\rho \left( 
	\frac{\PD u_i}{\PD t} + 
	u_j\frac{\PD u_i}{\PD x_j}
	\right)
	= 
	- \frac{\PD p}{\PD x_i}
	+ \frac{\PD}{\PD x_j} \left( \mu \frac{\PD u_i}{\PD x_j} \right)
	+ (j \times B)_i
\end{equation}

Dotting this with the velocity yields

\begin{equation}
	\boxed{
	u_i
	\left(
	\rho \left( 
	\frac{\PD u_i}{\PD t} + 
	u_j\frac{\PD u_i}{\PD x_j}
	\right)
	= 
	- \frac{\PD p}{\PD x_i}
	+ \frac{\PD}{\PD x_j} \left( \mu \frac{\PD u_i}{\PD x_j} \right)
	+ (j \times B)_i
	\right)
	}
\end{equation}

Let's analyze each term, but first, let's define
\begin{equation}
	E_K = \frac{1}{2} \rho u_i u_i
\end{equation}

\subsection{Unsteady term}
\begin{equation}
	\frac{\PD u_i u_i}{\PD t} =
	2 u_i \frac{\PD u_i}{\PD t}
\end{equation}
Therefore
\begin{equation}
	\boxed{
	u_i \rho \frac{\PD u_i}{\PD t} =
	\frac{\PD}{\PD t} \left( \frac{1}{2} \rho u_i u_i \right) =
	\frac{\PD E_K}{\PD t}
	}
\end{equation}

\subsection{Convection term}
\begin{equation}
	u_j \rho \frac{\PD u_i u_i}{\PD x_j} =
	2 \rho u_j u_i \frac{\PD u_i}{\PD x_j}
\end{equation}
Therefore
\begin{equation}
	\boxed{
	\rho u_i u_j \frac{\PD u_i}{\PD x_j} =
	\frac{1}{2} \rho u_j \frac{\PD u_i u_i}{\PD x_j}
	= u_j \frac{\PD E_K}{\PD x_j}
	}
\end{equation}

\subsection{Diffusion Term}
First, let
\begin{equation} \label{eq:strainRate}
	S_{ij} = \frac{1}{2}
	\left(
	\frac{\PD u_i}{\PD x_j} + 
	\frac{\PD u_j}{\PD x_i}
	\right)
\end{equation}
And note that
\begin{equation} \label{eq:strainIdentity}
	2 S_{ij} S_{ij} =
	\frac{1}{2} \left( \frac{\PD u_i}{\PD x_j} + \frac{\PD u_j}{\PD x_i} \right)
	\left( \frac{\PD u_i}{\PD x_j} + \frac{\PD u_j}{\PD x_i} \right)
	= \frac{1}{2} \left( \frac{\PD u_i}{\PD x_j} \frac{\PD u_i}{\PD x_j}
	+ 2 \frac{\PD u_i}{\PD x_j} \frac{\PD u_j}{\PD x_i} 
	+ \frac{\PD u_j}{\PD x_i} \frac{\PD u_j}{\PD x_i} \right)
	= \frac{\PD u_i}{\PD x_j} \frac{\PD u_i}{\PD x_j}
	+ \frac{\PD u_i}{\PD x_j} \frac{\PD u_j}{\PD x_i}
\end{equation}
since no free indexes exist and indexes may be swapped. In addition, we may write
\begin{equation}
	\frac{\PD}{\PD x_j} u_i \frac{\PD u_i}{\PD x_j} =
	\frac{\PD u_i}{\PD x_j}\frac{\PD u_i}{\PD x_j} + 
	u_i \frac{\PD^2 u_i}{\PD x_j^2}
\end{equation}
Therefore
\begin{equation}
	u_i \frac{\PD^2 u_i}{\PD x_j^2} = 
	\frac{\PD}{\PD x_j} \left[ u_i \frac{\PD u_i}{\PD x_j} \right] - 
	\frac{\PD u_i}{\PD x_j}\frac{\PD u_i}{\PD x_j}
\end{equation}
Adding and subtracting $\frac{\PD}{\PD x_j} u_i \frac{\PD u_j}{\PD x_i}$ on the RHS, we have
\begin{equation}
	u_i \frac{\PD^2 u_i}{\PD x_j^2} = 
	\frac{\PD}{\PD x_j} \left[ u_i \frac{\PD u_i}{\PD x_j} + u_i \frac{\PD u_j}{\PD x_i} \right] - 
	\frac{\PD}{\PD x_j} \left[ u_i \frac{\PD u_j}{\PD x_i} \right] - 
	\frac{\PD u_i}{\PD x_j}\frac{\PD u_i}{\PD x_j}
\end{equation}
Using \ref{eq:strainRate}, and expanding $\frac{\PD}{\PD x_j} u_i \frac{\PD u_j}{\PD x_i}$, we have
\begin{equation}
	u_i \frac{\PD^2 u_i}{\PD x_j^2} = 
	\frac{\PD}{\PD x_j} \left[ u_i 2 S_{ij} \right] - 
	\left( 
	\underbrace{u_i \frac{\PD^2 u_j}{\PD x_j x_i}}_{=0 \text{ (swap ij, $\PD_i u_i =0$)}}
	+
	\frac{\PD u_i}{\PD x_j} \frac{\PD u_j}{\PD x_i}
	\right) - 
	\frac{\PD u_i}{\PD x_j}\frac{\PD u_i}{\PD x_j}
\end{equation}
Using \ref{eq:strainIdentity}, we have
\begin{equation}
	u_i \frac{\PD^2 u_i}{\PD x_j^2} = 
	\frac{\PD}{\PD x_j} \left[ 2 u_i S_{ij} \right] - 
	2 S_{ij} S_{ij}
\end{equation}
Swapping the index of the first term on the RHS, and using $S_{ij}=S_{ji}$, we have
\begin{equation}
	u_i \frac{\PD^2 u_i}{\PD x_j^2} = 
	\frac{\PD}{\PD x_i} \left[ 2 u_j S_{ij} \right] - 
	2 S_{ij} S_{ij}
\end{equation}
Finally, we have
\begin{equation}
	\boxed{
	u_i \mu \frac{\PD^2 u_i}{\PD x_j^2} = 
	\frac{\PD}{\PD x_i} \left[ 2 \mu u_j S_{ij} \right] - 
	2 \mu S_{ij} S_{ij}
	}
\end{equation}


\subsection{Pressure term}
\begin{equation}
	\frac{\PD p u_i}{\PD x_i}
	=
	u_i \frac{\PD p}{\PD x_i}
	+
	p \frac{\PD u_i}{\PD x_i}
	=
	u_i \frac{\PD p}{\PD x_i}
\end{equation}

Therefore

\begin{equation}
	\boxed{
	u_i \frac{\PD p}{\PD x_i} =
	\frac{\PD (u_i p)}{\PD x_i}
	}
\end{equation}

\subsection{Lorentz Force term}
We will simply leave the Lorentz force as is
\begin{equation}
	\boxed{
	u_i (j \times B)_i = u_i (j \times B)_i
	}
\end{equation}

\section{Kinetic Energy Equation}
Putting it all together, we have

\begin{equation}
	\frac{\PD E_K}{\PD t} + 
	u_j \frac{\PD E_K}{\PD x_j} = 
	- \frac{\PD (u_i p)}{\PD x_i} + 
	\frac{\PD}{\PD x_i} \left[ 2 \mu u_j S_{ij} \right] - 
	2 \mu S_{ij} S_{ij}
\end{equation}
Combining the compressive viscous stress with pressure, we have
\begin{equation}
	\boxed{
	\underbrace{\frac{\PD E_K}{\PD t} + u_j \frac{\PD E_K}{\PD x_j}}_
	{\substack{\text{Time rate of change of $E_K$}\\\text{following the mean flow}}} +
	\underbrace{\frac{\PD}{\PD x_i} \left( u_i p - 2 \mu u_j S_{ij} \right)}_{\text{Transport}} = 
	- \underbrace{2 \mu S_{ij} S_{ij}}_{\text{Dissipation } \varepsilon} + 
	\underbrace{u_i (j \times B)_i}_{\text{Production } \mathcal P_{em}}
	}
\end{equation}
Where
\begin{equation}
	E_K = \frac{1}{2} \rho u_i u_i
\end{equation}

Note that there is no production term, $P$, since turbulent fluctuations were not accounted for.

\end{document}