\documentclass[landscape]{article}
\usepackage{graphicx}    % needed for including graphics e.g. EPS, PS
\usepackage{epstopdf}
\usepackage{amsmath}
\usepackage{hyperref}
\usepackage{xspace}
\usepackage{mathtools}
\usepackage{tikz}
\usepackage{epsfig}
\usepackage{float}
%\usepackage{natbib}
\usepackage{subfigure}
\usepackage{setspace}
\usepackage{tabularx,ragged2e,booktabs,caption}


% \setlength{\oddsidemargin}{0.1in}
% \setlength{\textwidth}{7.25in}

\setlength{\topmargin}{-1in}     %\topmargin: gap above header
\setlength{\headheight}{0in}     %\headheight: height of header
\setlength{\topskip}{0in}        %\topskip: between header and text
% \setlength{\headsep}{0in}        
\setlength{\textheight}{525pt}   %\textheight: height of main text
\setlength{\textwidth}{10in}    % \textwidth: width of text
\setlength{\oddsidemargin}{-0.5in}  % \oddsidemargin: odd page left margin
\setlength{\evensidemargin}{0in} %\evensidemargin : even page left margin
\setlength{\parindent}{0.25in}   %\parindent: indentation of paragraphs
\setlength{\parskip}{0pt}        %\parskip: gap between paragraphs
% \setlength{\voffset}{0.5in}


% Useful commands:

% \hfill		aligns-right everything right of \hfill

\begin{document}
\doublespacing
\title{Magnetohydrodynamic Object-Oriented Numerical Solver (MOONS)}
\author{C. Kawczynski \\
Department of Mechanical and Aerospace Engineering \\
University of California Los Angeles, USA\\
}
% \maketitle

\section{CG symmetry prefactor}

We would like to look to pre-multiply a diagonal matrix $D$ to the equation $Ax=b$ such that $S = DA$ is symmetric. We cannot simply use $A^T Ax = A^T b$, since this drastically increases the condition number, leading to unnacceptable convergence.

Let

\[
A = 
\left[
\begin{array}{ccccccccc}
a_{1,1} & a_{1,2} & a_{1,3} & a_{1,5} & a_{1,5} \\
a_{2,1} & a_{2,2} & a_{2,3} & a_{2,5} & a_{2,5} \\
a_{3,1} & a_{3,2} & a_{3,3} & a_{3,5} & a_{3,5} \\
a_{4,1} & a_{4,2} & a_{4,3} & a_{4,5} & a_{4,5} \\
a_{5,1} & a_{5,2} & a_{5,3} & a_{5,5} & a_{5,5} \\
\end{array}
\right]
, \qquad
D = 
\left[
\begin{array}{ccccccccc}
c_{1} &   &   &   &  0 \\
  & c_{2} &   &   &   \\
  &   & c_{3} &   &   \\
  &   &   & c_{4} &   \\
 0 &   &   &   & c_{5} \\
\end{array}
\right]
\]

Multiplying these we get

\[
DA = 
\left[
\begin{array}{ccccccccc}
c_1 a_{1,1} & c_1 a_{1,2} & c_1 a_{1,3} & c_1 a_{1,4} & c_1 a_{1,5} \\
c_2 a_{2,1} & c_2 a_{2,2} & c_2 a_{2,3} & c_2 a_{2,4} & c_2 a_{2,5} \\
c_3 a_{3,1} & c_3 a_{3,2} & c_3 a_{3,3} & c_3 a_{3,4} & c_3 a_{3,5} \\
c_4 a_{4,1} & c_4 a_{4,2} & c_4 a_{4,3} & c_4 a_{4,4} & c_4 a_{4,5} \\
c_5 a_{5,1} & c_5 a_{5,2} & c_5 a_{5,3} & c_5 a_{5,4} & c_5 a_{5,5} \\
\end{array}
\right]
\]

We can solve for $c_i,\forall i$ by insisting that $(DA) = (DA)^T$. 
% Let's first assume $c_1 = 1$ (this is simply a refernce).

The number of equations we have, in general, is $(n-1)+(n-2)+ \dots + 1$, which by Gauss' theorem is $\frac{(n-1)((n-1)+1)}{2} = \frac{n(n-1)}{2}$ where $n$ is the number of equations.

The first $n-1$ set of equations that must be solved is

\begin{equation}
  c_1 a_{1,2} = c_2 a_{2,1}, \qquad \qquad
  c_2 a_{2,3} = c_3 a_{3,2}, \qquad \qquad
  c_3 a_{3,4} = c_4 a_{4,3}
\end{equation}

Or, in general
\begin{equation}
  c_i a_{i,i+1} = c_{i+1} a_{i+2,i}, i=1,n-1
\end{equation}

The next set is

\begin{equation}
  c_1 a_{1,3} = c_3 a_{3,1}, \qquad \qquad
  c_2 a_{2,4} = c_4 a_{4,2}, \qquad \qquad
  c_3 a_{3,5} = c_5 a_{5,3}
\end{equation}

Or, in general
\begin{equation}
  c_i a_{i,i+2} = c_{i+2} a_{i+2,i}, i=1,n-2
\end{equation}

It is already clear that all of the equations that must be solved are

\begin{equation}
  c_i a_{i,i+k} = c_{i+k} a_{i+k,i}, i=1,n-1, k=1,n-1
\end{equation}

Since this can be applied recursively, we have

\begin{equation}
  c_{i+k} = c_i a_{i,i+k}/a_{i+k,i}, i=1,n-1, k=1,n-1
\end{equation}




\end{document}