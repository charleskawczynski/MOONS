\documentclass[11pt]{article}
% \documentclass[3p,twocolumn,10pt]{elsarticle}
\usepackage{graphicx}    % needed for including graphics e.g. EPS, PS
\usepackage{epstopdf}
\usepackage{amsmath}
\usepackage{amssymb}
\usepackage{hyperref}
\usepackage{xspace}
\usepackage{mathtools}
\usepackage{tikz}
\usepackage{epsfig}
\usepackage{float}
%\usepackage{natbib}
\usepackage{subfigure}
\usepackage{setspace}
\usepackage{tabularx,ragged2e,booktabs,caption}

\usepackage{xcolor}
\usepackage{xspace}
\usepackage{longtable}
\usepackage{mathtools}
\usepackage{natbib}
\usepackage{setspace}
\usepackage{ragged2e}
\usepackage{etoolbox}
\usepackage{geometry}


\newcommand{\B}{\mathbf{B}}
\newcommand{\BT}{\tilde{\mathbf{B}}}
\renewcommand{\H}{\mathbf{H}}
\newcommand{\C}{\mathbf{C}}
\newcommand{\U}{\mathbf{u}}
\newcommand{\SI}{\sigma}
\newcommand{\M}{\mu}
\newcommand{\curl}{\nabla \times}

\newcommand{\A}{\mathbf{A}}
\newcommand{\PD}{\partial}
\newcommand{\MC}{\mathcal}
\newcommand{\BM}{\frac{\B}{\mu}}
\newcommand{\J}{\mathbf{j}}
\newcommand{\E}{\mathbf{E}}
\newcommand{\N}{\mathbf{n}}
\newcommand{\JS}{\frac{\J}{\sigma}}
\newcommand{\JoS}{\frac{\J^1}{\sigma}}
\newcommand{\JSS}{\frac{\J^2}{\sigma}}
\newcommand{\SII}{\sigma^{-1}}
\newcommand{\MO}{\overline{\mu}}
\newcommand{\SO}{\overline{\sigma}}
\newcommand{\DOT}{\text{\textbullet}}

\newcommand{\Rei}{\frac{1}{Re}}
\newcommand{\Remi}{\frac{1}{Re_m}}
\newcommand{\NRemi}{\frac{N}{Re_m}}

% \setlength{\textfloatsep}{0.1cm}
\newcommand{\volume}{\mathop{\ooalign{\hfil$V$\hfil\cr\kern0.08em--\hfil\cr}}\nolimits}
\newcommand{\figWidth}{0.45\textwidth}
\newcommand{\figSuffix}{_png.png}
\newcommand{\pushright}[1]{\ifmeasuring@#1\else\omit\hfill$\displaystyle#1$\fi\ignorespaces}
\newcommand{\hfillMath}{\hskip \textwidth minus \textwidth}
% \renewcommand{\arraystretch}{0.0} % Removes vertical spaces before/after tabularx


\setlength{\oddsidemargin}{0.1in}
\setlength{\textwidth}{7.25in}

\setlength{\topmargin}{-1in}     %\topmargin: gap above header
\setlength{\headheight}{0in}     %\headheight: height of header
\setlength{\topskip}{0in}        %\topskip: between header and text
\setlength{\headsep}{0in}        
\setlength{\textheight}{692pt}   %\textheight: height of main text
\setlength{\textwidth}{7.5in}    % \textwidth: width of text
\setlength{\oddsidemargin}{-0.5in}  % \oddsidemargin: odd page left margin
\setlength{\evensidemargin}{0in} %\evensidemargin : even page left margin
\setlength{\parindent}{0.25in}   %\parindent: indentation of paragraphs
\setlength{\parskip}{0pt}        %\parskip: gap between paragraphs
\setlength{\voffset}{0.5in}

% Useful commands:

% \hfill    aligns-right everything right of \hfill

\begin{document}
\doublespacing
\title{Shercliff and Hunt flow - velocity decomposition}
\author{C. Kawczynski \\
Department of Mechanical and Aerospace Engineering \\
University of California Los Angeles, USA\\
}
\maketitle

\section{Higher order attempt}
\section{Low but finite Rem Shercliff and Hunt flow}
Here, we derived the equations for a finite magnetic Reynolds number ($Re_m$) Shercliff / Hunt (SH) flow. The dimensionless momentum and induction equations are

\begin{equation}\begin{aligned}
\PD_t u_i+u_j\PD_j u_i =-\PD_i p+ Re^{-1} \PD_{jj} u_i + Re_m^{-1} N (B_j \PD_j B_i - \tfrac{1}{2} \PD_i B_j B_j) \\
\PD_t B_i + \PD_j (u_j B_i - u_i B_j) = Re_m^{-1} \PD_{jj} B_i
\end{aligned} \end{equation}

Where $\U,\B,p,Re,N,Re_m$ are the velocity, magnetic field, pressure, Reynolds, interaction and magnetic Reynolds number respectively. First, let's let the mechanical pressure absorb the magnetic pressure and seek the steady solution.
\begin{equation}\begin{aligned}
u_j\PD_j u_i =-\PD_i p+ Re^{-1} \PD_{jj} u_i + Re_m^{-1} N (B_j \PD_j B_i) \\
Re_m^{-1} \PD_{jj} B_i = \PD_j (u_j B_i - u_i B_j)
\end{aligned} \end{equation}
Since we are interested in when $Re_m$ is small, and the induction equation is linear, we'll combine them by factors of $Re_m$. There is documentation in MOONS how this is done. To be clear we may decompse $\B$ in one of two ways:
\begin{equation}\begin{aligned}
	\B = \B^0 + \B^1 + \B^2 + \hdots , \\
	\B = \B^0 + Re_m \B^* + Re_m^2 \B^{**} + \hdots \\
	\J = \curl \B = \curl \B^* + Re_m \curl \B^{**} + Re_m^2 \curl \B^{***} + \hdots \\
	\J \times \B = (B_j^0+Re_m B_j^*+Re_m^2 B_j^{**} + \hdots) \PD_j (B_i^*+Re_m^1 B_i^{**}+Re_m^2 B_i^{***} + \hdots) \\
	\J \times \B = (\curl \B^* + Re_m \curl \B^{**} + Re_m^2 \curl \B^{***} + \hdots) \times (\B^0 + Re_m \B^* + Re_m^2 \B^{**} + \hdots) \\
\end{aligned} \end{equation}

Now, let's rewrite the induction equation and Lorentz force with this expansion
\begin{equation}\begin{aligned}
u_j\PD_j u_i =-\PD_i p+ Re^{-1} \PD_{jj} u_i + N (B_j^0+Re_m B_j^*) \PD_j (B_i^* + Re_m B_i^{**}) \\
\PD_{jj} B_i^{*}  = \PD_j (u_j B_i^{0} - u_i B_j^{0}) \\
\PD_{jj} B_i^{**} = \PD_j (u_j B_i^{*} - u_i B_j^{*}) \\
\end{aligned} \end{equation}

The third equation corresponds to the secondary magnetic field. Neglecting the $Re_m^2$ term, we have

\begin{equation}\boxed{\begin{aligned}
u_j\PD_j u_i =-\PD_i p+ Re^{-1} \PD_{jj} u_i + N B_j^0 \PD_j B_i^* + N Re_m ( B_j^0 \PD_j B_i^{**} + B_j^* \PD_j B_i^* ) \\
\PD_{jj} B_i^{*}  = \PD_j (u_j B_i^{0} - u_i B_j^{0}) \\
\PD_{jj} B_i^{**} = \PD_j (u_j B_i^{*} - u_i B_j^{*}) \\
\end{aligned}}\end{equation}

Notice that with this formulation, we may set $Re_m$ to be zero \textit{exactly} and still be able to solve for have a physical MHD result since $\B^*$ has absorbed $Re_m$ in the denominator of the Lorentz force term in the momentum equation. This is the low $Re_m$ formulation.

\section{End of finite Rem formulation in B, start of finite Rem formulation in U}
Now, let's expand velocity and pressure with respect to $Re_m$.
\begin{equation}\begin{aligned}
	\U = \U^0 + Re_m \U^* + Re_m^2 \U^{**} + \hdots \\
	p = p^0 + Re_m p^* + Re_m^2 p^{**} + \hdots
\end{aligned} \end{equation}
Plug into the momentum and induction equations.
\begin{equation}\begin{aligned}
(u_j^0+Re_m u_j^*)\PD_j (u_i^0+Re_m u_i^*) =-\PD_i p^0-Re_m\PD_i p^* \\ + Re^{-1} \PD_{jj} (u_i^0+Re_m u_i^*) + N B_j^0 \PD_j B_i^* + N Re_m ( B_j^0 \PD_j B_i^{**} + B_j^* \PD_j B_i^* ) \\
\PD_{jj} B_i^{*}  = \PD_j ((u_j^0+Re_m u_j^*) B_i^{0} - (u_i^0+Re_m u_i^*) B_j^{0}) \\
\PD_{jj} B_i^{**} = \PD_j ((u_j^0+Re_m u_j^*) B_i^{*} - (u_i^0+Re_m u_i^*) B_j^{*}) \\
\end{aligned}\end{equation}
Expanding out and neglect $Re_m^2$ order terms.
\begin{equation}\boxed{\begin{aligned}
u_j^0 \PD_j u_i^0 + Re_m u_j^0 \PD_j u_i^* + Re_m u_j^* \PD_j u_i^0 = -\PD_i p^0 - Re_m\PD_i p^*+  \\
Re^{-1} \PD_{jj} u_i^0 + Re_m Re^{-1} \PD_{jj} u_i^*
+ N B_j^0 \PD_j B_i^* + N Re_m ( B_j^0 \PD_j B_i^{**} + B_j^* \PD_j B_i^* )
\\
\PD_{jj} B_i^{*}  = \PD_j ((u_j^0+Re_m u_j^*) B_i^{0} - (u_i^0+Re_m u_i^*) B_j^{0}) \\
\PD_{jj} B_i^{**} = \PD_j ((u_j^0+Re_m u_j^*) B_i^{*} - (u_i^0+Re_m u_i^*) B_j^{*}) \\
\end{aligned}}\end{equation}


\section{Separate Rem equations}

Let's collect terms of $Re_m$ to end up with two equations for momentum and two equations for induction

Equations for $Re_m^0$
\begin{equation}\boxed{\begin{aligned}
u_j^0 \PD_j u_i^0 = -\PD_i p^0 + Re^{-1} \PD_{jj} u_i^0 + N B_j^0 \PD_j B_i^* \\
\PD_{jj} B_i^* = \PD_j ( u_j^0 B_i^0 - u_i^0 B_j^0)
\end{aligned}}\end{equation}

Equations for $Re_m^1$
\begin{equation}\boxed{\begin{aligned}
u_j^0 \PD_j u_i^* + u_j^* \PD_j u_i^0 = -\PD_i p^*+ Re^{-1} \PD_{jj} u_i^* + N Re_m ( B_j^0 \PD_j B_i^{**} + B_j^* \PD_j B_i^* ) \\
\PD_{jj} B_i^{*}  = Re_m \PD_j (u_j^* B_i^{0} - u_i^* B_j^{0}) \\
\PD_{jj} B_i^{**} = Re_m \PD_j (u_j^* B_i^{*} - u_i^* B_j^{*}) \\
\end{aligned}}\end{equation}

\section{Apply duct flow conditions}
Let's consider fully developed duct flow, where $\PD_x()$ except for pressure, which we'll assume varies linearly. 
Keeping in mind to group magnetic pressure with mechanical pressure, we have

Equations for $Re_m^0$
\begin{equation}\begin{aligned}
v^0 \PD_y u^0 + w^0 \PD_z u^0 = -\PD_x p^0 + Re^{-1} (\PD_{yy} u^0 + \PD_{zz} u^0 ) + N (B_y^0 \PD_y B_x^* + B_z^0 \PD_z B_x^*) \\
v^0 \PD_y v^0 + w^0 \PD_z v^0 = -\PD_y p^0 + Re^{-1} (\PD_{yy} v^0 + \PD_{zz} v^0 ) + N  B_z^0 \PD_z B_y^* \\
v^0 \PD_y w^0 + w^0 \PD_z w^0 = -\PD_z p^0 + Re^{-1} (\PD_{yy} w^0 + \PD_{zz} w^0 ) + N  B_y^0 \PD_y B_z^* \\
\PD_{yy} B_x^* + \PD_{zz} B_x^* = \PD_y ( v^0 B_x^0 - u^0 B_y^0) - \PD_z ( w^0 B_x^0 - u^0 B_z^0) \\
\PD_{yy} B_y^* + \PD_{zz} B_y^* = \PD_z ( w^0 B_y^0 - v^0 B_z^0) \\
\PD_{yy} B_z^* + \PD_{zz} B_z^* = \PD_y ( v^0 B_z^0 - w^0 B_y^0) \\
\end{aligned}\end{equation}

Equations for $Re_m^1$
\begin{equation}\begin{aligned}
v^* \PD_y u^0 + w^* \PD_z u^0 = -\PD_x p^*+ Re^{-1} (\PD_{yy} u^* + \PD_{zz} u^*) + N Re_m (B_z^0 \PD_z B_x^{**} + B_y^* \PD_y B_x^* + B_z^* \PD_z B_x^* ) \\
v^* \PD_y v^0 + w^* \PD_z v^0 = -\PD_y p^*+ Re^{-1} (\PD_{yy} v^* + \PD_{zz} v^*) + N Re_m (B_z^0 \PD_z B_y^{**} + B_y^* \PD_y B_y^* + B_z^* \PD_z B_y^* ) \\
v^* \PD_y w^0 + w^* \PD_z w^0 = -\PD_z p^*+ Re^{-1} (\PD_{yy} w^* + \PD_{zz} w^*) + N Re_m (B_z^0 \PD_z B_z^{**} + B_y^* \PD_y B_z^* + B_z^* \PD_z B_z^* ) \\
Re_m^{-1} \nabla^2 B_x^{*}  = \PD_y (v^* B_x^{0} - u^* B_y^{0}) + \PD_z (w^* B_x^{0} - u^* B_z^{0}) \\
Re_m^{-1} \nabla^2 B_y^{*}  = \PD_z (w^* B_y^{0} - v^* B_z^{0}) \\
Re_m^{-1} \nabla^2 B_z^{*}  = \PD_y (v^* B_z^{0} - w^* B_y^{0}) \\
Re_m^{-1} \nabla^2 B_x^{**} = \PD_y (v^* B_x^{*} - u^* B_y^{*}) + \PD_z (w^* B_x^{*} - u^* B_z^{*}) \\
Re_m^{-1} \nabla^2 B_y^{**} = \PD_z (w^* B_y^{*} - v^* B_z^{*}) \\
Re_m^{-1} \nabla^2 B_z^{**} = \PD_y (v^* B_z^{*} - w^* B_y^{*}) \\
\end{aligned}\end{equation}

\section{Separate by SH and secondary flow}
Applying $B_x^0=0,B_y^0=0,v^0=0, w^0=0,B_z=\text{uniform}$ yields the SH governing equations and secondary flow governing equations

Equations for $Re_m^0$
\begin{equation}\boxed{\begin{aligned}
0 = -\PD_x p^0 + Re^{-1} (\PD_{yy} u^0 + \PD_{zz} u^0 ) + N B_z^0 \PD_z B_x^* \\
\PD_{yy} B_x^* + \PD_{zz} B_x^* = - \PD_z (u^0 B_z^0) \\
\end{aligned}} \qquad \text{Shercliff / Hunt governing equations}\end{equation}

Equations for $Re_m^1$
\begin{equation}\begin{aligned}
v^* \PD_y u^0 + w^* \PD_z u^0 = -\PD_x p^*+ Re^{-1} (\PD_{yy} u^* + \PD_{zz} u^*) + N Re_m (B_z^0 \PD_z B_x^{**} + B_y^* \PD_y B_x^* + B_z^* \PD_z B_x^* ) \\
0                             = -\PD_y p^*+ Re^{-1} (\PD_{yy} v^* + \PD_{zz} v^*) + N Re_m (B_z^0 \PD_z B_y^{**} + B_y^* \PD_y B_y^* + B_z^* \PD_z B_y^* ) \\
0                             = -\PD_z p^*+ Re^{-1} (\PD_{yy} w^* + \PD_{zz} w^*) + N Re_m (B_z^0 \PD_z B_z^{**} + B_y^* \PD_y B_z^* + B_z^* \PD_z B_z^* ) \\
Re_m^{-1} \nabla^2 B_x^{*}  =-\PD_z (u^* B_z^{0}) \\
Re_m^{-1} \nabla^2 B_y^{*}  =-\PD_z (v^* B_z^{0}) \\
Re_m^{-1} \nabla^2 B_z^{*}  = \PD_y (v^* B_z^{0}) \\
Re_m^{-1} \nabla^2 B_x^{**} = \PD_y (v^* B_x^{*} - u^* B_y^{*}) - \PD_z (u^* B_z^{*}) \\
Re_m^{-1} \nabla^2 B_y^{**} = \PD_z (w^* B_y^{*} - v^* B_z^{*}) \\
Re_m^{-1} \nabla^2 B_z^{**} = \PD_y (v^* B_z^{*} - w^* B_y^{*}) \\
\end{aligned}\end{equation}

Where $u^0, B_x^*$ are solutions from low $Re_m$ SH flows. With 4 unknowns $v,w,p,B_y^*,B_z^*$ and the above 4 equations
plus $\PD_j B_j = 0, \PD_j u_j = 0$ we can solve for the secondary flow. Let's look more closely at the induction equation.

\begin{equation}\begin{aligned}
0 =  B_y^{*} \PD_y u^0 + u^0 \PD_y B_y^{*} + u^0 \PD_z B_z^{*} + B_z^{*} \PD_z u^0 + u^* \PD_z B_z^0 + B_z^0 \PD_z u^* \\
0 =  B_y^{*} \PD_y u^0 + u^0 \PD_y B_y^{*} + u^0 \PD_z B_z^{*} + B_z^{*} \PD_z u^0 + B_z^0 \PD_z u^* \\
0 =  B_y^{*} \PD_y u^0 + u^0 (\PD_y B_y^{*} + \PD_z B_z^{*}) + B_z^{*} \PD_z u^0 + B_z^0 \PD_z u^* \\
0 =  B_y^{*} \PD_y u^0 + B_z^{*} \PD_z u^0 + B_z^0 \PD_z u^* \\
\end{aligned}\end{equation}

So the governing equations are

\begin{equation}\begin{aligned}
v^* \PD_y u^0 + w^* \PD_z u^0 = -\PD_x p^* + Re^{-1} (\PD_{yy} u^* + \PD_{zz} u^*) + N (B_y^* \PD_y B_x^* + B_z^* \PD_z B_x^*) \\
0 = -\PD_y p^* + Re^{-1} (\PD_{yy} v^* + \PD_{zz} v^*) + N B_z^* \PD_z B_y^* \\
0 = -\PD_z p^* + Re^{-1} (\PD_{yy} w^* + \PD_{zz} w^*) + N B_y^* \PD_y B_z^* \\
0 =  B_y^{*} \PD_y u^0 + B_z^{*} \PD_z u^0 + B_z^0 \PD_z u^* \\
\end{aligned}\end{equation}
Where
\begin{equation}\begin{aligned}
B_z^0 = 1 \\
u^0 = u^{Shercliff /  Hunt} \\
B_x^* = B_x^{Shercliff /  Hunt} \\
u^*,v^*,w^*,p^*,B_y^*,B_z^* \rightarrow \text{due to finite $Re_m$ effects} \\
\end{aligned}\end{equation}



Now, we analyze the physics. Sine axial velocity has the largest component, the largest induced magnetic field is also axial.

The strength of this induced magnetic field is of order $Re_m$. The total magnetic field becomes slightly out of 
the transverse plane as the low $Re_m$ is relaxed. This makes the Lorentz force slightly out of the transverse plane
and "swirls" the fluid. We will assume here that this transverse velocity induces an out of plane magnetic field that 
is negligibly small. So we may neglect the induction equation. 

\begin{equation}\begin{aligned}
v^* \PD_y u^0 + w^* \PD_z u^0 = -\PD_x p^* + Re^{-1} (\PD_{yy} u^* + \PD_{zz} u^*) + N (B_y^* \PD_y B_x^* + B_z^* \PD_z B_x^*) \\
0 = -\PD_y p^* + Re^{-1} (\PD_{yy} v^* + \PD_{zz} v^*) + N B_z^* \PD_z B_y^* \\
0 = -\PD_z p^* + Re^{-1} (\PD_{yy} w^* + \PD_{zz} w^*) + N B_y^* \PD_y B_z^* \\
\end{aligned}\end{equation}
We can see here that the axial velocity is decoupled from the transverse component. We can solve the transverse component
and use the transverse velocity solution to solve the axial velocity solution. 
Let's take the curl and look at vorticity $(\omega_x = \PD_y w - \PD_z v)$
\begin{equation}\begin{aligned}
0 = Re^{-1} (\PD_{yy} \omega_x^* + \PD_{zz} \omega_x^*) 
+ N \left[ \PD_y (B_y^* \PD_y B_z^*) 
- \PD_z (B_z^* \PD_z B_y^*) \right] \\
\end{aligned}\end{equation}

\end{document}