\documentclass[11pt]{article}
\usepackage{graphicx}    % needed for including graphics e.g. EPS, PS
\usepackage{epstopdf}
\usepackage{amsmath}
\usepackage{hyperref}
\usepackage{xspace}
\usepackage{mathtools}
\usepackage{tikz}
\usepackage{epsfig}
\usepackage{float}
%\usepackage{natbib}
\usepackage{subfigure}
\usepackage{setspace}
\usepackage{tabularx,ragged2e,booktabs,caption}

\newcommand{\B}{\mathbf{B}}
\newcommand{\C}{\mathbf{C}}
\newcommand{\U}{\mathbf{u}}
\newcommand{\SI}{\sigma}
\newcommand{\M}{\mu}

\newcommand{\A}{\mathbf{A}}
\newcommand{\PD}{\partial}
\newcommand{\MC}{\mathcal}
\newcommand{\BM}{\frac{\B}{\mu}}
\newcommand{\J}{\mathbf{j}}
\newcommand{\E}{\mathbf{E}}
\newcommand{\N}{\mathbf{n}}
\newcommand{\JS}{\frac{\J}{\sigma}}
\newcommand{\JoS}{\frac{\J^1}{\sigma}}
\newcommand{\JSS}{\frac{\J^2}{\sigma}}
\newcommand{\SII}{\sigma^{-1}}
\newcommand{\MO}{\overline{\mu}}
\newcommand{\SO}{\overline{\sigma}}
\newcommand{\DOT}{\text{\textbullet}}

\setlength{\oddsidemargin}{0.1in}
\setlength{\textwidth}{7.25in}

\setlength{\topmargin}{-1in}     %\topmargin: gap above header
\setlength{\headheight}{0in}     %\headheight: height of header
\setlength{\topskip}{0in}        %\topskip: between header and text
\setlength{\headsep}{0in}        
\setlength{\textheight}{692pt}   %\textheight: height of main text
\setlength{\textwidth}{7.5in}    % \textwidth: width of text
\setlength{\oddsidemargin}{-0.5in}  % \oddsidemargin: odd page left margin
\setlength{\evensidemargin}{0in} %\evensidemargin : even page left margin
\setlength{\parindent}{0.25in}   %\parindent: indentation of paragraphs
\setlength{\parskip}{0pt}        %\parskip: gap between paragraphs
\setlength{\voffset}{0.5in}


% Useful commands:

% \hfill		aligns-right everything right of \hfill

\begin{document}
\doublespacing
\title{Magnetohydrodynamic Object-Oriented Numerical Solver (MOONS)}
\author{C. Kawczynski \\
Department of Mechanical and Aerospace Engineering \\
University of California Los Angeles, USA\\
}
% \maketitle

\section{Induction equation by orders of induced magnetic field}
The dimensional induction equation is
\begin{equation} \label{eq:ind_dim}
  \frac{\PD \B}{\PD t} 
  =
  (\nabla \times \U \times \B)
  - \\
  \left(
  \nabla \times
  \left\{
  \frac{1}{\sigma}
  \nabla \times
  \frac{\B}{\mu}
  \right\}
  \right)
\end{equation}
Non-dimensionalizing by $t,u,\nabla , \mu, \sigma$, we have
\begin{equation} \label{eq:ind}
  \frac{\PD \B}{\PD t} 
  =
  (\nabla \times \U \times \B)
  - \\
  \frac{1}{Re_m}
  \left(
  \nabla \times
  \left\{
  \frac{1}{\sigma}
  \nabla \times
  \frac{\B}{\mu}
  \right\}
  \right)
  ,\qquad Re_m = \sigma \mu LU
\end{equation}
Note that in the above equation, only $t,u,\nabla , \mu, \sigma$ have been non-dimensionalized and not $\B$. Let
\begin{equation}
  G(\B)
  =
  \frac{\PD \B}{\PD t} 
  -
  (\nabla \times \U \times \B)
  +
  \frac{1}{Re_m}
  \left(
  \nabla \times
  \left\{
  \frac{1}{\sigma}
  \nabla \times
  \frac{\B}{\mu}
  \right\}
  \right)
  =
  \mathbf{0}
\end{equation}
Since this equation applies to the external applied magnetic field as well as any induced magnetic fields, we may write (generally)
\begin{equation}
	\sum_{k=0}^{\infty} G(\B^k)
	=
	\mathbf{0}
\end{equation}
Considering the first two terms we have
\begin{equation}
  G(\B^0)
  +
  G(\B^1)
  =
  \mathbf{0}
\end{equation}
For a uniform and steady external field ($\B^0$), this simplifies to
\begin{equation}
  \frac{\PD \B^1}{\PD t} 
  -
  (\nabla \times \U \times \B^1)
  +
  \frac{1}{Re_m}
  \left(
  \nabla \times
  \left\{
  \frac{1}{\sigma}
  \nabla \times
  \frac{\B^1}{\mu}
  \right\}
  \right)
  -
  (\nabla \times \U \times \B^0)
  =
  \mathbf{0}
\end{equation}
This is similar to the formulation with normalization, but the strength of the induced magnetic field is $\B^1$.

\end{document}