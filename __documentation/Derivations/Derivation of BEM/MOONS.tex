\documentclass[11pt]{article}
\usepackage{graphicx}    % needed for including graphics e.g. EPS, PS
\usepackage{epstopdf}
\usepackage{amsmath}
\usepackage{hyperref}
\usepackage{xspace}
\usepackage{mathtools}
\usepackage{tikz}
\usepackage{epsfig}
\usepackage{float}
%\usepackage{natbib}
\usepackage{subfigure}
\usepackage{setspace}
\usepackage{tabularx,ragged2e,booktabs,caption}


\setlength{\oddsidemargin}{0.1in}
\setlength{\textwidth}{7.25in}

\setlength{\topmargin}{-1in}     %\topmargin: gap above header
\setlength{\headheight}{0in}     %\headheight: height of header
\setlength{\topskip}{0in}        %\topskip: between header and text
\setlength{\headsep}{0in}        
\setlength{\textheight}{692pt}   %\textheight: height of main text
\setlength{\textwidth}{7.5in}    % \textwidth: width of text
\setlength{\oddsidemargin}{-0.5in}  % \oddsidemargin: odd page left margin
\setlength{\evensidemargin}{0in} %\evensidemargin : even page left margin
\setlength{\parindent}{0.25in}   %\parindent: indentation of paragraphs
\setlength{\parskip}{0pt}        %\parskip: gap between paragraphs
\setlength{\voffset}{0.5in}
\newcommand{\myBullet}{\text{\textbullet}}

% Useful commands:

% \hfill		aligns-right everything right of \hfill

\begin{document}
\doublespacing
\title{Magnetohydrodynamic Object-Oriented Numerical Solver (MOONS)}
\author{C. Kawczynski \\
Department of Mechanical and Aerospace Engineering \\
University of California Los Angeles, USA\\
}
% \maketitle

\section{Boundary Element Method (BEM)}
Here, I follow the work of \cite{Iskakov2004}. Consider boundary $\Gamma$. The tangential component of the electric field on the boundary is
\begin{equation}
  E_{\tau} = -\mathbf{\tau} \myBullet (\mathbf{u} \times \mathbf{B})_{\Gamma} + 
  Re_m^{-1} \mathbf{\tau} \myBullet (\nabla \times \mathbf{B})_{\Gamma}
\end{equation}
If we assume $\mathbf{u}_{\Gamma} \myBullet \mathbf{n} = 0$ then this simplifies to
\begin{equation}
  E_{\tau} = -\mathbf{\tau} \myBullet (\mathbf{u} \times (B_n \mathbf{n})_{\Gamma}) + 
  Re_m^{-1} \mathbf{\tau} \myBullet (\nabla \times \mathbf{B})_{\Gamma}
\end{equation}
% \begin{equation}
%   (\nabla \times \mathbf{B})_{\Gamma} = \lim_{r \rightarrow \Gamma} (\nabla \times \mathbf{B}(\mathbf{r}))
% \end{equation}
The first term requires knowledge of $B_n$ only and therefore requires no special difficulty. The second term may be approximated using a 2nd order accurate stencil, but the tangential component of the magnetic field is still unknown. This will be determined from the normal component and from proper boundary conditions.
\section{Integral formulation at the boundary}
The tangential component of the magnetic field depends on the values of its normal component everywhere at the boundary.

Consider matching a potential field as an elliptic problem in $\Omega^c$, the complementary domain of $\Omega$
\begin{equation}
  \mathbf{B} = -\nabla \phi, \qquad \Delta \phi = 0.
\end{equation}
Where $\phi: \Omega^c \rightarrow R$ is the potential function. At infinity, the physical condition for the magnetic field is
\begin{equation}
  \phi \rightarrow O(r^{-2}), \qquad r \rightarrow \infty
\end{equation}
The normal component of B is known, which implies a Neumann BC on the potential:
\begin{equation}
  \frac{\partial \phi}{\partial n}_{\Gamma} = - B_n \qquad (B_n: \Gamma \rightarrow R)
\end{equation}
Expressing this as surface integrals over the fundamental solution $\Delta G = \delta(\mathbf{x},\mathbf{y})$
\begin{equation}
  G(\mathbf{x},\mathbf{y}) = \frac{-1}{4 \pi |\mathbf{x}-\mathbf{y}|}
\end{equation}

Denoting an open ball, $B$, of radius $R$ such that $\overline{\Omega}\in B$, as a consequence of Green's theorem, the magnetic potential at the boundary satisfies
\begin{equation}
  \phi(\mathbf{x}) = 
  -2 \int_{\Gamma} \left( \phi(\mathbf{y})
  \frac{\partial G}{\partial n} + B_n(\mathbf{y}) G \right) ds(\mathbf{y})
  +2 \int_{\partial B_R} \left( \phi(\mathbf{y})
  \frac{\partial G}{\partial \tilde{n}} - \frac{\partial \phi(\mathbf{y}}{\partial \tilde{n}} G \right) ds(\mathbf{y})
\end{equation}
Where $n$ is the coordinate along the outward normal to $\Omega$ and $\tilde{n}$ is the coordinate along the outward normal to the ball. The second term vanishes when $R\rightarrow \infty$ (keeping the center of the ball fixed). So we may write
\begin{equation} \label{eq:phi}
  \phi(\mathbf{x}) = 
  -2 \int_{\Gamma} \left( \phi(\mathbf{y})
  \frac{\partial G}{\partial n} + B_n(\mathbf{y}) G \right) ds(\mathbf{y})
\end{equation}
Consequently, the tangential component of the magnetic field on $\Gamma$ along the unit vector $\tau$ is 
\begin{equation}
  B_{\tau} = - \mathbf{\tau} \myBullet \nabla \phi(\mathbf{x}) = 
  2 \mathbf{\tau} \myBullet
  \int_{\Gamma}
  \left(
  \phi(\mathbf{y}) \frac{\partial G}{\partial n}
  + B_n(\mathbf{y}) \nabla_x G
  \right) ds(\mathbf{y})
\end{equation}
The discrete form of this may be assembeled as 
\begin{equation}
  \frac{1}{2} \Phi = A \Phi + C B_n
\end{equation}
With the resulting solution
\begin{equation}
  \left(\frac{I}{2} - A\right)\Phi = C B_n, \qquad
  \rightarrow
  \Phi = \left(\frac{I}{2} - A\right)^{-1} C B_n
\end{equation}
This can be computed once at the beginning of the sim. Returning to \ref{eq:phi}, and $G(\mathbf{x}_i,\mathbf{y}) = \frac{-1}{4\pi |\mathbf{x} - \mathbf{y}|}$ we may write the discrete form as
\begin{equation}
\begin{split}
\phi(\mathbf{x}_i) & = -2 \sum_j \phi_j \int_{S_j} \frac{\partial G(\mathbf{x}_i,\mathbf{y})}{\partial n} ds(\mathbf{y})
           -2 \sum_j B_{nj} \int_{S_j} G(\mathbf{x}_i,\mathbf{y}) ds(\mathbf{y}) \\
       & = \frac{1}{2 \pi} 
       \left[
       \sum_j \phi_j \int_{S_j} \frac{\partial}{\partial n} \frac{1}{|\mathbf{x}_i-\mathbf{y}|} ds(\mathbf{y})
     + \sum_j B_{nj} \int_{S_j} \frac{1}{|\mathbf{x}_i-\mathbf{y}|} ds(\mathbf{y})
       \right] \\
\end{split}
\end{equation}


\begin{thebibliography}{1}
\bibitem{Iskakov2004} Iskakov, A. B., Descombes, S. and Dormy, E. An integro-differential formulation for magnetic induction in bounded domains: boundary element–finite volume method. J. Comput. Phys. 197, 540–554 (2004)..
\end{thebibliography}

\end{document}