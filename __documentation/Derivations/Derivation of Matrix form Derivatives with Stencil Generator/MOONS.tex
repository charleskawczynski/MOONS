\documentclass[landscape]{article}
\usepackage{graphicx}    % needed for including graphics e.g. EPS, PS
\usepackage{epstopdf}
\usepackage{amsmath}
\usepackage{hyperref}
\usepackage{xspace}
\usepackage{mathtools}
\usepackage{tikz}
\usepackage{epsfig}
\usepackage{float}
%\usepackage{natbib}
\usepackage{subfigure}
\usepackage{setspace}
\usepackage{tabularx,ragged2e,booktabs,caption}


% \setlength{\oddsidemargin}{0.1in}
% \setlength{\textwidth}{7.25in}

\setlength{\topmargin}{-1in}     %\topmargin: gap above header
\setlength{\headheight}{0in}     %\headheight: height of header
\setlength{\topskip}{0in}        %\topskip: between header and text
% \setlength{\headsep}{0in}        
\setlength{\textheight}{525pt}   %\textheight: height of main text
\setlength{\textwidth}{10in}    % \textwidth: width of text
\setlength{\oddsidemargin}{-0.5in}  % \oddsidemargin: odd page left margin
\setlength{\evensidemargin}{0in} %\evensidemargin : even page left margin
\setlength{\parindent}{0.25in}   %\parindent: indentation of paragraphs
\setlength{\parskip}{0pt}        %\parskip: gap between paragraphs
% \setlength{\voffset}{0.5in}


% Useful commands:

% \hfill		aligns-right everything right of \hfill

\begin{document}
\doublespacing
\title{Magnetohydrodynamic Object-Oriented Numerical Solver (MOONS)}
\author{C. Kawczynski \\
Department of Mechanical and Aerospace Engineering \\
University of California Los Angeles, USA\\
}
\maketitle

\section{Matrix form}
The equations from the python stencil generator can be put into matrix form as follows:

\[ Af = \left[
\begin{array}{ccccccccc}
L_{1} & D_{1}    & U_{1}     &           &           &           &         \\
L_{2} & D_{2}    & U_{2}     &           &           &           &         \\
      & \ddots   & \ddots    & \ddots    &           &           &         \\
      &          & L_{i-1}   & D_{i-1}   & U_{i-1}   &           &         \\
      &          &           & \ddots    & \ddots    & \ddots    &         \\
      &          &           &           & L_{s-3}   & D_{s-3}   & U_{s-3} \\
      &          &           &           & L_{s-2}   & D_{s-2}   & U_{s-2} \\
\end{array} \right] 
\left[ \begin{array}{c}
f_{2} \\ \vdots \\ f_{i-1} \\ f_{i} \\ f_{i+1} \\ \vdots \\ f_{s-1}
\end{array} \right]
=
\left[ \begin{array}{c}
f'_{2} \\ \vdots \\ f'_{i-1} \\ f'_{i} \\ f'_{i+1} \\ \vdots \\ f'_{s-1}
\end{array} \right]
\]

Note that the field $f$ starts at the second index which refers to the boundary value for node data (due to one ghost node) and the first interior cell for cell centered data (due to the ghost cell). Indexing between the function and diagonals must be
\begin{equation}
  L_{i-1} : f_{i-1}
  , \qquad
  D_{i-1} : f_{i}
  , \qquad
  U_{i-1} : f_{i+1}
  , \qquad
  i \in (2,s-1)
\end{equation}

Results shown in the next section were obtained by running the python stencil maker and matching the appropriate indexes of the function and its coefficients.

\section{Node Data}
For node data, one-sided difference schemes must be used at the boundary.
\subsection{1st Derivative}
Interior:
\begin{equation} 
f^{{(1)}}_{i} = \left(- \frac{\Delta h_{{i}}}{\Delta h_{{i-1}} \left(\Delta h_{{i-1}} + \Delta h_{{i}}\right)}\right)f_{{i-1}}+ \left(\frac{- \Delta h_{{i-1}} + \Delta h_{{i}}}{\Delta h_{{i-1}} \Delta h_{{i}}}\right)f_{{i}}+ \left(\frac{\Delta h_{{i-1}}}{\Delta h_{{i}} \left(\Delta h_{{i-1}} + \Delta h_{{i}}\right)}\right)f_{{i+1}}
  , \qquad
  i \in (3,s-2)
 \end{equation} 
Front boundary:
\begin{equation} 
f^{{(1)}}_{i} = 
\left(- \frac{\Delta h_{{i+1}} + 2 \Delta h_{{i}}}{\Delta h_{{i}} \left(\Delta h_{{i+1}} + \Delta h_{{i}}\right)}\right) f_{{i}} + \left(\frac{\Delta h_{{i+1}} + \Delta h_{{i}}}{\Delta h_{{i+1}} \Delta h_{{i}}}\right) f_{{i+1}} + \left(- \frac{\Delta h_{{i}}}{\Delta h_{{i+1}} \left(\Delta h_{{i+1}} + \Delta h_{{i}}\right)}\right) f_{{i+2}}
 \end{equation} 
Back boundary:
\begin{equation} 
f^{{(1)}}_{i} = \left(\frac{\Delta h_{{i-1}}}{\Delta h_{{i-2}} \left(\Delta h_{{i-1}} + \Delta h_{{i-2}}\right)}\right)f_{{i-2}}+ \left(- \frac{\Delta h_{{i-1}} + \Delta h_{{i-2}}}{\Delta h_{{i-1}} \Delta h_{{i-2}}}\right)f_{{i-1}}+ \left(\frac{2 \Delta h_{{i-1}} + \Delta h_{{i-2}}}{\Delta h_{{i-1}} \left(\Delta h_{{i-1}} + \Delta h_{{i-2}}\right)}\right)f_{{i}}
 \end{equation} 

\subsection{2nd Derivative}
Interior:
\begin{equation} 
f^{{(2)}}_{i} = \left(\frac{2}{\Delta h_{{i-1}} \left(\Delta h_{{i-1}} + \Delta h_{{i}}\right)}\right)f_{{i-1}}+ \left(- \frac{2}{\Delta h_{{i-1}} \Delta h_{{i}}}\right)f_{{i}}+ \left(\frac{2}{\Delta h_{{i}} \left(\Delta h_{{i-1}} + \Delta h_{{i}}\right)}\right)f_{{i+1}}
, \qquad
  i \in (3,s-2)
 \end{equation} 
Front boundary:
\begin{equation} 
f^{{(2)}}_{i} = \left(\frac{2}{\Delta h_{{i}} \left(\Delta h_{{i+1}} + \Delta h_{{i}}\right)}\right)f_{{i}}+ \left(- \frac{2}{\Delta h_{{i+1}} \Delta h_{{i}}}\right)f_{{i+1}}+ \left(\frac{2}{\Delta h_{{i+1}} \left(\Delta h_{{i+1}} + \Delta h_{{i}}\right)}\right)f_{{i+2}}
 \end{equation} 
Back boundary:
\begin{equation} 
f^{{(2)}}_{i} = \left(\frac{2}{\Delta h_{{i-2}} \left(\Delta h_{{i-1}} + \Delta h_{{i-2}}\right)}\right)f_{{i-2}}+ \left(- \frac{2}{\Delta h_{{i-1}} \Delta h_{{i-2}}}\right)f_{{i-1}}+ \left(\frac{2}{\Delta h_{{i-1}} \left(\Delta h_{{i-1}} + \Delta h_{{i-2}}\right)}\right)f_{{i}}
 \end{equation}

\newpage
\section{Cell centered Data}
For cell centered data, a centered difference schemes must be used at the boundary but with modified stencil near the boundary. In order to achieve this, the ghost and first interior cell are interpolated to give the boundary value, which is used with the modified stencil.

\subsection{1st Derivative}
Interior and boundaries with modified stencil:
\begin{equation} 
f^{{(1)}}_{i} = \left(- \frac{\Delta h_{{i}}}{\Delta h_{{i-1}} \left(\Delta h_{{i-1}} + \Delta h_{{i}}\right)}\right)f_{{i-1}}+ \left(\frac{- \Delta h_{{i-1}} + \Delta h_{{i}}}{\Delta h_{{i-1}} \Delta h_{{i}}}\right)f_{{i}}+ \left(\frac{\Delta h_{{i-1}}}{\Delta h_{{i}} \left(\Delta h_{{i-1}} + \Delta h_{{i}}\right)}\right)f_{{i+1}}
  , \qquad
  i \in (3,s-2)
 \end{equation} 

\subsection{2nd Derivative}
Interior and boundaries with modified stencil:
\begin{equation} 
f^{{(2)}}_{i} = \left(\frac{2}{\Delta h_{{i-1}} \left(\Delta h_{{i-1}} + \Delta h_{{i}}\right)}\right)f_{{i-1}}+ \left(- \frac{2}{\Delta h_{{i-1}} \Delta h_{{i}}}\right)f_{{i}}+ \left(\frac{2}{\Delta h_{{i}} \left(\Delta h_{{i-1}} + \Delta h_{{i}}\right)}\right)f_{{i+1}}
, \qquad
  i \in (3,s-2)
 \end{equation} 

\section{Cell centered Data (alternative form)}
\subsection{1st Derivative}
\begin{equation} 
f^{{(1)}}_{i} = \left(\frac{\alpha_{{i+1}}}{\alpha_{{i-1}} \left(\alpha_{{i+1}} - \alpha_{{i-1}}\right)}\right)f_{{i-1}}+ \left(- \frac{\alpha_{{i+1}} + \alpha_{{i-1}}}{\alpha_{{i+1}} \alpha_{{i-1}}}\right)f_{{i}}+ \left(- \frac{\alpha_{{i-1}}}{\alpha_{{i+1}} \left(\alpha_{{i+1}} - \alpha_{{i-1}}\right)}\right)f_{{i+1}}
 \end{equation} 

\subsection{2nd Derivative}
\begin{equation} 
f^{{(2)}}_{i} = \left(- \frac{2}{\alpha_{{i-1}} \left(\alpha_{{i+1}} - \alpha_{{i-1}}\right)}\right)f_{{i-1}}+ \left(\frac{2}{\alpha_{{i+1}} \alpha_{{i-1}}}\right)f_{{i}}+ \left(\frac{2}{\alpha_{{i+1}} \left(\alpha_{{i+1}} - \alpha_{{i-1}}\right)}\right)f_{{i+1}}
 \end{equation} 



\end{document}