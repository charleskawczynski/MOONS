\documentclass[landscape]{article}
\usepackage{graphicx}    % needed for including graphics e.g. EPS, PS
\usepackage{epstopdf}
\usepackage{amsmath}
\usepackage{hyperref}
\usepackage{xspace}
\usepackage{mathtools}
\usepackage{tikz}
\usepackage{epsfig}
\usepackage{float}
%\usepackage{natbib}
\usepackage{subfigure}
\usepackage{setspace}
\usepackage{tabularx,ragged2e,booktabs,caption}


% \setlength{\oddsidemargin}{0.1in}
% \setlength{\textwidth}{7.25in}

\setlength{\topmargin}{-1in}     %\topmargin: gap above header
\setlength{\headheight}{0in}     %\headheight: height of header
\setlength{\topskip}{0in}        %\topskip: between header and text
% \setlength{\headsep}{0in}        
\setlength{\textheight}{525pt}   %\textheight: height of main text
\setlength{\textwidth}{10in}    % \textwidth: width of text
\setlength{\oddsidemargin}{-0.5in}  % \oddsidemargin: odd page left margin
\setlength{\evensidemargin}{0in} %\evensidemargin : even page left margin
\setlength{\parindent}{0.25in}   %\parindent: indentation of paragraphs
\setlength{\parskip}{0pt}        %\parskip: gap between paragraphs
% \setlength{\voffset}{0.5in}


% Useful commands:

% \hfill		aligns-right everything right of \hfill

\begin{document}
\doublespacing
\title{Magnetohydrodynamic Object-Oriented Numerical Simulation (MOONS)}
\author{C. Kawczynski \\
Department of Mechanical and Aerospace Engineering \\
University of California Los Angeles, USA\\
}
\maketitle

\section{Matrix form}
The equations from the python stencil generator can be put into matrix form as follows:

\[ Af = \left[
\begin{array}{ccccccccc}
D_{1} & U_{1} &  &   &   &   &   &   & 0 \\
L_{1} & D_{2} & U_{2} & UU  &   &   &   &   &   \\
  & L_{2} & D_{3} & U_{3} &   &   &   &   &   \\
  &   & \ddots & \ddots & \ddots &   &   &   &   \\
  &   &   & L_{i-1} & D_{i} & U_{i} &   &   &   \\
  &   &   &   & \ddots & \ddots & \ddots &   &   \\
  &   &   &   &   & L_{s-3} & D_{s-2} & U_{s-2} &   \\
  &   &   &   &   & LL & L_{s-2} & D_{s-1} & U_{s-1} \\
0 &   &   &   &   &   &  & L_{s-1} & D_{s}
\end{array} \right] \left[ \begin{array}{c}
f_{1} \\ f_{2} \\ \vdots \\ f_{i-1} \\ f_{i} \\ f_{i+1} \\ \vdots \\ f_{s-1} \\ f_{s}
\end{array} \right] 
= \left( \frac{\partial f}{\partial h} \right)_{i} 
\]
The derivative on all ghost points are zero, so let's use the interior only:
\[ Af = \left[
\begin{array}{ccccccccc}
0 & 0 &  &   &   &   &   &   & 0 \\
L_{1} & D_{2} & U_{2} & UU  &   &   &   &   &   \\
  & L_{2} & D_{3} & U_{3} &   &   &   &   &   \\
  &   & \ddots & \ddots & \ddots &   &   &   &   \\
  &   &   & L_{i-1} & D_{i} & U_{i} &   &   &   \\
  &   &   &   & \ddots & \ddots & \ddots &   &   \\
  &   &   &   &   & L_{s-3} & D_{s-2} & U_{s-2} &   \\
  &   &   &   &   & LL & L_{s-2} & D_{s-1} & U_{s-1} \\
0 &   &   &   &   &   &  & 0 & 0
\end{array} \right] \left[ \begin{array}{c}
f_{1} \\ f_{2} \\ \vdots \\ f_{i-1} \\ f_{i} \\ f_{i+1} \\ \vdots \\ f_{s-1} \\ f_{s}
\end{array} \right] 
= \left( \frac{\partial f}{\partial h} \right)_{i} 
\]
Reducing the size of this matrix and re-numbering, we have
\[ Af = \left[
\begin{array}{ccccccccc}
L_{1} & D_{1} & U_{1} &  &   &   &   &   &   \\
  & L_{2} & D_{2} & U_{2} &   &   &   &   &   \\
  &   & \ddots & \ddots & \ddots &   &   &   &   \\
  &   &   & L_{i-1} & D_{i-1} & U_{i-1} &   &   &   \\
  &   &   &   & \ddots & \ddots & \ddots &   &   \\
  &   &   &   &   & L_{s-3} & D_{s-3} & U_{s-3} &   \\
  &   &   &   &   & & L_{s-2} & D_{s-2} & U_{s-2} \\
\end{array} \right] \left[ \begin{array}{c}
f_{2} \\ \vdots \\ f_{i-1} \\ f_{i} \\ f_{i+1} \\ \vdots \\ f_{s-1}
\end{array} \right] 
= \left( \frac{\partial f}{\partial h} \right)_{i} 
\]

Which means that indexing between the function and diagonals must be
\begin{equation}
  L_{i-1} : f_{i-1}
  , \qquad
  D_{i-1} : f_{i}
  , \qquad
  U_{i-1} : f_{i+1}
  , \qquad
  i \in (2,s-1)
\end{equation}

Results shown in the next section were obtained by running the python stencil maker and matching the appropriate indexes of the function and its coefficients.

\subsection{1st Derivative (node data)}
For node data, one-sided difference schemes must be used at the boundary.
\subsubsection{Generator Results}
Interior:
\begin{equation} 
f^{{(1)}}_{i} = \left(- \frac{\Delta h_{{i}}}{\Delta h_{{i-1}} \left(\Delta h_{{i-1}} + \Delta h_{{i}}\right)}\right)f_{{i-1}}+ \left(\frac{- \Delta h_{{i-1}} + \Delta h_{{i}}}{\Delta h_{{i-1}} \Delta h_{{i}}}\right)f_{{i}}+ \left(\frac{\Delta h_{{i-1}}}{\Delta h_{{i}} \left(\Delta h_{{i-1}} + \Delta h_{{i}}\right)}\right)f_{{i+1}}
  , \qquad
  i \in (2,s-1)
 \end{equation} 
Front boundary:
\begin{equation} 
f^{{(1)}}_{i} = 
\left(- \frac{\Delta h_{{i+1}} + 2 \Delta h_{{i}}}{\Delta h_{{i}} \left(\Delta h_{{i+1}} + \Delta h_{{i}}\right)}\right) f_{{i}} + \left(\frac{\Delta h_{{i+1}} + \Delta h_{{i}}}{\Delta h_{{i+1}} \Delta h_{{i}}}\right) f_{{i+1}} + \left(- \frac{\Delta h_{{i}}}{\Delta h_{{i+1}} \left(\Delta h_{{i+1}} + \Delta h_{{i}}\right)}\right) f_{{i+2}}
 \end{equation} 
Back boundary:
\begin{equation} 
f^{{(1)}}_{i} = \left(\frac{\Delta h_{{i-1}}}{\Delta h_{{i-2}} \left(\Delta h_{{i-1}} + \Delta h_{{i-2}}\right)}\right)f_{{i-2}}+ \left(- \frac{\Delta h_{{i-1}} + \Delta h_{{i-2}}}{\Delta h_{{i-1}} \Delta h_{{i-2}}}\right)f_{{i-1}}+ \left(\frac{2 \Delta h_{{i-1}} + \Delta h_{{i-2}}}{\Delta h_{{i-1}} \left(\Delta h_{{i-1}} + \Delta h_{{i-2}}\right)}\right)f_{{i}}
 \end{equation} 

\subsubsection{Matrix form}
Interior matrix:
\begin{equation}
  L_{i-1} = \left(- \frac{\Delta h_{{i}}}{\Delta h_{{i-1}} \left(\Delta h_{{i-1}} + \Delta h_{{i}}\right)}\right)
  , \qquad
  D_{i-1} = \left(\frac{- \Delta h_{{i-1}} + \Delta h_{{i}}}{\Delta h_{{i-1}} \Delta h_{{i}}}\right)
  , \qquad 
  U_{i-1} = \left(\frac{\Delta h_{{i-1}}}{\Delta h_{{i}} \left(\Delta h_{{i-1}} + \Delta h_{{i}}\right)}\right)
  , \qquad
  i \in (2,s-1)
\end{equation}
Front boundary:

\begin{equation}
  L_{1} = \left(- \frac{\Delta h_{{i+1}} + 2 \Delta h_{{i}}}{\Delta h_{{i}} \left(\Delta h_{{i+1}} + \Delta h_{{i}}\right)}\right)
  :f_2
  , \qquad
  D_{1} = \left(\frac{\Delta h_{{i+1}} + \Delta h_{{i}}}{\Delta h_{{i+1}} \Delta h_{{i}}}\right)
  :f_3
  , \qquad 
  U_{1} = \left(- \frac{\Delta h_{{i}}}{\Delta h_{{i+1}} \left(\Delta h_{{i+1}} + \Delta h_{{i}}\right)}\right)
  :f_4
\end{equation}
Back boundary:

\begin{equation}
  L_{s-2} = \left(\frac{\Delta h_{{i-1}}}{\Delta h_{{i-2}} \left(\Delta h_{{i-1}} + \Delta h_{{i-2}}\right)}\right)
  , \qquad
  D_{s-2} = \left(- \frac{\Delta h_{{i-1}} + \Delta h_{{i-2}}}{\Delta h_{{i-1}} \Delta h_{{i-2}}}\right)
  , \qquad 
  U_{s-2} = \left(\frac{2 \Delta h_{{i-1}} + \Delta h_{{i-2}}}{\Delta h_{{i-1}} \left(\Delta h_{{i-1}} + \Delta h_{{i-2}}\right)}\right)
\end{equation}

\subsection{2nd Derivative}
\subsubsection{Generator Results}
\subsubsection{Matrix form}
\begin{multline} 
\begin{aligned} 
f^{{(2)}}_{i} = \left(\frac{2}{\Delta h_{{i}} \left(\Delta h_{{i+1}} + \Delta h_{{i}}\right)}\right)f_{i}+ \\ \left(- \frac{2}{\Delta h_{{i+1}} \Delta h_{{i}}}\right)f_{{i+1}}+ \\ \left(\frac{2}{\Delta h_{{i+1}} \left(\Delta h_{{i+1}} + \Delta h_{{i}}\right)}\right)f_{{i+2}}
 \end{aligned}
 \end{multline} 


\end{document}