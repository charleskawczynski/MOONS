\documentclass[11pt]{article}
\usepackage{graphicx}    % needed for including graphics e.g. EPS, PS
\usepackage{epstopdf}
\usepackage{amsmath}
\usepackage{hyperref}
\usepackage{xspace}
\usepackage{mathtools}
\usepackage{tikz}
\usepackage{epsfig}
\usepackage{float}
%\usepackage{natbib}
\usepackage{subfigure}
\usepackage{setspace}
\usepackage{tabularx,ragged2e,booktabs,caption}


\setlength{\oddsidemargin}{0.1in}
\setlength{\textwidth}{7.25in}

\setlength{\topmargin}{-1in}     %\topmargin: gap above header
\setlength{\headheight}{0in}     %\headheight: height of header
\setlength{\topskip}{0in}        %\topskip: between header and text
\setlength{\headsep}{0in}        
\setlength{\textheight}{692pt}   %\textheight: height of main text
\setlength{\textwidth}{7.5in}    % \textwidth: width of text
\setlength{\oddsidemargin}{-0.5in}  % \oddsidemargin: odd page left margin
\setlength{\evensidemargin}{0in} %\evensidemargin : even page left margin
\setlength{\parindent}{0.25in}   %\parindent: indentation of paragraphs
\setlength{\parskip}{0pt}        %\parskip: gap between paragraphs
\setlength{\voffset}{0.5in}


% Useful commands:

% \hfill		aligns-right everything right of \hfill

\begin{document}
\doublespacing
\title{Magnetohydrodynamic Object-Oriented Numerical Solver (MOONS)}
\author{C. Kawczynski \\
Department of Mechanical and Aerospace Engineering \\
University of California Los Angeles, USA\\
}
\maketitle

\section{Induction Equation by Orders of Rem with Intrinsic Normalization}
The dimensionless induction equation is
\begin{equation}
  \label{eq:ind}
  \frac{\partial \pmb{B}}{\partial t} 
  =
  (\nabla \times \pmb{u} \times \pmb{B})
  - \\
  \frac{1}{Re_m}
  \left(
  \nabla \times
  \left\{
  \frac{1}{\sigma}
  \nabla \times
  \frac{\pmb{B}}{\mu}
  \right\}
  \right)
\end{equation}
Note that in the above equation, only $t,u,\nabla , \mu, \sigma$ have been non-dimensionalized and not $B$. Let

\begin{equation}
  G(\pmb{B})
  =
  \frac{\partial \pmb{B}}{\partial t} 
  -
  (\nabla \times \pmb{u} \times \pmb{B})
  +
  \frac{1}{Re_m}
  \left(
  \nabla \times
  \left\{
  \frac{1}{\sigma}
  \nabla \times
  \frac{\pmb{B}}{\mu}
  \right\}
  \right)
  =
  \pmb{0}
\end{equation}

Since this equation applies to the external applied magnetic field as well as any induced magnetic fields, we may write (generally)

\begin{equation}
	\sum_{k=0}^{\infty} G(\pmb{B}^k)
	=
	\pmb{0}
\end{equation}

Considering the first two terms we have
\begin{equation}
  G(\pmb{B}^0)
  +
  G(\pmb{B}^{ind})
  =
  \pmb{0}
\end{equation}

For a uniform and steady external field ($\pmb{B}^0$), this simplifies to


\begin{equation}
  \frac{\partial \pmb{B}^{ind}}{\partial t} 
  -
  (\nabla \times \pmb{u} \times \pmb{B}^{ind})
  +
  \frac{1}{Re_m}
  \left(
  \nabla \times
  \left\{
  \frac{1}{\sigma}
  \nabla \times
  \frac{\pmb{B}^{ind}}{\mu}
  \right\}
  \right)
  -
  (\nabla \times \pmb{u} \times \pmb{B}^0)
  =
  \pmb{0}
\end{equation}

This is similar to the formulation without the intrinsic normalization, however, multiplying by $Re_m$ is not necessary during post processing nor in the momentum equation. This is clearly more straightforward.

\end{document}