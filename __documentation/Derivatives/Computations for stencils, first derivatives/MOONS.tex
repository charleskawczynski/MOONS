\documentclass[11pt]{article}
\newcommand{\VAR}{Success}


\begin{document}
\doublespacing
\MOONSTITLE
\maketitle

\section{Stencils for first derivatives}
This document uses results from the non-uniform grid stencil results.

\subsection{Staggered}
By staggered derivative, what is meant is

if the field $f\in CC$ then $\frac{\PD f}{\PD h}\in N$.

Likewise

if the field $f\in N$ then $\frac{\PD f}{\PD h}\in CC$.

Where $CC$ and $N$ mean cell center and cell corner data. To put in words, if data lives on the primary grid, then its derivative lives on the dual grid. That is what a staggered derivative means in MOONS.

\begin{equation}
	f_{p,i+1/2}' = \frac{f_{p,i+1}-f_{p,i}}{h_{p,i+1}-h_{p,i}} \qquad i=1,sf-1
\end{equation}

Where the subscript $p$ means primary grid, and $sf$ means the size of $f$. Note that the half index is used. To remove the half index, we may write this as

\begin{equation}
	f_{d,i+gt}' = \frac{f_{p,i+1}-f_{p,i}}{h_{p,i+1}-h_{p,i}} \qquad i=1,sf-1
\end{equation}

Where
 \begin{equation}
   gt = gridType =
  \begin{cases}
      0 & \text{if $f \in $ cell center} \\
      1 & \text{if $f \in $ cell corner} \\
   \end{cases}
\end{equation}
Note that the result, $f'$, lives on the dual grid (hence the subscript $d$). Also note that $f'$ is not defined for the ghost cell ($f_1'$). MOONS sets ghost cells to be zero when computing derivatives, the applyBCs module sets these cells to enforce BCs.


\subsection{Collocated}
For the collocated case (when both $f$ and $\frac{\PD f}{\PD h}$ live on the primary grid), MOONS uses the results from the non-uniform grid-stencils.

THESE FORMULAS NEED TO BE CHECKED FOR CC VS NODE DATA SINCE THERE EXISTS A GHOST CELL. THIS VERY WELL MAY CHANGE THE NECESSARY INDEX FOR CC VS N DATA. THIS SHOULD BE SIMPLE TO FIX, SIMPLY ADD A gt TO THE INDEXING FOR $\alpha_k$ and $\beta_j$.

\subsubsection{Central differencing}

Use the non-uniform grid stencil results with

\begin{equation}
	\alpha_k = -(x_{i} - x_{i-1}) = \Delta x_{i} \qquad k = -1
\end{equation}
\begin{equation}
	\beta_j = (x_{i+1} - x_{i}) = \Delta x_{i+1} \qquad j = 1
\end{equation}


\subsubsection{Forward differencing (soon to be obsolete)}

Use the non-uniform grid stencil results with

\begin{equation}
	\alpha_k = -(x_{i+1} - x_{i}) = \Delta x_{i+1} \qquad k = 1
\end{equation}
\begin{equation}
	\beta_j = x_{i+2} - x_{i} = (x_{i+2} - x_{i+1}) + (x_{i+1} - x_{i}) = \Delta x_{i+1} + \Delta x_{i+2} \qquad j = 2
\end{equation}

\subsubsection{Backward differencing (soon to be obsolete)}

Use the non-uniform grid stencil results with

\begin{equation}
	\alpha_k = -(x_{i} - x_{i-1}) = -\Delta x_{i} \qquad k = -1
\end{equation}
\begin{equation}
	\beta_j = -(x_{i} - x_{i-2}) = -(\Delta x_{i} + \Delta x_{i-1}) \qquad j = -2
\end{equation}

\end{document}