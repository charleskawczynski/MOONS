\documentclass[11pt]{article}
\usepackage{graphicx}    % needed for including graphics e.g. EPS, PS
\usepackage{epstopdf}
\usepackage{amsmath}
\usepackage{hyperref}
\usepackage{xspace}
\usepackage{mathtools}
\usepackage{tikz}
\usepackage{epsfig}
\usepackage{float}
\usepackage{natbib}
\usepackage{subfigure}
\usepackage{setspace}
\usepackage{tabularx,ragged2e,booktabs,caption}


\setlength{\oddsidemargin}{0.1in}
\setlength{\textwidth}{7.25in}

\setlength{\topmargin}{-1in}     %\topmargin: gap above header
\setlength{\headheight}{0in}     %\headheight: height of header
\setlength{\topskip}{0in}        %\topskip: between header and text
\setlength{\headsep}{0in}        
\setlength{\textheight}{692pt}   %\textheight: height of main text
\setlength{\textwidth}{7.5in}    % \textwidth: width of text
\setlength{\oddsidemargin}{-0.5in}  % \oddsidemargin: odd page left margin
\setlength{\evensidemargin}{0in} %\evensidemargin : even page left margin
\setlength{\parindent}{0.25in}   %\parindent: indentation of paragraphs
\setlength{\parskip}{0pt}        %\parskip: gap between paragraphs
\setlength{\voffset}{0.5in}


% Useful commands:

% \hfill		aligns-right everything right of \hfill

\begin{document}
\doublespacing
\title{Magnetohydrodynamic Object-Oriented Numerical Solver (MOONS)}
\author{C. Kawczynski \\
Department of Mechanical and Aerospace Engineering \\
University of California Los Angeles, USA\\
}
% \maketitle

We would like to transpose the cell centered derivative operator

\subsection{Cell centered data. System:
\texorpdfstring{$ (s-2 \times s) \times (s \times 1) = s-2 \times 1$}{}}

\[ Af = \left[
\begin{array}{ccccccccc}
0 & 0    &       &           &           &           &       &         & \\
L_{1} & D_{1}    & U_{1}     &           &           &           &       &         & \\
      & L_{2} & D_{2}    & U_{2}     &           &           &           &         & \\
      &       & \ddots   & \ddots    & \ddots    &           &           &         & \\
      &       &          & L_{i-1}   & D_{i-1}   & U_{i-1}   &           &         & \\
      &       &          &           & \ddots    & \ddots    & \ddots    &         & \\
      &       &          &           &           & L_{s-3}   & D_{s-3}   & U_{s-3} & \\
      &      &       &          &           &           & L_{s-2}   & D_{s-2}   & U_{s-2} \\
  &      &       &           &           &           &       &      0  & 0 \\
\end{array} \right] 
\left[ \begin{array}{c}
f_{1} \\ f_{2} \\ \vdots \\ f_{i-1} \\ f_{i} \\ f_{i+1} \\ \vdots \\ f_{s-1} \\ f_{s}
\end{array} \right]
\]

We get

\[ A^Tf = \left[
\begin{array}{ccccccccc}
0 & L_{1}    &       &           &           &           &       &         & \\
0 & D_{1}    & L_{2}     &           &           &           &       &         & \\
      & U_{1} & D_{2}    & L_{3}     &           &           &           &         & \\
      &       & \ddots   & \ddots    & \ddots    &           &           &         & \\
      &       &          & L_{i-2}   & D_{i-1}   & U_{i}     &           &         & \\
      &       &          &           & \ddots    & \ddots    & \ddots    &         & \\
      &       &          &           &           & U_{s-4}   & D_{s-3}   & L_{s-2} & \\
      &      &       &          &           &           & U_{s-3}   & D_{s-2}   & 0 \\
  &      &       &           &           &           &       &      U_{s-2}  & 0 \\
\end{array} \right] 
\left[ \begin{array}{c}
f_{1} \\ f_{2} \\ \vdots \\ f_{i-1} \\ f_{i} \\ f_{i+1} \\ \vdots \\ f_{s-1} \\ f_{s}
\end{array} \right]
\]


\end{document}