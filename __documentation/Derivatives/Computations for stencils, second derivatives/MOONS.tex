\documentclass[11pt]{article}
\newcommand{\PSCHAIN}{..}
\edef\PSCHAIN{\PSCHAIN/LATEX_INCLUDES}

\newcommand{\rootdir}{\PSCHAIN}
\newcommand{\bibdir}{\rootdir/includes/bib}

\usepackage{graphicx}
\usepackage{adjustbox}
% \usepackage{epstopdf} % Does not work with pdflatex
\usepackage{amsmath}
\usepackage{amssymb}
\usepackage{hyperref}
\usepackage{xspace}
\usepackage{mathtools}
\usepackage{tikz}
\usepackage{epsfig}
\usepackage{float}
\usepackage{subfigure}
\usepackage{setspace}
\usepackage{tabularx}
\usepackage{multirow}
\usepackage{ragged2e}
\usepackage{booktabs}
\usepackage{caption}
% \usepackage{subcaption}
\usepackage{xcolor}
\usepackage{longtable}
% \usepackage{natbib} % Sometimes causes errors
\usepackage{etoolbox}
\usepackage{geometry}
\usepackage{esint}
\usepackage{ifxetex}
\usepackage{ifluatex}
% \usepackage{siunitx}
\usepackage{numprint}
\usepackage{empheq}
\usepackage{slashbox}


 % For local use
\newcommand{\figWidth}{0.45\textwidth}
%\newcommand{\figW}{6.5in}
%\newcommand{\figH}{5.5in}
\newcommand{\figW}{5.in}
\newcommand{\figH}{4.in}
%\newcommand{\figW}{6.825in}
%\newcommand{\figH}{5.775in}
\newcommand{\ffigW}{3.1in}
\newcommand{\ffigH}{3.1in}

\newcommand{\captionWidth}{0.45}
\newcommand{\subFigPad}{0.52\textwidth} % good for subcaption
\newcommand{\TWOFIGCOLUMN}{0.52\textwidth} % good for subcaption
\newcommand{\TWOFIGCOLUMNCAPTION}{0.52\textwidth} % good for subcaption
% \newcommand{\subFigPad}{0.45} % good for subfig
\newcommand{\miniPageSize}{.1\textwidth}

\newcommand{\subFigSize}{16em} % good for subcaption
\newcommand{\figSize}{0.45\textwidth}


% \renewcommand{\contcaption}{%
%   \expandafter\addtocounter\expandafter{\@captype alt}{\m@ne}% Step alias cntr back
%   \expandafter\refstepcounter\expandafter{\@captype alt}% Make reference
%   \@contcaption\@captype}

% \definecolor{darkGreen}{rgb}{0.0, 0.5, 0.0}
\newcommand{\COR}{\color{red}}
\newcommand{\COG}{\color{darkGreen}}
\newcommand{\COM}{\color{magenta}}
\newcommand{\COK}{\color{black}}
\newcommand{\COB}{\color{blue}}

\newcommand{\toth}{\tiny{Toth, Gabor. "The $\DIV \B = 0$ constraint in shock-capturing magnetohydrodynamics codes." J. Comput. Phys. 161.2 (2000): 605-652.}}
\newcommand{\pattison}{\tiny{M. J. Pattison, K. N. Premnath, N. B. Morley, M. Abdou, Fusion Eng. Des. 83 (2008) 557-572.}}
\newcommand{\gujStella}{\tiny{G. Guj, F. Stella, J. Comput. Phys. 298 (1993) 286-298.}}
\newcommand{\bandaru}{\tiny{V. Bandaru, J. Pracht, T. Boeck, J. Schumacher, Theor. Comp. Fluid Dyn. (2015).}}
\newcommand{\guermond}{\tiny{Guermond, J. L., J. Léorat, and C. Nore. "A new Finite Element Method for magneto-dynamical problems: two-dimensional results." European Journal of Mechanics-B/Fluids 22.6 (2003): 555-579.}}
\newcommand{\kawczynski}{\tiny{C. Kawczynski, S. Smolentsev, M. Abdou. "A Lid-Driven Cavity MHD Flow Numerical Study at Moderate Magnetic Reynolds Number using Proper Magnetic Boundary Conditions". In Progress.}}
\newcommand{\roache}{\tiny{P. J. Roache, Annu. Rev. Fluid Mech. 29 (1997) 123-160.}}
\newcommand{\hunt}{\tiny{J. C. R. Hunt, J. Fluid Mech. 21 (1965) 577-590.}}
\newcommand{\shercliff}{\tiny{J. A. Shercliff, Proc. Camb. Phil. Soc. 49 (1953) 136.}}
\newcommand{\turner}{\tiny{Turner, Larry R. "Electromagnetic computations for fusion devices." IEEE Transactions on Magnetics 26.2 (1990): 847-852.}}
\newcommand{\kotulskiA}{\tiny{Kotulski, Joseph D., et al. "Electromagnetic analysis of forces and torques on the baseline and enhanced ITER shield modules due to plasma disruption." IEEE Transactions on Plasma Science 38.4 (2010): 1047.}}
\newcommand{\kotulskiB}{\tiny{Kotulski, Joseph D., Rebecca Coats, and Michael Ulrickson. "The analysis of the electromagnetic loads on selected ITER blanket shield modules due to induced eddy and halo currents." Fusion Engineering (SOFE), 2011 IEEE/NPSS 24th Symposium on. IEEE, 2011.}}
\newcommand{\iskakov}{\tiny{Iskakov, A. B., S. Descombes, and E. Dormy. "An integro-differential formulation for magnetic induction in bounded domains: boundary element-finite volume method." J. Comput. Phys. 197.2 (2004): 540-554.}}
\newcommand{\mikeulrickson}{\tiny{We received numerical data from Dr. Mike Ulrickson, computed from the plasma code DINA.}}
\newcommand{\Leriche}{Leriche, E. \& Gavrilakis, S. Direct numerical simulation of the flow in a lid-driven cubical cavity. Phys. Fluids 12, (2000).}
\newcommand{\Brackbill}{\tiny{Brackbill, Jeremiah U., and Daniel C. Barnes. "The effect of nonzero $\DIV \B = 0$ on the numerical solution of the magnetohydrodynamic equations." Journal of Computational Physics 35.3 (1980): 426-430.}}
\newcommand{\Leuer}{\tiny{Leuer, J. A., and J. C. Wesley. "ITER plasma start-up modeling." Fusion Engineering, 1993., 15th IEEE/NPSS Symposium on. Vol. 2. IEEE, 1993.}}

% \newcommand{\NJ}{New Jersey}
% \newcommand{\UCC}{Union County College}
% \newcommand{\GSR}{Graduate Student Researcher}
% \newcommand{\ME}{Mechanical Engineering}
% \newcommand{\NJSGC}{New Jersey Space Grant Consortium}
% % \newcommand{\NASA}{National Aeronautics and Space Administation}
% \newcommand{\NASA}{NASA}
% \newcommand{\SAS}{School of Arts and Sciences}
% % \newcommand{\UCLA}{University of California Los Angeles}
% \newcommand{\UCLA}{UCLA}
% \newcommand{\IDRE}{Institute for Digital Research and Education}
% \newcommand{\MAE}{Department of Mechanical and Aerospace Engineering}
% \newcommand{\UCLAMAE}{Henri Samueli School Of Engineering And Applied Sciences}
% % \newcommand{\MHD}{Magnetohydrodynamic}
% \newcommand{\MHDs}{Magnetohydrodynamics}
% % \newcommand{\RUTGERS}{Rutgers University, The State University of New Jersey}
% \newcommand{\RUTGERS}{Rutgers University}
% \newcommand{\RU}{Rutgers University}
% \newcommand{\US}{United States}

\newcommand{\height}{0.4}
\newcommand{\radius}{0.1}
\newcommand{\offSet}{12}
\newcommand{\Deltah}{1.6666}
\lstset{language=[90]Fortran,
  basicstyle=\ttfamily,
  keywordstyle=\color{red},
  commentstyle=\color{green},
  morecomment=[l]{!\ }% Comment only with space after !
}
\newif\ifxetexorluatex
\ifxetex
  \renewcommand{\C}{\mathbf{C}}
  \renewcommand{\U}{\mathbf{u}}
  \renewcommand{\G}{\mathbf{g}}
\else
  \newcommand{\C}{\mathbf{C}}
  \newcommand{\U}{\mathbf{u}}
  \newcommand{\G}{\mathbf{g}}
\fi
\renewcommand{\H}{\mathbf{H}}

\newcommand{\Rem}{Re_m}
\newcommand{\Rm}{$Re_m$}
\newcommand{\B}{\mathbf{B}}
\newcommand{\cleanB}{\xi}
\newcommand{\R}{\mathbf{r}}
\newcommand{\T}{\mathbf{T}}
\newcommand{\BBx}{\bar{B_x}}
\newcommand{\BBy}{\bar{B_y}}
\newcommand{\BBz}{\bar{B_z}}
\newcommand{\PB}{\bar{p}}
\newcommand{\UBx}{\bar{u}}
\newcommand{\VBy}{\bar{v}}
\newcommand{\WBz}{\bar{w}}
\newcommand{\E}{\mathbf{E}}
\newcommand{\J}{\mathbf{j}}
\newcommand{\A}{\mathbf{A}}
\newcommand{\N}{\mathbf{n}}
\newcommand{\F}{\mathbf{f}}
\newcommand{\X}{\mathbf{x}}
\newcommand{\Y}{\mathbf{y}}
\newcommand{\zero}{\mathbf{0}}
\newcommand{\ttheta}{\tilde{\theta}}
\newcommand{\M}{{\mu_m}}
\newcommand{\SII}{\SI^{-1}}
\newcommand{\JS}{\frac{\J}{\sigma}}
\newcommand{\MO}{\overline{\M}}
\newcommand{\SO}{\overline{\sigma}}
\newcommand{\JSS}{\frac{\J^2}{\sigma}}

\newcommand{\KE}{\left(\frac{1}{2} \U \DOT \U\right)}
\newcommand{\ME}{\left(\frac{1}{2} \B \DOT \B\right)}
\newcommand{\EPS}{\varepsilon}

\newcommand{\SIG}{\sigma}
\newcommand{\PH}{physical}

\newcommand{\SOVAC}{\SO_{v}}


\newcommand{\NPB}{\num[round-mode=places,round-precision=2]}
\newcommand{\NPU}{\num[round-mode=places,round-precision=5]}
\newcommand{\NPJ}{\num[round-mode=places,round-precision=5]}
\newcommand{\NPS}{\num[scientific-notation=true,round-mode=places,round-precision=2]}
\newcommand{\NPP}{\num[round-mode=places,round-precision=2]}

\newcommand{\MAT}{\left[ \begin{array}}
\newcommand{\EMAT}{\end{array} \right]}
\newcommand{\ARRAY}{\begin{array}}
\newcommand{\EARRAY}{\end{array}}

% \newcommand{\DOT}{\textbullet} % Causes warnings
% \newcommand{\DOT}{\text{\textbullet}}
\newcommand{\DOT}{\ensuremath{\bullet}}
% \newcommand{\textbullet}{}
\newcommand{\CROSS}{\times}
\newcommand{\BUN}{\otimes}
\newcommand{\DEL}{\nabla}
\newcommand{\CURL}{\DEL \CROSS}
\newcommand{\DIV}{\DEL \DOT}
\newcommand{\PD}{\partial}
\newcommand{\MAC}{\mathcal}
\newcommand{\GRAD}{\nabla}
\newcommand{\RA}{\rightarrow}

\newcommand{\EQSIZE}{\footnotesize}

\newcommand{\MC}{\multicolumn}
\newcommand{\MR}{\multirow}
\newcommand{\LL}{\raggedright}
\newcommand{\RR}{\raggedleft}
\newcommand{\CE}{\centering}
\newcommand{\TW}{\textwidth}
\renewcommand{\TH}{\textheight}
\newcommand{\TWP}{em\textwidth}
\newcommand{\qquadmany}{\qquad\qquad\qquad\qquad\qquad\qquad}
\newcommand{\hfillMath}{\hskip \textwidth minus \textwidth}

\newcommand{\tenDots}{..........}
\newcommand{\twentyDots}{\tenDots\tenDots}
\newcommand{\fiftyDots}{\twentyDots\twentyDots\tenDots}
\newcommand{\sixtyDots}{\twentyDots\twentyDots\twentyDots}
\newcommand{\seventyDots}{\fiftyDots\twentyDots}
\newcommand{\hundredDots}{\fiftyDots\fiftyDots}

% \restylefloat{table} % Results in errors in prospectus presentation


\newtoggle{includeImages}
\toggletrue{includeImages}
% \togglefalse{includeImages}

\newcommand{\IG}{\includegraphics}
\newcommand{\MP}{\minipage}
\newcommand{\EMP}{\endminipage}
\newcommand{\FIG}{\figure}
\newcommand{\EFIG}{\endfigure}
\newcommand{\SFIG}{\subfigure}
\newcommand{\ESFIG}{\endsubfigure}

\newcommand{\BLANKFIGURE}{\begin{figure*}[!htp]\centering\IG[width=.3\TW]{\figdir/blank.png}\end{figure*}}
\newcommand{\BLANKGRAPHIC}{\IG[width=.3\TW]{\figdir/blank.png}}
% \renewcommand{\BLANKGRAPHIC}{}
% \renewcommand{\BLANKFIGURE}{}

\newcommand{\FIGTYPE}{_BW}
\renewcommand{\FIGTYPE}{} % uncomment if color
\newcommand{\FIGEXT}{png}
% \renewcommand{\FIGEXT}{eps} % uncomment for .eps (takes forever to build)

\newcommand{\MOONSTITLE}{\title{MHD Object-Oriented Numerical Solver (MOONS)}
\author{C. Kawczynski \\
Department of Mechanical and Aerospace Engineering \\
University of California Los Angeles, USA\\
}}


% % Numbering from: https://www.sharelatex.com/learn-scripts/images/f/fc/Layout-dimensions.png
\setlength{\hoffset}{-.5in}       % 1
\setlength{\voffset}{0in}        % 2
\setlength{\oddsidemargin}{0in}  % 3
\setlength{\topmargin}{-.5in}     % 4
\setlength{\headheight}{0in}     % 5
\setlength{\headsep}{0in}        % 6
\setlength{\textheight}{700pt}   % 7
\setlength{\textwidth}{550pt}    % 8
\setlength{\marginparsep}{0in}   % 9
% \setlength{\marginparwidth}{0in} % 10
\setlength{\footskip}{0in}       % 11 a (see fig)
% \setlength{\marginparpush}{0in}  % 11 b (see fig)
% \setlength{\paperwidth}{0in}     % 11 e (see fig)
% \setlength{\paperheight}{390pt}  % 11 f (see fig)
 % For local use
% % Numbering from: https://www.sharelatex.com/learn-scripts/images/f/fc/Layout-dimensions.png
\setlength{\hoffset}{0in}       % 1
\setlength{\voffset}{0in}        % 2
\setlength{\oddsidemargin}{0in}  % 3
\setlength{\topmargin}{0in}     % 4
\setlength{\headheight}{0in}     % 5
\setlength{\headsep}{0in}        % 6
\setlength{\textheight}{350pt}   % 7
\setlength{\textwidth}{5in}    % 8
\setlength{\marginparsep}{0in}   % 9
\setlength{\marginparwidth}{0in} % 10
\setlength{\footskip}{0in}       % 11 a (see fig)
\setlength{\marginparpush}{0in}  % 11 b (see fig)
\setlength{\hoffset}{0in}        % 11 c (see fig)
\setlength{\voffset}{0in}        % 11 d (see fig)
\setlength{\paperwidth}{0in}     % 11 e (see fig)
\setlength{\paperheight}{390pt}  % 11 f (see fig)
 % For local use
% \input{\rootdir/includes/diagrams/coordinates.tex} % For local use
% \input{\rootdir/includes/margins/margins.tex} % For local use




\newcommand{\height}{0.4}
\newcommand{\radius}{0.1}
\newcommand{\offSet}{12}
\newcommand{\Deltah}{3}

\begin{document}
\doublespacing
\MOONSTITLE
\maketitle

\section{Grid}
Cell corner and cell centered data (and their indexing) are shown below.

\begin{figure}[H]
  \begin{center}
    \begin{tikzpicture}
      \draw [black] (0,0) to (2*\Deltah,0);
      \draw [black,dashed] (2*\Deltah,0) to (4*\Deltah,0);
      \draw [black] (4*\Deltah,0) to (6*\Deltah,0);

      % Nodes
      \draw [black] (0*\Deltah,-\height) to (0+0*\Deltah,\height);
      \draw [black] (1*\Deltah,-\height) to (0+1*\Deltah,\height);
      \draw [black] (2*\Deltah,-\height) to (0+2*\Deltah,\height);
      \draw [black] (4*\Deltah,-\height) to (0+4*\Deltah,\height);
      \draw [black] (5*\Deltah,-\height) to (0+5*\Deltah,\height);
      \draw [black] (6*\Deltah,-\height) to (0+6*\Deltah,\height);

      % CCs
      \draw [black] (0*\Deltah+\Deltah/2,0) circle [radius=\radius];
      \draw [black] (1*\Deltah+\Deltah/2,0) circle [radius=\radius];
      \draw [black] (4*\Deltah+\Deltah/2,0) circle [radius=\radius];
      \draw [black] (5*\Deltah+\Deltah/2,0) circle [radius=\radius];

      \draw (0*\Deltah,0) node [black,below=\offSet] {$f_{n,1}$};
      \draw (1*\Deltah,0) node [black,below=\offSet] {$f_{n,2}$};
      \draw (2*\Deltah,0) node [black,below=\offSet] {$f_{n,3}$};
      \draw (4*\Deltah,0) node [black,below=\offSet] {$f_{n,sn-2}$};
      \draw (5*\Deltah,0) node [black,below=\offSet] {$f_{n,sn-1}$};
      \draw (6*\Deltah,0) node [black,below=\offSet] {$f_{n,sn}$};

      \draw (0*\Deltah+\Deltah/2,0) node [black,below=\offSet] {$f_{c,1}$};
      \draw (1*\Deltah+\Deltah/2,0) node [black,below=\offSet] {$f_{c,2}$};
      \draw (5*\Deltah-\Deltah/2,0) node [black,below=\offSet] {$f_{c,sc-1}$};
      \draw (6*\Deltah-\Deltah/2,0) node [black,below=\offSet] {$f_{c,sc}$};

    \end{tikzpicture}
    \caption{Index convention for node / face centered data}
  \end{center}
\end{figure}

\section{Stencils for 2nd derivatives}
MOONS computes
\begin{equation}
\frac{\partial}{\partial h} \left( k \frac{\partial u}{\partial h} \right)
\end{equation}
on a staggered grid for both cell corner and cell centered data. Let subscript $p$ and $d$ represent the primary and dual grid of the staggered grid respectively. If the data lives on the primary grid, then the result also lives on the primary grid, and the coefficient, $k$, lives on the primary grid (which must be interpolated to the dual grid). This derivative is computed as follows.

\begin{equation}
\frac{\partial}{\partial h} \left( k \frac{\partial u}{\partial h} \right)
 = \frac{\frac{u_{i+1}-u_{i}}{h_{p,i+1}-h_{p,i}} k_{p,i+1/2} - \frac{u_{i}-u_{i-1}}{h_{p,i}-h_{p,i-1}} k_{p,i-1/2}}{h_{p,i+1/2} - h_{p,i-1/2}}
\end{equation}
This form of the derivative is explicitely clear and consistent (since all references are to the primary grid). Beforehand, we must be clear about the indexes. Note that $\frac{\partial}{\partial h} \left( k \frac{\partial u}{\partial h} \right)$ lives on integers of $i$, whether it is computed on CC or node data. We seek to remove the half indexes because we cannot program with them.

Since
\begin{equation}
  \Delta h_{c,1} = h_{c,2} - h_{c,1}
\end{equation}
\begin{equation}
  \Delta h_{n,1} = h_{n,2} - h_{n,1}
\end{equation}
Let $\Delta h_i = h_{i+1} - h_i$ for both grids

\begin{figure}[h!]
  \begin{center}
    \begin{tikzpicture}
      \draw [black] (0,0) to (2*\Deltah,0);
      \draw [black,dashed] (2*\Deltah,0) to (4*\Deltah,0);
      \draw [black] (4*\Deltah,0) to (6*\Deltah,0);

      % Nodes
      \draw [black] (0*\Deltah,-\height) to (0+0*\Deltah,\height);
      \draw [black] (1*\Deltah,-\height) to (0+1*\Deltah,\height);
      \draw [black] (2*\Deltah,-\height) to (0+2*\Deltah,\height);
      \draw [black] (4*\Deltah,-\height) to (0+4*\Deltah,\height);
      \draw [black] (5*\Deltah,-\height) to (0+5*\Deltah,\height);
      \draw [black] (6*\Deltah,-\height) to (0+6*\Deltah,\height);

      % CCs
      \draw [black] (0*\Deltah+\Deltah/2,0) circle [radius=\radius];
      \draw [black] (1*\Deltah+\Deltah/2,0) circle [radius=\radius];
      \draw [black] (4*\Deltah+\Deltah/2,0) circle [radius=\radius];
      \draw [black] (5*\Deltah+\Deltah/2,0) circle [radius=\radius];

      \draw (0*\Deltah,0) node [black,below=\offSet] {$f_{n,1}$};
      \draw (1*\Deltah,0) node [black,below=\offSet] {$f_{n,2}$};
      \draw (2*\Deltah,0) node [black,below=\offSet] {$f_{n,3}$};
      \draw (4*\Deltah,0) node [black,below=\offSet] {$f_{n,sn-2}$};
      \draw (5*\Deltah,0) node [black,below=\offSet] {$f_{n,sn-1}$};
      \draw (6*\Deltah,0) node [black,below=\offSet] {$f_{n,sn}$};

      \draw (0*\Deltah+\Deltah/2,0) node [black,below=\offSet] {$f_{c,1}$};
      \draw (1*\Deltah+\Deltah/2,0) node [black,below=\offSet] {$f_{c,2}$};
      \draw (5*\Deltah-\Deltah/2,0) node [black,below=\offSet] {$f_{c,sc-1}$};
      \draw (6*\Deltah-\Deltah/2,0) node [black,below=\offSet] {$f_{c,sc}$};

    \end{tikzpicture}
    \caption{Index convention for node / face centered data}
  \end{center}
\end{figure}
\subsection{Cell Centered (CC) data}
We seek to remove the half indexes and replace them with the dual grid index. For uniform properties, we have

\begin{equation}
\frac{\partial}{\partial h} \left( \frac{\partial u}{\partial h} \right)
 =
 \frac{\frac{u_{i+1}-u_{i}}{h_{p,i+1}-h_{p,i}} - \frac{u_{i}-u_{i-1}}{h_{p,i}-h_{p,i-1}}}{h_{p,i+1/2} - h_{p,i-1/2}}
\end{equation}

Let's assume $u\in CC$, and $i=2$, then we have

\begin{equation}
 \left\{\frac{\partial}{\partial h} \left( \frac{\partial u}{\partial h} \right) \right\}_{c,2}
 =
 \frac{
 \frac{u_{c,3}-u_{c,2}}{h_{c,3}-h_{c,2}} -
 \frac{u_{c,2}-u_{c,1}}{h_{c,2}-h_{c,1}}
 }{h_{c,2+1/2} - h_{c,2-1/2}}
\end{equation}

Here we ask, what is the above equation in terms of the dual grid, which is clearly

\begin{equation}
 \left\{\frac{\partial}{\partial h} \left( \frac{\partial u}{\partial h} \right) \right\}_{c,2}
 = \frac{ \frac{u_{c,3}-u_{c,2}}{h_{c,3}-h_{c,2}} -  \frac{u_{c,2}-u_{c,1}}{h_{c,2}-h_{c,1}} }{h_{n,k+1} - h_{n,k}}
 =
 \frac{ \frac{u_{c,3}-u_{c,2}}{h_{c,3}-h_{c,2}} -  \frac{u_{c,2}-u_{c,1}}{h_{c,2}-h_{c,1}} }{\Delta h_{n,k}}
 =
 \frac{ \frac{u_{c,3}-u_{c,2}}{h_{c,3}-h_{c,2}} -  \frac{u_{c,2}-u_{c,1}}{h_{c,2}-h_{c,1}} }{\Delta h_{n,2}}
\end{equation}

We can see from the index convention, the correct value for $k$ is 2. Therefore, for cell centered data the relationship between the primary (CC) and dual (N) grids, we can relate the indexes as

\begin{equation}
  h_{p,i+1/2} - h_{p,i-1/2} = \Delta h_{d,i}
\end{equation}

\subsection{Cell Corner (N) data}
We seek to remove the half indexes and replace them with the dual grid index. For uniform properties, we have

\begin{equation}
\frac{\partial}{\partial h} \left( \frac{\partial u}{\partial h} \right)
 =
 \frac{\frac{u_{i+1}-u_{i}}{h_{p,i+1}-h_{p,i}} - \frac{u_{i}-u_{i-1}}{h_{p,i}-h_{p,i-1}}}{h_{p,i+1/2} - h_{p,i-1/2}}
\end{equation}

Let's assume $u\in N$, and $i=2$, then we have

\begin{equation}
 \left\{\frac{\partial}{\partial h} \left( \frac{\partial u}{\partial h} \right) \right\}_{n,2}
 =
 \frac{
 \frac{u_{n,3}-u_{n,2}}{h_{n,3}-h_{n,2}} -
 \frac{u_{n,2}-u_{n,1}}{h_{n,2}-h_{n,1}}
 }{h_{n,2+1/2} - h_{n,2-1/2}}
\end{equation}

Here we ask, what is the above equation in terms of the dual grid, which is clearly

\begin{equation}
 \left\{\frac{\partial}{\partial h} \left( \frac{\partial u}{\partial h} \right) \right\}_{n,2}
 = \frac{ \frac{u_{n,3}-u_{n,2}}{h_{n,3}-h_{n,2}} -  \frac{u_{n,2}-u_{n,1}}{h_{n,2}-h_{n,1}} }{h_{c,k+1} - h_{c,k}}
 =
 \frac{ \frac{u_{n,3}-u_{n,2}}{h_{n,3}-h_{n,2}} -  \frac{u_{n,2}-u_{n,1}}{h_{n,2}-h_{n,1}} }{\Delta h_{c,k}}
 =
 \frac{ \frac{u_{n,3}-u_{n,2}}{h_{n,3}-h_{n,2}} -  \frac{u_{n,2}-u_{n,1}}{h_{n,2}-h_{n,1}} }{\Delta h_{c,1}}
\end{equation}

We can see from the index convention, the correct value for $k$ is 1. Therefore, for cell centered data the relationship between the primary (N) and dual (CC) grids, we can relate the indexes as

\begin{equation}
  h_{p,i+1/2} - h_{p,i-1/2} = \Delta h_{d,i-1}
\end{equation}


This means we may write a general form of the dual grid indexes, based on the grid type ($gt$), as

\begin{equation}
  h_{p,i+1/2} - h_{p,i-1/2} = \Delta h_{d,i-1+gt} = h_{d,i+gt} - h_{d,i-1+gt}
  , \qquad
 gt =
\begin{cases}
    0 & \text{if $u \in $ N} \\
    1 & \text{if $u \in $ CC} \\
 \end{cases}
\end{equation}


% Let $i+1/2$ of the primary grid be $i+1$ of the dual grid (this can easily be seen by replacing $i$ with 1). This means that if $\textcolor{green}{i=2}$ then $i-1+1=2-1+1=2$ refers to the \textit{second} index of $\Delta h_{d,i-1+1}$ which is $\textcolor{purple}{\Delta h_{d,2}}$ (which, again is what we want). This verifies the index for the N data case.

\section{General form without half indexes}
Now, we are equipped to write a more general equation
\begin{equation}
\frac{\partial}{\partial h} \left( k \frac{\partial u}{\partial h} \right)
 = \frac{\frac{u_{i+1}-u_{i}}{h_{p,i+1}-h_{p,i}} k_{d,i+gt} - \frac{u_{i}-u_{i-1}}{h_{p,i}-h_{p,i-1}} k_{d,i-1+gt}}{h_{d,i+gt} - h_{d,i-1+gt}}
\end{equation}

Now we have successfully removed the half indexes, and this form is more easily programmable. Note that $k$ has also adopted the index convention. Now we have

\begin{equation}
\frac{\partial}{\partial h} \left( k \frac{\partial u}{\partial h} \right)
 = \frac{\frac{u_{i+1}-u_{i}}{\Delta h_{p,i}} k_{d,i+gt} - \frac{u_{i}-u_{i-1}}{\Delta h_{p,i-1}} k_{d,i-1+gt}}{\Delta h_{d,i-1+gt}}
\end{equation}

\begin{equation}
 = \frac{u_{i+1}-u_{i}}{ \Delta h_{p,i} \Delta h_{d,i-1+gt}} k_{d,i+gt} - \frac{u_{i}-u_{i-1}}{ \Delta h_{p,i-1}\Delta h_{d,i-1+gt}} k_{d,i-1+gt}
\end{equation}

\begin{equation}
\boxed{
\frac{\partial}{\partial h} \left( k \frac{\partial u}{\partial h} \right)
 = \left( \frac{k_{d,i-1+gt}}{\Delta h_{p,i-1} \Delta h_{d,i-1+gt}} \right) u_{i - 1} -
   \left( \frac{k_{d,i-1+gt}}{\Delta h_{p,i-1} \Delta h_{d,i-1+gt}} + \frac{k_{d,i+gt}}{\Delta h_{p,i} \Delta h_{d,i-1+gt}} \right) u_{i} +
   \left( \frac{k_{d,i+gt}}{\Delta h_{p,i} \Delta h_{d,i-1+gt}} \right) u_{i+1}
 }
\end{equation}

In the case of uniform properties

\begin{equation}
\boxed{
\frac{\partial}{\partial h} \left( \frac{\partial u}{\partial h} \right)
 = \left( \frac{1}{\Delta h_{p,i-1} \Delta h_{d,i-1+gt}} \right) u_{i - 1} -
   \left( \frac{1}{\Delta h_{p,i-1} \Delta h_{d,i-1+gt}} + \frac{1}{\Delta h_{p,i} \Delta h_{d,i-1+gt}} \right) u_{i} +
   \left( \frac{1}{\Delta h_{p,i} \Delta h_{d,i-1+gt}} \right) u_{i+1}
 }
\end{equation}


\section{Computing k}
The coefficient must be linearly interpolated when $u \in N$, otherwise and ordinary average suffices. The formulation of $k_{d,i}$ may be determined from

\begin{equation}
  \frac{k_{p,i+1} - k_{p,i}}{h_{p,i+1} - h_{p,i}}
   =
  \frac{k_{p,i+1/2} - k_{p,i}}{h_{p,i+1/2} - h_{p,i}}
   =
  \frac{k_{d,i+gt} - k_{p,i}}{h_{d,i+gt} - h_{p,i}}
\end{equation}
Therefore we may compute $k_{p,i+1/2} = k_{d,i+gt}$ to be

\begin{equation}
  k_{p,i+1/2} = k_{d,i+gt} = k_{p,i} - \frac{h_{d,i+gt} - h_{p,i}}{\Delta h_{p,i}} (k_{p,i+1} - k_{p,i})
\end{equation}

Note that if $u \in N$, then we have $gt=1$ and $\Delta h_{p,i}= 2(h_{d,i+1} - h_{p,i})$ and so

\begin{equation}
  k_{p,i+1/2} = k_{d,i} = k_{p,i} + \frac{\Delta h_{p,i} / 2}{\Delta h_{p,i}} (k_{p,i+1} - k_{p,i}) = \frac{1}{2} (k_{p,i+1} + k_{p,i})
\end{equation}

Which verifies the case when the linearly interpolation becomes a simple average.


\end{document}