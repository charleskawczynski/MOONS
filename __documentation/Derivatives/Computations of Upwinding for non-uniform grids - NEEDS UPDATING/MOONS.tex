\documentclass[11pt]{article}
\usepackage{graphicx}    % needed for including graphics e.g. EPS, PS
\usepackage{epstopdf}
\usepackage{amsmath}
\usepackage{hyperref}
\usepackage{xspace}
\usepackage{mathtools}
\usepackage{tikz}
\usepackage{epsfig}
\usepackage{float}
\usepackage{natbib}
\usepackage{subfigure}
\usepackage{setspace}
\usepackage{tabularx,ragged2e,booktabs,caption}


\setlength{\oddsidemargin}{0.1in}
\setlength{\textwidth}{7.25in}

\setlength{\topmargin}{-1in}     %\topmargin: gap above header
\setlength{\headheight}{0in}     %\headheight: height of header
\setlength{\topskip}{0in}        %\topskip: between header and text
\setlength{\headsep}{0in}        
\setlength{\textheight}{692pt}   %\textheight: height of main text
\setlength{\textwidth}{7.5in}    % \textwidth: width of text
\setlength{\oddsidemargin}{-0.5in}  % \oddsidemargin: odd page left margin
\setlength{\evensidemargin}{0in} %\evensidemargin : even page left margin
\setlength{\parindent}{0.25in}   %\parindent: indentation of paragraphs
\setlength{\parskip}{0pt}        %\parskip: gap between paragraphs
\setlength{\voffset}{0.5in}


% Useful commands:

% \hfill		aligns-right everything right of \hfill

\begin{document}
\doublespacing
\title{Magnetohydrodynamic Object-Oriented Numerical Solver (MOONS)}
\author{C. Kawczynski \\
Department of Mechanical and Aerospace Engineering \\
University of California Los Angeles, USA\\
}
\maketitle

\section{Computations of Upwinding for non-uniform grids}

Ultimately, we wish to write a formula to express

\begin{equation}
	advect() = \frac{\partial(u_i u_j)}{\partial x_j}
	=
	\gamma upwind()_i
	+
	(1-\gamma) CD2()_i
\end{equation}

MOONS already has the $CD2()$ routine. We wish to write an $upwind()$ formula for non-uniform grids

\begin{equation}
	upwind()_i = \frac{k_r u_r - k_l u_l}{\Delta x_i}
\end{equation}

Following 3.12 we set

\begin{equation}
	k_r = \frac{v_{i,j}+v_{i+1,j}}{2} \qquad
	k_l = \frac{v_{i,j-1}+v_{i+1,j-1}}{2}
\end{equation}

Equation 3.1 describes the form of the hybrid scheme, therefore, we may write the normal and mixed derivatives as

\begin{equation}
	\frac{\partial(u^2)}{\partial x} = \gamma upwind() + (1-\gamma) CD2()
	=
	CD2() + \gamma (upwind()-CD2())
\end{equation}

The $CD2()$ has already been developed. So let's look at the upwind only:

\begin{equation}
	upwind() = \frac{k_r u_r - k_l u_l}{\Delta x}
\end{equation}

\begin{equation}
	u_r =
	\begin{cases}
	u_i , k_r>0 \\
	u_{i+1} , k_r <0
	\end{cases}
	\qquad
	\qquad
	u_l =
	\begin{cases}
	u_{i-1} , k_l>0 \\
	u_{i} , k_l <0
	\end{cases}
\end{equation}

\begin{equation}
	k_r = \frac{u_{i,j}+u{i+1,j}}{2}
	\qquad
	k_l = \frac{u_{i-1,j}+u{i+1,j}}{2}
\end{equation}

And

\begin{equation}
	k_r = \frac{v_{i,j}+v{i+1,j}}{2}
	\qquad
	k_l = \frac{v_{i+1,j}+v{i+1,j}}{2}
\end{equation}




\end{document}


