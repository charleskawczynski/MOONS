\documentclass[11pt]{article}
\newcommand{\PSCHAIN}{..}
\edef\PSCHAIN{\PSCHAIN/LATEX_INCLUDES}

\newcommand{\rootdir}{\PSCHAIN}
\newcommand{\VAR}{Success}


\newcommand{\height}{0.4}
\newcommand{\radius}{0.1}
\newcommand{\offSet}{12}
\newcommand{\Deltah}{3}

\begin{document}
\doublespacing
\MOONSTITLE
\maketitle

\section{Derivation of stencils in matrix form}

The previous form of derivatives were not computationally efficient. It was realized that saving coefficients of each term in derivatives will be the most computationally efficient and not require much memory demands.

Cell corner and cell centered data (and their indexing) are shown below.

\begin{figure}[H]
  \begin{center}
    \begin{tikzpicture}
      \draw [black] (0,0) to (2*\Deltah,0);
      \draw [black,dashed] (2*\Deltah,0) to (4*\Deltah,0);
      \draw [black] (4*\Deltah,0) to (6*\Deltah,0);

      % Nodes
      \draw [black] (0*\Deltah,-\height) to (0+0*\Deltah,\height);
      \draw [black] (1*\Deltah,-\height) to (0+1*\Deltah,\height);
      \draw [black] (2*\Deltah,-\height) to (0+2*\Deltah,\height);
      \draw [black] (4*\Deltah,-\height) to (0+4*\Deltah,\height);
      \draw [black] (5*\Deltah,-\height) to (0+5*\Deltah,\height);
      \draw [black] (6*\Deltah,-\height) to (0+6*\Deltah,\height);

      % CCs
      \draw [black] (0*\Deltah+\Deltah/2,0) circle [radius=\radius];
      \draw [black] (1*\Deltah+\Deltah/2,0) circle [radius=\radius];
      \draw [black] (4*\Deltah+\Deltah/2,0) circle [radius=\radius];
      \draw [black] (5*\Deltah+\Deltah/2,0) circle [radius=\radius];

      \draw (0*\Deltah,0) node [black,below=\offSet] {$f_{n,1}$};
      \draw (1*\Deltah,0) node [black,below=\offSet] {$f_{n,2}$};
      \draw (2*\Deltah,0) node [black,below=\offSet] {$f_{n,3}$};
      \draw (4*\Deltah,0) node [black,below=\offSet] {$f_{n,sn-2}$};
      \draw (5*\Deltah,0) node [black,below=\offSet] {$f_{n,sn-1}$};
      \draw (6*\Deltah,0) node [black,below=\offSet] {$f_{n,sn}$};

      \draw (0*\Deltah+\Deltah/2,0) node [black,below=\offSet] {$f_{c,1}$};
      \draw (1*\Deltah+\Deltah/2,0) node [black,below=\offSet] {$f_{c,2}$};
      \draw (5*\Deltah-\Deltah/2,0) node [black,below=\offSet] {$f_{c,sc-1}$};
      \draw (6*\Deltah-\Deltah/2,0) node [black,below=\offSet] {$f_{c,sc}$};

    \end{tikzpicture}
    \caption{Index convention for node / face centered data}
  \end{center}
\end{figure}

We would like to get the coefficients of the derivative operator, $A$, when in the form
\begin{equation} \label{eq:Af=b}
  Af = g
\end{equation}
The size, and shape, of $A$ will depend on whether the input and output are collocated or not.

Note that for both cell center and cell corner data,

\begin{equation}
  \Delta h_{j} = h_{j+1}-h_{j}
\end{equation}

\section{Staggered derivatives}
\subsection{CC to N}
Consider the first derivative of staggered cell corner data of the familiar form:
\begin{equation}
  g_{n,j} = \left( \frac{\partial f_{c}}{\partial h} \right)_{n,j} = \frac{f_{c,j+1}-f_{c,j}}{\Delta h_{c,j}}
\end{equation}

First, let's discuss the change in size of the vector. The size of $f_c$ is $sc$, and the size of $\left( \frac{\partial f_{c}}{\partial h} \right)_{n}$ is $sn$ (from the above figure). Which means that the multiplication in \ref{eq:Af=b} must be $(sn \times sc) \times (sc \times 1) = (sn \times 1)$. We may also write this as an interior multiplication as $(sn-2 \times sc) \times (sc \times 1) = (sn-2 \times 1)$

\[
Af =
\left[
\begin{array}{ccccccccc}
D_{1} & U_{1} &  &   &   &   &   &   & 0 \\
  & D_{2} & U_{2} &   &   &   &   &   &   \\
  &   & D_{3} & U_{3} &   &   &   &   &   \\
  &   &  & \ddots & \ddots &   &   &   &   \\
  &   &   &   & D_{j} & U_{j} &   &   &   \\
  &   &   &   &  & \ddots & \ddots &   &   \\
  &   &   &   &   &   & D_{sc-2} & U_{sc-2} &   \\
 0 &   &   &   &   &   &   & D_{sc-1} & U_{sc-1} \\
\end{array}
\right]
\left[ \begin{array}{c}
f_{c,1} \\ f_{c,2} \\ \vdots \\ f_{c,j-1} \\ f_{c,j} \\ f_{c,j+1} \\ \vdots \\ f_{c,sc-1} \\ f_{c,sc}
\end{array} \right]
=
\left[ \begin{array}{c}
g_{n,2} \\ g_{n,3} \\ \vdots \\ g_{n,j-1} \\ g_{n,j} \\ g_{n,j+1} \\ \vdots \\ g_{n,sn-2} \\ g_{n,sn-1}
\end{array} \right]
 = \left( \frac{\partial f_{c}}{\partial h} \right)_{n,j}
\]

From the above diagram, it is clear that

\begin{equation}
  D_j = \frac{-1}{\Delta h_{c,j}}, \qquad
  U_j = \frac{1}{\Delta h_{c,j}}, \qquad
  \Delta h_{c,j} = h_{c,j+1}-h_{c,j}
\end{equation}

\subsection{N to CC}
Consider the first derivative of staggered cell corner data of the familiar form:
\begin{equation}
  g_{c,j} = \left( \frac{\partial f_{n}}{\partial h} \right)_{c,j} = \frac{f_{n,j+1}-f_{n,j}}{\Delta h_{n,j}}
\end{equation}

First, let's discuss the change in size of the vector. The size of $f_n$ is $sn$, and the size of $\left( \frac{\partial f_{n}}{\partial h} \right)_{c}$ is $sc$. Which means that the multiplication in \ref{eq:Af=b} must be $(sc \times sn) \times (sn \times 1) = (sc \times 1)$.

\[
Af =
\left[
\begin{array}{ccccccccc}
D_{1} & U_{1} &  &   &   &   &   &   & 0 \\
  & D_{2} & U_{2} &   &   &   &   &   &   \\
  &   & D_{3} & U_{3} &   &   &   &   &   \\
  &   &  & \ddots & \ddots &   &   &   &   \\
  &   &   &   & D_{j} & U_{j} &   &   &   \\
  &   &   &   &  & \ddots & \ddots &   &   \\
  &   &   &   &   &   & D_{sn-2} & U_{sn-2} &   \\
 0 &   &   &   &   &   &   & D_{sn-1} & U_{sn-1} \\
\end{array}
\right]
\left[ \begin{array}{c}
f_{n,1} \\ f_{n,2} \\ \vdots \\ f_{n,j-1} \\ f_{n,j} \\ f_{n,j+1} \\ \vdots \\ f_{n,sn-1} \\ f_{n,sn}
\end{array} \right]
=
\left[ \begin{array}{c}
g_{c,1} \\ g_{c,2} \\ \vdots \\ g_{c,j-1} \\ g_{c,j} \\ g_{c,j+1} \\ \vdots \\ g_{c,sc-1} \\ g_{c,sc}
\end{array} \right]
 = \left( \frac{\partial f_{c}}{\partial h} \right)_{n,j}
\]

From the above diagram, it is clear that

\begin{equation}
  D_j = \frac{-1}{\Delta h_{n,j}}, \qquad
  U_j = \frac{1}{\Delta h_{n,j}}
\end{equation}


\section{Collocated derivatives}

\subsection{1st Derivatives}

\subsection{2nd Derivatives}
Consider the second derivative of collocated data of the familiar form:
\begin{equation}
  g_{p,j} =
\frac{\partial}{\partial h} \left( k \frac{\partial f}{\partial h} \right)
 = \frac{\frac{f_{j+1}-f_{j}}{h_{p,j+1}-h_{p,j}} k_{p,j+1/2} - \frac{f_{j}-f_{j-1}}{h_{p,j}-h_{p,j-1}} k_{p,j-1/2}}{h_{p,j+1/2} - h_{p,j-1/2}}
\end{equation}

This simplifies to

\begin{equation}
\frac{\partial}{\partial h} \left( k \frac{\partial f}{\partial h} \right)
 = \left( \frac{k_{d,j-1+gt}}{\Delta h_{p,j-1} \Delta h_{d,j-1+gt}} \right) f_{j - 1} -
   \left( \frac{k_{d,j-1+gt}}{\Delta h_{p,j-1} \Delta h_{d,j-1+gt}} + \frac{k_{d,j+gt}}{\Delta h_{p,j} \Delta h_{d,j-1+gt}} \right) f_{j} +
   \left( \frac{k_{d,j+gt}}{\Delta h_{p,j} \Delta h_{d,j-1+gt}} \right) f_{j+1}
\end{equation}

Where
\begin{equation}
 gt =
\begin{cases}
    0 & \text{if $u \in $ N} \\
    1 & \text{if $u \in $ CC} \\
 \end{cases}
\end{equation}

And in the case of uniform properties, we have

\begin{equation}
\frac{\partial}{\partial h} \left( \frac{\partial f}{\partial h} \right)
 = \left( \frac{1}{\Delta h_{p,j-1} \Delta h_{d,j-1+gt}} \right) f_{j - 1} -
   \left( \frac{1}{\Delta h_{p,j-1} \Delta h_{d,j-1+gt}} + \frac{1}{\Delta h_{p,j} \Delta h_{d,j-1+gt}} \right) f_{j} +
   \left( \frac{1}{\Delta h_{p,j} \Delta h_{d,j-1+gt}} \right) f_{j+1}
\end{equation}


\[
Af =
\left[
\begin{array}{ccccccccc}
D_{1} & U_{1} &  &   &   &   &   &   & 0 \\
L_{1} & D_{2} & U_{2} &   &   &   &   &   &   \\
  & L_{2} & D_{3} & U_{3} &   &   &   &   &   \\
  &   & \ddots & \ddots & \ddots &   &   &   &   \\
  &   &   & L_{j-1} & D_{j} & U_{j} &   &   &   \\
  &   &   &   & \ddots & \ddots & \ddots &   &   \\
  &   &   &   &   & L_{s-3} & D_{s-2} & U_{s-2} &   \\
  &   &   &   &   &   & L_{s-2} & D_{s-1} & U_{s-1} \\
0 &   &   &   &   &   &   & L_{s-1} & D_{s}
\end{array}
\right]
\left[ \begin{array}{c}
f_{p,1} \\ f_{p,2} \\ \vdots \\ f_{p,j-1} \\ f_{p,j} \\ f_{p,j+1} \\ \vdots \\ f_{p,s-1} \\ f_{p,s}
\end{array} \right]
 = \left( \frac{\partial f_{p}}{\partial h} \right)_{p,j}
\]

From the above equations, it is clear that

\begin{equation}
  D_j = -\left( \frac{1}{\Delta h_{p,j-1} \Delta h_{d,j-1+gt}} + \frac{1}{\Delta h_{p,j} \Delta h_{d,j-1+gt}} \right), \qquad j \in (2,s-1)
\end{equation}

\begin{equation}
  L_j = \left( \frac{1}{\Delta h_{p,j-1} \Delta h_{d,j-1+gt}} \right), \qquad
  U_j = \left( \frac{1}{\Delta h_{p,j} \Delta h_{d,j-1+gt}} \right), \qquad j \in (2,s-1)
\end{equation}

\section{2nd derivative via composition}

The 3-point stencil to achieve 2nd order accuracy is


\[ Af = \left[
\begin{array}{ccccccccc}
L_{1} & D_{1}    & U_{1}     &           &           &           &         \\
L_{2} & D_{2}    & U_{2}     &           &           &           &         \\
      & \ddots   & \ddots    & \ddots    &           &           &         \\
      &          & L_{i-1}   & D_{i-1}   & U_{i-1}   &           &         \\
      &          &           & \ddots    & \ddots    & \ddots    &         \\
      &          &           &           & L_{s-3}   & D_{s-3}   & U_{s-3} \\
      &          &           &           & L_{s-2}   & D_{s-2}   & U_{s-2} \\
\end{array} \right]
\left[ \begin{array}{c}
f_{2} \\ \vdots \\ f_{i-1} \\ f_{i} \\ f_{i+1} \\ \vdots \\ f_{s-1}
\end{array} \right]
=
\left[ \begin{array}{c}
f'_{2} \\ \vdots \\ f'_{i-1} \\ f'_{i} \\ f'_{i+1} \\ \vdots \\ f'_{s-1}
\end{array} \right]
\]

\subsection{Node derivatives}
We have

\[
Af =
\frac{\partial}{\partial h} \left( \frac{\partial f}{\partial h} \right)
=
\left[
\begin{array}{ccccccccc}
D_{1} & U_{1} &  &   &   &   &   &   & 0 \\
  & D_{2} & U_{2} &   &   &   &   &   &   \\
  &   & D_{3} & U_{3} &   &   &   &   &   \\
  &   &  & \ddots & \ddots &   &   &   &   \\
  &   &   &   & D_{j} & U_{j} &   &   &   \\
  &   &   &   &  & \ddots & \ddots &   &   \\
  &   &   &   &   &   & D_{sc-2} & U_{sc-2} &   \\
 0 &   &   &   &   &   &   & D_{sc-1} & U_{sc-1} \\
\end{array}
\right]
\left[ \begin{array}{c}
k_{c,1} \\ k_{c,2} \\ \vdots \\ k_{c,j-1} \\ k_{c,j} \\ k_{c,j+1} \\ \vdots \\ k_{c,sc-1} \\ k_{c,sc}
\end{array} \right]
\]

\[
\left[
\begin{array}{ccccccccc}
D_{1} & U_{1} &  &   &   &   &   &   & 0 \\
  & D_{2} & U_{2} &   &   &   &   &   &   \\
  &   & D_{3} & U_{3} &   &   &   &   &   \\
  &   &  & \ddots & \ddots &   &   &   &   \\
  &   &   &   & D_{j} & U_{j} &   &   &   \\
  &   &   &   &  & \ddots & \ddots &   &   \\
  &   &   &   &   &   & D_{sn-2} & U_{sn-2} &   \\
 0 &   &   &   &   &   &   & D_{sn-1} & U_{sn-1} \\
\end{array}
\right]
\]

Where $g \in CC$. Multiplying the first two we have
\[
Af =
\frac{\partial}{\partial h} \left( \frac{\partial f}{\partial h} \right)
=
\left[
\begin{array}{ccccccccc}
D_{1} k_{1} & U_{1} k_{2} &  &   &   &   &   &   & 0 \\
  & D_{2} k_{2} & U_{2} k_{3} &   &   &   &   &   &   \\
  &   & D_{3} k_{3} & U_{3} k_{4} &   &   &   &   &   \\
  &   &  & \ddots & \ddots &   &   &   &   \\
  &   &   &   & D_{j} k_{j} & U_{j} k_{j+1} &   &   &   \\
  &   &   &   &  & \ddots & \ddots &   &   \\
  &   &   &   &   &   & D_{sc-2} k_{sc-1} & U_{sc-2} k_{sc-1} &   \\
 0 &   &   &   &   &   &   & D_{sc-1} k_{sc-1} & U_{sc-1} k_{sc} \\
\end{array}
\right]
\]

\[
\left[
\begin{array}{ccccccccc}
D_{1} & U_{1} &  &   &   &   &   &   & 0 \\
  & D_{2} & U_{2} &   &   &   &   &   &   \\
  &   & D_{3} & U_{3} &   &   &   &   &   \\
  &   &  & \ddots & \ddots &   &   &   &   \\
  &   &   &   & D_{j} & U_{j} &   &   &   \\
  &   &   &   &  & \ddots & \ddots &   &   \\
  &   &   &   &   &   & D_{sn-2} & U_{sn-2} &   \\
 0 &   &   &   &   &   &   & D_{sn-1} & U_{sn-1} \\
\end{array}
\right]
\]

First note that, currently in MOONS $sn = sc + 1$. Next, multiplying the two matrices of size
$(sc-1 \times sc) \times (sn-1 \times sn) = (sn-2 \times sn-1) \times (sn-1 \times sn) = (sn-2 \times sn)$ to operate on data of size $sn \times 1$ results in $sn-2 \times 1$ (only the interior/up to boundary data for node data), we have

\begin{equation}
  Af =
\end{equation}
\[
\left[
\begin{array}{ccccccccc}
D_1D_1k_1&(D_1k_1U_1+U_1k_2D_2) &U_1k_2U_2          &           &                             &                    0                      \\
         &D_2D_2k_2           &(D_2U_2k_2+D_3U_2k_3)&U_2k_3U_3  &                             &                                         \\
         &\ddots             &\ddots     &                             &                    &                                         \\
         &D_j D_j k_j&(D_jU_jk_j+D_{j+1}U_{j}k_{j+1})&U_jk_{j+1}U_{j+1}   &                                         \\
         &\ddots                       &\ddots              &                                         \\
0        &D_{sn-2} D_{sn-2} k_{sn-2}&(D_{sn-2}U_{sn-2}k_{sn-2}+D_{sn-1}U_{sn-2}k_{sn-1})&U_{sn-2}k_{sn-1}U_{sn-1}   &                                         \\
\end{array}
\right]
\]


\section{Boundaries}
One-sided difference equations must be used at the boundaries in order to maintain 2nd order accuracy. Ghost cells are linearly extrapolated, which allows for simple BCs for the PPE. The extra data point can be used to our advantage because the boundary value is then easily obtained.

For CC data, a weighted interior difference equation is used with half of the cell distance is used and the boundary value is computed by linear interpolation. For N data, a one-sided difference equation is used with the boundary value, again computed by linear interpolation. Each of these formulas are discussed in the next section.

For convenience, the grid is shown again here:

\begin{figure}[h!]
  \begin{center}
    \begin{tikzpicture}
      \draw [black] (0,0) to (2*\Deltah,0);
      \draw [black,dashed] (2*\Deltah,0) to (4*\Deltah,0);
      \draw [black] (4*\Deltah,0) to (6*\Deltah,0);

      % Nodes
      \draw [black] (0*\Deltah,-\height) to (0+0*\Deltah,\height);
      \draw [black] (1*\Deltah,-\height) to (0+1*\Deltah,\height);
      \draw [black] (2*\Deltah,-\height) to (0+2*\Deltah,\height);
      \draw [black] (4*\Deltah,-\height) to (0+4*\Deltah,\height);
      \draw [black] (5*\Deltah,-\height) to (0+5*\Deltah,\height);
      \draw [black] (6*\Deltah,-\height) to (0+6*\Deltah,\height);

      % CCs
      \draw [black] (0*\Deltah+\Deltah/2,0) circle [radius=\radius];
      \draw [black] (1*\Deltah+\Deltah/2,0) circle [radius=\radius];
      \draw [black] (4*\Deltah+\Deltah/2,0) circle [radius=\radius];
      \draw [black] (5*\Deltah+\Deltah/2,0) circle [radius=\radius];

      \draw (0*\Deltah,0) node [black,below=\offSet] {$f_{n,1}$};
      \draw (1*\Deltah,0) node [black,below=\offSet] {$f_{n,2}$};
      \draw (2*\Deltah,0) node [black,below=\offSet] {$f_{n,3}$};
      \draw (4*\Deltah,0) node [black,below=\offSet] {$f_{n,sn-2}$};
      \draw (5*\Deltah,0) node [black,below=\offSet] {$f_{n,sn-1}$};
      \draw (6*\Deltah,0) node [black,below=\offSet] {$f_{n,sn}$};

      \draw (0*\Deltah+\Deltah/2,0) node [black,below=\offSet] {$f_{c,1}$};
      \draw (1*\Deltah+\Deltah/2,0) node [black,below=\offSet] {$f_{c,2}$};
      \draw (5*\Deltah-\Deltah/2,0) node [black,below=\offSet] {$f_{c,sc-1}$};
      \draw (6*\Deltah-\Deltah/2,0) node [black,below=\offSet] {$f_{c,sc}$};

    \end{tikzpicture}
    \caption{Index convention for node / face centered data}
  \end{center}
\end{figure}

\subsection{Boundary treatment for CC data}


\subsection{Boundary treatment for N data}


\end{document}