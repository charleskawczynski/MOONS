\documentclass[11pt]{article}
\newcommand{\PSCHAIN}{..}
\edef\PSCHAIN{\PSCHAIN/LATEX_INCLUDES}

\newcommand{\rootdir}{\PSCHAIN}
\newcommand{\VAR}{Success}



\begin{document}
\doublespacing
\MOONSTITLE
\maketitle

\section{Stretching function for non-uniform grids}
MOONS implements several stretching functions. These stretching factors were adopted from \cite{pletcher2012computational}.

\subsection{Roberts Transformation 1}

This transformation clusters more points near $y=0$ as the stretching parameter $\beta \rightarrow 1$.

\begin{equation}
	y
	=
	h
	\frac{(\beta+1)-(\beta-1) \left[ (\beta+1)/(\beta-1)^{1-\bar{y}} \right] }
	{\left[ (\beta+1)/(\beta-1)^{1-\bar{y}} \right]+1}
\end{equation}

\subsection{Roberts Transformation 2}
For this transformation, if $\alpha=0$, the mesh will be refined near $y=h$ only, whereas if $\alpha= \frac{1}{2}$, the mesh will be refined equally near $y=0$ and $y=h$.

\begin{equation}
	y
	=
	h
	\frac{
	(\beta + 2 \alpha)
	\left[ \frac{\beta+1}{\beta-1}^{(\bar{y}-\alpha)/(1-\alpha)} \right] - \beta + 2 \alpha}
	{
	(2\alpha+1)
	\left(1+\left[ \frac{\beta+1}{\beta-1}^{(\bar{y}-\alpha)/(1-\alpha)} \right]
	\right)
	}
\end{equation}

Where $h$ is the length of the domain. Note that

\begin{equation}
	\alpha = 0 \qquad \rightarrow \text{mesh will be refined near $y=h$ only}
\end{equation}
\begin{equation}
	\alpha = 1/2 \qquad \rightarrow \text{mesh will be refined near $y=h$ and $y=0$}
\end{equation}
\begin{equation}
	\bar{y} = \text{uniformly spaced grid}
\end{equation}

If there exists a boundary layer of thickness $\delta$, then an adequate stretching parameter, $\beta$, may be chosen as

\begin{equation}
	\beta = \left( 1 - \frac{\delta}{h} \right)^{-1/2}
	\qquad \qquad
	\qquad \qquad
	0 < \frac{\delta}{h} < 1
\end{equation}


\subsection{Roberts Transformation 3}
In this transformation, $\tau$ is the stretching parameter, which varies from zero (no stretching) to large values that produce the most refinement near $y=y_c$.

\begin{equation}
	y
	=
	y_c
	\left\{
	1
	+
	\frac{\sinh[\tau (\bar{y}-B)]}
	{\sinh(\tau B)}
	\right\}
\end{equation}

Where

\begin{equation}
	B
	=
	\frac{1}{2\tau}
	\ln
	\left[
	\frac{1+(e^\tau-1) (y_c/h)}{1+(e^{-\tau}-1) (y_c/h)}
	\right]
	\qquad \qquad \qquad \qquad
	0 <\tau < \infty
\end{equation}

\input{\rootdir/includes/include_bib.tex}





\end{document}