\documentclass[11pt]{article}
\usepackage{graphicx}    % needed for including graphics e.g. EPS, PS
\usepackage{epstopdf}
\usepackage{amsmath}
\usepackage{hyperref}
\usepackage{xspace}
\usepackage{mathtools}
\usepackage{tikz}
\usepackage{epsfig}
\usepackage{float}
\usepackage{natbib}
\usepackage{caption}
\usetikzlibrary{graphs}
\usepackage{subfigure}
\usepackage{setspace}
\usetikzlibrary{plotmarks}
\usetikzlibrary{decorations.pathmorphing}
\usetikzlibrary{graphs,graphs.standard,quotes}
\usepackage{ifthen}
\usepackage{tabularx,ragged2e,booktabs,caption}
\usepackage{bbding}

\tikzset{snake it/.style={decorate, decoration=snake}}

\newcommand{\height}{0.4}
\newcommand{\radius}{0.1}
\newcommand{\offSet}{12}
\newcommand{\Deltah}{2}
\newcommand{\dx}{1.5}
\newcommand{\dxc}{\dx*2}
\newcommand{\w}{1}

\newcommand{\qq}{$\qquad \qquad \qquad \qquad \quad$}
\newcommand{\qe}{$\qquad$}

\setlength{\oddsidemargin}{0.1in}
\setlength{\textwidth}{7.25in}

\setlength{\topmargin}{-1in}     %\topmargin: gap above header
\setlength{\headheight}{0in}     %\headheight: height of header
\setlength{\topskip}{0in}        %\topskip: between header and text
\setlength{\headsep}{0in}        
\setlength{\textheight}{692pt}   %\textheight: height of main text
\setlength{\textwidth}{7.5in}    % \textwidth: width of text
\setlength{\oddsidemargin}{-0.5in}  % \oddsidemargin: odd page left margin
\setlength{\evensidemargin}{0in} %\evensidemargin : even page left margin
\setlength{\parindent}{0.25in}   %\parindent: indentation of paragraphs
\setlength{\parskip}{0pt}        %\parskip: gap between paragraphs
\setlength{\voffset}{0.5in}

% Useful commands:

% \hfill		aligns-right everything right of \hfill

\begin{document}
\doublespacing
\title{Magnetohydrodynamic Object-Oriented Numerical Solver (MOONS)}
\author{C. Kawczynski \\
Department of Mechanical and Aerospace Engineering \\
University of California Los Angeles, USA\\
}
% \maketitle

\section{Embed extract indexes}

Consider a domain (see figure below) with fluid (middle) and solid (outside) portions.

\begin{figure}[h!] \label{fig:pic}
  \begin{center}
    \begin{tikzpicture}

    % Horizontal line
    \draw [black] (-\dx,0) to (9*\dx,0);

    % Ghost data & boundary
    \draw [snake it] (-\dx,-\height) -- ++(0,2*\height);
    \draw plot [only marks, mark=square*] coordinates {(-0.5*\dx,0)};

    % Dark walls
    \draw[line width=\w] (0,-\height)    -- ++(0,2*\height);
    \draw[line width=\w] (2*\dx,-\height)    -- ++(0,2*\height);
    \draw[line width=\w] (6*\dx,-\height)    -- ++(0,2*\height);
    \draw[line width=\w] (8*\dx,-\height)    -- ++(0,2*\height);

    % Slanted lines
    \foreach \y in {0,1,...,6}{ \draw[line width=\w] (-0.2,-\height+\y*0.1)    -- ++(0.2,0.2);}
    \foreach \y in {0,1,...,6}{ \draw[line width=\w] (2*\dx-0.2,-\height+\y*0.1)    -- ++(0.2,0.2);}
    \foreach \y in {0,1,...,6}{ \draw[line width=\w] (6*\dx,-\height+\y*0.1)    -- ++(0.2,+0.2);}
    \foreach \y in {0,1,...,6}{ \draw[line width=\w] (8*\dx,-\height+\y*0.1)    -- ++(0.2,+0.2);}

    % interior data
    \foreach \x in {0,1,...,6}{
          \draw (\x*\dx+\dx,-\height) -- ++(0,2*\height);
          \draw (\x*\dx+\dx*0.5,0) circle [radius=\radius];
    }
    \draw (7*\dx+\dx/2,0) circle [radius=\radius];

    % Ghost data & boundary
    \draw plot [only marks, mark=square*] coordinates {(8*\dx + 0.5*\dx,0)};
    \draw [snake it] (9*\dx,-\height) -- ++(0,2*\height);

    \end{tikzpicture}
    \caption*{ghost \qquad \quad Solid \qq Fluid \qq Solid \qquad \quad ghost}
  \end{center}
\end{figure}
\vspace{-1em}
\begin{itemize}
    \item The first letter (N,C) indicates the data type (node,cell center) along a given direction.
    \item The second letter (I,E) indicates whether ghost points are included or excluded.
    \item The second letter may be (B), which indicates the boundary point is included (for node data).
    \item Let $N$ be the total number of cells (should be clear from next section).
\end{itemize}

\subsection{Conservation}
\begin{equation}\begin{aligned}
    N = N_{wbot} + N_{wtop} + N_i = 2 + 2 + 4 = 8 \\
\end{aligned}\end{equation}
\subsection{NI}
\begin{equation}\begin{aligned}
    NI1 = 4 = N_{wbot} + 2  \qquad \text{\Checkmark} \\
\end{aligned}\end{equation}
\begin{equation}\begin{aligned}
    NI2 = 8 = N - N_{wtop} + x  \qquad \text{\Checkmark} \\
    = 8 - 2 + x, \qquad \rightarrow x = 2 \\
    = N - N_{wtop} + 2
\end{aligned}\end{equation}

\subsection{CE}
\begin{equation}\begin{aligned}
    CE1 = 4 = N_{wbot} + 2  \qquad \text{\Checkmark} \\
\end{aligned}\end{equation}
\begin{equation}\begin{aligned}
    CE2 = 7 = N - N_{wtop} + x  \qquad \text{\Checkmark} \\
    = 7 = 8 - 2 + x, \qquad \rightarrow x = 1
\end{aligned}\end{equation}

\subsection{CI}
\begin{equation}\begin{aligned}
    CI1 = 3 = N_{wbot} + 1  \qquad \text{\Checkmark} \\
\end{aligned}\end{equation}
\begin{equation}\begin{aligned}
    CI2 = 8 = N - N_{wtop} + x, \qquad \rightarrow x = 2  \qquad \text{\Checkmark} \\
    = 8 - 2
\end{aligned}\end{equation}


\end{document}