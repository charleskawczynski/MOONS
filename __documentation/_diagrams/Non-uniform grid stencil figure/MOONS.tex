\documentclass[landscape, 12pt]{report}
\usepackage{graphicx}    % needed for including graphics e.g. EPS, PS
\usepackage{epstopdf}
\usepackage{amsmath}
\usepackage{hyperref}
\usepackage{xspace}
\usepackage{mathtools}
\usepackage{tikz}
\usepackage{epsfig}
\usepackage{float}
\usepackage{natbib}
\usepackage{subfigure}
\usepackage{setspace}
\usepackage{tabularx,ragged2e,booktabs,caption}

\newcommand{\height}{0.5}
\newcommand{\radius}{0.1}
\newcommand{\offSet}{20}
\newcommand{\dx}{1.6666}

\setlength{\oddsidemargin}{0.1in}
\setlength{\textwidth}{7.25in}

\setlength{\topmargin}{-1in}     %\topmargin: gap above header
\setlength{\headheight}{0in}     %\headheight: height of header
\setlength{\topskip}{0in}        %\topskip: between header and text
\setlength{\headsep}{0in}        
\setlength{\textheight}{692pt}   %\textheight: height of main text
\setlength{\textwidth}{7.5in}    % \textwidth: width of text
\setlength{\oddsidemargin}{-0.5in}  % \oddsidemargin: odd page left margin
\setlength{\evensidemargin}{0in} %\evensidemargin : even page left margin
\setlength{\parindent}{0.25in}   %\parindent: indentation of paragraphs
\setlength{\parskip}{0pt}        %\parskip: gap between paragraphs
\setlength{\voffset}{0.5in}


% Useful commands:

% \hfill		aligns-right everything right of \hfill

\begin{document}
\doublespacing
\title{Magnetohydrodynamic Object-Oriented Numerical Solver (MOONS)}
\author{C. Kawczynski \\
Department of Mechanical and Aerospace Engineering \\
University of California Los Angeles, USA\\
}
\maketitle

\section{Non-Uniform Grid Stencil Results Figure}

Note that $k$ and or $j$ may be positive or negative. Similarly, $\alpha_k$ and or $\beta_j$ may be positive or negative.

\begin{figure}[h!]
  \begin{center}
    \begin{tikzpicture}[scale=3]
      \draw [->] (0,0) -- ++(3*\dx,0);

      \draw [->] (0,-\height/2) -- ++(1*\dx,0) node[midway,anchor=north]  {$\alpha_k$};
      \draw [->] (0,-\height*1.5) -- ++(2*\dx,0) node[midway,anchor=north]  {$\beta_j$};

      % Nodes
      \draw [black] (0*\dx,-2*\height)   node [black,anchor=south east] {$x_{i}$} -- ++(0,3*\height) node [black,anchor=west] {$f_{i}$};
      \draw [black] (1*\dx,-\height)   node [black,anchor=south west] {$x_{i+k}$} -- ++(0,2*\height)   node [black,anchor=west] {$f_{i+k}$};
      \draw [black] (2*\dx,-\height)   node [black,anchor=south west] {$x_{i+j}$} -- ++(0,2*\height)   node [black,anchor=west] {$f_{i+j}$};

    \end{tikzpicture}
    \caption{Stencil possibility 1}
  \end{center}
\end{figure}


\begin{figure}[h!]
  \begin{center}
    \begin{tikzpicture}[scale=3]
      \draw [<->] (-\dx/2,0) -- ++(3*\dx,0);

      \draw [<-] (0,-\height/2) -- ++(1*\dx,0) node[midway,anchor=north]  {$\alpha_k$};
      \draw [->] (\dx,-\height*1.5) -- ++(\dx,0) node[midway,anchor=north]  {$\beta_j$};

      % Nodes
      \draw [black] (1*\dx,-2*\height)   node [black,anchor=south east] {$x_{i}$} -- ++(0,3*\height) node [black,anchor=west] {$f_{i}$};
      \draw [black] (0*\dx,-\height)   node [black,anchor=south east] {$x_{i+k}$} -- ++(0,2*\height)   node [black,anchor=west] {$f_{i+k}$};
      \draw [black] (2*\dx,-\height)   node [black,anchor=south west] {$x_{i+j}$} -- ++(0,2*\height)   node [black,anchor=west] {$f_{i+j}$};

    \end{tikzpicture}
    \caption{Stencil possibility 2}
  \end{center}
\end{figure}


\end{document}