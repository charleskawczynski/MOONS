\documentclass[11pt]{article}
\newcommand{\PSCHAIN}{..}
\edef\PSCHAIN{\PSCHAIN/LATEX_INCLUDES}

\newcommand{\rootdir}{\PSCHAIN}
\newcommand{\VAR}{Success}



\begin{document}
\doublespacing
\MOONSTITLE
\maketitle

\section{1D Diffusion for code verification}
The governing equation is

\begin{equation}
	\frac{\PD T}{\PD t} = \alpha \frac{\PD^2 T}{\PD x^2} + \Phi(x,t)
\end{equation}

With the general solution

\begin{equation}
	T(x,t) = \int_0^t \int_0^L G(x,\xi,t-\tau) \Phi(\xi,\tau) d\xi d\tau
	+
	\int_0^L G(x,\xi,t) f(\xi) d\xi
\end{equation}

Where $f(\xi)$ is the initial condition and

\begin{equation}
	G(x,\xi,t) = \frac{2}{L} \sum_{n=1}^{\infty}
	\sin(n\pi x/L)
	\sin(n\pi \xi/L)
	e^{-\lambda_n t}
\end{equation}

\begin{equation}
	\lambda_n = \frac{\alpha n^2\pi^2}{L^2}
\end{equation}

For the particular problem of interest, we have

\begin{equation}
	\Phi(x,t) = \Phi(t) = e^{-t} \qquad \qquad \qquad f(\xi) = 0
\end{equation}

Which yields

\begin{equation}
	T(x,t) = \int_0^t \int_0^L G(x,\xi,t-\tau) e^{-\tau} d\xi d\tau
\end{equation}

Combining the equations, and then the exponential terms, we have

\begin{equation}
	T(x,t) = \int_0^t \int_0^L
	\frac{2}{L} \sum_{n=1}^{\infty}
	\sin(n\pi x/L)
	\sin(n\pi \xi/L)
	e^{-\lambda_n (t-\tau)}
	e^{-\tau}
	d\xi d\tau
\end{equation}

We may exchange the order of the summation and the integral to get

\begin{equation}
	T(x,t)
	=
	\sum_{n=1}^{\infty}
	\frac{2}{L}
	\int_0^t \int_0^L
	\sin(n\pi x/L)
	\sin(n\pi \xi/L)
	e^{-\lambda_n (t-\tau)}
	e^{-\tau}
	d\xi d\tau
\end{equation}

\begin{equation}
	=
	\sum_{n=1}^{\infty}
	\sin(n\pi x/L)
	\frac{2}{L}
	\int_0^t \int_0^L
	\sin(n\pi \xi/L)
	e^{-\lambda_n (t-\tau)}
	e^{-\tau}
	d\xi d\tau
\end{equation}

Using Wolfram Alpha

\begin{equation}
	\frac{2}{L}
	\int_0^t \int_0^L
	\sin(n\pi \xi/L)
	e^{-\lambda_n (t-\tau)}
	e^{-\tau}
	d\xi d\tau
	=
	- \frac{2 (\cos(\pi n) - 1) e^{-\lambda_n t} (e^{(\lambda_n-1)t}-1)}{\pi (\lambda_n-1)n}
\end{equation}

\begin{equation}
	=
	- \frac{2 ((-1)^n - 1) (e^{(\lambda_n-1)t-\lambda_n t}-e^{-\lambda_n t})}{\pi (\lambda_n-1)n}
	=
	\frac{2 (1 - (-1)^n) (e^{-t}-e^{-\lambda_n t})}{\pi (\lambda_n-1)n}
\end{equation}

Therefore

\begin{equation}
	\boxed{
	T(x,t)
	=
	\sum_{n=1}^{\infty}
	\sin \left( \frac{n\pi x}{L} \right)
	\frac{2 (1 - (-1)^n) (e^{-t}-e^{-\lambda_n t})}{\pi (\lambda_n-1)n}
	}
\end{equation}


\begin{equation}
	\boxed{
	\lambda_n = \frac{\alpha n^2\pi^2}{L^2}
	}
\end{equation}

\end{document}