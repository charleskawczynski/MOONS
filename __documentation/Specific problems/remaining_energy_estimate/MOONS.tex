\documentclass[11pt]{article}
\newcommand{\VAR}{Success}


\begin{document}
\doublespacing
\MOONSTITLE
\maketitle

\section{Remaining error estimate}
Consider a simulation where the difference between the SS energy and instantaneous energy is decaying with time. We may write

\begin{equation}
	\frac{\PD E}{dt} = B e^{-n t}
\end{equation}

\begin{equation}
	n = - \frac{1}{t} \frac{\PD E}{dt} = B e^{- t}
\end{equation}

From this, we can estimate the energy at time $\theta$, from known information at time $T$, as

\begin{equation}
	E(SS) = E(T) + \int_{T}^{\theta} \frac{\PD E(\tau)}{dt} d\tau
\end{equation}

Plugging in and integrating we have

\begin{equation} \begin{aligned}
	E(SS) &= E(T) + \int_{T}^{\theta} \frac{\PD E}{dt} d\tau \\
	      &= E(T) + \int_{T}^{\theta} B e^{-n t} d\tau \\
	      &= E(T) + \frac{B}{n} (e^{-n T} - e^{-n \theta}), \qquad \text{let } \theta \rightarrow \infty \\
	      &= E(T) + \frac{B}{n} e^{-n T} \\
\end{aligned} \end{equation}

Let

\begin{equation} \begin{aligned}
	\varepsilon &= |E(SS)-E(T)| = \frac{B}{n} e^{-n t}
\end{aligned} \end{equation}

$B$ and $n$ may be estimated by extrapolating $\frac{\PD E}{dt}$ back to $t=0$ from $T$ on a log scale. Following this, we have



\end{document}