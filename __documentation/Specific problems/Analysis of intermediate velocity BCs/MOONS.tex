\documentclass[11pt]{article}
\usepackage{graphicx}    % needed for including graphics e.g. EPS, PS
\usepackage{epstopdf}
\usepackage{amsmath}
\usepackage{hyperref}
\usepackage{xspace}
\usepackage{mathtools}
\usepackage{tikz}
\usepackage{epsfig}
\usepackage{float}
\usepackage{natbib}
\usepackage{subfigure}
\usepackage{setspace}
\usepackage{tabularx,ragged2e,booktabs,caption}


\setlength{\oddsidemargin}{0.1in}
\setlength{\textwidth}{7.25in}

\setlength{\topmargin}{-1in}     %\topmargin: gap above header
\setlength{\headheight}{0in}     %\headheight: height of header
\setlength{\topskip}{0in}        %\topskip: between header and text
\setlength{\headsep}{0in}
\setlength{\textheight}{692pt}   %\textheight: height of main text
\setlength{\textwidth}{7.5in}    % \textwidth: width of text
\setlength{\oddsidemargin}{-0.5in}  % \oddsidemargin: odd page left margin
\setlength{\evensidemargin}{0in} %\evensidemargin : even page left margin
\setlength{\parindent}{0.25in}   %\parindent: indentation of paragraphs
\setlength{\parskip}{0pt}        %\parskip: gap between paragraphs
\setlength{\voffset}{0.5in}


% Useful commands:

% \hfill		aligns-right everything right of \hfill

\begin{document}
\doublespacing
\title{Magnetohydrodynamic Object-Oriented Numerical Solver (MOONS)}
\author{C. Kawczynski \\
Department of Mechanical and Aerospace Engineering \\
University of California Los Angeles, USA\\
}
\maketitle

\section{Intermediate velocity BCs}
The ith component of the momentum equation is
\begin{equation}
	\frac{\PD u_i}{\PD t} +
	=
	- \frac{\PD (u_i u_j)}{\PD x_j}
	- \frac{\PD p}{\PD x_i}
	+ \frac{1}{Re}
	\frac{\PD^2 u_i}{\PD x_j^2}
	+ \frac{1}{Fr^2}
	g_i
	+ \frac{Gr}{Re^2}
	T g_i
	+ \frac{Ha^2}{Re}
	(j \times B)_i
\end{equation}

And applying explicit Euler, we have

\begin{equation}
	u_i^{n+1}
	=
	u_i^n
	+
	\Delta t
	\left\{
	- \frac{\PD (u_i u_j)}{\PD x_j}
	- \frac{\PD p}{\PD x_i}
	+ \frac{1}{Re}
	\frac{\PD^2 u_i}{\PD x_j^2}
	+ \frac{1}{Fr^2}
	g_i
	+ \frac{Gr}{Re^2}
	T g_i
	+ \frac{Ha^2}{Re}
	(j \times B)_i
	\right\}
\end{equation}

Using the projection method, and treating pressure implicitly, we have

\begin{equation}
	u_i^{n+1}
	=
	u_i^n
	+
	\Delta t
	F_i^n
	-
	\Delta t
	\frac{\PD p^{n+1}}{\PD x_i}
\end{equation}

Where

\begin{equation}
	F_i
	=
	- \frac{\PD (u_i u_j)}{\PD x_j}
	+ \frac{1}{Re}
	\frac{\PD^2 u_i}{\PD x_j^2}
	+ \frac{1}{Fr^2}
	g_i
	+ \frac{Gr}{Re^2}
	T g_i
	+ \frac{Ha^2}{Re}
	(j \times B)_i
\end{equation}

Taking the divergence of this, we have

\begin{equation}
	0
	=
	\frac{\PD }{\PD x_i}
	\left(
	u_i^n
	+
	\Delta t
	F_i^n
	\right)
	-
	\Delta t
	\frac{\PD^2 p^{n+1}}{\PD x_i^2}
\end{equation}

Which yields Laplace's equation for pressure:

\begin{equation}
	\frac{\PD^2 p^{n+1}}{\PD x_i^2}
	=
	\frac{1}{\Delta t}
	\frac{\PD }{\PD x_i}
	\left(
	u_i^n
	+
	\Delta t
	F_i^n
	\right)
\end{equation}

\end{document}