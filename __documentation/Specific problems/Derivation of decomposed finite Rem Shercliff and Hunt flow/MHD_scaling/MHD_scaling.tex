\documentclass[11pt]{article}
% \documentclass[3p,twocolumn,10pt]{elsarticle}
\usepackage{graphicx}    % needed for including graphics e.g. EPS, PS
\usepackage{epstopdf}
\usepackage{amsmath}
\usepackage{amssymb}
\usepackage{hyperref}
\usepackage{xspace}
\usepackage{mathtools}
\usepackage{tikz}
\usepackage{epsfig}
\usepackage{float}
%\usepackage{natbib}
\usepackage{subfigure}
\usepackage{setspace}
\usepackage{tabularx,ragged2e,booktabs,caption}

\usepackage{xcolor}
\usepackage{xspace}
\usepackage{longtable}
\usepackage{mathtools}
\usepackage{natbib}
\usepackage{setspace}
\usepackage{ragged2e}
\usepackage{etoolbox}
\usepackage{geometry}


\newcommand{\B}{\mathbf{B}}
\newcommand{\BS}{B^{SH}}
\newcommand{\BT}{\tilde{B}}
\renewcommand{\H}{\mathbf{H}}
\newcommand{\C}{\mathbf{C}}
\newcommand{\U}{\mathbf{u}}
\newcommand{\SI}{\sigma}
\newcommand{\M}{\mu}
\newcommand{\curl}{\nabla \times}

\newcommand{\A}{\mathbf{A}}
\newcommand{\PD}{\partial}
\newcommand{\MC}{\mathcal}
\newcommand{\BM}{\frac{\B}{\mu}}
\newcommand{\J}{\mathbf{j}}
\newcommand{\E}{\mathbf{E}}
\newcommand{\N}{\mathbf{n}}
\newcommand{\JS}{\frac{\J}{\sigma}}
\newcommand{\JoS}{\frac{\J^1}{\sigma}}
\newcommand{\JSS}{\frac{\J^2}{\sigma}}
\newcommand{\SII}{\sigma^{-1}}
\newcommand{\MO}{\overline{\mu}}
\newcommand{\SO}{\overline{\sigma}}
\newcommand{\DOT}{\text{\textbullet}}

\newcommand{\Rei}{\frac{1}{Re}}
\newcommand{\Remi}{\frac{1}{Re_m}}
\newcommand{\NRemi}{\frac{N}{Re_m}}

% \setlength{\textfloatsep}{0.1cm}
\newcommand{\volume}{\mathop{\ooalign{\hfil$V$\hfil\cr\kern0.08em--\hfil\cr}}\nolimits}
\newcommand{\figWidth}{0.45\textwidth}
\newcommand{\figSuffix}{_png.png}
\newcommand{\pushright}[1]{\ifmeasuring@#1\else\omit\hfill$\displaystyle#1$\fi\ignorespaces}
\newcommand{\hfillMath}{\hskip \textwidth minus \textwidth}
% \renewcommand{\arraystretch}{0.0} % Removes vertical spaces before/after tabularx


\setlength{\oddsidemargin}{0.1in}
\setlength{\textwidth}{7.25in}

\setlength{\topmargin}{-1in}     %\topmargin: gap above header
\setlength{\headheight}{0in}     %\headheight: height of header
\setlength{\topskip}{0in}        %\topskip: between header and text
\setlength{\headsep}{0in}        
\setlength{\textheight}{692pt}   %\textheight: height of main text
\setlength{\textwidth}{7.5in}    % \textwidth: width of text
\setlength{\oddsidemargin}{-0.5in}  % \oddsidemargin: odd page left margin
\setlength{\evensidemargin}{0in} %\evensidemargin : even page left margin
\setlength{\parindent}{0.25in}   %\parindent: indentation of paragraphs
\setlength{\parskip}{0pt}        %\parskip: gap between paragraphs
\setlength{\voffset}{0.5in}

% Useful commands:

% \hfill    aligns-right everything right of \hfill

\begin{document}
\doublespacing
\title{Scaling for the MHD equations}
\author{C. Kawczynski \\
Department of Mechanical and Aerospace Engineering \\
University of California Los Angeles, USA\\
}
\maketitle

\section{Dimensional equations}
The dimensional momentum and induction equations for uniform material properties are

\begin{equation}\begin{aligned}
\PD_t u_i+u_j\PD_j u_i = -\frac{1}{\rho} \PD_i p+ \nu \PD_{jj} u_i + \frac{1}{\rho \mu} (B_j \PD_j B_i - \tfrac{1}{2} \PD_i B_j B_j) \\
\PD_t B_i = \frac{1}{\sigma \mu} \PD_{jj} B_i - \PD_j (u_j B_i - u_i B_j)
\end{aligned}\end{equation}
Grouping mechanical and magnetic pressure, we have
\begin{equation}\boxed{\begin{aligned}
\PD_t u_i+u_j\PD_j u_i = -\frac{1}{\rho} \PD_i p+ \nu \PD_{jj} u_i + \frac{1}{\rho \mu} B_j \PD_j B_i \\
\PD_t B_i = \frac{1}{\sigma \mu} \PD_{jj} B_i - \PD_j (u_j B_i - u_i B_j)
\end{aligned}}\end{equation}

\section{Flexible scaling}
Scales:
\begin{equation}\begin{aligned}
	[x,y,z] = L, \qquad
	[t] = t_c, \qquad
	[u] = U, \qquad
	[B] = B, \qquad
	[P] = \frac{\rho U L}{t_c}
\end{aligned} \end{equation}
Before algebra
\begin{equation}\begin{aligned}
\frac{U}{t_c} \PD_t u_i+ \frac{U^2}{L} u_j\PD_j u_i =-\frac{\rho U L}{\rho t_c L}\PD_i p+ \frac{U \nu}{L^2} \PD_{jj} u_i + \frac{B^2}{\rho L \mu} B_j \PD_j B_i \\
\frac{1}{t_c} \PD_t B_i + \frac{U}{L} \PD_j (u_j B_i - u_i B_j) = \frac{1}{\sigma \mu L^2} \PD_{jj} B_i
\end{aligned} \end{equation}
After algebra
\begin{equation}\begin{aligned}
\PD_t u_i+ \frac{U t_c}{L} u_j\PD_j u_i =-\PD_i p + \frac{\nu t_c}{L^2} \PD_{jj} u_i + \frac{B^2 t_c}{\rho \mu U L} B_j \PD_j B_i \\
\PD_t B_i + \frac{U t_c}{L} \PD_j (u_j B_i - u_i B_j) = \frac{t_c}{\sigma \mu L^2} \PD_{jj} B_i
\end{aligned} \end{equation}
Simplified
\begin{equation}\boxed{\begin{aligned}
\PD_t u_i+ C u_j\PD_j u_i = - \PD_i p+ D \PD_{jj} u_i + E B_j \PD_j B_i \\
\PD_t B_i + C \PD_j (u_j B_i - u_i B_j) = A \PD_{jj} B_i
\end{aligned}}\end{equation}
Where
\begin{equation}\boxed{\begin{aligned}
	C = \frac{U t_c}{L}, \qquad
	D = \frac{\nu t_c}{L^2}, \qquad
	E = \frac{B^2 t_c}{\rho \mu U L}, \qquad
	A = \frac{t_c}{\sigma \mu L^2}, \qquad
\end{aligned}}\end{equation}
Relation to dimensionless groups:
\begin{equation}\begin{aligned}
	\frac{C}{D} = Re, \qquad
	\frac{C}{A} = Re_m, \qquad
	\frac{E}{D} = \frac{Ha^2}{Re_m}, \qquad
	\frac{E}{C} = \frac{Ha^2}{Re_m} \frac{1}{Re}, \qquad
	\frac{E}{A} = \frac{Ha^2}{Re}, \qquad
\end{aligned} \end{equation}
% Details
% \begin{equation}\begin{aligned}
% 	\frac{C}{D} = Re, \qquad
% 	\frac{C}{A} = Re_m, \qquad
% 	\frac{E}{D} = \frac{B^2 L^2 \sigma}{\rho \nu \mu U \sigma L} = Ha^2 \frac{1}{\mu U \sigma L} = \frac{Ha^2}{Re_m}, \qquad
% 	\frac{E}{C} = \frac{E D}{C D} = \frac{E}{D} \frac{D}{C} = \frac{Ha^2}{Re_m} \frac{1}{Re}, \qquad
% \end{aligned} \end{equation}
Simplification for specific time scales:
\begin{equation}\begin{aligned}
	t_c = \frac{L}{U} \rightarrow \qquad C = 1, D = \frac{1}{Re}, A = \frac{1}{Re_m}, E = \frac{Ha^2}{Re Re_m} \\
	t_c = \frac{L^2}{\nu} \rightarrow \qquad D = 1, C = Re, E = \frac{Ha^2}{Re_m}, A = \frac{Re}{Re_m} \\
	t_c = \frac{\rho \mu U L}{B^2} \rightarrow \qquad E = 1, D = \frac{Re_m}{Ha^2}, A = \frac{Re}{Ha^2}, C = \frac{Re Re_m}{Ha^2} \\
	t_c = \sigma \mu L^2 \rightarrow \qquad A = 1,E = \frac{Ha^2}{Re}, C = Re_m, D = \frac{Re_m}{Re} \\
\end{aligned} \end{equation}

\section{Shercliff scaling 1}
Scales:
\begin{equation}\begin{aligned}
	[x,y,z] = L, \qquad
	[t] = L^2 / \nu, \qquad
	[u] = U, \qquad
	[B] = U (\sigma \rho \nu)^{1/2} \times \mu \quad \text{(Shercliff uses H)}, \qquad
	[P] = \rho U \nu / L
\end{aligned} \end{equation}
Before algebra
\begin{equation}\begin{aligned}
\frac{U \nu}{L^2} \PD_t u_i+ \frac{U^2}{L} u_j\PD_j u_i =-\frac{\rho \nu U}{\rho L^2}\PD_i p+ \frac{U \nu}{L^2} \PD_{jj} u_i + \frac{U^2 (\sigma \rho \nu) \mu^2}{\rho L \mu} B_j \PD_j B_i \\
\frac{\nu}{L^2} \PD_t B_i + \frac{U}{L} \PD_j (u_j B_i - u_i B_j) = \frac{1}{\sigma \mu L^2} \PD_{jj} B_i
\end{aligned} \end{equation}
Simplified
\begin{equation}\begin{aligned}
\PD_t u_i+ Re u_j\PD_j u_i =-\PD_i p+ \PD_{jj} u_i + Re_m B_j \PD_j B_i \\
Re^{-1} \PD_t B_i + \PD_j (u_j B_i - u_i B_j) = Re_m^{-1} \PD_{jj} B_i
\end{aligned} \end{equation}

\section{My scaling choice}
Scales:
\begin{equation}\begin{aligned}
	[x,y,z] = L, \qquad
	[t] = L / U, \qquad
	[u] = U, \qquad
	[B] = U (\sigma \rho \nu)^{1/2} \times \mu \quad \text{(Shercliff uses H)}, \qquad
	[P] = \rho U \nu / L
\end{aligned} \end{equation}
Before algebra
\begin{equation}\begin{aligned}
\frac{U^2}{L} \PD_t u_i+ \frac{U^2}{L} u_j\PD_j u_i =-\frac{\rho \nu U}{\rho L^2}\PD_i p+ \frac{U \nu}{L^2} \PD_{jj} u_i + \frac{U^2 (\sigma \rho \nu) \mu^2}{\rho L \mu} B_j \PD_j B_i \\
\frac{U}{L} \PD_t B_i + \frac{U}{L} \PD_j (u_j B_i - u_i B_j) = \frac{1}{\sigma \mu L^2} \PD_{jj} B_i
\end{aligned} \end{equation}
Simplified
\begin{equation}\begin{aligned}
\PD_t u_i + u_j \PD_j u_i = -Re^{-1}\PD_i p + Re^{-1} \PD_{jj} u_i + Re^{-1} Re_m B_j \PD_j B_i \\
\PD_t B_i + \PD_j (u_j B_i - u_i B_j) = Re_m^{-1} \PD_{jj} B_i
\end{aligned} \end{equation}

\section{Shercliff scaling 2}
Scales:
\begin{equation}\begin{aligned}
	[x,y,z] = L, \qquad
	[t] = L^2 / \nu, \qquad
	[u] = U, \qquad
	[B] = B, \qquad
	[P] = \rho U \nu / L
\end{aligned} \end{equation}
Before algebra
\begin{equation}\begin{aligned}
\frac{U \nu}{L^2} \PD_t u_i+ \frac{U^2}{L} u_j\PD_j u_i =-\frac{\rho \nu U}{\rho L^2}\PD_i p+ \frac{U \nu}{L^2} \PD_{jj} u_i + \frac{B^2}{\rho L \mu} B_j \PD_j B_i \\
\frac{\nu}{L^2} \PD_t B_i + \frac{U}{L} \PD_j (u_j B_i - u_i B_j) = \frac{1}{\sigma \mu L^2} \PD_{jj} B_i
\end{aligned} \end{equation}
Simplified
\begin{equation}\begin{aligned}
\PD_t u_i+ Re u_j\PD_j u_i =-\PD_i p+ \PD_{jj} u_i + \frac{Ha^2}{Re_m} B_j \PD_j B_i \\
Re^{-1} \PD_t B_i + \PD_j (u_j B_i - u_i B_j) = Re_m^{-1} \PD_{jj} B_i
\end{aligned} \end{equation}

\section{Shercliff scaling 3}
Scales:
\begin{equation}\begin{aligned}
	[x,y,z] = L, \qquad
	[t] = L^2 / \nu, \qquad
	[u] = U, \qquad
	[B] = B \text{ (for $\B$)}, \qquad
	[j] = \frac{1}{L} U (\sigma \rho \nu)^{1/2} \text{ (for $\J$)}, \qquad
	[P] = \rho U \nu / L
\end{aligned} \end{equation}
Before algebra
\begin{equation}\begin{aligned}
\frac{U \nu}{L^2} \PD_t u_i+ \frac{U^2}{L} u_j\PD_j u_i =-\frac{\rho \nu U}{\rho L^2}\PD_i p+ \frac{U \nu}{L^2} \PD_{jj} u_i + \frac{B U (\sigma \rho \nu)^{1/2} \mu}{\rho L \mu} B_j \PD_j B_i \\
\frac{B \nu}{L^2} \PD_t B_i + \frac{B U}{L} \PD_j (u_j B_i - u_i B_j) = \frac{U (\sigma \rho \nu)^{1/2} \mu}{\sigma \mu L^2} \PD_{jj} B_i
\end{aligned} \end{equation}
Simplified
\begin{equation}\begin{aligned}
\PD_t u_i+ Re u_j\PD_j u_i =-\PD_i p+ \PD_{jj} u_i + Ha B_j \PD_j B_i \\
\PD_t B_i + Re \PD_j (u_j B_i - u_i B_j) = \frac{Re}{Ha} \PD_{jj} B_i \\
\frac{Ha}{Re} \PD_t B_i + Ha \PD_j (u_j B_i - u_i B_j) = \PD_{jj} B_i \\
\PD_t B_i + Re \PD_j (u_j B_i - u_i B_j) = \frac{Ha}{Re_m} \PD_{jj} B_i
\end{aligned} \end{equation}

\section{Hartmann Reynolds magnetic Reynolds number proof}
The definition for $Ha$ is
\begin{equation}\begin{aligned}
Ha^2 = B^2 L^2 \frac{\sigma}{\rho \nu} \\
\end{aligned} \end{equation}
We can relate the scales of the Lorentz and advection terms
\begin{equation}\begin{aligned}
\frac{U^2}{L} = \frac{B^2}{\rho \mu L} \\
B^2 = U^2 \rho \mu \\
\end{aligned} \end{equation}
Using this in the definition of the Hartmann number yields

\begin{equation}\begin{aligned}
Ha^2 = B^2 L^2 \frac{\sigma}{\rho \nu} \\
Ha^2 = (U^2 \rho \mu) L^2 \frac{\sigma}{\rho \nu} \\
Ha^2 = U^2 \mu L^2 \frac{\sigma}{\nu} \\
Ha^2 = Re U \mu L \sigma \\
Ha^2 = Re Re_m \\
\end{aligned} \end{equation}

\end{document}