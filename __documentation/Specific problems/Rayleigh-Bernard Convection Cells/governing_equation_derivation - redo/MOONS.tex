\documentclass[11pt]{article}
\newcommand{\PSCHAIN}{..}
\edef\PSCHAIN{\PSCHAIN/LATEX_INCLUDES}

\newcommand{\rootdir}{\PSCHAIN}
\newcommand{\VAR}{Success}



\begin{document}
\doublespacing
\MOONSTITLE
% \maketitle

\section{Governing equations}
The dimensional momentum equation is
\begin{equation}\begin{aligned}
\PD_t u_i+u_j\PD_j u_i = -\frac{1}{\rho} \PD_i p - g_i + \frac{\mu}{\rho} \PD_{jj} u_i \\
\end{aligned} \end{equation}
Expanding, neglecting viscous effects, and using the material derivative we have
\begin{equation}\begin{aligned}
D_t u = -\frac{1}{\rho} \PD_x p     \\
D_t v = -\frac{1}{\rho} \PD_y p     \\
D_t w = -\frac{1}{\rho} \PD_z p - g \\
\end{aligned} \end{equation}
Using Boussinesq approximation,
\begin{equation}\begin{aligned}
g \rho = g \rho_0 - g \rho_0 \alpha \ttheta
\end{aligned} \end{equation}
Here, $\ttheta, \alpha$ are temperature and coefficient of expansion.
Plugging in $\rho$ for gravity, assuming $\rho=\rho_0$ for other terms and absorbing $\theta$ independent $g$ term into pressure, we have
\begin{equation}\begin{aligned}
D_t u = -\frac{1}{\rho} \PD_x P     \\
D_t v = -\frac{1}{\rho} \PD_y P     \\
D_t w = -\frac{1}{\rho} \PD_z P + \gamma \ttheta \\
\end{aligned} \end{equation}
Where $\rho$ is constant, $\gamma=g\alpha$ and $P=p + g\rho z$.

Eq. for conduction of heat is
\begin{equation}\begin{aligned}
D_t \ttheta = \kappa (\PD_{xx} \ttheta+\PD_{yy} \ttheta+\PD_{zz} \ttheta) \\
\end{aligned} \end{equation}
Domain is $0 \le z \le \xi$, at which $\U=\zero$ and $\PD_z \ttheta = \beta = \text{constant}$. If the equilibrium is stable, then $\beta$ is positive. If unstable with a higher temperature below, then $\beta$ is negative.
Let the temperature be written as the equilibrium temperature and departure from equilibrium $\theta = \ttheta - \Theta$. Substituting this and absorbing $\Theta$ in momentum yields
\begin{equation}\begin{aligned}
D_t u = -\frac{1}{\rho} \PD_x \omega     \\
D_t v = -\frac{1}{\rho} \PD_y \omega     \\
D_t w = -\frac{1}{\rho} \PD_z \omega + \gamma \theta \\
\end{aligned} \end{equation}
Where $\omega = P - \rho \gamma \int \Theta dz$.
Applying the same substitution to the energy equation, and assuming $\theta = \theta(z)$, we have
\begin{equation}\begin{aligned}
D_t \ttheta = & \PD_t \theta + \PD_t \Theta + w \PD_z \theta + w \PD_z \Theta, \qquad \text{assume eq. temp steady \& zero slope} \\
            = & \PD_t \theta + w \PD_z \theta  \\
            = & \PD_t \theta + w \beta  \\
\end{aligned} \end{equation}
Assuming $u,v,w,\theta$ are small, neglecting 2nd order terms, and including incompressibility, we finally have
\begin{equation}\begin{aligned}
\PD_x u + \PD_y v + \PD_z w = 0 \\
\PD_t u = -\frac{1}{\rho} \PD_x \omega     \\
\PD_t v = -\frac{1}{\rho} \PD_y \omega     \\
\PD_t w = -\frac{1}{\rho} \PD_z \omega + \gamma \theta \\
\PD_t \theta + w \beta = \kappa \PD_{zz} \theta \\
\end{aligned} \end{equation}
This constitutes our problem.

\newpage
\section{governing equations}
\begin{equation}\begin{aligned}
\PD_x u + \PD_y v + \PD_z w = 0 \\
\PD_t u = -\frac{1}{\rho} \PD_x \omega     \\
\PD_t v = -\frac{1}{\rho} \PD_y \omega     \\
\PD_t w = -\frac{1}{\rho} \PD_z \omega + \gamma \theta \\
\PD_t \theta + w \beta = \kappa \PD_{zz} \theta \\
\end{aligned} \end{equation}

Assuming small quantities are proportional to
\begin{equation}\begin{aligned}
e^{nt} e^{ilx} e^{imy} e^{isz} \\
\end{aligned} \end{equation}
We have
\begin{equation}\begin{aligned}
il u + im v + is w = 0 \\
n u = -\frac{1}{\rho} il \omega \\
n v = -\frac{1}{\rho} im \omega \\
n w = -\frac{1}{\rho} is \omega + \gamma \theta \\
n \theta + w \beta = \kappa (-s^2-l^2 -m^2) \theta \\
\end{aligned} \end{equation}
By elimination, we can derive an equation for $w$
\begin{equation}\begin{aligned}
il u + im v + \PD_z w = 0 \\
u = -\frac{1}{n \rho} il \omega \\
v = -\frac{1}{n \rho} im \omega \\
. \\
il (-\frac{1}{n \rho} il \omega) + im (-\frac{1}{n \rho} im \omega) + \PD_z w = 0 \\
\frac{l^2}{n \rho} \omega + \frac{m^2}{n \rho} \omega + \PD_z w = 0 \\
(l^2 + m^2)\omega + n \rho \PD_z w = 0 \\
. \\
\omega + \frac{n \rho}{l^2 + m^2} \PD_z w = 0 \\
\PD_z \omega + \frac{n \rho}{l^2 + m^2} \PD_z^2 w = 0 \\
%
. \\
w = -\frac{1}{n \rho} \PD_z \omega + \frac{1}{n}\gamma \theta \\
\PD_z \omega = \rho \gamma \theta - n \rho w \\
. \\
\rho \gamma \theta - n \rho w + \frac{n \rho}{l^2 + m^2} \PD_z^2 w = 0 \\
\gamma \theta - n w + \frac{n}{l^2 + m^2} \PD_z^2 w = 0 \\
\frac{n}{l^2 + m^2} \PD_z^2 w = n w - \gamma \theta \\
\end{aligned} \end{equation}
The energy equation terms may be collected:
\begin{equation}\begin{aligned}
n \theta + w \beta = \kappa (\PD_{zz}-l^2 -m^2) \theta \\
w \beta + \left\{n + \kappa (l^2 +m^2-\PD_{zz}) \right\} \theta = 0 \\
\end{aligned} \end{equation}

Our final form is
\begin{equation}\begin{aligned}
w \beta + \left\{n + \kappa (l^2 +m^2-\PD_{zz}) \right\} \theta = 0 \\
\frac{n}{l^2 + m^2} \PD_z^2 w = n w - \gamma \theta \\
\end{aligned} \end{equation}
Or
\begin{equation}\begin{aligned}
w \beta + \left\{n + \kappa (l^2 +m^2-\PD_{zz}) \right\} \theta = 0 \\
n (w (l^2 + m^2) - \PD_z^2 w) - \gamma (l^2 + m^2) \theta = 0 \\
\end{aligned} \end{equation}

Considering the BCs, Note that the $\theta$ BC is for the departure temperature, ($u=0,v=0,w=0,\theta=0$ on $z=0,\xi$), we may assume that these quantities are proportional to sine functions $\sin(sz), s=q\pi/\xi, q=$integer to get
\begin{equation}\begin{aligned}
w \beta + \left\{n + \kappa (l^2 +m^2+s^2) \right\} \theta = 0 \\
n (l^2 + m^2 + s^2) w - \gamma (l^2 + m^2) \theta = 0 \\
\end{aligned} \end{equation}
The equation determining $n$ found by combining these equations
\begin{equation}\begin{aligned}
w = \frac{\gamma (l^2 + m^2) \theta}{n (l^2 + m^2 + s^2)} \\
\frac{\gamma (l^2 + m^2) \theta}{n (l^2 + m^2 + s^2)} \beta + \left\{n + \kappa (l^2 +m^2+s^2) \right\} \theta = 0 \\
\gamma (l^2 + m^2) \beta + \left\{n + \kappa (l^2 +m^2+s^2) \right\} n (l^2 + m^2 + s^2) = 0 \\
\gamma (l^2 + m^2) \beta + n^2 (l^2 + m^2 + s^2) + n \kappa (l^2 +m^2+s^2)^2 = 0 \\
n^2 (l^2 + m^2 + s^2) + n \kappa (l^2 +m^2+s^2)^2 + \beta \gamma (l^2 + m^2) = 0 \\
n^2 + n \kappa (l^2 +m^2+s^2) + \beta \gamma \frac{l^2 + m^2}{l^2 + m^2 + s^2} = 0 \\
\end{aligned} \end{equation}

Finally:

\begin{equation}\begin{aligned}
n = -\kappa (l^2 +m^2+s^2) \pm \frac{1}{2} \sqrt{\kappa^2 (l^2 +m^2+s^2)^2-4 \beta \gamma \frac{l^2 + m^2}{l^2 + m^2 + s^2}}  \\
\end{aligned} \end{equation}
For $\kappa=0$ there is no conduction, so that each element of fluid retains its temperature and density.

If $\beta$ is positive, equilibrium is reached
\begin{equation}\begin{aligned}
n = \pm i \sqrt{\beta \gamma \frac{l^2 + m^2}{l^2 + m^2 + s^2}}  \\
\end{aligned} \end{equation}

If $\beta$ is negative, disturbances increase with time
\begin{equation}\begin{aligned}
n = \pm \sqrt{\beta \gamma \frac{l^2 + m^2}{l^2 + m^2 + s^2}}  \\
\end{aligned} \end{equation}


\end{document}