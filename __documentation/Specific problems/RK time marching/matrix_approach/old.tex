% \begin{equation}
%   \PD_t \X = f(\X),
% \end{equation}
% \begin{equation}
%   % \DEL \DOT \X = 0
%   g(\X) = 0
% \end{equation}


% \section{RK4 method wikipedia}
% The governing equations are
% \begin{equation}\begin{aligned}
%   \PD_t u = f(t,u) \\
% \end{aligned}\end{equation}
% The formal RK4 method can be outlined as
% \begin{equation}\begin{aligned}
%   u_{n+1} = u_n + \frac{\Delta t}{6} (k_1 + 2 k_2 + 2k_3 + k_4) \\
%   t_{n+1} = t_n + \Delta t \\
%   k_1 = f(t_n,u_n) \\
%   k_2 = f\left(t_n + \frac{\Delta t}{2} , u_n + \frac{\Delta t}{2} k_1 \right) \\
%   k_3 = f\left(t_n + \frac{\Delta t}{2} , u_n + \frac{\Delta t}{2} k_3 \right) \\
%   k_4 = f\left(t_n + \Delta t , u_n + \Delta t k_3 \right) \\
% \end{aligned}\end{equation}
% \section{Low storage RK3 method Nikitin}
% The method is outlined as follows
% \begin{equation}\begin{aligned}
%   \PD_t \U = F(t,\U) \\
%   \frac{\U'     -\U_n}{\tau} = \alpha_1 \F_n + \beta_1 \F' + \gamma_1 \F'', \quad \F_n = \F(t_n,\U_n)           , \quad \alpha_1 = \frac{2}{3}, \quad \beta_1 =     0      , \quad \gamma_1 = 0           \\
%   \frac{\U''    -\U_n}{\tau} = \alpha_2 \F_n + \beta_2 \F' + \gamma_2 \F'', \quad \F'  = \F(t_n+2\tau/3,\U')    , \quad \alpha_2 = \frac{1}{3}, \quad \beta_2 = \frac{1}{3}, \quad \gamma_2 = 0           \\
%   \frac{\U_{n+1}-\U_n}{\tau} = \alpha_3 \F_n + \beta_3 \F' + \gamma_3 \F'', \quad \F'' = \F(t_n+2\tau/3,\U_n'') , \quad \alpha_3 = \frac{1}{4}, \quad \beta_3 =     0      , \quad \gamma_3 = \frac{3}{4} \\
%   t_{n+1} = t_n + \tau \\
% \end{aligned}\end{equation}
% In matrix form, we may write this as
% \begin{equation}
% \MAT{c c c c}
% A & G \\
% D & 0 \\
% \EMAT
% \MAT{c} \U' \\ \U'' \\ \U^{n+1} \EMAT =
% \MAT{c c c c}
% \alpha_1 & \beta_1 & \gamma_1 \\
% \alpha_2 & \beta_2 & \gamma_2 \\
% \alpha_3 & \beta_3 & \gamma_3 \\
% \EMAT
% \MAT{c} \F_n \\ \F' \\ \F'' \EMAT
% \end{equation}



% \section{Desired equations to solve}
% The semi-discretized equations are
% \begin{multline}
% \frac{\U^{n+1}-\U^n}{\Delta t} - \theta_u \ReInv L\U^{n+1} = - G p^{n+1} + (1-\theta_u) \ReInv L\U^n + \R, \\
% \R = - D(\U \U^T) + \Al (C\B) \CROSS \B, \\
%  D \U^{n+1} = 0, \\
% \end{multline}
% \begin{multline}
% \frac{\B^{n+1}-\B^n}{\Delta t} + \theta_B \RemInv C (\left[ \SO^{-1} C\B^{n+1}) \right] = - G\correctB^{n+1} - (1-\theta_B) \RemInv C (\left[ \SO^{-1} C\B^n) \right] + \T, \\
% \T = C (\U \CROSS \B), \\
%  D \B^{n+1} = 0. \\
% \end{multline}


% \section{My matrix approach}
% From the Perot analysis, the equations we wish to solve are
% \begin{equation}
% \MAT{c c c c}
% A_u & G & 0   & 0 \\
% D   & 0 & 0   & 0 \\
% 0   & 0 & A_B & G \\
% 0   & 0 & D   & 0 \\
% \EMAT
% \MAT{c} \U^{n+1} \\ p^{n+1} \\ \B^{n+1} \\ \correctB^{n+1} \\ \EMAT =
% \MAT{c} \MAC R \\ 0  \\ \MAC T \\ 0  \\ \EMAT +
% \MAT{c} BC_1 \\ BC_2 \\BC_3 \\ BC_4 \\ \EMAT,
% \end{equation}
% \begin{equation}\LL\begin{aligned}
% & A_u  = \frac{1}{\Delta t} \left(I - \Delta t \theta_u \ReInv L \right), \\
% & A_B  = \frac{1}{\Delta t} \left(I + \Delta t \theta_B \RemInv C (\SO^{-1} C) \right), \\
% & \MAC R = \frac{1}{\Delta t} \left(I + \Delta t (1-\theta_u) \ReInv L \right) \U^n + \frac{3}{2}\R^n - \frac{1}{2}\R^{n-1}. \\
% & \MAC T = \frac{1}{\Delta t} [I - \Delta t(1-\theta_B) \RemInv C ( \SO^{-1} C) ]\B^n + \frac{3}{2} \T^n - \frac{1}{2} \T^{n-1}. \\
% \end{aligned} \end{equation}
