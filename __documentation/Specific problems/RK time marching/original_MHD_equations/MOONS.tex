\documentclass[11pt]{article}
\newcommand{\PSCHAIN}{..}
\edef\PSCHAIN{\PSCHAIN/LATEX_INCLUDES}

\newcommand{\rootdir}{\PSCHAIN}
\newcommand{\VAR}{Success}



\newcommand{\ReInv}{Re^{-1}}
\newcommand{\RemInv}{Re_m^{-1}}
\newcommand{\Al}{N Re_m^{-1}}
\newcommand{\Interaction}{N}

\begin{document}

\MOONSTITLE
% \maketitle

\section{Desired equations to solve}
\begin{equation}
  \PD_t \U - \ReInv \DEL^2 \U
  = - \DEL p + \left( \DEL \DOT (\U \U^T) \\ + \Al
  (\CURL \B) \CROSS \B \right)
\end{equation}
\begin{equation}
  \DEL \DOT \U = 0
\end{equation}
\begin{equation}
  \PD_t \B + \RemInv \DEL \CROSS ( \SO^{-1} \DEL \CROSS \B ) = -\DEL \phi + \left( \DEL \CROSS \U \CROSS \B \right)
\end{equation}
\begin{equation}
  \DEL \DOT \B = 0
\end{equation}

\section{Vector form}
Let
\begin{equation}
\X = \MAT{c} \U \\ \B \\ \EMAT, \qquad
\zeta = \MAT{c} p \\ \phi \\ \EMAT, \qquad
\theta = \MAT{c} \theta_u \\ \theta_B \\ \EMAT
\end{equation}
and
\begin{equation}
\MAC L(\theta) =
\MAT{c c}
- \ReInv \DEL^2 & 0 \\
0 & \RemInv \DEL \CROSS ( \SO^{-1} \DEL \CROSS ) \\
\EMAT
\theta
\end{equation}
\begin{equation}
\MAC F(\theta) =
\MAT{c}
\ReInv \DEL^2 \U \\
\RemInv \DEL \CROSS ( \SO^{-1} \DEL \CROSS ) \\
\EMAT
(1-\theta)
+
\MAT{c}
\DEL \DOT (\U \U^T) + \Al (\CURL \B) \CROSS \B \\
\DEL \CROSS \U \CROSS \B \\
\EMAT
\end{equation}
Our equations then become:
\begin{equation}
  \PD_t \X + \underbrace{\MAC L(\theta)}_{\text{linear operator}} \X = \underbrace{- \DEL \zeta}_{\text{enforces constraint}} + \underbrace{\MAC F(\theta)}_{\text{explicitly treated}}
\end{equation}
\begin{equation}
  \DEL \DOT \X = 0
\end{equation}

\begin{equation}\begin{aligned}
  \PD_t \X + \MAC L(\theta) \X = - \DEL \zeta + \MAC F(\theta) \\
  \DEL \DOT \X = 0
\end{aligned}\end{equation}
Alternatively, we may write
\begin{equation}\begin{aligned}
  \PD_t \X = - \DEL \zeta + \MAC Q, \qquad \MAC Q = \MAC F(\theta) - \MAC L(\theta) \X \\
  \DEL \DOT \X = 0
\end{aligned}\end{equation}


\end{document}