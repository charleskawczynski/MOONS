\documentclass[11pt]{article}
\newcommand{\bibdir}{\rootdir/includes/bib}

\usepackage{graphicx}
\usepackage{adjustbox}
% \usepackage{epstopdf} % Does not work with pdflatex
\usepackage{amsmath}
\usepackage{amssymb}
\usepackage{hyperref}
\usepackage{xspace}
\usepackage{mathtools}
\usepackage{tikz}
\usepackage{epsfig}
\usepackage{float}
\usepackage{subfigure}
\usepackage{setspace}
\usepackage{tabularx}
\usepackage{multirow}
\usepackage{ragged2e}
\usepackage{booktabs}
\usepackage{caption}
% \usepackage{subcaption}
\usepackage{xcolor}
\usepackage{longtable}
% \usepackage{natbib} % Sometimes causes errors
\usepackage{etoolbox}
\usepackage{geometry}
\usepackage{esint}
\usepackage{ifxetex}
\usepackage{ifluatex}
% \usepackage{siunitx}
\usepackage{numprint}
\usepackage{empheq}
\usepackage{slashbox}


 % For local use
\newcommand{\figWidth}{0.45\textwidth}
%\newcommand{\figW}{6.5in}
%\newcommand{\figH}{5.5in}
\newcommand{\figW}{5.in}
\newcommand{\figH}{4.in}
%\newcommand{\figW}{6.825in}
%\newcommand{\figH}{5.775in}
\newcommand{\ffigW}{3.1in}
\newcommand{\ffigH}{3.1in}

\newcommand{\captionWidth}{0.45}
\newcommand{\subFigPad}{0.52\textwidth} % good for subcaption
\newcommand{\TWOFIGCOLUMN}{0.52\textwidth} % good for subcaption
\newcommand{\TWOFIGCOLUMNCAPTION}{0.52\textwidth} % good for subcaption
% \newcommand{\subFigPad}{0.45} % good for subfig
\newcommand{\miniPageSize}{.1\textwidth}

\newcommand{\subFigSize}{16em} % good for subcaption
\newcommand{\figSize}{0.45\textwidth}


% \renewcommand{\contcaption}{%
%   \expandafter\addtocounter\expandafter{\@captype alt}{\m@ne}% Step alias cntr back
%   \expandafter\refstepcounter\expandafter{\@captype alt}% Make reference
%   \@contcaption\@captype}

% \definecolor{darkGreen}{rgb}{0.0, 0.5, 0.0}
\newcommand{\COR}{\color{red}}
\newcommand{\COG}{\color{darkGreen}}
\newcommand{\COM}{\color{magenta}}
\newcommand{\COK}{\color{black}}
\newcommand{\COB}{\color{blue}}

\newcommand{\toth}{\tiny{Toth, Gabor. "The $\DIV \B = 0$ constraint in shock-capturing magnetohydrodynamics codes." J. Comput. Phys. 161.2 (2000): 605-652.}}
\newcommand{\pattison}{\tiny{M. J. Pattison, K. N. Premnath, N. B. Morley, M. Abdou, Fusion Eng. Des. 83 (2008) 557-572.}}
\newcommand{\gujStella}{\tiny{G. Guj, F. Stella, J. Comput. Phys. 298 (1993) 286-298.}}
\newcommand{\bandaru}{\tiny{V. Bandaru, J. Pracht, T. Boeck, J. Schumacher, Theor. Comp. Fluid Dyn. (2015).}}
\newcommand{\guermond}{\tiny{Guermond, J. L., J. Léorat, and C. Nore. "A new Finite Element Method for magneto-dynamical problems: two-dimensional results." European Journal of Mechanics-B/Fluids 22.6 (2003): 555-579.}}
\newcommand{\kawczynski}{\tiny{C. Kawczynski, S. Smolentsev, M. Abdou. "A Lid-Driven Cavity MHD Flow Numerical Study at Moderate Magnetic Reynolds Number using Proper Magnetic Boundary Conditions". In Progress.}}
\newcommand{\roache}{\tiny{P. J. Roache, Annu. Rev. Fluid Mech. 29 (1997) 123-160.}}
\newcommand{\hunt}{\tiny{J. C. R. Hunt, J. Fluid Mech. 21 (1965) 577-590.}}
\newcommand{\shercliff}{\tiny{J. A. Shercliff, Proc. Camb. Phil. Soc. 49 (1953) 136.}}
\newcommand{\turner}{\tiny{Turner, Larry R. "Electromagnetic computations for fusion devices." IEEE Transactions on Magnetics 26.2 (1990): 847-852.}}
\newcommand{\kotulskiA}{\tiny{Kotulski, Joseph D., et al. "Electromagnetic analysis of forces and torques on the baseline and enhanced ITER shield modules due to plasma disruption." IEEE Transactions on Plasma Science 38.4 (2010): 1047.}}
\newcommand{\kotulskiB}{\tiny{Kotulski, Joseph D., Rebecca Coats, and Michael Ulrickson. "The analysis of the electromagnetic loads on selected ITER blanket shield modules due to induced eddy and halo currents." Fusion Engineering (SOFE), 2011 IEEE/NPSS 24th Symposium on. IEEE, 2011.}}
\newcommand{\iskakov}{\tiny{Iskakov, A. B., S. Descombes, and E. Dormy. "An integro-differential formulation for magnetic induction in bounded domains: boundary element-finite volume method." J. Comput. Phys. 197.2 (2004): 540-554.}}
\newcommand{\mikeulrickson}{\tiny{We received numerical data from Dr. Mike Ulrickson, computed from the plasma code DINA.}}
\newcommand{\Leriche}{Leriche, E. \& Gavrilakis, S. Direct numerical simulation of the flow in a lid-driven cubical cavity. Phys. Fluids 12, (2000).}
\newcommand{\Brackbill}{\tiny{Brackbill, Jeremiah U., and Daniel C. Barnes. "The effect of nonzero $\DIV \B = 0$ on the numerical solution of the magnetohydrodynamic equations." Journal of Computational Physics 35.3 (1980): 426-430.}}
\newcommand{\Leuer}{\tiny{Leuer, J. A., and J. C. Wesley. "ITER plasma start-up modeling." Fusion Engineering, 1993., 15th IEEE/NPSS Symposium on. Vol. 2. IEEE, 1993.}}

% \newcommand{\NJ}{New Jersey}
% \newcommand{\UCC}{Union County College}
% \newcommand{\GSR}{Graduate Student Researcher}
% \newcommand{\ME}{Mechanical Engineering}
% \newcommand{\NJSGC}{New Jersey Space Grant Consortium}
% % \newcommand{\NASA}{National Aeronautics and Space Administation}
% \newcommand{\NASA}{NASA}
% \newcommand{\SAS}{School of Arts and Sciences}
% % \newcommand{\UCLA}{University of California Los Angeles}
% \newcommand{\UCLA}{UCLA}
% \newcommand{\IDRE}{Institute for Digital Research and Education}
% \newcommand{\MAE}{Department of Mechanical and Aerospace Engineering}
% \newcommand{\UCLAMAE}{Henri Samueli School Of Engineering And Applied Sciences}
% % \newcommand{\MHD}{Magnetohydrodynamic}
% \newcommand{\MHDs}{Magnetohydrodynamics}
% % \newcommand{\RUTGERS}{Rutgers University, The State University of New Jersey}
% \newcommand{\RUTGERS}{Rutgers University}
% \newcommand{\RU}{Rutgers University}
% \newcommand{\US}{United States}

\newcommand{\height}{0.4}
\newcommand{\radius}{0.1}
\newcommand{\offSet}{12}
\newcommand{\Deltah}{1.6666}
\lstset{language=[90]Fortran,
  basicstyle=\ttfamily,
  keywordstyle=\color{red},
  commentstyle=\color{green},
  morecomment=[l]{!\ }% Comment only with space after !
}
\newif\ifxetexorluatex
\ifxetex
  \renewcommand{\C}{\mathbf{C}}
  \renewcommand{\U}{\mathbf{u}}
  \renewcommand{\G}{\mathbf{g}}
\else
  \newcommand{\C}{\mathbf{C}}
  \newcommand{\U}{\mathbf{u}}
  \newcommand{\G}{\mathbf{g}}
\fi
\renewcommand{\H}{\mathbf{H}}

\newcommand{\Rem}{Re_m}
\newcommand{\Rm}{$Re_m$}
\newcommand{\B}{\mathbf{B}}
\newcommand{\cleanB}{\xi}
\newcommand{\R}{\mathbf{r}}
\newcommand{\T}{\mathbf{T}}
\newcommand{\BBx}{\bar{B_x}}
\newcommand{\BBy}{\bar{B_y}}
\newcommand{\BBz}{\bar{B_z}}
\newcommand{\PB}{\bar{p}}
\newcommand{\UBx}{\bar{u}}
\newcommand{\VBy}{\bar{v}}
\newcommand{\WBz}{\bar{w}}
\newcommand{\E}{\mathbf{E}}
\newcommand{\J}{\mathbf{j}}
\newcommand{\A}{\mathbf{A}}
\newcommand{\N}{\mathbf{n}}
\newcommand{\F}{\mathbf{f}}
\newcommand{\X}{\mathbf{x}}
\newcommand{\Y}{\mathbf{y}}
\newcommand{\zero}{\mathbf{0}}
\newcommand{\ttheta}{\tilde{\theta}}
\newcommand{\M}{{\mu_m}}
\newcommand{\SII}{\SI^{-1}}
\newcommand{\JS}{\frac{\J}{\sigma}}
\newcommand{\MO}{\overline{\M}}
\newcommand{\SO}{\overline{\sigma}}
\newcommand{\JSS}{\frac{\J^2}{\sigma}}

\newcommand{\KE}{\left(\frac{1}{2} \U \DOT \U\right)}
\newcommand{\ME}{\left(\frac{1}{2} \B \DOT \B\right)}
\newcommand{\EPS}{\varepsilon}

\newcommand{\SIG}{\sigma}
\newcommand{\PH}{physical}

\newcommand{\SOVAC}{\SO_{v}}


\newcommand{\NPB}{\num[round-mode=places,round-precision=2]}
\newcommand{\NPU}{\num[round-mode=places,round-precision=5]}
\newcommand{\NPJ}{\num[round-mode=places,round-precision=5]}
\newcommand{\NPS}{\num[scientific-notation=true,round-mode=places,round-precision=2]}
\newcommand{\NPP}{\num[round-mode=places,round-precision=2]}

\newcommand{\MAT}{\left[ \begin{array}}
\newcommand{\EMAT}{\end{array} \right]}
\newcommand{\ARRAY}{\begin{array}}
\newcommand{\EARRAY}{\end{array}}

% \newcommand{\DOT}{\textbullet} % Causes warnings
% \newcommand{\DOT}{\text{\textbullet}}
\newcommand{\DOT}{\ensuremath{\bullet}}
% \newcommand{\textbullet}{}
\newcommand{\CROSS}{\times}
\newcommand{\BUN}{\otimes}
\newcommand{\DEL}{\nabla}
\newcommand{\CURL}{\DEL \CROSS}
\newcommand{\DIV}{\DEL \DOT}
\newcommand{\PD}{\partial}
\newcommand{\MAC}{\mathcal}
\newcommand{\GRAD}{\nabla}
\newcommand{\RA}{\rightarrow}

\newcommand{\EQSIZE}{\footnotesize}

\newcommand{\MC}{\multicolumn}
\newcommand{\MR}{\multirow}
\newcommand{\LL}{\raggedright}
\newcommand{\RR}{\raggedleft}
\newcommand{\CE}{\centering}
\newcommand{\TW}{\textwidth}
\renewcommand{\TH}{\textheight}
\newcommand{\TWP}{em\textwidth}
\newcommand{\qquadmany}{\qquad\qquad\qquad\qquad\qquad\qquad}
\newcommand{\hfillMath}{\hskip \textwidth minus \textwidth}

\newcommand{\tenDots}{..........}
\newcommand{\twentyDots}{\tenDots\tenDots}
\newcommand{\fiftyDots}{\twentyDots\twentyDots\tenDots}
\newcommand{\sixtyDots}{\twentyDots\twentyDots\twentyDots}
\newcommand{\seventyDots}{\fiftyDots\twentyDots}
\newcommand{\hundredDots}{\fiftyDots\fiftyDots}

% \restylefloat{table} % Results in errors in prospectus presentation


\newtoggle{includeImages}
\toggletrue{includeImages}
% \togglefalse{includeImages}

\newcommand{\IG}{\includegraphics}
\newcommand{\MP}{\minipage}
\newcommand{\EMP}{\endminipage}
\newcommand{\FIG}{\figure}
\newcommand{\EFIG}{\endfigure}
\newcommand{\SFIG}{\subfigure}
\newcommand{\ESFIG}{\endsubfigure}

\newcommand{\BLANKFIGURE}{\begin{figure*}[!htp]\centering\IG[width=.3\TW]{\figdir/blank.png}\end{figure*}}
\newcommand{\BLANKGRAPHIC}{\IG[width=.3\TW]{\figdir/blank.png}}
% \renewcommand{\BLANKGRAPHIC}{}
% \renewcommand{\BLANKFIGURE}{}

\newcommand{\FIGTYPE}{_BW}
\renewcommand{\FIGTYPE}{} % uncomment if color
\newcommand{\FIGEXT}{png}
% \renewcommand{\FIGEXT}{eps} % uncomment for .eps (takes forever to build)

\newcommand{\MOONSTITLE}{\title{MHD Object-Oriented Numerical Solver (MOONS)}
\author{C. Kawczynski \\
Department of Mechanical and Aerospace Engineering \\
University of California Los Angeles, USA\\
}}


% % Numbering from: https://www.sharelatex.com/learn-scripts/images/f/fc/Layout-dimensions.png
\setlength{\hoffset}{-.5in}       % 1
\setlength{\voffset}{0in}        % 2
\setlength{\oddsidemargin}{0in}  % 3
\setlength{\topmargin}{-.5in}     % 4
\setlength{\headheight}{0in}     % 5
\setlength{\headsep}{0in}        % 6
\setlength{\textheight}{700pt}   % 7
\setlength{\textwidth}{550pt}    % 8
\setlength{\marginparsep}{0in}   % 9
% \setlength{\marginparwidth}{0in} % 10
\setlength{\footskip}{0in}       % 11 a (see fig)
% \setlength{\marginparpush}{0in}  % 11 b (see fig)
% \setlength{\paperwidth}{0in}     % 11 e (see fig)
% \setlength{\paperheight}{390pt}  % 11 f (see fig)
 % For local use
% % Numbering from: https://www.sharelatex.com/learn-scripts/images/f/fc/Layout-dimensions.png
\setlength{\hoffset}{0in}       % 1
\setlength{\voffset}{0in}        % 2
\setlength{\oddsidemargin}{0in}  % 3
\setlength{\topmargin}{0in}     % 4
\setlength{\headheight}{0in}     % 5
\setlength{\headsep}{0in}        % 6
\setlength{\textheight}{350pt}   % 7
\setlength{\textwidth}{5in}    % 8
\setlength{\marginparsep}{0in}   % 9
\setlength{\marginparwidth}{0in} % 10
\setlength{\footskip}{0in}       % 11 a (see fig)
\setlength{\marginparpush}{0in}  % 11 b (see fig)
\setlength{\hoffset}{0in}        % 11 c (see fig)
\setlength{\voffset}{0in}        % 11 d (see fig)
\setlength{\paperwidth}{0in}     % 11 e (see fig)
\setlength{\paperheight}{390pt}  % 11 f (see fig)
 % For local use
% \input{\rootdir/includes/diagrams/coordinates.tex} % For local use
% \input{\rootdir/includes/margins/margins.tex} % For local use



\newcommand{\ReInv}{Re^{-1}}
\newcommand{\RemInv}{Re_m^{-1}}
\newcommand{\Al}{N Re_m^{-1}}
\newcommand{\Interaction}{N}

\begin{document}
\doublespacing
\MOONSTITLE
% \maketitle

\section{Formalizing time marching - block LU decomposition}
Here, we attempt to formalize the time marching schemes used for the MHD equations using the same framework in \cite{Perot1993}. In this analysis, the temporal order of accuracy is analyzed by comparing the fractional step method with the time discretization in the light of block LU decomposition. The dimensionless MHD equations are
\begin{equation}\begin{aligned}
\PD_t \U + \DEL \DOT (\U \U) = - \DEL p + \ReInv \DEL^2 \U + \Interaction \J \CROSS \B, \qquad \J = \RemInv \CURL \B, \\
 \DIV \U = 0, \\
\PD_t \B + \RemInv \CURL \left[ \SO^{-1} \CURL \B \right] = \CURL (\U \CROSS \B), \\
 \DIV \B = 0.
\end{aligned} \end{equation}
We may cast this set of equations as
\begin{equation}\begin{aligned}
\PD_t \U + D (\U \U) = - G p + \ReInv L \U + \Al C(\B) \CROSS \B, \\
 D \U = 0, \\
\PD_t \B + \RemInv C (\left[ \SO^{-1} C(\B)) \right] = C (\U \CROSS \B), \\
 D \B = 0.
\end{aligned} \end{equation}
Here, $L,D,G,C$ are the discrete Laplacian, divergence, gradient and curl operators.

\newpage
\section{Review of Perot analysis}
In \cite{Perot1993}, the general form of the analyzed equations are put into the form
\[
\MAT{c c}
A & G \\
D & 0 \\
\EMAT
\MAT{c}
\U^{n+1} \\
 p^{n+1} \\
\EMAT
=
\MAT{c}
\R \\
0  \\
\EMAT +
\MAT{c}
BC_1 \\
BC_2 \\
\EMAT \label{eq:blockLU}
\]
A $\theta$-implicit diffusion, Adams-Bashforth advection and implicit pressure results in
\begin{equation}\begin{aligned} \label{eq:TD_desired}
\frac{\U^{n+1}-\U^n}{\Delta t} - \theta \ReInv L(\U^{n+1}) = - G p^{n+1} + (1-\theta)\ReInv L(\U^n) + \frac{3}{2}\F^n - \frac{1}{2}\F^{n-1} + BC_1 , \\
\F^n = - D(\U^n \U^n), \\
 D \U^{n+1} = 0 + BC_2, \\
\end{aligned} \end{equation}
From this, we can recover $A$ and $\R$ from the block LU decomposition:
\begin{equation}\begin{aligned}
A  = \frac{1}{\Delta t} \left(I - \Delta t \theta \ReInv L \right), \\
\R = \frac{1}{\Delta t} \left(I + \Delta t (1-\theta) \ReInv L \right) \U^n + \frac{3}{2}\F^n - \frac{1}{2}\F^{n-1}. \\
\end{aligned} \end{equation}
Equation \ref{eq:TD_desired} can be approximated using a fractional step method, resulting in:
\begin{equation}\begin{aligned} \label{eq:TD_FSM}
\frac{\hat{\U}-\U^n}{\Delta t} - \theta \ReInv L(\hat{\U}) = (1-\theta)\ReInv L(\U^n) + \frac{3}{2}\F^n - \frac{1}{2}\F^{n-1}, \\
L(p^{n+1}) = \frac{1}{\Delta t} D(\hat{\U}), \\
\U^{n+1} = \hat{\U} - \Delta t G (p^{n+1}). \\
\end{aligned} \end{equation}
The error of the fractional step method is $\Delta t \theta \ReInv L(G(p^{n+1}))$, which can be found by adding \ref{eq:TD_FSM} and comparing with \ref{eq:TD_desired}.
If we approximate \ref{eq:blockLU} with
\[
\MAT{c c}
A & (AB)G \\
D & 0 \\
\EMAT
\MAT{c}
\U^{n+1} \\
 p^{n+1} \\
\EMAT
=
\MAT{c}
\R \\
0  \\
\EMAT +
\MAT{c}
BC_1 \\
BC_2 \\
\EMAT
\]
and factorize, we get
\[
\MAT{c c}
A & 0 \\
D & -DBG \\
\EMAT
\underbrace{
\MAT{c c}
I & BG \\
0 & I  \\
\EMAT
\MAT{c}
\U^{n+1} \\
 p^{n+1} \\
\EMAT}_{
\MAT{c}
\hat{\U} \\
 p^{n+1} \\
\EMAT}
=
\MAT{c}
\R \\
0  \\
\EMAT +
\MAT{c}
BC_1 \\
BC_2 \\
\EMAT
\]
Ideally $B = A^{-1}$. Here, we solve the outer problem first (for $\hat{\U}$) then the inner problem (for $\U^{n+1}$). For example, if we use

\begin{equation}\begin{aligned}
B = \Delta t I, \quad \rightarrow \quad \text{1st order error term} \\
B = \Delta t \left[I + \Delta t \frac{\theta}{Re} L \right], \quad \rightarrow \quad \text{2nd order error term} \\
B = \Delta t \left[I + \Delta t \frac{\theta}{Re} L + \left(\Delta t \frac{\theta}{Re} L \right)^2 \right], \quad \rightarrow \quad \text{3rd order error term} \\
B = A^{-1}, \quad \rightarrow \quad \text{Uzawa method, requires nested iterations (typically avoided)}
\end{aligned} \end{equation}
Then

\newpage
\section{Review of Kim and Moin - Fractional Step Method}
Equations 3, the divergence of equation 4, and equation 4 are:
\begin{equation}\begin{aligned} \label{eq:TD_FSM_KM}
\frac{\hat{\U}-\U^n}{\Delta t} = \frac{1}{2} \ReInv L(\U^n+\hat{\U}) + \frac{3}{2}\F^n - \frac{1}{2}\F^{n-1}, \\
\frac{\U^{n+1}-\hat{\U}}{\Delta t} =  - G (\phi^{n+1}), \\
D(\U^{n+1}) = 0.
\end{aligned} \end{equation}
Note: $p = \phi + \frac{\Delta t}{2} \ReInv L \phi$. To analyze the time accuracy, we add (and combine) the first two equations to get
\begin{equation}\begin{aligned}
& \frac{\U^{n+1}-\hat{\U}}{\Delta t} + \frac{\hat{\U}-\U^n}{\Delta t} = - G (\phi^{n+1}) + \frac{1}{2} \ReInv L(\U^n+\hat{\U}) + \frac{3}{2}\F^n - \frac{1}{2}\F^{n-1}, \\
& \frac{\U^{n+1}-\U^n}{\Delta t} = - G (\phi^{n+1}) + \frac{1}{2} \ReInv L\left(\U^{n+1}+ \Delta t G\phi^{n+1} \right) + \frac{1}{2} \ReInv L\U^n + \frac{3}{2}\F^n - \frac{1}{2}\F^{n-1}, \\
& \frac{\U^{n+1}-\U^n}{\Delta t} = - G (\phi^{n+1}) + \frac{\Delta t}{2} \ReInv L G\phi^{n+1} + \frac{1}{2} \ReInv L(\U^n+\U^{n+1}) + \frac{3}{2}\F^n - \frac{1}{2}\F^{n-1}, \\
& \frac{\U^{n+1}-\U^n}{\Delta t} = - G \left(p^{n+1} - \frac{\Delta t}{2} \ReInv L \phi^{n+1}\right) + \frac{\Delta t}{2} \ReInv L G\phi^{n+1} + \frac{1}{2} \ReInv L(\U^n+\U^{n+1}) + \frac{3}{2}\F^n - \frac{1}{2}\F^{n-1}, \\
& \frac{\U^{n+1}-\U^n}{\Delta t} = - G (p^{n+1}) + \frac{\Delta t}{2} \ReInv G L \phi^{n+1} + \frac{\Delta t}{2} \ReInv L G\phi^{n+1} + \frac{1}{2} \ReInv L(\U^n+\U^{n+1}) + \frac{3}{2}\F^n - \frac{1}{2}\F^{n-1}, \\
\end{aligned} \end{equation}
% \begin{equation}\begin{aligned}
% & \frac{\U^{n+1}-\U^n}{\Delta t} = - G (\phi^{n+1}) + \frac{1}{2} \ReInv L(\U^n + \hat{\U}) + \frac{3}{2}\F^n - \frac{1}{2}\F^{n-1}, \\
% & \frac{\U^{n+1}-\U^n}{\Delta t} = - G (\phi^{n+1}) + \frac{1}{2} \ReInv L(\U^n + (\U^{n+1} + \Delta t G \phi^{n+1})) + \frac{3}{2}\F^n - \frac{1}{2}\F^{n-1}, \\
% & \frac{\U^{n+1}-\U^n}{\Delta t} = - G (\phi^{n+1}) + \frac{1}{2} \ReInv L(\U^n + \U^{n+1}) + \frac{\Delta t}{2} \ReInv LG \phi^{n+1} + \frac{3}{2}\F^n - \frac{1}{2}\F^{n-1}, \\
% & \frac{\U^{n+1}-\U^n}{\Delta t} = - G \left(\phi^{n+1} - \frac{\Delta t}{2} \ReInv L \phi^{n+1} \right) + \frac{1}{2} \ReInv L(\U^n + \U^{n+1}) + \frac{3}{2}\F^n - \frac{1}{2}\F^{n-1}, \\
% & \frac{\U^{n+1}-\U^n}{\Delta t} = - G \left(p^{n+1} - \frac{\Delta t}{2} \ReInv L \phi^{n+1}  - \frac{\Delta t}{2} \ReInv L \phi^{n+1} \right) + \frac{1}{2} \ReInv L(\U^n + \U^{n+1}) + \frac{3}{2}\F^n - \frac{1}{2}\F^{n-1}, \\
% \end{aligned} \end{equation}
Since the relation between $p$ and $\phi$ includes a Laplacian term, we must
\begin{equation}\begin{aligned}
L \phi^{n+1} = \frac{1}{\Delta t} D\hat{\U}, \\
\end{aligned} \end{equation}

\newpage
\section{Perot analysis applied to MHD equations}
\begin{equation}\begin{aligned}
\frac{\U^{n+1}-\U^n}{\Delta t} - \theta \ReInv L(\U^{n+1}) = - G p^{n+1} + \F^n , \\
\F^n = (1-\theta) \ReInv L(\U^n) - D(\U^n \U^n) + \Al C(\B^n) \CROSS \B^n, \\
 D \U^{n+1} = 0, \\
\frac{\B^{n+1}-\B^n}{\Delta t} + \theta \RemInv C (\left[ \SO^{-1} C(\B^{n+1})) \right] = \T^n, \\
\T^n = C (\U^n \CROSS \B^n) - (1-\theta) \RemInv C (\left[ \SO^{-1} C(\B^n)) \right], \\
 D \B^{n+1} = 0.
\end{aligned} \end{equation}

In matrix form, we have
\[
\MAT{c c c c}
A_u & G & 0   & 0 \\
D   & 0 & 0   & 0 \\
0   & 0 & A_B & G \\
0   & 0 & D   & 0 \\
\EMAT
\MAT{c}
\U^{n+1} \\
 p^{n+1} \\
\B^{n+1} \\
\cleanB^{n+1} \\
\EMAT
=
\MAT{c}
\R \\
0  \\
\T \\
0  \\
\EMAT +
\MAT{c}
BC_1 \\
BC_2 \\
BC_3 \\
BC_4 \\
\EMAT
\]


\input{\rootdir/includes/include_bib.tex}




\end{document}