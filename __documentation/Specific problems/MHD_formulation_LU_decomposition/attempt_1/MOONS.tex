\documentclass[landscape]{article}
\newcommand{\VAR}{Success}

\usepackage[a4paper,margin=1in,landscape]{geometry} % For local use

\newcommand{\ReInv}{Re^{-1}}
\newcommand{\RemInv}{Re_m^{-1}}
\newcommand{\Al}{N Re_m^{-1}}
\newcommand{\Interaction}{N}

\begin{document}
\doublespacing
\MOONSTITLE
% \maketitle

\section{Formalizing time marching - block LU decomposition}
Here, we attempt to formalize the time marching schemes used for the MHD equations using the same framework in \cite{Perot1993}. In this analysis, the temporal order of accuracy is analyzed by comparing the fractional step method with the time discretization in the light of block LU decomposition. The dimensionless MHD equations are
\begin{equation}\begin{aligned}
\PD_t \U + \DEL \DOT (\U \U) = - \DEL p + \ReInv \DEL^2 \U + \Interaction \J \CROSS \B, \qquad \J = \RemInv \CURL \B, \\
 \DIV \U = 0, \\
\PD_t \B + \RemInv \CURL \left[ \SO^{-1} \CURL \B \right] = \CURL (\U \CROSS \B), \\
 \DIV \B = 0.
\end{aligned} \end{equation}
We may cast this set of equations as
\begin{equation}\begin{aligned}
\PD_t \U + D (\U \U) = - G p + \ReInv L \U + \Al C(\B) \CROSS \B, \\
 D \U = 0, \\
\PD_t \B + \RemInv C (\left[ \SO^{-1} C(\B)) \right] = C (\U \CROSS \B), \\
 D \B = 0.
\end{aligned} \end{equation}
Here, $L,D,G,C$ are the discrete Laplacian, divergence, gradient and curl operators.

\newpage
\section{Review of Perot analysis}
In \cite{Perot1993}, the general form of the analyzed equations are put into the form
\[
\MAT{c c}
A & G \\
D & 0 \\
\EMAT
\MAT{c}
\U^{n+1} \\
 p^{n+1} \\
\EMAT
=
\MAT{c}
\R \\
0  \\
\EMAT +
\MAT{c}
BC_1 \\
BC_2 \\
\EMAT \label{eq:blockLU}
\]
A $\theta$-implicit diffusion, Adams-Bashforth advection and implicit pressure results in
\begin{equation}\begin{aligned} \label{eq:TD_desired}
\frac{\U^{n+1}-\U^n}{\Delta t} - \theta \ReInv L(\U^{n+1}) = - G p^{n+1} + (1-\theta)\ReInv L(\U^n) + \frac{3}{2}\F^n - \frac{1}{2}\F^{n-1} + BC_1 , \\
\F^n = - D(\U^n \U^n), \\
 D \U^{n+1} = 0 + BC_2, \\
\end{aligned} \end{equation}
From this, we can recover $A$ and $\R$ from the block LU decomposition:
\begin{equation}\begin{aligned}
A  = \frac{1}{\Delta t} \left(I - \Delta t \theta \ReInv L \right), \\
\R = \frac{1}{\Delta t} \left(I + \Delta t (1-\theta) \ReInv L \right) \U^n + \frac{3}{2}\F^n - \frac{1}{2}\F^{n-1}. \\
\end{aligned} \end{equation}
Equation \ref{eq:TD_desired} can be approximated using a fractional step method, resulting in:
\begin{equation}\begin{aligned} \label{eq:TD_FSM}
\frac{\hat{\U}-\U^n}{\Delta t} - \theta \ReInv L(\hat{\U}) = (1-\theta)\ReInv L(\U^n) + \frac{3}{2}\F^n - \frac{1}{2}\F^{n-1}, \\
L(p^{n+1}) = \frac{1}{\Delta t} D(\hat{\U}), \\
\U^{n+1} = \hat{\U} - \Delta t G (p^{n+1}). \\
\end{aligned} \end{equation}
The error of the fractional step method is $\Delta t \theta \ReInv L(G(p^{n+1}))$, which can be found by adding \ref{eq:TD_FSM} and comparing with \ref{eq:TD_desired}.
If we approximate \ref{eq:blockLU} with
\[
\MAT{c c}
A & (AB)G \\
D & 0 \\
\EMAT
\MAT{c}
\U^{n+1} \\
 p^{n+1} \\
\EMAT
=
\MAT{c}
\R \\
0  \\
\EMAT +
\MAT{c}
BC_1 \\
BC_2 \\
\EMAT
\]
and factorize, we get
\[
\MAT{c c}
A & 0 \\
D & -DBG \\
\EMAT
\underbrace{
\MAT{c c}
I & BG \\
0 & I  \\
\EMAT
\MAT{c}
\U^{n+1} \\
 p^{n+1} \\
\EMAT}_{
\MAT{c}
\hat{\U} \\
 p^{n+1} \\
\EMAT}
=
\MAT{c}
\R \\
0  \\
\EMAT +
\MAT{c}
BC_1 \\
BC_2 \\
\EMAT
\]
Further simplified, this results in a series of operations
\begin{equation}\LL\begin{aligned}
A\hat{\U} = \R + BC_1 \\
\Delta t DG p^{n+1} = D\hat{\U} - BC_2 \\
\U^{n+1} = \hat{\U} - \Delta t G p^{n+1} \\
\end{aligned} \end{equation}

Ideally $B = A^{-1}$. Here, we solve the outer problem first (for $\hat{\U}$) then the inner problem (for $\U^{n+1}$). For example, if we use

\begin{equation}\begin{aligned}
B = \Delta t I, \quad \rightarrow \quad \text{1st order error term} \\
B = \Delta t \left[I + \Delta t \frac{\theta}{Re} L \right], \quad \rightarrow \quad \text{2nd order error term} \\
B = \Delta t \left[I + \Delta t \frac{\theta}{Re} L + \left(\Delta t \frac{\theta}{Re} L \right)^2 \right], \quad \rightarrow \quad \text{3rd order error term} \\
B = A^{-1}, \quad \rightarrow \quad \text{Uzawa method, requires nested iterations (typically avoided)}
\end{aligned} \end{equation}
Then

\newpage
\section{Review of Kim and Moin - Fractional Step Method}
\underline{NOTE}: It seems that there is a typo in Kim and Moin's paper. The defined $p$-$\phi$ relationship is stated as $p = \phi + \frac{\Delta t}{2} \ReInv L \phi$, when cross referencing this with Perot's paper, and by a later paper by Moin (and confirming ourselves), it seems that the correct expression is $p = \phi - \frac{\Delta t}{2} \ReInv L \phi$.

Equations 3, the divergence of equation 4, and equation 4 are:
\begin{equation}\begin{aligned} \label{eq:TD_FSM_KM}
\frac{\hat{\U}-\U^n}{\Delta t} = \frac{1}{2} \ReInv L(\U^n+\hat{\U}) + \frac{3}{2}\F^n - \frac{1}{2}\F^{n-1}, \\
\frac{\U^{n+1}-\hat{\U}}{\Delta t} =  - G (\phi^{n+1}), \\
D(\U^{n+1}) = 0.
\end{aligned} \end{equation}
To analyze the time accuracy, we add (and combine) the first two equations to get
\begin{equation}\begin{aligned}
& \frac{\U^{n+1}-\hat{\U}}{\Delta t} + \frac{\hat{\U}-\U^n}{\Delta t} = - G (\phi^{n+1}) + \frac{1}{2} \ReInv L(\U^n+\hat{\U}) + \frac{3}{2}\F^n - \frac{1}{2}\F^{n-1}, \\
& \frac{\U^{n+1}-\U^n}{\Delta t} = - G (\phi^{n+1}) + \frac{1}{2} \ReInv L\left(\U^{n+1}+ \Delta t G\phi^{n+1} \right) + \frac{1}{2} \ReInv L\U^n + \frac{3}{2}\F^n - \frac{1}{2}\F^{n-1}, \\
& \frac{\U^{n+1}-\U^n}{\Delta t} = - G (\phi^{n+1}) + \frac{\Delta t}{2} \ReInv L G\phi^{n+1} + \frac{1}{2} \ReInv L(\U^n+\U^{n+1}) + \frac{3}{2}\F^n - \frac{1}{2}\F^{n-1}, \\
& \frac{\U^{n+1}-\U^n}{\Delta t} = - G \underbrace{\left(\phi^{n+1} - \frac{\Delta t}{2} \ReInv L \phi^{n+1} \right)}_{p^{n+1}} + \frac{1}{2} \ReInv L(\U^n+\U^{n+1}) + \frac{3}{2}\F^n - \frac{1}{2}\F^{n-1}, \qquad \text{ assuming $L,G$ are commutative} \\
\end{aligned} \end{equation}

For general $\theta$ treatment of diffusion, the first equation is slightly different
\begin{equation}\begin{aligned}
\frac{\hat{\U}-\U^n}{\Delta t} = \ReInv L((1-\theta)\U^n+\theta\hat{\U}) + \frac{3}{2}\F^n - \frac{1}{2}\F^{n-1}, \\
\end{aligned} \end{equation}
To analyze the time accuracy, we add (and combine) the first two equations to get
\begin{equation}\begin{aligned}
& \frac{\U^{n+1}-\hat{\U}}{\Delta t} + \frac{\hat{\U}-\U^n}{\Delta t} = - G (\phi^{n+1}) + \ReInv L((1-\theta)\U^n+\theta\hat{\U}) + \frac{3}{2}\F^n - \frac{1}{2}\F^{n-1}, \\
& \frac{\U^{n+1}-\U^n}{\Delta t} = - G (\phi^{n+1}) + \theta \ReInv L(\U^{n+1}+ \Delta t G\phi^{n+1}) + (1-\theta) \ReInv L\U^n + \frac{3}{2}\F^n - \frac{1}{2}\F^{n-1}, \\
& \frac{\U^{n+1}-\U^n}{\Delta t} = - G (\phi^{n+1}) + \theta \Delta t \ReInv L G\phi^{n+1} + \ReInv L(\theta\U^n+(1-\theta)\U^{n+1}) + \frac{3}{2}\F^n - \frac{1}{2}\F^{n-1}, \\
& \frac{\U^{n+1}-\U^n}{\Delta t} = - G \underbrace{\left(\phi^{n+1} - \theta \Delta t \ReInv L \phi^{n+1} \right)}_{p^{n+1}} + \ReInv L(\theta\U^n+(1-\theta)\U^{n+1}) + \frac{3}{2}\F^n - \frac{1}{2}\F^{n-1}, \qquad \text{ assuming $L,G$ are commutative} \\
\end{aligned} \end{equation}


\newpage
\section{Perot analysis applied to MHD equations}
The second order temporally accurate finite difference equations we wish to solve are
\begin{equation}\begin{aligned}
\frac{\U^{n+1}-\U^n}{\Delta t} - \theta_u \ReInv L(\U^{n+1}) = - G p^{n+1} + (1-\theta_u) \ReInv L(\U^n) + \frac{3}{2} \R^n - \frac{1}{2} \R^{n-1} , \\
\R^n = - D(\U^n \U^n) + \Al C(\B^n) \CROSS \B^n, \\
 D \U^{n+1} = 0, \\
\frac{\B^{n+1}-\B^n}{\Delta t} + \theta_B \RemInv C (\left[ \SO^{-1} C\B^{n+1}) \right] = - G\cleanB^{n+1} - (1-\theta_B) \RemInv C (\left[ \SO^{-1} C\B^n) \right] + \frac{3}{2} \T^n - \frac{1}{2} \T^{n-1}, \\
\T^n = C (\U^n \CROSS \B^n), \\
 D \B^{n+1} = 0.
\end{aligned} \end{equation}
We may write this in matrix form as
\[
\MAT{c c c c}
A_u & G & 0   & 0 \\
D   & 0 & 0   & 0 \\
0   & 0 & A_B & G \\
0   & 0 & D   & 0 \\
\EMAT
\MAT{c}
\U^{n+1} \\
 p^{n+1} \\
\B^{n+1} \\
\cleanB^{n+1} \\
\EMAT
=
\MAT{c}
\MAC R \\
0  \\
\MAC T \\
0  \\
\EMAT +
\MAT{c}
BC_1 \\
BC_2 \\
BC_3 \\
BC_4 \\
\EMAT
\]
\begin{equation}\LL\begin{aligned}
& A_u  = \frac{1}{\Delta t} \left(I - \Delta t \theta_u \ReInv L \right), \\
& A_B  = \frac{1}{\Delta t} \left(I + \Delta t \theta_B \RemInv C (\SO^{-1} C) \right), \\
& \MAC R = \frac{1}{\Delta t} \left(I + \Delta t (1-\theta_u) \ReInv L \right) \U^n + \frac{3}{2}\R^n - \frac{1}{2}\R^{n-1}. \\
& \MAC T = \frac{1}{\Delta t} [I - \Delta t(1-\theta_B) \RemInv C ( \SO^{-1} C) ]\B^n + \frac{3}{2} \T^n - \frac{1}{2} \T^{n-1}. \\
\end{aligned} \end{equation}
Now, consider the approximation (which we've seen can achieve 2nd order from Perot):
\[
\MAT{c c c c}
A_u & A_u\Theta_u G    &    0      &              0    \\
D   &        0         &    0      &              0    \\
0   &        0         &    A_B    & A_B \Theta_B G    \\
0   &        0         &    D      &              0    \\
\EMAT
\MAT{c}
\U^{n+1} \\
 p^{n+1} \\
\B^{n+1} \\
\cleanB^{n+1} \\
\EMAT
=
\MAT{c}
\MAC R \\
0  \\
\MAC T \\
0  \\
\EMAT +
\MAT{c}
BC_1 \\
BC_2 \\
BC_3 \\
BC_4 \\
\EMAT
\]
Factorizing (using \textquote{wxMaxima} software), we get
% wxMaxima website: http://andrejv.github.io/wxmaxima/
\[
\MAT{c c c c}
I      &     0           &    0      &          0    \\
D/A_u  &     I           &    0      &          0    \\
0      &     0           &    I      &          0    \\
0      &     0           &    D/A_B  &          I    \\
\EMAT
\MAT{c c c c}
A_u & A_u G \Theta_u    &    0      &          0              \\
0   &    -DG\Theta_u    &    0      &          0              \\
0   &     0             &    A_B    & A_B G \Theta_B          \\
0   &     0             &    0      &          -DG\Theta_B    \\
\EMAT
\MAT{c}
\U^{n+1} \\
 p^{n+1} \\
\B^{n+1} \\
\cleanB^{n+1} \\
\EMAT
=
\MAT{c}
\MAC R \\
0  \\
\MAC T \\
0  \\
\EMAT +
\MAT{c}
BC_1 \\
BC_2 \\
BC_3 \\
BC_4 \\
\EMAT
\]
Re-arranging a bit, we get
\[
\MAT{c c c c}
A_u    &     0           &    0      &          0      \\
D      &  -D\Theta_u G   &    0      &          0      \\
0      &     0           &    A_B    &          0      \\
0      &     0           &    D      &   -D\Theta_B G  \\
\EMAT
\underbrace{
\MAT{c c c c}
I   &     \Theta_u G    &    0      &          0       \\
0   &     I             &    0      &          0       \\
0   &     0             &    I      &     \Theta_B G   \\
0   &     0             &    0      &          I       \\
\EMAT
\MAT{c}
\U^{n+1} \\
 p^{n+1} \\
\B^{n+1} \\
\cleanB^{n+1} \\
\EMAT}_{
\MAT{c}
\hat{\U} \\
 p^{n+1} \\
\hat{\B} \\
\cleanB^{n+1} \\
\EMAT
}
=
\MAT{c}
\MAC R \\
0  \\
\MAC T \\
0  \\
\EMAT +
\MAT{c}
BC_1 \\
BC_2 \\
BC_3 \\
BC_4 \\
\EMAT
\]
Further simplified, this results in a series of operations, which can be viewed as first solving the outer problem:
\[
\MAT{c c c c}
A_u    &     0           &    0      &          0      \\
D      &  -D\Theta_u G   &    0      &          0      \\
0      &     0           &    A_B    &          0      \\
0      &     0           &    D      &   -D\Theta_B G  \\
\EMAT
\MAT{c}
\hat{\U} \\
 p^{n+1} \\
\hat{\B} \\
\cleanB^{n+1} \\
\EMAT
=
\MAT{c}
\MAC R \\
0  \\
\MAC T \\
0  \\
\EMAT +
\MAT{c}
BC_1 \\
BC_2 \\
BC_3 \\
BC_4 \\
\EMAT,
\]
then the inner problem:
\[
\MAT{c c c c}
I   &     \Theta_u G    &    0      &          0       \\
0   &     I             &    0      &          0       \\
0   &     0             &    I      &     \Theta_B G   \\
0   &     0             &    0      &          I       \\
\EMAT
\MAT{c}
\U^{n+1} \\
 p^{n+1} \\
\B^{n+1} \\
\cleanB^{n+1} \\
\EMAT =
\MAT{c}
\hat{\U} \\
 p^{n+1} \\
\hat{\B} \\
\cleanB^{n+1} \\
\EMAT.
\]
The sequence of operations are:
\begin{equation}\LL\begin{aligned}
A_u \hat{\U} = \MAC R + BC_1 \\
D \Theta_u G p^{n+1} = D\hat{\U} - BC_2 \\
\U^{n+1} = \hat{\U} - \Theta_u G p^{n+1} \\
A_B \hat{\B} = \MAC T + BC_3 \\
D \Theta_B G \cleanB^{n+1} = D\hat{\B} - BC_4 \\
\B^{n+1} = \hat{\B} - \Theta_B G \cleanB^{n+1} \\
\end{aligned} \end{equation}
Again, the matrix $\Theta$ may be
\begin{equation}\begin{aligned}
\Theta = \Delta t I, \quad \rightarrow \quad \text{1st order error term} \\
\Theta = A^{-1}, \quad \rightarrow \quad \text{Uzawa method, requires nested iterations (typically avoided)}
\end{aligned} \end{equation}
As noted in Perot, the 1st order error term may be circumvented by absorbing the error in the pressure (and, we assume, correction parameter) gradient

\newpage
\section{Fractional step method for full BC paper}
The second order temporally accurate finite difference equations we wish to solve are
\begin{equation}\begin{aligned}
\frac{\U^{n+1}-\U^n}{\Delta t} - \theta_u \ReInv L(\U^{n+1}) = - G p^{n+1} + (1-\theta_u) \ReInv L(\U^n) + \frac{3}{2} \R^n - \frac{1}{2} \R^{n-1} , \\
\R^n = - D(\U^n \U^n) + \Al C(\B^n) \CROSS \B^n, \\
 D \U^{n+1} = 0, \\
\frac{\B^{n+1}-\B^n}{\Delta t} + \theta_B \RemInv C (\left[ \SO^{-1} C\B^{n+1}) \right] = - G\cleanB^{n+1} - (1-\theta_B) \RemInv C (\left[ \SO^{-1} C\B^n) \right] + \frac{3}{2} \T^n - \frac{1}{2} \T^{n-1}, \\
\T^n = C (\U^n \CROSS \B^n), \\
 D \B^{n+1} = 0.
\end{aligned} \end{equation}
We may write this in matrix form as
\[
\MAT{c c c c}
A_u & G & 0   & 0 \\
D   & 0 & 0   & 0 \\
0   & 0 & A_B & G \\
0   & 0 & D   & 0 \\
\EMAT
\MAT{c}
\U^{n+1} \\
 p^{n+1} \\
\B^{n+1} \\
\cleanB^{n+1} \\
\EMAT
=
\MAT{c}
\MAC R \\
0  \\
\MAC T \\
0  \\
\EMAT +
\MAT{c}
BC_1 \\
BC_2 \\
BC_3 \\
BC_4 \\
\EMAT
\]
\begin{equation}\LL\begin{aligned}
& A_u  = \frac{1}{\Delta t} \left(I - \Delta t \theta_u \ReInv L \right), \\
& A_B  = \frac{1}{\Delta t} \left(I + \Delta t \theta_B \RemInv C (\SO^{-1} C) \right), \\
& \MAC R = \frac{1}{\Delta t} \left(I + \Delta t (1-\theta_u) \ReInv L \right) \U^n + \frac{3}{2}\R^n - \frac{1}{2}\R^{n-1}. \\
& \MAC T = \frac{1}{\Delta t} [I - \Delta t(1-\theta_B) \RemInv C ( \SO^{-1} C) ]\B^n + \frac{3}{2} \T^n - \frac{1}{2} \T^{n-1}. \\
\end{aligned} \end{equation}
Now, consider the approximation (which we've seen can achieve 2nd order from Perot):
\[
\MAT{c c c c}
A_u & A_u\Theta_u G    &    0      &              0    \\
D   &        0         &    0      &              0    \\
0   &        0         &    A_B    & A_B \Theta_B G    \\
0   &        0         &    D      &              0    \\
\EMAT
\MAT{c}
\U^{n+1} \\
 p^{n+1} \\
\B^{n+1} \\
\cleanB^{n+1} \\
\EMAT
=
\MAT{c}
\MAC R \\
0  \\
\MAC T \\
0  \\
\EMAT +
\MAT{c}
BC_1 \\
BC_2 \\
BC_3 \\
BC_4 \\
\EMAT
\]
Factorizing (using \textquote{wxMaxima} software), we get
% wxMaxima website: http://andrejv.github.io/wxmaxima/
\[
\MAT{c c c c}
I      &     0           &    0      &          0    \\
D/A_u  &     I           &    0      &          0    \\
0      &     0           &    I      &          0    \\
0      &     0           &    D/A_B  &          I    \\
\EMAT
\MAT{c c c c}
A_u & A_u G \Theta_u    &    0      &          0              \\
0   &    -DG\Theta_u    &    0      &          0              \\
0   &     0             &    A_B    & A_B G \Theta_B          \\
0   &     0             &    0      &          -DG\Theta_B    \\
\EMAT
\MAT{c}
\U^{n+1} \\
 p^{n+1} \\
\B^{n+1} \\
\cleanB^{n+1} \\
\EMAT
=
\MAT{c}
\MAC R \\
0  \\
\MAC T \\
0  \\
\EMAT +
\MAT{c}
BC_1 \\
BC_2 \\
BC_3 \\
BC_4 \\
\EMAT
\]
Re-arranging a bit, we get
\[
\MAT{c c c c}
A_u    &     0           &    0      &          0      \\
D      &  -D\Theta_u G   &    0      &          0      \\
0      &     0           &    A_B    &          0      \\
0      &     0           &    D      &   -D\Theta_B G  \\
\EMAT
\underbrace{
\MAT{c c c c}
I   &     \Theta_u G    &    0      &          0       \\
0   &     I             &    0      &          0       \\
0   &     0             &    I      &     \Theta_B G   \\
0   &     0             &    0      &          I       \\
\EMAT
\MAT{c}
\U^{n+1} \\
 p^{n+1} \\
\B^{n+1} \\
\cleanB^{n+1} \\
\EMAT}_{
\MAT{c}
\hat{\U} \\
 p^{n+1} \\
\hat{\B} \\
\cleanB^{n+1} \\
\EMAT
}
=
\MAT{c}
\MAC R \\
0  \\
\MAC T \\
0  \\
\EMAT +
\MAT{c}
BC_1 \\
BC_2 \\
BC_3 \\
BC_4 \\
\EMAT
\]
Further simplified, this results in a series of operations, which can be viewed as first solving the outer problem:
\[
\MAT{c c c c}
A_u    &     0           &    0      &          0      \\
D      &  -D\Theta_u G   &    0      &          0      \\
0      &     0           &    A_B    &          0      \\
0      &     0           &    D      &   -D\Theta_B G  \\
\EMAT
\MAT{c}
\hat{\U} \\
 p^{n+1} \\
\hat{\B} \\
\cleanB^{n+1} \\
\EMAT
=
\MAT{c}
\MAC R \\
0  \\
\MAC T \\
0  \\
\EMAT +
\MAT{c}
BC_1 \\
BC_2 \\
BC_3 \\
BC_4 \\
\EMAT,
\]
then the inner problem:
\[
\MAT{c c c c}
I   &     \Theta_u G    &    0      &          0       \\
0   &     I             &    0      &          0       \\
0   &     0             &    I      &     \Theta_B G   \\
0   &     0             &    0      &          I       \\
\EMAT
\MAT{c}
\U^{n+1} \\
 p^{n+1} \\
\B^{n+1} \\
\cleanB^{n+1} \\
\EMAT =
\MAT{c}
\hat{\U} \\
 p^{n+1} \\
\hat{\B} \\
\cleanB^{n+1} \\
\EMAT.
\]
The sequence of operations are:
\begin{equation}\LL\begin{aligned}
A_u \hat{\U} = \MAC R + BC_1 \\
D \Theta_u G p^{n+1} = D\hat{\U} - BC_2 \\
\U^{n+1} = \hat{\U} - \Theta_u G p^{n+1} \\
A_B \hat{\B} = \MAC T + BC_3 \\
D \Theta_B G \cleanB^{n+1} = D\hat{\B} - BC_4 \\
\B^{n+1} = \hat{\B} - \Theta_B G \cleanB^{n+1} \\
\end{aligned} \end{equation}
Again, the matrix $\Theta$ may be
\begin{equation}\begin{aligned}
\Theta = \Delta t I, \quad \rightarrow \quad \text{1st order error term} \\
\Theta = A^{-1}, \quad \rightarrow \quad \text{Uzawa method, requires nested iterations (typically avoided)}
\end{aligned} \end{equation}
As noted in Perot, the 1st order error term may be circumvented by absorbing the error in the pressure (and, we assume, correction parameter) gradient

\input{\rootdir/includes/include_bib.tex}




\end{document}