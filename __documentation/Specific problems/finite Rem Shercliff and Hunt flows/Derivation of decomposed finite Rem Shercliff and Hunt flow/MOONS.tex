\documentclass[11pt]{article}
% \documentclass[3p,twocolumn,10pt]{elsarticle}
\usepackage{graphicx}    % needed for including graphics e.g. EPS, PS
\usepackage{epstopdf}
\usepackage{amsmath}
\usepackage{amssymb}
\usepackage{hyperref}
\usepackage{xspace}
\usepackage{mathtools}
\usepackage{tikz}
\usepackage{epsfig}
\usepackage{float}
%\usepackage{natbib}
\usepackage{subfigure}
\usepackage{setspace}
\usepackage{tabularx,ragged2e,booktabs,caption}

\usepackage{xcolor}
\usepackage{xspace}
\usepackage{longtable}
\usepackage{mathtools}
\usepackage{natbib}
\usepackage{setspace}
\usepackage{ragged2e}
\usepackage{etoolbox}
\usepackage{geometry}


\newcommand{\PATHTOUSEFULCOMMANDS}{../../../../useful_commands.tex}

\input{\PATHTOUSEFULCOMMANDS}

\newcommand{\BS}{B^{SH}}
\newcommand{\BT}{\tilde{B}}

% \setlength{\textfloatsep}{0.1cm}
\newcommand{\volume}{\mathop{\ooalign{\hfil$V$\hfil\cr\kern0.08em--\hfil\cr}}\nolimits}
\newcommand{\figWidth}{0.45\textwidth}
\newcommand{\figSuffix}{_png.png}
\newcommand{\pushright}[1]{\ifmeasuring@#1\else\omit\hfill$\displaystyle#1$\fi\ignorespaces}
\newcommand{\hfillMath}{\hskip \textwidth minus \textwidth}
% \renewcommand{\arraystretch}{0.0} % Removes vertical spaces before/after tabularx


\setlength{\oddsidemargin}{0.1in}
\setlength{\textwidth}{7.25in}

\setlength{\topmargin}{-1in}     %\topmargin: gap above header
\setlength{\headheight}{0in}     %\headheight: height of header
\setlength{\topskip}{0in}        %\topskip: between header and text
\setlength{\headsep}{0in}
\setlength{\textheight}{692pt}   %\textheight: height of main text
\setlength{\textwidth}{7.5in}    % \textwidth: width of text
\setlength{\oddsidemargin}{-0.5in}  % \oddsidemargin: odd page left margin
\setlength{\evensidemargin}{0in} %\evensidemargin : even page left margin
\setlength{\parindent}{0.25in}   %\parindent: indentation of paragraphs
\setlength{\parskip}{0pt}        %\parskip: gap between paragraphs
\setlength{\voffset}{0.5in}

% Useful commands:

% \hfill    aligns-right everything right of \hfill

\begin{document}
\doublespacing
\title{Secondary flow in finite Rem Shercliff and Hunt analog}
\author{C. Kawczynski \\
Department of Mechanical and Aerospace Engineering \\
University of California Los Angeles, USA\\
}
\maketitle

\section{Before higher order attempt}
\section{Low but finite Rem Shercliff and Hunt flow}
The dimensional momentum and induction equations for uniform material properties are

\begin{equation}\begin{aligned}
\PD_t u_i+u_j\PD_j u_i = -\frac{1}{\rho} \PD_i p+ \nu \PD_{jj} u_i + \frac{1}{\rho} (B_j \PD_j B_i - \tfrac{1}{2} \PD_i B_j B_j) \\
\PD_t B_i = \frac{1}{\sigma \mu} \PD_{jj} B_i - \PD_j (u_j B_i - u_i B_j)
\end{aligned} \end{equation}

Non-dimensionalizing using

\begin{equation}\begin{aligned}
	[u] = U, \qquad
	[B] = U (\sigma \rho \nu)^{1/2} \times \mu \quad \text{(Shercliff uses H)}, \qquad
	[P] = \rho U \nu / L, \qquad
	[t] = L^2 / \nu
\end{aligned} \end{equation}

The dimensionless momentum and induction equations are

\begin{equation}\begin{aligned}
\frac{U \nu}{L^2} \PD_t u_i+ \frac{U^2}{L} u_j\PD_j u_i =-\frac{\rho \nu U}{\rho L^2}\PD_i p+ \frac{U \nu}{L^2} \PD_{jj} u_i + \frac{U^2 (\sigma \rho \nu) \mu^2}{\rho L \mu} (B_j \PD_j B_i - \tfrac{1}{2} \PD_i B_j B_j) \\
\frac{\nu U (\sigma \rho \nu)^{1/2}}{L^2} \PD_t B_i + \frac{U^2 (\sigma \rho \nu)^{1/2}}{L} \PD_j (u_j B_i - u_i B_j) = \frac{U (\sigma \rho \nu)^{1/2}}{\sigma \mu L^2} \PD_{jj} B_i
\end{aligned} \end{equation}
Simplifying, we have
\begin{equation}\begin{aligned}
\PD_t u_i+ Re u_j\PD_j u_i =-\PD_i p+ \PD_{jj} u_i + Re_m (B_j \PD_j B_i - \tfrac{1}{2} \PD_i B_j B_j) \\
Re^{-1} \PD_t B_i + \PD_j (u_j B_i - u_i B_j) = Re_m^{-1} \PD_{jj} B_i
\end{aligned} \end{equation}

Instead, using Hartmann number for $B$, that is, the scale for $B$ is instead $L^{-1} (\rho \nu / \sigma)^{1/2}$, we have

\begin{equation}\begin{aligned}
\frac{U \nu}{L^2} \PD_t u_i+ \frac{U^2}{L} u_j\PD_j u_i =-\frac{\rho \nu U}{\rho L^2}\PD_i p+ \frac{U \nu}{L^2} \PD_{jj} u_i + \frac{L^{-2} (\rho \nu / \sigma)}{\rho L \mu} (B_j \PD_j B_i - \tfrac{1}{2} \PD_i B_j B_j) \\
\frac{\nu U (\sigma \rho \nu)^{1/2}}{L^2} \PD_t B_i + \frac{U^2 (\sigma \rho \nu)^{1/2}}{L} \PD_j (u_j B_i - u_i B_j) = \frac{U (\sigma \rho \nu)^{1/2}}{\sigma \mu L^2} \PD_{jj} B_i
\end{aligned} \end{equation}
Simplifying, we have
\begin{equation}\begin{aligned}
\PD_t u_i+ Re u_j\PD_j u_i =-\PD_i p+ \PD_{jj} u_i + Re_m^{-1} (B_j \PD_j B_i - \tfrac{1}{2} \PD_i B_j B_j) \\
Re^{-1} \PD_t B_i + \PD_j (u_j B_i - u_i B_j) = Re_m^{-1} \PD_{jj} B_i
\end{aligned} \end{equation}

Where $\U,\B,p,Re,N,Re_m$ are the velocity, magnetic field, pressure, Reynolds, interaction and magnetic Reynolds number respectively. First, let's let the mechanical pressure absorb the magnetic pressure and seek the steady solution.
\begin{equation}\begin{aligned}
u_j\PD_j u_i =-\PD_i p+ Re^{-1} \PD_{jj} u_i + Re_m^{-1} Ha^2 Re^{-1} (B_j \PD_j B_i) \\
Re_m^{-1} \PD_{jj} B_i = \PD_j (u_j B_i - u_i B_j)
\end{aligned} \end{equation}
Since we are interested in when $Re_m$ is small, and the induction equation is linear, we'll combine them by factors of $Re_m$. There is documentation in MOONS how this is done. To be clear we may decompse $\B$ in one of two ways:
\begin{equation}\begin{aligned}
	\B = \B^0 + \B^1 + \B^2 + \hdots , \\
	\B = \B^0 + Re_m \B^* + Re_m^2 \B^{**} + \hdots \\
	\J = \CURL \B = \CURL \B^* + Re_m \CURL \B^{**} + \hdots \\
	\J \times \B = B_j \PD_j B_i - \underbrace{\tfrac{1}{2} \PD_j (B_i B_i)}_{\text{let pressure absorb}} \\
	= B_j \PD_j B_i \\
	= (B_j^0+Re_m B_j^* + \hdots) \PD_j (B_i^*+Re_m B_i^{**} + \hdots) \\
\end{aligned} \end{equation}
Now, let's rewrite the induction equation and Lorentz force with this expansion and neglect $Re_m^2$ terms

\begin{equation}\begin{aligned}
u_j\PD_j u_i =-\PD_i p+ Re^{-1} \PD_{jj} u_i + N (B_j^0+Re_m B_j^*) \PD_j B_i^*, \qquad \qquad \text{ neglecting } \B^{**}  \\
\PD_{jj} B_i^{*}  = \PD_j (u_j (B_i^0 + Re_m B_i^*) - u_i (B_j^0+Re_m B_j^*)), \qquad \qquad \text{equation for } \B, \text{ neglecting } \B^{**} \\
\end{aligned} \end{equation}

Note that $\B^{**}$ has been neglected in both the induction and momentum equations. Our final result is:

\begin{equation}\boxed{\begin{aligned}
u_j\PD_j u_i =-\PD_i p+ Re^{-1} \PD_{jj} u_i + N (B_j^0+Re_m B_j^*) \PD_j B_i^* \\
\PD_{jj} B_i^* = \PD_j (u_j (B_i^0 + Re_m B_i^{*}) - u_i (B_j^0+Re_m B_j^{*}))
\end{aligned}}\end{equation}

Notice that with this formulation, we may set $Re_m$ to be zero \textit{exactly} and still be able to solve for have a physical MHD result since $\B^*$ has absorbed $Re_m$ in the denominator of the Lorentz force term in the momentum equation. This is the low $Re_m$ formulation.

\section{End of finite Rem formulation in B, start of finite Rem formulation in U}
Now, let's expand velocity and pressure with respect to $Re_m$.
\begin{equation}\begin{aligned}
	\U = \U^0 + Re_m \U^* + Re_m^2 \U^{**} + \hdots \\
	p = p^0 + Re_m p^* + Re_m^2 p^{**} + \hdots
\end{aligned} \end{equation}
Plug into the momentum and induction equations.
\small\begin{equation}\begin{aligned}
(u_j^0+Re_m u_j^*)\PD_j (u_i^0+Re_m u_i^*) =-\PD_i p^0-Re_m\PD_i p^*+ Re^{-1} \PD_{jj} (u_i^0+Re_m u_i^*) + N (B_j^0+Re_m B_j^*) \PD_j B_i^* \\
\PD_{jj} B_i^* =  \PD_j ((u_j^0+Re_m u_j^*) (B_i^0 + Re_m B_i^{*}) - (u_i^0+Re_m u_i^*) (B_j^0+Re_m B_j^{*}))
\end{aligned}\end{equation}\normalsize
Expanding out and neglect $Re_m^2$ order terms.
\begin{equation}\boxed{\begin{aligned}
u_j^0 \PD_j u_i^0 +
Re_m u_j^0 \PD_j u_i^* +
Re_m u_j^* \PD_j u_i^0
= -\PD_i p^0 - Re_m\PD_i p^*+ Re^{-1} \PD_{jj} u_i^0 + Re_m Re^{-1} \PD_{jj} u_i^* + N B_j^0 \PD_j B_i^* + Re_m N B_j^* \PD_j B_i^* \\
\PD_{jj} B_i^* = \PD_j ( u_j^0 B_i^0 + Re_m u_j^0 B_i^{*} + Re_m u_j^* B_i^0 - u_i^0 B_j^0 - Re_m u_i^0 B_j^{*} - Re_m u_i^* B_j^0)
\end{aligned}}\end{equation}
Let's collect terms of $Re_m$ to end up with two equations for momentum and two equations for induction

\section{Separating \texorpdfstring{$\B^*$}{} into \texorpdfstring{$\B^{\text{Shercliff}}$}{} and \texorpdfstring{$\tilde{\B}$}{}}

Let's decompose the induced magnetic field into
\begin{equation}
	\B^* = \mathbf{B}^{SH} + Re_m \tilde{\mathbf{B}}
\end{equation}
Doing so yields
\tiny\begin{equation}\boxed{\begin{aligned}
u_j^0 \PD_j u_i^0 +
Re_m u_j^0 \PD_j u_i^* +
Re_m u_j^* \PD_j u_i^0
= -\PD_i p^0 - Re_m\PD_i p^*+ Re^{-1} \PD_{jj} u_i^0 + Re_m Re^{-1} \PD_{jj} u_i^* + N B_j^0 \PD_j (\BS_i+Re_m \BT_i) + Re_m N (\BS_j+Re_m \BT_j) \PD_j (\BS_i+Re_m \BT_i) \\
\PD_{jj} (\BS_i+Re_m \BT_i) = \PD_j ( u_j^0 B_i^0 + Re_m u_j^0 (\BS_i+Re_m \BT_i) + Re_m u_j^* B_i^0 - u_i^0 B_j^0 - Re_m u_i^0 (\BS_j+Re_m \BT_j) - Re_m u_i^* B_j^0)
\end{aligned}}\end{equation}\normalsize
Again, neglecting the higher order terms, we have

\tiny\begin{equation}\begin{aligned}
u_j^0 \PD_j u_i^0 +
Re_m u_j^0 \PD_j u_i^* +
Re_m u_j^* \PD_j u_i^0
= -\PD_i p^0 - Re_m\PD_i p^*+ Re^{-1} \PD_{jj} u_i^0 + Re_m Re^{-1} \PD_{jj} u_i^* + N B_j^0 \PD_j (\BS_i+Re_m \BT_i) + Re_m N \BS_j \PD_j \BS_i \\
\PD_{jj} (\BS_i+Re_m \BT_i) = \PD_j ( u_j^0 B_i^0 + Re_m u_j^0 \BS_i + Re_m u_j^* B_i^0 - u_i^0 B_j^0 - Re_m u_i^0 \BS_j - Re_m u_i^* B_j^0)
\end{aligned}\end{equation}\normalsize

\section{Separation into Rem 0 and Rem 1}
Equations for $Re_m^0$
\begin{equation}\boxed{\begin{aligned}
u_j^0 \PD_j u_i^0 = -\PD_i p^0 + Re^{-1} \PD_{jj} u_i^0 + N B_j^0 \PD_j \BS_i \\
\PD_{jj} \BS_i = \PD_j ( u_j^0 B_i^0 - u_i^0 B_j^0)
\end{aligned}}\end{equation}

Equations for $Re_m^1$
\begin{equation}\boxed{\begin{aligned}
u_j^0 \PD_j u_i^* + u_j^* \PD_j u_i^0 = -\PD_i p^*+ Re^{-1} \PD_{jj} u_i^* + N ( B_j^0 \PD_j \BT_i + \BS_j \PD_j \BS_i ) \\
\PD_{jj} \BT_i = \PD_j ( u_j^0 \BS_i + u_j^* B_i^0 - u_i^0 \BS_j - u_i^* B_j^0)
\end{aligned}}\end{equation}

\section{Apply duct flow conditions}
Let's expand and apply $B_x^0 = 0,B_y^0 = 0, v^0 = 0, w^0 = 0$ and $\PD_x() = 0$ except for pressure, which we'll assume varies linearly.
Keeping in mind to group magnetic pressure with mechanical pressure, we have

Equations for $Re_m^0$
\begin{equation}\begin{aligned}
0 = -\PD_x p^0 + Re^{-1} (\PD_{yy} u^0 + \PD_{zz} u^0) + N (B_z^0 \PD_z \BS_x) \\
0 = -\PD_y p^0 + Re^{-1} (\PD_{yy} v^0 + \PD_{zz} v^0) + N (B_z^0 \PD_z \BS_y) \\
0 = -\PD_z p^0 + Re^{-1} (\PD_{yy} w^0 + \PD_{zz} w^0) + N (B_z^0 \PD_z \BS_z) \\
\PD_{yy} \BS_x + \PD_{zz} \BS_x = - \PD_y ( u^0 B_y^0) - \PD_z ( u^0 B_z^0) \\
\PD_{yy} \BS_y + \PD_{zz} \BS_y = - \PD_z ( v^0 B_z^0) \\
\PD_{yy} \BS_z + \PD_{zz} \BS_z = - \PD_y ( w^0 B_y^0) \\
\end{aligned}\end{equation}

Equations for $Re_m^1 $
\small\begin{equation}\begin{aligned}
v^* \PD_y u^0 + w^* \PD_z u^0 = -\PD_x p^*+ Re^{-1} (\PD_{yy} u^* + \PD_{zz} u^*) + N ( B_z^0 \PD_z \BT_x + \BS_y \PD_y \BS_x + \BS_z \PD_z \BS_x ) \\
0                             = -\PD_y p^*+ Re^{-1} (\PD_{yy} v^* + \PD_{zz} v^*) + N ( B_z^0 \PD_z \BT_y + \BS_y \PD_y \BS_y + \BS_z \PD_z \BS_y ) \\
0                             = -\PD_z p^*+ Re^{-1} (\PD_{yy} w^* + \PD_{zz} w^*) + N ( B_z^0 \PD_z \BT_z + \BS_y \PD_y \BS_z + \BS_z \PD_z \BS_z ) \\
\PD_{jj} \BT_x =-\PD_y ( u^0 \BS_y) + \PD_z ( u^0 \BS_z - u^* B_z^0) \\
\PD_{jj} \BT_y =-\PD_z ( v^* B_z^0) \\
\PD_{jj} \BT_z = \PD_y ( v^* B_z^0) \\
\end{aligned}\end{equation}\normalsize

\section{Separate by SH and secondary flow}
Using applied magnetic field conditions and SH solution $B_x^0=0,B_y^0=0,v^0=0,w^0=0,B_z=\text{uniform}$ yields the SH governing equations and secondary flow governing equations

Equations for $Re_m^0$
\begin{equation}\boxed{\begin{aligned}
0 = -\PD_x p^0 + Re^{-1} (\PD_{yy} u^0 + \PD_{zz} u^0) + N B_z^0 \PD_z \BS_x \\
\PD_{yy} \BS_x + \PD_{zz} \BS_x = - \PD_z (u^0 B_z^0) \\
\end{aligned}} \qquad \text{Shercliff / Hunt governing equations}\end{equation}
Scaling $\PD_x p^0$ by $Re$ and multiplying by $Re$ yields
\begin{equation}\boxed{\begin{aligned}
1 = (\PD_{yy} u^0 + \PD_{zz} u^0) + Ha^2 B_z^0 \PD_z \BS_x \\
\PD_{yy} \BS_x + \PD_{zz} \BS_x = - \PD_z (u^0 B_z^0) \\
\end{aligned}} \qquad \text{Shercliff / Hunt governing equations}\end{equation}


Equations for $Re_m^1$
\begin{equation}\begin{aligned}
v^* \PD_y u^0 + w^* \PD_z u^0 = -\PD_x p^*+ Re^{-1} (\PD_{yy} u^* + \PD_{zz} u^*) + N ( B_z^0 \PD_z \BT_x + \BS_y \PD_y \BS_x + \BS_z \PD_z \BS_x ) \\
0 = -\PD_y p^*+ Re^{-1} (\PD_{yy} v^* + \PD_{zz} v^*) + N ( B_z^0 \PD_z \BT_y + \BS_z \PD_z \BS_y ) \\
0 = -\PD_z p^*+ Re^{-1} (\PD_{yy} w^* + \PD_{zz} w^*) + N ( B_z^0 \PD_z \BT_z + \BS_y \PD_y \BS_z) \\
\PD_{yy} \BT_x + \PD_{zz} \BT_x =-\PD_y ( u^0 \BS_y) - \PD_z ( u^0 \BS_z + u^* B_z^0) \\
\PD_{yy} \BT_y + \PD_{zz} \BT_y = \PD_z ( v^* B_z^0 ) \\
\PD_{yy} \BT_z + \PD_{zz} \BT_z = \PD_y ( v^* B_z^0 ) \\
\end{aligned}\end{equation}
Noting that $\BS_y = 0, \BS_z = 0$, we have

Equations for $Re_m^1$
\begin{equation}\begin{aligned}
v^* \PD_y u^0 + w^* \PD_z u^0 = -\PD_x p^*+ Re^{-1} (\PD_{yy} u^* + \PD_{zz} u^*) + N B_z^0 \PD_z \BT_x \\
0 = -\PD_y p^*+ Re^{-1} (\PD_{yy} v^* + \PD_{zz} v^*) + N B_z^0 \PD_z \BT_y \\
0 = -\PD_z p^*+ Re^{-1} (\PD_{yy} w^* + \PD_{zz} w^*) + N B_z^0 \PD_z \BT_z \\
\PD_{yy} \BT_x + \PD_{zz} \BT_x =-\PD_z ( u^* B_z^0 ) \\
\PD_{yy} \BT_y + \PD_{zz} \BT_y = \PD_z ( v^* B_z^0 ) \\
\PD_{yy} \BT_z + \PD_{zz} \BT_z = \PD_y ( v^* B_z^0 ) \\
\end{aligned}\end{equation}

It is clear that the transverse directions are decoupled from the streamwise direction.
Let's focus on the transverse direction, and return to streamwise later. Also, let's assume $B_z^0$. We have

\begin{equation}\begin{aligned}
0 = -\PD_y p^*+ Re^{-1} (\PD_{yy} v^* + \PD_{zz} v^*) + N B_z^0 \PD_z \BT_y \\
0 = -\PD_z p^*+ Re^{-1} (\PD_{yy} w^* + \PD_{zz} w^*) + N B_z^0 \PD_z \BT_z \\
\PD_{yy} \BT_y + \PD_{zz} \BT_y = B_z^0 \PD_z v^* \\
\PD_{yy} \BT_z + \PD_{zz} \BT_z = B_z^0 \PD_y v^* \\
\end{aligned}\end{equation}

Taking the curl of these equations we have

\begin{equation}\begin{aligned}
0 = Re^{-1} (\PD_{yy} \omega_x^* + \PD_{zz} \omega_x^*) + N B_z^0 (\PD_{yz} \BT_z - \PD_{zz} \BT_y) \\
0 = \PD_y (\PD_{yy} \BT_z + \PD_{zz} \BT_z - B_z^0 \PD_y v^*) - \PD_z (\PD_{yy} \BT_y + \PD_{zz} \BT_y - B_z^0 \PD_z v^*) \\
\PD_y (\PD_{yy} \BT_z + \PD_{zz} \BT_z) - \PD_z (\PD_{yy} \BT_y + \PD_{zz} \BT_y) = B_z^0 (\PD_{yy} v^* - \PD_{zz} v^*), \qquad \text{modifying above} \\
\end{aligned}\end{equation}

Writing in terms of current

\begin{equation}\begin{aligned}
0 = Re^{-1} (\PD_{yy} \omega_x^* + \PD_{zz} \omega_x^*) + N B_z^0 \PD_{z} j_x \\
\PD_{yy} j_x + \PD_{zz} j_x = B_z^0 (\PD_{yy} v^* - \PD_{zz} v^*) \\
\end{aligned}\end{equation}
Or, for $\B_z^0 = 1$, we have

\begin{equation}\boxed{\begin{aligned}
\PD_t \omega_x = \PD_{yy} \omega_x^* + \PD_{zz} \omega_x^* + Ha^2 \PD_{z} j_x \\
\PD_t j_x + \PD_{yy} j_x + \PD_{zz} j_x = (\PD_{yy} v^* - \PD_{zz} v^*) \\
\end{aligned}}\end{equation}

\section{Assumed solution}

Assuming the solutions are of the form

\begin{equation}\begin{aligned}
\omega_x = \bar{\omega_x} e^{i(\alpha y + \beta z - \theta t)} \\
j_x      = \bar{j_x}      e^{i(\alpha y + \beta z - \theta t)} \\
v^*      = \bar{v}        e^{i(\alpha y + \beta z - \theta t)} \\
\alpha = \alpha_r, \qquad \beta = \beta_r, \qquad \theta = \theta_r + i \theta_i,  \\
\end{aligned}\end{equation}

\begin{equation}\begin{aligned}
\{\omega_x,j_x,v^*\}(y,z,t) = \{\bar{\omega_x},\bar{j_x},\bar{v}\}(y,z) + \{\hat{\omega_x},\hat{j_x},\hat{v}\} (y,z) e^{i(\alpha y + \beta z - \theta t)} \\
\alpha = \alpha_r, \qquad \beta = \beta_r, \qquad \theta = \theta_r + i \theta_i,  \\
\end{aligned}\end{equation}

We can substitute to get

\begin{equation}\begin{aligned}
-i\theta \bar{\omega_x} = - (\alpha^2 \bar{\omega_x} + \beta^2 \bar{\omega_x}) + Ha^2 i \beta \bar{j_x} \\
i\theta \bar{j_x} + \alpha^2 \bar{j_x} + \beta^2 \bar{j_x} = \alpha^2 \bar{v} + \beta^2 \bar{v} \\
\end{aligned}\end{equation}

Taking the derivative of the momentum equation w.r.t $z$, we may combine the two equations
after solving for the Lorentz force term in the induction equation

\begin{equation}\begin{aligned}
-i\theta \PD_{zt} \bar{\omega_x} = - (\alpha^2 \PD_{zyy} \bar{\omega_x} + \beta^2 \PD_{zzz} \bar{\omega_x}) + Ha^2 i \beta \PD_{zz} \bar{j_x} \\
\beta^2 \PD_{zz} \bar{j_x} = (\alpha^2 \PD_{yy} \bar{v} + \beta^2 \PD_{zz} \bar{v}) - i\theta \PD_t \bar{j_x} - \alpha^2 \PD_{yy} \bar{j_x} \\
\beta \PD_{zz} \bar{j_x} = \frac{\alpha^2}{\beta} \PD_{yy} \bar{v} + \beta \PD_{zz} \bar{v} - i \frac{\theta}{\beta} \PD_t \bar{j_x} - \frac{\alpha^2}{\beta} \PD_{yy} \bar{j_x} \\
\end{aligned}\end{equation}
Combining and expanding $\theta$, we have
\begin{equation}\begin{aligned}
-i(\theta_r + i\theta_i) \PD_{zt} \bar{\omega_x} = - (\alpha^2 \PD_{zyy} \bar{\omega_x} + \beta^2 \PD_{zzz} \bar{\omega_x}) + Ha^2 i \left( \frac{\alpha^2}{\beta} \PD_{yy} \bar{v} + \beta \PD_{zz} \bar{v} - i \frac{(\theta_r + i\theta_i)}{\beta} \PD_t \bar{j_x} - \frac{\alpha^2}{\beta} \PD_{yy} \bar{j_x} \right) \\
-i(\theta_r + i\theta_i) \beta \PD_{zt} \bar{\omega_x} = - \beta (\alpha^2 \PD_{zyy} \bar{\omega_x} + \beta^2 \PD_{zzz} \bar{\omega_x}) + Ha^2 i \left( \alpha^2 \PD_{yy} \bar{v} + \beta^2 \PD_{zz} \bar{v} - i (\theta_r + i\theta_i) \PD_t \bar{j_x} - \alpha^2 \PD_{yy} \bar{j_x} \right) \\
(-i\theta_r + \theta_i) \beta \PD_{zt} \bar{\omega_x} = - \beta (\alpha^2 \PD_{zyy} \bar{\omega_x} + \beta^2 \PD_{zzz} \bar{\omega_x}) + Ha^2 \theta_r \PD_t \bar{j_x} + Ha^2 i \left( \alpha^2 \PD_{yy} \bar{v} + \beta^2 \PD_{zz} \bar{v} + \theta_i \PD_t \bar{j_x} - \alpha^2 \PD_{yy} \bar{j_x} \right) \\
\end{aligned}\end{equation}
Comparing the real and imaginary parts of this equation, we have
\begin{equation}\begin{aligned}
 \theta_i \beta \PD_{zt} \bar{\omega_x} = - \beta (\alpha^2 \PD_{zyy} \bar{\omega_x} + \beta^2 \PD_{zzz} \bar{\omega_x}) + Ha^2 \theta_r \PD_t \bar{j_x}, \qquad \text{real} \\
-\theta_r \beta \PD_{zt} \bar{\omega_x} = Ha^2 \left( \alpha^2 \PD_{yy} \bar{v} + \beta^2 \PD_{zz} \bar{v} + \theta_i \PD_t \bar{j_x} - \alpha^2 \PD_{yy} \bar{j_x} \right), \qquad \text{imag} \\
\end{aligned}\end{equation}

\end{document}