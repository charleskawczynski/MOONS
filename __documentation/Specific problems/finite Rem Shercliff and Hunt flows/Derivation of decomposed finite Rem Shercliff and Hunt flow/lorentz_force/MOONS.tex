\documentclass[11pt]{article}
% \documentclass[3p,twocolumn,10pt]{elsarticle}
\usepackage{graphicx}    % needed for including graphics e.g. EPS, PS
\usepackage{epstopdf}
\usepackage{amsmath}
\usepackage{amssymb}
\usepackage{hyperref}
\usepackage{xspace}
\usepackage{mathtools}
\usepackage{tikz}
\usepackage{epsfig}
\usepackage{float}
%\usepackage{natbib}
\usepackage{subfigure}
\usepackage{setspace}
\usepackage{tabularx,ragged2e,booktabs,caption}

\usepackage{xcolor}
\usepackage{xspace}
\usepackage{longtable}
\usepackage{mathtools}
\usepackage{natbib}
\usepackage{setspace}
\usepackage{ragged2e}
\usepackage{etoolbox}
\usepackage{geometry}


\newcommand{\B}{\mathbf{B}}
\newcommand{\BT}{\tilde{\mathbf{B}}}
\renewcommand{\H}{\mathbf{H}}
\newcommand{\C}{\mathbf{C}}
\newcommand{\U}{\mathbf{u}}
\newcommand{\SI}{\sigma}
\newcommand{\M}{\mu}
\newcommand{\curl}{\nabla \times}

\newcommand{\A}{\mathbf{A}}
\newcommand{\PD}{\partial}
\newcommand{\MC}{\mathcal}
\newcommand{\BM}{\frac{\B}{\mu}}
\newcommand{\J}{\mathbf{j}}
\newcommand{\E}{\mathbf{E}}
\newcommand{\N}{\mathbf{n}}
\newcommand{\JS}{\frac{\J}{\sigma}}
\newcommand{\JoS}{\frac{\J^1}{\sigma}}
\newcommand{\JSS}{\frac{\J^2}{\sigma}}
\newcommand{\SII}{\sigma^{-1}}
\newcommand{\MO}{\overline{\mu}}
\newcommand{\SO}{\overline{\sigma}}
\newcommand{\DOT}{\text{\textbullet}}

\newcommand{\Rei}{\frac{1}{Re}}
\newcommand{\Remi}{\frac{1}{Re_m}}
\newcommand{\NRemi}{\frac{N}{Re_m}}

% \setlength{\textfloatsep}{0.1cm}
\newcommand{\volume}{\mathop{\ooalign{\hfil$V$\hfil\cr\kern0.08em--\hfil\cr}}\nolimits}
\newcommand{\figWidth}{0.45\textwidth}
\newcommand{\figSuffix}{_png.png}
\newcommand{\pushright}[1]{\ifmeasuring@#1\else\omit\hfill$\displaystyle#1$\fi\ignorespaces}
\newcommand{\hfillMath}{\hskip \textwidth minus \textwidth}
% \renewcommand{\arraystretch}{0.0} % Removes vertical spaces before/after tabularx


\setlength{\oddsidemargin}{0.1in}
\setlength{\textwidth}{7.25in}

\setlength{\topmargin}{-1in}     %\topmargin: gap above header
\setlength{\headheight}{0in}     %\headheight: height of header
\setlength{\topskip}{0in}        %\topskip: between header and text
\setlength{\headsep}{0in}        
\setlength{\textheight}{692pt}   %\textheight: height of main text
\setlength{\textwidth}{7.5in}    % \textwidth: width of text
\setlength{\oddsidemargin}{-0.5in}  % \oddsidemargin: odd page left margin
\setlength{\evensidemargin}{0in} %\evensidemargin : even page left margin
\setlength{\parindent}{0.25in}   %\parindent: indentation of paragraphs
\setlength{\parskip}{0pt}        %\parskip: gap between paragraphs
\setlength{\voffset}{0.5in}

% Useful commands:

% \hfill    aligns-right everything right of \hfill

\begin{document}
\doublespacing
\title{Secondary flow in finite Rem Shercliff and Hunt analog}
\author{C. Kawczynski \\
Department of Mechanical and Aerospace Engineering \\
University of California Los Angeles, USA\\
}
\maketitle

\section{Low but finite Rem Shercliff and Hunt flow}
Here, we derived the equations for a finite magnetic Reynolds number ($Re_m$) Shercliff / Hunt (SH) flow. The dimensionless momentum and induction equations are

\begin{equation}\begin{aligned}
\PD_t u_i+u_j\PD_j u_i =-\PD_i p+ Re^{-1} \PD_{jj} u_i + Re_m^{-1} N (B_j \PD_j B_i - \tfrac{1}{2} \PD_i B_j B_j) \\
\PD_t B_i = Re_m^{-1} \PD_{jj} B_i - \PD_j (u_j B_i - u_i B_j)
\end{aligned} \end{equation}

Where $\U,\B,p,Re,N,Re_m$ are the velocity, magnetic field, pressure, Reynolds, interaction and magnetic Reynolds number respectively. First, let's let the mechanical pressure absorb the magnetic pressure and seek the steady solution.
\begin{equation}\begin{aligned}
u_j\PD_j u_i =-\PD_i p+ Re^{-1} \PD_{jj} u_i + Re_m^{-1} N (B_j \PD_j B_i) \\
Re_m^{-1} \PD_{jj} B_i = \PD_j (u_j B_i - u_i B_j)
\end{aligned} \end{equation}
Since we are interested in when $Re_m$ is small, and the induction equation is linear, we'll combine them by factors of $Re_m$. There is documentation in MOONS how this is done. To be clear we may decompse $\B$ in one of two ways:

\begin{equation}\begin{aligned}
	\B = \B^0 + \B^1 + \B^2 + \hdots , \\
	\B = \B^0 + Re_m \B^* + Re_m^2 \B^{**} + \hdots \\
	\J = \curl \B = \curl \B^* + Re_m \curl \B^{**} + \hdots \\
	\J \times \B = B_j \PD_j B_i - \underbrace{\tfrac{1}{2} \PD_j (B_i B_i)}_{\text{let pressure absorb}} \\
	= B_j \PD_j B_i \\
	= (B_j^0+Re_m B_j^* + \hdots) \PD_j (B_i^*+Re_m^1 B_i^{**} + \hdots) \\
\end{aligned} \end{equation}
Setting $\PD_x() = 0$ and looking at the $x$-component of the curl of $\J \times \B$, we have
\begin{equation}\begin{aligned}
	\left [\curl (\J \times \B) \right]_x = 
	\PD_y ((Re_m B_x^* + Re_m B_y^* + B_z^0+Re_m B_z^*) (\PD_y B_z+\PD_z B_z)) - \\
	\PD_z ((Re_m B_x^* + Re_m B_y^* + B_z^0+Re_m B_z^*) (\PD_y B_y+\PD_z B_y))
\end{aligned} \end{equation}
And the $x$-compontent of $\J \times \B$ is 
\begin{equation}\begin{aligned}
	(\J \times \B)_x = 
	(B_y^0+Re_m B_y^*) \PD_y (B_y^*+Re_m^1 B_y^{**}) + (B_z^0+Re_m B_z^*) \PD_z (B_x^*+Re_m^1 B_x^{**}) \\
\end{aligned} \end{equation}

\end{document}