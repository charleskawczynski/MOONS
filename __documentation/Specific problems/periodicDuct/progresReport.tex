\documentclass[11pt]{article}
\usepackage{graphicx}    % needed for including graphics e.g. EPS, PS
\usepackage{epstopdf}
\usepackage{amsmath}
\usepackage{hyperref}
\usepackage{xspace}
\usepackage{mathtools}
\usepackage{tikz}
\usepackage{epsfig}
\usepackage{float}
\usepackage{natbib}
\usepackage{subfigure}
\usepackage{setspace}
\usepackage{tabularx,ragged2e,booktabs,caption}
\usepackage{etoolbox}

\newcommand{\figS}{22.1em}
\newcommand{\BfigS}{50em}
\newcommand{\ffigS}{15.1em}
\newcommand{\figH}{\figS}
\newcommand{\figW}{\figS}
\newcommand{\BfigH}{\BfigS}
\newcommand{\BfigW}{\BfigS}
\newcommand{\ffigH}{\ffigS}
\newcommand{\ffigW}{\ffigS}


\setlength{\oddsidemargin}{0.1in}
\setlength{\textwidth}{7.25in}

\setlength{\topmargin}{-1in}     %\topmargin: gap above header
\setlength{\headheight}{0in}     %\headheight: height of header
\setlength{\topskip}{0in}        %\topskip: between header and text
\setlength{\headsep}{0in}        
\setlength{\textheight}{692pt}   %\textheight: height of main text
\setlength{\textwidth}{7.5in}    % \textwidth: width of text
\setlength{\oddsidemargin}{-0.5in}  % \oddsidemargin: odd page left margin
\setlength{\evensidemargin}{0in} %\evensidemargin : even page left margin
\setlength{\parindent}{0.25in}   %\parindent: indentation of paragraphs
\setlength{\parskip}{0pt}        %\parskip: gap between paragraphs
\setlength{\voffset}{0.5in}

\newtoggle{plotUstream}
\toggletrue{plotUstream}
\togglefalse{plotUstream}

% Useful commands:

% \hfill		aligns-right everything right of \hfill

\begin{document}
\doublespacing
\title{Results of Dynamic MHD Periodic Duct for finite \texorpdfstring{$Re_m$}}
\author{S. Smolentsev, C. Kawczynski \\
Department of Mechanical and Aerospace Engineering \\
University of California Los Angeles, USA\\
}
\maketitle


\section{Case E: Fully Coupled 3 Component Magnetic field in a Periodic Duct}

Here, we made the following assumptions:

\begin{itemize}
\item Uniform $\sigma$, uniform $\mu$
\item A uniform applied magnetic field of the form $B^0 = (B_x^0,B_y^0,B_z^0) = (e^{-t},e^{-t},1)$
\item We considered the domain enclosure of size $-10.1 < x < 10.1 \qquad -1.1 < y < 1.1 \qquad -1.1 < z < 1.1$
\item The number of cells in the fluid and walls were $N_{fluid} = (75,45,45)$ \qquad $N_{wall} = (11,11,11)$
\end{itemize}

\begin{figure}[H]
 \centering
  \includegraphics[width=\figW,height=\figH]{geometry.eps}
   \caption[Optional ]{Geometry}
\end{figure}

For this problem, we solved the momentum and induction equations. For boundary conditions used were the following

\begin{equation}
	\pmb{u},
	\pmb{B},
	p = \text{periodic}
	\qquad
	x_{min},x_{max}
	\qquad \qquad
\end{equation}

\begin{equation}
	\pmb{u} = \pmb{u}_{walls} = 0
	\qquad
	\frac{\partial p}{\partial n} = 0
	\qquad
	\frac{\partial B_{n}}{\partial n} = 0
	\qquad
	B_{tangent} = 0
	\qquad
	y_{min},y_{max},
	z_{min},z_{max}
\end{equation}

Using the $\frac{\partial ()}{\partial x} = 0$, the momentum equations become

\begin{equation}
	\frac{\partial u}{\partial t} 
	+ \frac{\partial (u v)}{\partial y}
	+ \frac{\partial (u w)}{\partial z}
	= 
	- \frac{\partial p}{\partial x}
	+ \frac{1}{Re}
	\left(
	\frac{\partial^2 u}{\partial y^2}
	+\frac{\partial^2 u}{\partial z^2}
	\right)
	+ \frac{Ha^2}{Re}
	(j \times B)_x
\end{equation}
\begin{equation}
	\frac{\partial v}{\partial t} 
	+ \frac{\partial (v v)}{\partial y}
	+ \frac{\partial (v w)}{\partial z}
	= 
	- \frac{\partial p}{\partial y}
	+ \frac{1}{Re}
	\left(
	\frac{\partial^2 v}{\partial y^2}
	+\frac{\partial^2 v}{\partial z^2}
	\right)
	+ \frac{Ha^2}{Re}
	(j \times B)_y
\end{equation}
\begin{equation}
	\frac{\partial w}{\partial t} 
	+ \frac{\partial (w v)}{\partial y}
	+ \frac{\partial (w w)}{\partial z}
	= 
	- \frac{\partial p}{\partial z}
	+ \frac{1}{Re}
	\left(
	\frac{\partial^2 w}{\partial y^2}
	+\frac{\partial^2 w}{\partial z^2}
	\right)
	+ \frac{Ha^2}{Re}
	(j \times B)_z
\end{equation}

Using the uniform $\sigma$, uniform $\mu$ the induction equations become

\begin{equation}
	\frac{\partial B_i}{\partial t} 
	=
	- \frac{\partial B_i^0}{\partial t} 
	- \frac{\partial}{\partial x_j} (u_j B_i^0 - u_i B_j^0) 
	- \frac{\partial}{\partial x_j} (u_j B_i - u_i B_j) 
	+
	\frac{1}{Re_m}
	\frac{\partial^2 B_i}{\partial x_j^2}
\end{equation}

Using the $\frac{\partial ()}{\partial x} = 0$, we have

\begin{equation}
	\frac{\partial B_x}{\partial t} 
	=
	- \frac{\partial B_x^0}{\partial t} 
	- \frac{\partial}{\partial y} (v B_x - u B_y)
	- \frac{\partial}{\partial z} (w B_x - u B_z)
	- \frac{\partial}{\partial y} (v B_x^0 - u B_y^0)
	- \frac{\partial}{\partial z} (w B_x^0 - u B_z^0)	
	+
	\frac{1}{Re_m}
	\left(
	\frac{\partial^2 B_x}{\partial y^2}
	+ \frac{\partial^2 B_x}{\partial z^2}
	\right)
\end{equation}
\begin{equation}
	\frac{\partial B_y}{\partial t} 
	=
	- \frac{\partial B_y^0}{\partial t} 
	- \frac{\partial}{\partial z} (w B_y - v B_z)
	- \frac{\partial}{\partial z} (w B_y^0 - v B_z^0)	
	+
	\frac{1}{Re_m}
	\left(
	\frac{\partial^2 B_y}{\partial y^2}
	+ \frac{\partial^2 B_y}{\partial z^2}
	\right)
\end{equation}
\begin{equation}
	\frac{\partial B_z}{\partial t} 
	=
	- \frac{\partial B_z^0}{\partial t} 
	- \frac{\partial}{\partial y} (v B_z - w B_y)
	- \frac{\partial}{\partial y} (v B_z^0 - w B_y^0)
	+
	\frac{1}{Re_m}
	\left(
	\frac{\partial^2 B_z}{\partial y^2}
	+ \frac{\partial^2 B_z}{\partial z^2}
	\right)
\end{equation}

Using the applied magnetic field, we have

\begin{equation}
	\frac{\partial B_x}{\partial t} 
	=
	e^{-t}
	- \frac{\partial}{\partial y} (v B_x - u B_y)
	- \frac{\partial}{\partial z} (w B_x - u B_z)
	+ e^{-t} \frac{\partial}{\partial y} (v - u)
	- \frac{\partial}{\partial z} (-w e^{-t} - u)	
	+
	\frac{1}{Re_m}
	\left(
	\frac{\partial^2 B_x}{\partial y^2}
	+ \frac{\partial^2 B_x}{\partial z^2}
	\right)
\end{equation}
\begin{equation}
	\frac{\partial B_y}{\partial t} 
	=
	e^{-t}
	- \frac{\partial}{\partial z} (w B_y - v B_z)
	+ \frac{\partial}{\partial z} (w e^{-t} + v)	
	+
	\frac{1}{Re_m}
	\left(
	\frac{\partial^2 B_y}{\partial y^2}
	+ \frac{\partial^2 B_y}{\partial z^2}
	\right)
\end{equation}
\begin{equation}
	\frac{\partial B_z}{\partial t} 
	=
	- \frac{\partial}{\partial y} (v B_z - w B_y)
	- \frac{\partial}{\partial y} (v + w e^{-t})
	+
	\frac{1}{Re_m}
	\left(
	\frac{\partial^2 B_z}{\partial y^2}
	+ \frac{\partial^2 B_z}{\partial z^2}
	\right)
\end{equation}

\section{Energy}
The plots have been organized using colors=(black,red,blue,green) $\rightarrow$ Re=(100,500,1000,10000) and different symbols have been used for difference magnetic Reynolds numbers.

Not all simulations reached a maximum energy. In fact, many simulations seem like motion and induced magnetic field have just begun to develop. Particularly, those with low $Ha$ number (10).

\subsection{Kinetic}
\begin{figure}[H]
 \centering
  \subfigure[standard scale] {\includegraphics[width=\figW,height=\figH]{KU/KU.eps}}
  \subfigure[semi-log y scale] {\includegraphics[width=\figW,height=\figH]{KU/KU_log.eps}}
   \caption[Optional ]{Kinetic Energy vs Time Step}
\end{figure}

\subsection{Magnetic}
\begin{figure}[H]
 \centering 
  \subfigure[standard scale] {\includegraphics[width=\figW,height=\figH]{KB/KB.eps}}
  \subfigure[semi-log y scale] {\includegraphics[width=\figW,height=\figH]{KB/KB_log.eps}}
   \caption[Optional ]{Magnetic Energy vs Time Step}
\end{figure}


\section{Given Ha=10}
\subsection{Given Rm=1}

\subsubsection{Re = 100}

\begin{figure}[H]
 \centering
  \subfigure[Velocity Field] {\includegraphics[width=\figW,height=\figH]{Ha10Rem1Re100/U.eps}}
  \subfigure[Current Field] {\includegraphics[width=\figW,height=\figH]{Ha10Rem1Re100/J.eps}}
   \caption[Optional ]{}
\end{figure}

\iftoggle{plotUstream}{
\begin{figure}[H]
 \centering \subfigure[]
   {\includegraphics[width=\figW,height=\figH]{Ha10Rem1Re100/Ustream.png}}
   \caption[Optional ]{Velocity stream traces}
\end{figure}
}

\subsubsection{Re = 500}

\begin{figure}[H]
 \centering
  \subfigure[Velocity Field] {\includegraphics[width=\figW,height=\figH]{Ha10Rem1Re500/U.eps}}
  \subfigure[Current Field] {\includegraphics[width=\figW,height=\figH]{Ha10Rem1Re500/J.eps}}
   \caption[Optional ]{}
\end{figure}

\iftoggle{plotUstream}{
\begin{figure}[H]
 \centering \subfigure[]
   {\includegraphics[width=\figW,height=\figH]{Ha10Rem1Re500/Ustream.png}}
   \caption[Optional ]{Velocity stream traces}
\end{figure}
}

\subsubsection{Re = 1000}

\begin{figure}[H]
 \centering
  \subfigure[Velocity Field] {\includegraphics[width=\figW,height=\figH]{Ha10Rem1Re1000/U.eps}}
  \subfigure[Current Field] {\includegraphics[width=\figW,height=\figH]{Ha10Rem1Re1000/J.eps}}
   \caption[Optional ]{}
\end{figure}

\iftoggle{plotUstream}{
\begin{figure}[H]
 \centering \subfigure[]
   {\includegraphics[width=\figW,height=\figH]{Ha10Rem1Re1000/Ustream.png}}
   \caption[Optional ]{Velocity stream traces}
\end{figure}
}

\subsubsection{Re = 10000}

\begin{figure}[H]
 \centering
  \subfigure[Velocity Field] {\includegraphics[width=\figW,height=\figH]{Ha10Rem1Re10000/U.eps}}
  \subfigure[Current Field] {\includegraphics[width=\figW,height=\figH]{Ha10Rem1Re10000/J.eps}}
   \caption[Optional ]{}
\end{figure}

\iftoggle{plotUstream}{
\begin{figure}[H]
 \centering \subfigure[]
   {\includegraphics[width=\figW,height=\figH]{Ha10Rem1Re10000/Ustream.png}}
   \caption[Optional ]{Velocity stream traces}
\end{figure}
}

\subsection{Given Rm=10}
\subsubsection{Re = 100}

\begin{figure}[H]
 \centering
  \subfigure[Velocity Field] {\includegraphics[width=\figW,height=\figH]{Ha10Rem10Re100/U.eps}}
  \subfigure[Current Field] {\includegraphics[width=\figW,height=\figH]{Ha10Rem10Re100/J.eps}}
   \caption[Optional ]{}
\end{figure}

\iftoggle{plotUstream}{
\begin{figure}[H]
 \centering \subfigure[]
   {\includegraphics[width=\figW,height=\figH]{Ha10Rem10Re100/Ustream.png}}
   \caption[Optional ]{Velocity stream traces}
\end{figure}
}

\subsubsection{Re = 500}

\begin{figure}[H]
 \centering
  \subfigure[Velocity Field] {\includegraphics[width=\figW,height=\figH]{Ha10Rem10Re500/U.eps}}
  \subfigure[Current Field] {\includegraphics[width=\figW,height=\figH]{Ha10Rem10Re500/J.eps}}
   \caption[Optional ]{}
\end{figure}

\iftoggle{plotUstream}{
\begin{figure}[H]
 \centering \subfigure[]
   {\includegraphics[width=\figW,height=\figH]{Ha10Rem10Re500/Ustream.eps}}
   \caption[Optional ]{Velocity stream traces}
\end{figure}
}

\subsubsection{Re = 1000}

\begin{figure}[H]
 \centering
  \subfigure[Velocity Field] {\includegraphics[width=\figW,height=\figH]{Ha10Rem10Re1000/U.eps}}
  \subfigure[Current Field] {\includegraphics[width=\figW,height=\figH]{Ha10Rem10Re1000/J.eps}}
   \caption[Optional ]{}
\end{figure}

\iftoggle{plotUstream}{
\begin{figure}[H]
 \centering \subfigure[]
   {\includegraphics[width=\figW,height=\figH]{Ha10Rem10Re1000/Ustream.png}}
   \caption[Optional ]{Velocity stream traces}
\end{figure}
}

\subsubsection{Re = 10000}

\begin{figure}[H]
 \centering
  \subfigure[Velocity Field] {\includegraphics[width=\figW,height=\figH]{Ha10Rem10Re10000/U.eps}}
  \subfigure[Current Field] {\includegraphics[width=\figW,height=\figH]{Ha10Rem10Re10000/J.eps}}
   \caption[Optional ]{}
\end{figure}

\iftoggle{plotUstream}{
\begin{figure}[H]
 \centering \subfigure[]
   {\includegraphics[width=\figW,height=\figH]{Ha10Rem10Re10000/Ustream.png}}
   \caption[Optional ]{Velocity stream traces}
\end{figure}
}

\subsection{Given Rm=100}
\subsubsection{Re = 100}

\begin{figure}[H]
 \centering
  \subfigure[Velocity Field] {\includegraphics[width=\figW,height=\figH]{Ha10Rem100Re100/U.eps}}
  \subfigure[Current Field] {\includegraphics[width=\figW,height=\figH]{Ha10Rem100Re100/J.eps}}
   \caption[Optional ]{}
\end{figure}

\iftoggle{plotUstream}{
\begin{figure}[H]
 \centering \subfigure[]
   {\includegraphics[width=\figW,height=\figH]{Ha10Rem100Re100/Ustream.png}}
   \caption[Optional ]{Velocity stream traces}
\end{figure}
}

\subsubsection{Re = 500}
\begin{figure}[H]
 \centering
  \subfigure[Velocity Field] {\includegraphics[width=\figW,height=\figH]{Ha10Rem100Re500/U.eps}}
  \subfigure[Current Field] {\includegraphics[width=\figW,height=\figH]{Ha10Rem100Re500/J.eps}}
   \caption[Optional ]{}
\end{figure}

\subsubsection{Re = 1000}
\begin{figure}[H]
 \centering
  \subfigure[Velocity Field] {\includegraphics[width=\figW,height=\figH]{Ha10Rem100Re1000/U.eps}}
  \subfigure[Current Field] {\includegraphics[width=\figW,height=\figH]{Ha10Rem100Re1000/J.eps}}
   \caption[Optional ]{}
\end{figure}


\subsubsection{Re = 10000}
\begin{figure}[H]
 \centering
  \subfigure[Velocity Field] {\includegraphics[width=\figW,height=\figH]{Ha10Rem100Re10000/U.eps}}
  \subfigure[Current Field] {\includegraphics[width=\figW,height=\figH]{Ha10Rem100Re10000/J.eps}}
   \caption[Optional ]{}
\end{figure}

\section{Given Ha=100}

\subsection{Given Rm=1}
\subsubsection{Re = 100}

\begin{figure}[H]
 \centering
  \subfigure[Velocity Field] {\includegraphics[width=\figW,height=\figH]{Ha100Rem1Re100/U.eps}}
  \subfigure[Current Field] {\includegraphics[width=\figW,height=\figH]{Ha100Rem1Re100/J.eps}}
   \caption[Optional ]{}
\end{figure}

\iftoggle{plotUstream}{
\begin{figure}[H]
 \centering \subfigure[]
   {\includegraphics[width=\figW,height=\figH]{Ha100Rem1Re100/Ustream.png}}
   \caption[Optional ]{Velocity stream traces}
\end{figure}
}

\subsubsection{Re = 500}
\begin{figure}[H]
 \centering
  \subfigure[Velocity Field] {\includegraphics[width=\figW,height=\figH]{Ha100Rem1Re500/U.eps}}
  \subfigure[Current Field] {\includegraphics[width=\figW,height=\figH]{Ha100Rem1Re500/J.eps}}
   \caption[Optional ]{}
\end{figure}

\subsubsection{Re = 1000}
\begin{figure}[H]
 \centering
  \subfigure[Velocity Field] {\includegraphics[width=\figW,height=\figH]{Ha100Rem1Re1000/U.eps}}
  \subfigure[Current Field] {\includegraphics[width=\figW,height=\figH]{Ha100Rem1Re1000/J.eps}}
   \caption[Optional ]{}
\end{figure}

\subsubsection{Re = 10000}
\begin{figure}[H]
 \centering
  \subfigure[Velocity Field] {\includegraphics[width=\figW,height=\figH]{Ha100Rem1Re10000/U.eps}}
  \subfigure[Current Field] {\includegraphics[width=\figW,height=\figH]{Ha100Rem1Re10000/J.eps}}
   \caption[Optional ]{}
\end{figure}

\subsection{Given Rm=10}
\subsubsection{Re = 100}
\begin{figure}[H]
 \centering
  \subfigure[Velocity Field] {\includegraphics[width=\figW,height=\figH]{Ha100Rem10Re100/U.eps}}
  \subfigure[Current Field] {\includegraphics[width=\figW,height=\figH]{Ha100Rem10Re100/J.eps}}
   \caption[Optional ]{}
\end{figure}

\subsubsection{Re = 500}
\begin{figure}[H]
 \centering
  \subfigure[Velocity Field] {\includegraphics[width=\figW,height=\figH]{Ha100Rem10Re500/U.eps}}
  \subfigure[Current Field] {\includegraphics[width=\figW,height=\figH]{Ha100Rem10Re500/J.eps}}
   \caption[Optional ]{}
\end{figure}

\subsubsection{Re = 1000}
\begin{figure}[H]
 \centering
  \subfigure[Velocity Field] {\includegraphics[width=\figW,height=\figH]{Ha100Rem10Re1000/U.eps}}
  \subfigure[Current Field] {\includegraphics[width=\figW,height=\figH]{Ha100Rem10Re1000/J.eps}}
   \caption[Optional ]{}
\end{figure}

\subsubsection{Re = 10000}
\begin{figure}[H]
 \centering
  \subfigure[Velocity Field] {\includegraphics[width=\figW,height=\figH]{Ha100Rem10Re10000/U.eps}}
  \subfigure[Current Field] {\includegraphics[width=\figW,height=\figH]{Ha100Rem10Re10000/J.eps}}
   \caption[Optional ]{}
\end{figure}


\subsection{Given Rm=100}
\subsubsection{Given Re=100 **Problems start**}
Until now, all flows have been observed to be 2-dimensional in the sense that there are no changes with respect to the x-direction. In addition, most of the solutions have seemed smooth. For the Re=100 case, the solution shows what look like numerical instabilities, and it seems that the mesh must be refined. The rest of the cases will be carried out once I find a mesh that can suppress these instabilities.

\begin{figure}[H]
 \centering
  \includegraphics[width=\BfigW,height=\BfigH]{Ha100Rem100Re100/U.eps}
   \caption[Optional ]{Velocity Field}
\end{figure}

\begin{figure}[H]
 \centering
  \includegraphics[width=\BfigW,height=\BfigH]{Ha100Rem100Re100/V.eps}
   \caption[Optional ]{V-component of Velocity Field}
\end{figure}



\bibliographystyle{unsrt}
\bibliography{MHD,Math,Interface,fluids,CFD,handpicked}


\end{document}