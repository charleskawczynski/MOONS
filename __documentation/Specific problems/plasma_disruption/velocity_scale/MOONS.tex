\documentclass[11pt]{article}
% \documentclass[3p,twocolumn,10pt]{elsarticle}
\usepackage{graphicx}    % needed for including graphics e.g. EPS, PS
\usepackage{epstopdf}
\usepackage{amsmath}
\usepackage{amssymb}
\usepackage{hyperref}
\usepackage{xspace}
\usepackage{mathtools}
\usepackage{tikz}
\usepackage{epsfig}
\usepackage{float}
%\usepackage{natbib}
\usepackage{subfigure}
\usepackage{setspace}
\usepackage{tabularx,ragged2e,booktabs,caption}

\usepackage{xcolor}
\usepackage{xspace}
\usepackage{longtable}
\usepackage{mathtools}
\usepackage{natbib}
\usepackage{setspace}
\usepackage{ragged2e}
\usepackage{etoolbox}
\usepackage{geometry}

  \newcommand{\B}{\mathbf{B}}
  \renewcommand{\C}{\mathbf{C}}
  \renewcommand{\U}{\mathbf{u}}
  \newcommand{\E}{\mathbf{E}}
  \newcommand{\J}{\mathbf{j}}
  \newcommand{\A}{\mathbf{A}}
  \newcommand{\N}{\mathbf{n}}
  \renewcommand{\G}{\mathbf{g}}
  \newcommand{\F}{\mathbf{f}}
  \newcommand{\X}{\mathbf{x}}
  \newcommand{\Y}{\mathbf{y}}
\renewcommand{\H}{\mathbf{H}}

\newcommand{\DOT}{\text{\textbullet}}
\newcommand{\CROSS}{\times}
\newcommand{\DEL}{\nabla}
\newcommand{\CURL}{\DEL \CROSS}
\newcommand{\DIV}{\DEL \DOT}
\newcommand{\PD}{\partial}
\newcommand{\MAC}{\mathcal}

\newcommand{\MC}{\multicolumn}
\newcommand{\MR}{\multirow}

\newcommand{\PH}{physical}
\newcommand{\M}{{\mu_m}}
\newcommand{\SI}{\sigma}
\newcommand{\SII}{\SI^{-1}}
\newcommand{\JS}{\frac{\J}{\sigma}}
\newcommand{\MO}{\overline{\M}}
\newcommand{\SO}{\overline{\sigma}}
\newcommand{\RA}{\rightarrow}

\newcommand{\JSS}{\frac{\J^2}{\sigma}}

\newcommand{\qquadmany}{\qquad\qquad\qquad\qquad\qquad\qquad}




\newcommand{\volume}{\mathop{\ooalign{\hfil$V$\hfil\cr\kern0.08em--\hfil\cr}}\nolimits}
\newcommand{\figWidth}{0.45\textwidth}
\newcommand{\figSuffix}{_png.png}
\newcommand{\pushright}[1]{\ifmeasuring@#1\else\omit\hfill$\displaystyle#1$\fi\ignorespaces}
\newcommand{\hfillMath}{\hskip \textwidth minus \textwidth}

\setlength{\oddsidemargin}{0.1in}
\setlength{\textwidth}{7.25in}

\setlength{\topmargin}{-1in}     %\topmargin: gap above header
\setlength{\headheight}{0in}     %\headheight: height of header
\setlength{\topskip}{0in}        %\topskip: between header and text
\setlength{\headsep}{0in}        
\setlength{\textheight}{692pt}   %\textheight: height of main text
\setlength{\textwidth}{7.5in}    % \textwidth: width of text
\setlength{\oddsidemargin}{-0.5in}  % \oddsidemargin: odd page left margin
\setlength{\evensidemargin}{0in} %\evensidemargin : even page left margin
\setlength{\parindent}{0.25in}   %\parindent: indentation of paragraphs
\setlength{\parskip}{0pt}        %\parskip: gap between paragraphs
\setlength{\voffset}{0.5in}

% Useful commands:

% \hfill    aligns-right everything right of \hfill

\begin{document}
\doublespacing
\title{Plasma Disruption Scales}
\author{C. Kawczynski \\
Department of Mechanical and Aerospace Engineering \\
University of California Los Angeles, USA\\
}
\maketitle

\section{Velocity}
We best know the time rate change of $\B^0$. 
\begin{equation}
	\frac{B^0}{\tau} \sim \frac{J}{\sigma L}
\end{equation}
This will induce $\J$ and interact with $\B^0$ resulting in an imbalance in the momentum equation:
\begin{equation}
	\frac{U}{t} \sim \rho^{-1} J B
\end{equation}
The time to create an eddy of length $L$ we'll call the eddy turnover time $t=L/U$
\begin{equation}
	\frac{U^2}{L} \sim \rho^{-1} \left(\frac{B^0}{\tau} \sigma L \right) B
\end{equation}
Resulting in
\begin{equation}
	U \sim {B^0} B L \sqrt{\frac{\sigma}{\rho \tau}}
\end{equation}
In terms of PD for ITER,
\begin{equation}\begin{aligned}
	\B^0 = 5 \text{ [T]} \\
	\tau = 0.045 \text{ [s]} \\
	\rho = 450 \text{ [kg/m$^3$]} \\
	\sigma = 250 \text{ [$\Omega$ m]} \\
	\mu_m = 4 \pi \times 10^{-7} \text{ [H m$^{-1}$]} \\
	L = 10 \text{ [m]} \\
\end{aligned}\end{equation}
These values result in
\begin{equation}\begin{aligned}
	U = B^0 L \sqrt{\frac{\sigma}{\rho \tau}} = \text{ [m/s]} \\
	U = (5 T) (10 m) \sqrt{\frac{(250 \Omega m)}{(450 kg/m^3) (0.045 s)}} = \text{ [m/s]} \\
	Re_m = U L \mu_m \sigma =  \\
	Re_m = U (10 m) (4 \pi \times 10^{-7} H/m) (250 \Omega m) =  \\
\end{aligned}\end{equation}

\end{document}