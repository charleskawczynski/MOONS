\documentclass[11pt]{article}
% \documentclass[11pt,landscape]{article}
\newcommand{\PSCHAIN}{..}
\edef\PSCHAIN{\PSCHAIN/LATEX_INCLUDES}

\newcommand{\rootdir}{\PSCHAIN}
\newcommand{\VAR}{Success}



% Numbering from: https://www.sharelatex.com/learn-scripts/images/f/fc/Layout-dimensions.png
\setlength{\hoffset}{-.5in}       % 1
\setlength{\voffset}{0in}        % 2
\setlength{\oddsidemargin}{0in}  % 3
\setlength{\topmargin}{-.5in}     % 4
\setlength{\headheight}{0in}     % 5
\setlength{\headsep}{0in}        % 6
\setlength{\textheight}{700pt}   % 7
\setlength{\textwidth}{550pt}    % 8
\setlength{\marginparsep}{0in}   % 9
% \setlength{\marginparwidth}{0in} % 10
\setlength{\footskip}{0in}       % 11 a (see fig)
% \setlength{\marginparpush}{0in}  % 11 b (see fig)
% \setlength{\paperwidth}{0in}     % 11 e (see fig)
% \setlength{\paperheight}{390pt}  % 11 f (see fig)
 % For local use

\begin{document}
\doublespacing
\MOONSTITLE
\maketitle

\section{Decoupling of plasma-disruption governing equations}
In this document, decouple the plasma disruption governing equations
\begin{equation}\begin{aligned}
\PD_t u                 = -\frac{1}{\rho}\PD_x p+ \nu u'' + \frac{1}{\rho \M} B_z^0 B_x' \\
\PD_t B_x + \PD_t B_x^0 = \frac{1}{\sigma \M} B_x'' + B_z^0 u' \\
\end{aligned} \end{equation}
Let
\begin{equation}\begin{aligned}
G = \frac{1}{\rho}\PD_x p, \qquad
H = \nu, \qquad
A = \frac{1}{\rho \M} B_z^0, \qquad
C = \PD_t B_x^0, \qquad
K = \frac{1}{\sigma \M}, \qquad
D = B_z^0, \qquad
\end{aligned} \end{equation}
Then we have
\begin{equation}\begin{aligned}
\PD_t u                 = -G+ H u'' + A B_x' \\
\PD_t B_x + C = K B_x'' + D u' \\
\end{aligned} \end{equation}

\section{Variable seperation transformation}
We may apply the transformation
\begin{equation}\begin{aligned}
X = u + E B_x \\
Y = u - E B_x \\
\end{aligned} \end{equation}

\subsection{Transformation for X}
We have
\begin{equation}\begin{aligned}
\PD_t (X - E B_x)        = -G+ H (X'' - E B_x'') + A B_x' \\
\PD_t B_x + C = K B_x'' + D (X' - E B_x') \\
\end{aligned} \end{equation}
Distributing
\begin{equation}\begin{aligned}
\PD_t X - E \PD_t B_x = -G+ H X'' - H E B_x'' + A B_x' \\
\PD_t B_x + C = K B_x'' + D X' - D E B_x' \\
\end{aligned} \end{equation}
Combining with factor $F$ for second equation
\begin{equation}\begin{aligned}
\PD_t X - E \PD_t B_x + F \PD_t B_x + FC = -G+ H X'' - H E B_x'' + A B_x' + F K B_x'' + DF X' - D EF B_x'\\
\end{aligned} \end{equation}
To ensure $B_x$ dissappears, we must enforce
\begin{equation}\begin{aligned}
F = E, \qquad
FK = HE, \qquad
FDE = A, \qquad \\
FK = HF, \qquad
FDF = A, \qquad \\
K = H, \qquad
F = \sqrt{\frac{A}{D}}, \qquad \\
\end{aligned} \end{equation}
The $K=H$ requirement implies that $\nu = \nu_m$. Put differently, magnetic Prandtle ($Pr_m$) must equal unity ($Pr_m = 1$). So, we have
\begin{equation}\begin{aligned}
\PD_t X + FC = -G+ H X'' + DF X'\\
\end{aligned} \end{equation}
\subsection{Transformation for Y}
Reminder
\begin{equation}\begin{aligned}
\PD_t u                 = -G+ H u'' + A B_x' \\
\PD_t B_x + C = K B_x'' + D u' \\
\end{aligned} \end{equation}
We have
\begin{equation}\begin{aligned}
\PD_t (Y + E B_x)        = -G+ H (Y'' + E B_x'') + A B_x'' \\
\PD_t B_x + C = K B_x'' + D (Y' + E B_x') \\
\end{aligned} \end{equation}
Distributing
\begin{equation}\begin{aligned}
\PD_t Y + E \PD_t B_x = -G+ H Y'' + H E B_x'' + A B_x' \\
\PD_t B_x + C = K B_x'' + D Y' + D E B_x' \\
\end{aligned} \end{equation}
Combining with factor $-F$ for second equation
\begin{equation}\begin{aligned}
\PD_t Y + E \PD_t B_x - F \PD_t B_x - F C = -G+ H Y'' + H E B_x'' + A B_x' - F K B_x'' - D F Y' - D E F B_x'\\
\end{aligned} \end{equation}
To ensure $B_x$ dissappears, we must enforce
\begin{equation}\begin{aligned}
F = E, \qquad
FK = HE, \qquad
FDE = A, \qquad \\
FK = HF, \qquad
FDF = A, \qquad \\
K = H, \qquad
F = \sqrt{\frac{A}{D}}, \qquad \\
\end{aligned} \end{equation}
Again, the $K=H$ requirement implies that $\nu = \nu_m$. Put differently, magnetic Prandtle ($Pr_m$) must equal unity ($Pr_m = 1$). So, we have
\begin{equation}\begin{aligned}
\PD_t Y - FC = -G+ H Y'' - DF Y'\\
\end{aligned} \end{equation}

\section{Transformation summary}
Our decoupled equations are
\begin{equation}\begin{aligned}
\PD_t X + EC = -G+ H X'' + DE X'\\
\PD_t Y - EC = -G+ H Y'' - DE Y'\\
\end{aligned} \end{equation}
Or
\begin{equation}\begin{aligned}
\PD_t X - H X'' - DE X'  = (- EC -G)\\
\PD_t Y - H Y'' + DE Y'  = (+ EC -G)\\
\end{aligned} \end{equation}

With transformations
\begin{equation}\begin{aligned}
X = u + E B_x \\
Y = u - E B_x \\
E = \sqrt{\frac{A}{D}} \\
\end{aligned} \end{equation}
And inverse transformations
\begin{equation}\begin{aligned}
u   = \frac{1}{2} \left( X + Y \right) \\
B_x = \frac{1}{2E} \left( X - Y \right) \\
E = \sqrt{\frac{A}{D}} \\
\end{aligned} \end{equation}
\begin{equation}\begin{aligned}
G = \frac{1}{\rho}\PD_x p, \qquad
H = K = \nu = \frac{1}{\sigma \M}, \qquad
A = \frac{1}{\rho \M} B_z^0, \qquad
C = \PD_t B_x^0, \qquad
D = B_z^0.
\end{aligned} \end{equation}


\end{document}