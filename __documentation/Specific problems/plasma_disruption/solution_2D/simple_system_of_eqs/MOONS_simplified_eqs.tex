\documentclass[11pt]{article}
% \documentclass[3p,twocolumn,10pt]{elsarticle}
\usepackage{graphicx}    % needed for including graphics e.g. EPS, PS
\usepackage{epstopdf}
\usepackage{amsmath}
\usepackage{amssymb}
\usepackage{hyperref}
\usepackage{xspace}
\usepackage{mathtools}
\usepackage{tikz}
\usepackage{epsfig}
\usepackage{float}
%\usepackage{natbib}
\usepackage{subfigure}
\usepackage{setspace}
\usepackage{tabularx,ragged2e,booktabs,caption}

\usepackage{xcolor}
\usepackage{xspace}
\usepackage{longtable}
\usepackage{mathtools}
\usepackage{natbib}
\usepackage{setspace}
\usepackage{ragged2e}
\usepackage{etoolbox}
\usepackage{geometry}

  \newcommand{\B}{\mathbf{B}}
  \renewcommand{\C}{\mathbf{C}}
  \renewcommand{\U}{\mathbf{u}}
  \newcommand{\E}{\mathbf{E}}
  \newcommand{\J}{\mathbf{j}}
  \newcommand{\A}{\mathbf{A}}
  \newcommand{\N}{\mathbf{n}}
  \renewcommand{\G}{\mathbf{g}}
  \newcommand{\F}{\mathbf{f}}
  \newcommand{\X}{\mathbf{x}}
  \newcommand{\Y}{\mathbf{y}}
\renewcommand{\H}{\mathbf{H}}

\newcommand{\DOT}{\text{\textbullet}}
\newcommand{\CROSS}{\times}
\newcommand{\DEL}{\nabla}
\newcommand{\CURL}{\DEL \CROSS}
\newcommand{\DIV}{\DEL \DOT}
\newcommand{\PD}{\partial}
\newcommand{\MAC}{\mathcal}

\newcommand{\PH}{physical}
\newcommand{\M}{{\mu_m}}
\newcommand{\SI}{\sigma}
\newcommand{\SII}{\SI^{-1}}
\newcommand{\JS}{\frac{\J}{\sigma}}
\newcommand{\MO}{\overline{\M}}
\newcommand{\SO}{\overline{\sigma}}
\newcommand{\MC}{\mathcal}

\newcommand{\JSS}{\frac{\J^2}{\sigma}}

\newcommand{\qquadmany}{\qquad\qquad\qquad\qquad\qquad\qquad}




\newcommand{\volume}{\mathop{\ooalign{\hfil$V$\hfil\cr\kern0.08em--\hfil\cr}}\nolimits}
\newcommand{\figWidth}{0.45\textwidth}
\newcommand{\figSuffix}{_png.png}
\newcommand{\pushright}[1]{\ifmeasuring@#1\else\omit\hfill$\displaystyle#1$\fi\ignorespaces}
\newcommand{\hfillMath}{\hskip \textwidth minus \textwidth}

\setlength{\oddsidemargin}{0.1in}
\setlength{\textwidth}{7.25in}

\setlength{\topmargin}{-1in}     %\topmargin: gap above header
\setlength{\headheight}{0in}     %\headheight: height of header
\setlength{\topskip}{0in}        %\topskip: between header and text
\setlength{\headsep}{0in}
\setlength{\textheight}{692pt}   %\textheight: height of main text
\setlength{\textwidth}{8in}    % \textwidth: width of text
\setlength{\oddsidemargin}{-0.7in}  % \oddsidemargin: odd page left margin
\setlength{\evensidemargin}{0in} %\evensidemargin : even page left margin
\setlength{\parindent}{0.25in}   %\parindent: indentation of paragraphs
\setlength{\parskip}{0pt}        %\parskip: gap between paragraphs
\setlength{\voffset}{0.5in}

% Useful commands:

% \hfill    aligns-right everything right of \hfill

\begin{document}
\doublespacing
\title{Plasma Disruption 2D solution}
\author{C. Kawczynski \\
Department of Mechanical and Aerospace Engineering \\
University of California Los Angeles, USA\\
}
% \maketitle

\section{Solution of simplified system of equations}
Both sets of equations are in the form
\begin{equation}\begin{aligned}
A_1 E_1 + A_2 E_2 = H \\
A_1 F_1 + A_2 F_2 = H \\
A_2 E_2 = H - A_1 E_1 \\
A_2 = \frac{1}{E_2} (H - A_1 E_1) \\
A_1 F_1 + \frac{F_2}{E_2} (H - A_1 E_1) = H \\
A_1 F_1 + \frac{F_2}{E_2} H - \frac{F_2}{E_2} A_1 E_1 = H \\
A_1 \left( F_1 - \frac{F_2}{E_2} E_1 \right) = H -\frac{F_2}{E_2} H \\
A_1 \left( F_1 - \frac{F_2}{E_2} E_1 \right) = H \left( 1 - \frac{F_2}{E_2} \right) \\
A_1 = H \frac{\left( 1 - \frac{F_2}{E_2} \right)}{\left( F_1 - \frac{F_2}{E_2} E_1 \right)} \\
\boxed{A_1 = H \left( \frac{E_2 - F_2}{F_1 E_2 - F_2 E_1} \right) } \\
A_2 = \frac{1}{E_2} \left(H - A_1 E_1 \right) \\
A_2 = \frac{1}{E_2} \left(H - H \frac{( E_2 - F_2 )}{( F_1 E_2 - F_2 E_1)} E_1 \right) \\
A_2 = \frac{H}{E_2} \left(1 - \frac{(E_2 E_1 - F_2 E_1)}{(F_1 E_2 - F_2 E_1)} \right) \\
A_2 = \frac{H}{E_2} \left(\frac{F_1 E_2 - F_2 E_1}{F_1 E_2 - F_2 E_1} - \frac{(E_2 E_1 - F_2 E_1)}{(F_1 E_2 - F_2 E_1)} \right) \\
A_2 = \frac{H}{E_2} \left(\frac{F_1 E_2 - F_2 E_1 - (E_2 E_1 - F_2 E_1)}{F_1 E_2 - F_2 E_1}\right) \\
A_2 = \frac{H}{E_2} \left(\frac{F_1 E_2 - F_2 E_1 - E_2 E_1 + F_2 E_1}{F_1 E_2 - F_2 E_1}\right) \\
A_2 = \frac{H}{E_2} \left(\frac{F_1 E_2 - E_2 E_1}{F_1 E_2 - F_2 E_1}\right) \\
\boxed{A_2 = H \left(\frac{F_1 - E_1}{F_1 E_2 - F_2 E_1}\right)} \\
\end{aligned} \end{equation}

\begin{equation}\begin{aligned}
A_1 =& \frac{H}{\theta} (E_2 - F_2) \\
A_2 =& \frac{H}{\theta} (F_1 - E_1), \qquad \theta = F_1 E_2 - F_2 E_1 \\
\end{aligned} \end{equation}


\end{document}