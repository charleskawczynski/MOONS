\documentclass[11pt]{article}
\newcommand{\PSCHAIN}{..}
\edef\PSCHAIN{\PSCHAIN/LATEX_INCLUDES}

\newcommand{\rootdir}{\PSCHAIN}
\newcommand{\VAR}{Success}



\begin{document}
% \doublespacing
\MOONSTITLE
\maketitle
\section{Deriving the separable PDE}
The derived governing equation is
\begin{equation}\begin{aligned}
\PD_t B + (\PD_t B^0(t)) = R (\PD_{xx} B + \PD_{yy} B), \qquad B(x,y,t),\theta(t),R=\text{constant} \\
\end{aligned} \end{equation}
Assuming a superposition of homogeneous and non-homogeneous terms
\begin{equation}\begin{aligned}
B = B_h(x,y,t) + B_n(t) \\
\end{aligned} \end{equation}
Which satisfy the governing equations as follows
\begin{equation}\begin{aligned}
\PD_t B_h = R (\PD_{xx} B_h + \PD_{yy} B_h) \\
\PD_t B_n + \theta = 0 \\
\end{aligned} \end{equation}
The solution to the non-homogeneous solution is simply
\begin{equation}\begin{aligned}
B_n(t) = - \int_0^t \theta(\tau) d\tau \\
\end{aligned} \end{equation}
Then, assuming the homogeneous term is a solution of the form
\begin{equation}\begin{aligned}
B_h(x,y,t) = Q(t) F(x) G(y) \\
\end{aligned} \end{equation}
We get the following governing eqation
\begin{equation}\begin{aligned}
\dot{Q} FG = R Q( F'' G +  F G'') \\
\end{aligned} \end{equation}
Dividing by $QFG$ yields
\begin{equation}\begin{aligned}
\frac{\dot{Q}}{Q} = R \frac{F''}{F} - R \frac{G''}{G} \\
\end{aligned} \end{equation}
Moving the time-dependent terms to one side reveals that the RHS is independent of time, similarly the LHS is independent of space, and is therefore equal to a constant:
\begin{equation}\begin{aligned}
\frac{\dot{Q}}{Q} = R \frac{F''}{F} + R \frac{G''}{G} = \lambda \\
\end{aligned} \end{equation}
We may seek the solution to these problems separately:

\subsection{Spatial part}
The spatial part of the governing equation is
\begin{equation}\begin{aligned}
 \frac{F''}{F} + \frac{G''}{G} = \frac{1}{R} \lambda \\
\end{aligned} \end{equation}
Moving the x-dependent terms to one side, and y-dependent terms to the other, reveals that the RHS and LHS are independent of one another, therefore they are equal to a constant:
\begin{equation}\begin{aligned}
 \frac{F''}{F} = \frac{1}{R} \lambda - \frac{G''}{G} = - \gamma \\
\end{aligned} \end{equation}
Each of these equations can be solved separately.

\subsubsection{Spatial x-dependence}
The x-dependent equation is
\begin{equation}\begin{aligned}
 F'' = - F \gamma \\
\end{aligned} \end{equation}
Assume $F = Ae^{\alpha x}$, therefore we get
\begin{equation}\begin{aligned}
 \alpha^2 = - \gamma \\
 \alpha = \pm i \sqrt{\gamma} \\
\end{aligned} \end{equation}
Therefore
\begin{equation}\begin{aligned}
 F = A_1 e^{i\sqrt{\gamma} x} + A_2 e^{-i\sqrt{\gamma} x} \\
\end{aligned} \end{equation}
Applying BCs ($F(0)=0,F(L_x)=0$) yields
\begin{equation}\begin{aligned}
 F(0) = A_1 + A_2 = 0, \qquad A_2 = -A_1 \\
 F(L_x) = A_1 \left( e^{i\sqrt{\gamma} L_x} - e^{-i\sqrt{\gamma} L_x} \right) = 0 \\
              \left( e^{2i\sqrt{\gamma} L_x} - 1 \right) = 0 \\
              e^{2i\sqrt{\gamma} L_x} = 1 = e^{i2n\pi} \\
 2i\sqrt{\gamma} L_x = i2n\pi \\
 \gamma_n = \left( \frac{n\pi}{L_x} \right)^2, \qquad n=0,1,2,... \\
\end{aligned} \end{equation}
Therefore, there are an infinite number of constants $\gamma_n$. In summary:
\begin{equation}\begin{aligned}
 F(x) = A \sin(\sqrt{\gamma_n} x) \\
 \gamma_n = \left( \frac{n\pi}{L_x} \right)^2 \\
\end{aligned} \end{equation}

\subsubsection{Spatial y-dependence}
The y-dependent equation is
\begin{equation}\begin{aligned}
 \frac{1}{R} \lambda - \frac{G''}{G} = - \gamma_n \\
\end{aligned} \end{equation}
First, let
\begin{equation}
	\kappa = -\left( \frac{1}{R} \lambda + \gamma_n \right)
\end{equation}
The y-dependent equation becomes
\begin{equation}\begin{aligned}
 G'' = - G \kappa \\
\end{aligned} \end{equation}
Assume $G = C e^{\beta y}$, therefore we get
\begin{equation}\begin{aligned}
 \beta^2 = - \kappa \\
 \beta = \pm i \sqrt{\kappa} \\
\end{aligned} \end{equation}
Therefore
\begin{equation}\begin{aligned}
 G = C_1 e^{i\sqrt{\kappa} y} + C_2 e^{-i\sqrt{\kappa} y} \\
\end{aligned} \end{equation}
Applying BCs ($G(0)=0,G(L_y)=0$)
\begin{equation}\begin{aligned}
 G(0) = C_1 + C_2 = 0, \qquad C_2 = -C_1 \\
 G(L_y) = C_1 \left( e^{i\sqrt{\kappa} L_y} - e^{-i\sqrt{\kappa} L_y} \right) = 0 \\
              \left( e^{2i\sqrt{\kappa} L_y} - 1 \right) = 0 \\
              e^{2i\sqrt{\kappa_{m}} L_y} = 1 = e^{i2m\pi} \\
 2i\sqrt{\kappa_{m}} L_y = i2m\pi \\
 \kappa_{m} = \left( \frac{m\pi}{L_y} \right)^2, \qquad m=0,1,2,... \\
\end{aligned} \end{equation}
Therefore
\begin{equation}\begin{aligned}
	\kappa_{m} = \left( \frac{m\pi}{L_y} \right)^2 = -\left( \frac{1}{R} \lambda + \gamma_n \right) \\
	\lambda_{n,m} = - R \left( \left( \frac{n\pi}{L_x} \right)^2 + \left( \frac{m\pi}{L_y} \right)^2 \right) \\
	\gamma_n = \left( \frac{n\pi}{L_x} \right)^2 \\
\end{aligned} \end{equation}
And our y-distribution follows
\begin{equation}\begin{aligned}
 G(x) = C \sin(\sqrt{\kappa_{m}} y) \\
 \kappa_{m} = \left( \frac{m\pi}{L_y} \right)^2 \\
\end{aligned} \end{equation}
\subsection{Spatial summary}
Our spatial distribution follows
\begin{equation}\begin{aligned}
 F(x) = A \sin(\sqrt{\gamma_n} x) \\
 G(x) = C \sin(\sqrt{\kappa_{m}} y) \\
 \gamma_n = \left( \frac{n\pi}{L_x} \right)^2 \\
 \kappa_{m} = \left( \frac{m\pi}{L_y} \right)^2 \\
 \lambda_{n,m} = - R \left( \gamma_n + \kappa_{m} \right) \\
\end{aligned} \end{equation}
\begin{equation}\begin{aligned}
\frac{\dot{Q}}{Q} = \underbrace{R \frac{F''}{F}}_{-R\gamma_n} + R \frac{G''}{G} = \lambda_{n,m} \\
\end{aligned} \end{equation}

\subsection{Temporal part}
The original temporal equation is
\begin{equation}\begin{aligned}
\dot{Q} = Q \lambda_{n,m} \\
\end{aligned} \end{equation}
Assuming a solution of the form $Q = D_1 e^{\omega t}$, we have
\begin{equation}\begin{aligned}
\omega_{n,m} = \lambda_{n,m} \\
\end{aligned} \end{equation}

\section{General Solution and Initial Conditions}
The original superposition principle used is:
\begin{equation}\begin{aligned}
B = B_h(x,y,t) + B_n(t) \\
\end{aligned} \end{equation}
Which satisfy the governing equations as follows
\begin{equation}\begin{aligned}
\PD_t B_h = R (\PD_{xx} B_h + \PD_{yy} B_h) \\
\PD_t B_n + \theta = 0 \\
\end{aligned} \end{equation}
The general solution is therefore a superposition of all possible solutions of the total magnetic field:
\begin{equation}\begin{aligned}
B(x,y,t) = - \int_0^t \theta(\tau) d\tau +
\sum_{n=0}^{\infty} \sum_{m=0}^{\infty}
D_{n,m} e^{\lambda_{n,m} t} \sin(\sqrt{\gamma_n} x) \sin(\sqrt{\kappa_{m}} y)
 \\
 \gamma_n = \left( \frac{n\pi}{L_x} \right)^2 \\
 \kappa_{m} = \left( \frac{m\pi}{L_y} \right)^2 \\
 \lambda_{n,m} = - R \left( \gamma_n + \kappa_{m} \right) \\
\end{aligned} \end{equation}

\section{Duhamel's Principle}
Applying Duhamel's Principle, we have
\begin{equation}\begin{aligned}
B(x,y,t) = \int_{0}^{L_x} \int_{0}^{L_y} G(x,y,t-s;\xi,\zeta) (-\theta(t)) d\xi \\
\end{aligned} \end{equation}

\end{document}