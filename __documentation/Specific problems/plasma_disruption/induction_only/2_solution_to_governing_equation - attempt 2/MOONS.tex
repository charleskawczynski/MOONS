\documentclass[11pt]{article}
\newcommand{\PSCHAIN}{..}
\edef\PSCHAIN{\PSCHAIN/LATEX_INCLUDES}

\newcommand{\rootdir}{\PSCHAIN}
\newcommand{\VAR}{Success}



\begin{document}
% \doublespacing
\MOONSTITLE
\maketitle
\section{Deriving the separable PDE}
The derived governing equation is
\begin{equation}\begin{aligned}
\PD_t B^1 + \PD_t B^0 = R (\PD_{xx} B^1 + \PD_{yy} B^1), \qquad B^1(x,y,t),B^0(t),R=\text{constant} \\
\end{aligned} \end{equation}
Assuming a superposition of two terms
\begin{equation}\begin{aligned}
B = B^1(x,y,t) + B^0(t) \\
\end{aligned} \end{equation}
Our equation is then
\begin{equation}\begin{aligned}
\PD_t B = R (\PD_{xx} B + \PD_{yy} B), \qquad B(x,y,t),\theta(t),R=\text{constant} \\
\end{aligned} \end{equation}

Which satisfy the governing equations as follows
\begin{equation}\begin{aligned}
\PD_t B_h = R (\PD_{yy} B_h) \\
\PD_t B_n + \theta = R (\PD_{xx} B_n) \\
\end{aligned} \end{equation}

\section{Laplace Transform}
Applying the Laplace transform, we have
\begin{equation}\begin{aligned}
\MAC L \{f(t)\} = \int_0^{\infty} e^{-st} f(t) dt \\
\end{aligned} \end{equation}
For a simple function, say $\PD_t B^0 = \PD_t e^{-\omega t} = - \omega e^{-\omega t} $, we have
\begin{equation}\begin{aligned}
\MAC L \{ e^{-\omega t} \} = \frac{- \omega}{s+\omega} \\
\end{aligned} \end{equation}

\end{document}