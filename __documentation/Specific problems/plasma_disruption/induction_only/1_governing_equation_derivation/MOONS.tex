\documentclass[11pt]{article}
\newcommand{\PSCHAIN}{..}
\edef\PSCHAIN{\PSCHAIN/LATEX_INCLUDES}

\newcommand{\rootdir}{\PSCHAIN}
\newcommand{\VAR}{Success}



\begin{document}
% \doublespacing
\MOONSTITLE
\maketitle
\section{Governing equations}
Here, we derive the induction solution for a time varying field ($\PD_t B_z^0$) in a solid rectangular bar. Since the bar is stationary ($\U=\zero$), the only source term is the time varying field. The dimensional induction equation for uniform $\rho,\mu,\M,\sigma$ is
\begin{equation}\begin{aligned}
\PD_t B_i + \PD_t B_i^0 = \frac{1}{\sigma \M} \PD_{jj} B_i
\end{aligned} \end{equation}
Assuming pseudo-vacuum BCs, this problem simplifies to only one component with no variation along $z$ and with Dirichlet BCs for $B_z$ along the $x,y$ directions:
\begin{itemize}\setlength\itemsep{-1em}
	\item $B_x=0$,$B_y=0$,$\PD_z B_z = 0$
\end{itemize}
Expanding equations yields
\begin{equation}\begin{aligned}
\PD_t B_z + \PD_t B_z^0 = \frac{1}{\sigma \M} (\PD_{xx} B_z + \PD_{yy} B_z) \\
\end{aligned} \end{equation}
Or, removing the $z$ subscript, we have
\begin{equation}\begin{aligned}
\PD_t B + \PD_t B^0 = \frac{1}{\sigma \M} (\PD_{xx} B + \PD_{yy} B) \\
\end{aligned} \end{equation}
Let $R = \frac{1}{\sigma \M} = \text{constant}$, now we have:
\begin{equation}\begin{aligned}
\PD_t B + \theta = R (\PD_{xx} B + \PD_{yy} B), \qquad B(x,y,t),\theta(t),R=\text{constant} \\
\end{aligned} \end{equation}

\end{document}