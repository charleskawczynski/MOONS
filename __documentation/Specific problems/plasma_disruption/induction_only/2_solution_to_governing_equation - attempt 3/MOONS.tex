\documentclass[11pt]{article}
\newcommand{\PSCHAIN}{..}
\edef\PSCHAIN{\PSCHAIN/LATEX_INCLUDES}

\newcommand{\rootdir}{\PSCHAIN}
\newcommand{\VAR}{Success}



\begin{document}
% \doublespacing
\MOONSTITLE
\maketitle
\section{Deriving the separable PDE}
The derived governing equation is
\begin{equation}\begin{aligned}
\PD_t B^1 = R (\PD_{xx} B^1 + \PD_{yy} B^1) - \PD_t B^0, \qquad B^1(x,y,t),B^0(t),R=\text{constant} \\
\end{aligned} \end{equation}
\section{Sergey's Russian mathematics book}
Using Sergey's russian mathematics book (P. 145 for the problem statement, P. 149 for solution)
\begin{equation}\begin{aligned}
\PD_t T = a (\PD_{xx} T + \PD_{yy} T) + \Phi(x,y,t) \\
\end{aligned} \end{equation}
Under case 6, finite BCs are used:
\begin{equation}\begin{aligned}
0 \le x \le L_x \\
0 \le y \le L_y \\
T = f(x,y), \quad t=0 \\
T = 0 \in \PD \Omega \\
\end{aligned} \end{equation}
The solution to the equation is:
\begin{equation}\begin{aligned}
T(x,y,t)=
\int_0^{L_x} \int_0^{L_y} G(x,\xi,y,\zeta,t) f(\xi,\zeta) d \xi d \zeta
\\ +
\int_0^t \int_0^{L_x} \int_0^{L_y} G(x,\xi,y,\zeta,t-\tau) \Phi(\xi,\zeta,\tau) d \xi d \zeta d \tau
\\
G(x,\xi,y,\zeta,t) =
\frac{4}{L_x L_y} \sum_{n=1}^{\infty} \sum_{m=1}^{\infty}
e^{\left[ -\pi^2 a t \left( \frac{n^2}{L_x^2} + \frac{m^2}{L_y^2} \right) \right]}
\sin \left( \frac{n \pi x}{L_x} \right)
\sin \left( \frac{n \pi \xi}{L_x} \right)
\sin \left( \frac{m \pi y}{L_y} \right)
\sin \left( \frac{m \pi \zeta}{L_y} \right)
\\
\end{aligned} \end{equation}

\section{Applying the solution to our case}
The derived governing equation is
\begin{equation}\begin{aligned}
\PD_t B^1 = R (\PD_{xx} B^1 + \PD_{yy} B^1) - \PD_t B^0, \qquad B^1(x,y,t),B^0(t),R=\text{constant} \\
\end{aligned} \end{equation}
For our equation
\begin{equation}\begin{aligned}
T(x,y,t) = B(x,y,t) \\
f(\xi,\zeta) = 0 \\
\Phi(\xi,\zeta,\tau) = - \PD_t B^0 = \theta(t) \\
a = R \\
\end{aligned} \end{equation}
Therefore, we have
\begin{equation}\begin{aligned}
B(x,y,t) = \int_0^t \int_0^{L_x} \int_0^{L_y} G(x,\xi,y,\zeta,t-\tau) \theta(\tau) d \xi d \zeta d \tau \\
G(x,\xi,y,\zeta,t) =
\frac{4}{L_x L_y} \sum_{n=1}^{\infty} \sum_{m=1}^{\infty}
e^{-\pi^2 R t \left( \frac{n^2}{L_x^2} + \frac{m^2}{L_y^2} \right)}
\sin \left( \frac{n \pi x}{L_x} \right)
\sin \left( \frac{n \pi \xi}{L_x} \right)
\sin \left( \frac{m \pi y}{L_y} \right)
\sin \left( \frac{m \pi \zeta}{L_y} \right)
\\
\end{aligned} \end{equation}
Therefore, the electric current is
\begin{equation}\begin{aligned}
\J = \CURL \B \\
\end{aligned} \end{equation}

\end{document}