\documentclass[11pt]{article}
\newcommand{\PSCHAIN}{..}
\edef\PSCHAIN{\PSCHAIN/LATEX_INCLUDES}

\newcommand{\rootdir}{\PSCHAIN}
\newcommand{\VAR}{Success}



\begin{document}
\doublespacing
\MOONSTITLE
\maketitle

\section{Lorentz Force Scale}
The dimensional momentum equation is:
\begin{equation}
	\boxed{
	\rho \left( \frac{\PD \U}{\PD t} + (\U \DOT \DEL) \U \right) =
	-\DEL p + \mu \DEL^2 \U + \J \CROSS \B
	}
\end{equation}
Newton definition:
\begin{equation}\begin{aligned}
	[N] = \text{[kg m/s$^2$]} \\
\end{aligned}\end{equation}
The units are:
\begin{equation}\begin{aligned}
	[\U] = \text{[m/s]} \\
	[t] = \text{[s]} \\
	[x,y,z] = \text{[m]} \\
	[\rho] = \text{[kg/m$^3$]} \\
\end{aligned}\end{equation}
Therefore, the Lorentz force units are:
\begin{equation}\begin{aligned}
	[\J \CROSS \B] & = \frac{\text{m}}{\text{s}} \frac{1}{\text{s}} \frac{\text{kg}}{\text{m}^3} \\
	& = \frac{\text{kg}}{\text{m}^2 \text{s}^2} \\
	& = \frac{\text{kg} \text{ m}}{\text{s}^2} \frac{1}{\text{m}^3} \\
	& = [N] \frac{1}{\text{m}^3} \\
\end{aligned}\end{equation}

The Lorentz force is force per unit volume! Therefore, we must first multiply by the volume in which it is acting.


The implemented form follows
\begin{equation}\begin{aligned}
	[\H \BUN \B] & = \frac{\text{kg}}{\text{m}^3} \frac{\text{m}}{\text{s}} \frac{\text{m}}{\text{s}} = [\text{N}] \frac{1}{\text{m}^2} \\
	             & = \frac{B_c}{\M} B_c (\H^* \BUN \B^*) \\
	             & = \rho U_c^2 Al (\H^* \BUN \B^*) \\
\end{aligned}\end{equation}


\end{document}