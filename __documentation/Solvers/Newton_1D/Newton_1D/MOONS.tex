\documentclass[11pt]{article}
\newcommand{\PSCHAIN}{..}
\edef\PSCHAIN{\PSCHAIN/LATEX_INCLUDES}

\newcommand{\rootdir}{\PSCHAIN}
\newcommand{\VAR}{Success}



\begin{document}
\doublespacing
\MOONSTITLE
\maketitle

\section{Root solver for matching grid stretching parameter}

This document describes the method used to solve the nonlinear algebraic equation in MOONS to match the grid stretching parameter. The equation to solve is

\begin{equation}
	\frac{(\beta+2\alpha)\gamma_{sn} - \beta + 2\alpha}{(2\alpha+1)(1+\gamma_{sn})}
	-
	\left[
	\frac{(\beta+2\alpha)\gamma_{sn-1} - \beta + 2\alpha}{(2\alpha+1)(1+\gamma_{sn-1})}
	\right]
	=
	\frac{\Delta h}{h_{max}-h_{min}}
\end{equation}

Where $\beta$ is the unknown. Clearly, this equation is non-linear and must be solved numerically. The method I used is a sort of a binary search method. For ease of notation, I will write the solution as would be done in matlab. The same method can be used in Fortran with the intrinsic minloc() function.

\section{Problem statement}
We may write our equation by moving everything to one side:
\begin{equation}
	f(h) = root(h) = 0
\end{equation}

To solve for the solution $h^*$, make a vector, $h$

\begin{equation}
	h = \text{linspace}(h_{min},h_{max},N)
\end{equation}
\begin{equation}
	root(h) = \ldots
\end{equation}
\begin{equation}
	[\text{minVal} \, \text{minLoc}] = \text{min}(abs(root(h)))
\end{equation}
\begin{equation}
	h^* = h(minLoc)
\end{equation}

Note that the accuracy depends on the range and number of points used in the array, $h_{min}$, $h_{max}$, $N$. Therefore, the resolution is
\begin{equation}
	dh = \frac{h_{max}-h_{min}}{N}
\end{equation}

This means that we adjust searching for the root if the initial range is poor.
Since
\begin{equation}
	1 < \beta < \infty
\end{equation}
We have two asymptotic boundaries (1 refers to infinite stretching).

The next discussions are for advancing the search front in different scenarios given our knowledge of how $\beta$ is supposed to be used.

\subsection{Central search}
We may adjust the bounds of the solution to enhance the resolution by movind the boundary to $2 \Delta h$ from the initial solution.
\begin{equation}
	h_{min}^{n+1} = h^* - 2\Delta h
\end{equation}
\begin{equation}
	h_{max}^{n+1} = h^* + 2\Delta h
\end{equation}
Where $h^*$ is the solution at step $n$.

\subsection{Search Right}
Increase the search to the right, leave the search to the left narrow:
\begin{equation}
	h_{min}^{n+1} = h^* - 2\Delta h
\end{equation}
\begin{equation}
	h_{max}^{n+1} = h^* + 2N\Delta h
\end{equation}

\subsection{Search Left}
Increase the search to the left, leave the search to the right narrow:
\begin{equation}
	h_{min}^{n+1} = h^* - 2N\Delta h
\end{equation}
\begin{equation}
	h_{max}^{n+1} = h^* + 2\Delta h
\end{equation}

\subsection{Steep asymptotic search left (for \texorpdfstring{$\beta \rightarrow 1$}{beta approaching 1})}
If we are searching near a known asymptote, where the root becomes nearly vertical, we may want to make the search expand at a slower rate as the asymptote is reached.
\begin{equation}
	h_{min}^{n+1}=A-(A-h^*)\times(1-\gamma)
\end{equation}
\begin{equation}
	h_{max}^{n+1} = h^*+2\Delta h
\end{equation}
Where $A$ is the location of the asymptote and as $\gamma \rightarrow 1$, $h_{min}^{n+1} \rightarrow A$ and as $\gamma \rightarrow 0$, $h_{min}^{n+1} \rightarrow h_{min}^{n}$. I believe a value of $\gamma = 0.8$ should be a good conservative choice in order to slowly ascend/descend to the asymptote (in order to avoid NaNs by truncation error).

\subsection{Steep asymptotic search right (for \texorpdfstring{$\beta \rightarrow 1$}{beta approaching 1})}
For search right we have
\begin{equation}
	h_{min}^{n+1}=A+(A-h^*)\times(1-\gamma)
\end{equation}
\begin{equation}
	h_{max}^{n+1} = h^*-2\Delta h
\end{equation}

\section{Iterations}
This process can be repeated in order to reach a high precision. It has been my experience that, for a central search, this method is very effective at reaching high resolution in a very short time.


\end{document}