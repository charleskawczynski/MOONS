\documentclass[11pt]{article}
\newcommand{\bibdir}{\rootdir/includes/bib}

\usepackage{graphicx}
\usepackage{adjustbox}
% \usepackage{epstopdf} % Does not work with pdflatex
\usepackage{amsmath}
\usepackage{amssymb}
\usepackage{hyperref}
\usepackage{xspace}
\usepackage{mathtools}
\usepackage{tikz}
\usepackage{epsfig}
\usepackage{float}
\usepackage{subfigure}
\usepackage{setspace}
\usepackage{tabularx}
\usepackage{multirow}
\usepackage{ragged2e}
\usepackage{booktabs}
\usepackage{caption}
% \usepackage{subcaption}
\usepackage{xcolor}
\usepackage{longtable}
% \usepackage{natbib} % Sometimes causes errors
\usepackage{etoolbox}
\usepackage{geometry}
\usepackage{esint}
\usepackage{ifxetex}
\usepackage{ifluatex}
% \usepackage{siunitx}
\usepackage{numprint}
\usepackage{empheq}
\usepackage{slashbox}


 % For local use
\newcommand{\figWidth}{0.45\textwidth}
%\newcommand{\figW}{6.5in}
%\newcommand{\figH}{5.5in}
\newcommand{\figW}{5.in}
\newcommand{\figH}{4.in}
%\newcommand{\figW}{6.825in}
%\newcommand{\figH}{5.775in}
\newcommand{\ffigW}{3.1in}
\newcommand{\ffigH}{3.1in}

\newcommand{\captionWidth}{0.45}
\newcommand{\subFigPad}{0.52\textwidth} % good for subcaption
\newcommand{\TWOFIGCOLUMN}{0.52\textwidth} % good for subcaption
\newcommand{\TWOFIGCOLUMNCAPTION}{0.52\textwidth} % good for subcaption
% \newcommand{\subFigPad}{0.45} % good for subfig
\newcommand{\miniPageSize}{.1\textwidth}

\newcommand{\subFigSize}{16em} % good for subcaption
\newcommand{\figSize}{0.45\textwidth}


% \renewcommand{\contcaption}{%
%   \expandafter\addtocounter\expandafter{\@captype alt}{\m@ne}% Step alias cntr back
%   \expandafter\refstepcounter\expandafter{\@captype alt}% Make reference
%   \@contcaption\@captype}

% \definecolor{darkGreen}{rgb}{0.0, 0.5, 0.0}
\newcommand{\COR}{\color{red}}
\newcommand{\COG}{\color{darkGreen}}
\newcommand{\COM}{\color{magenta}}
\newcommand{\COK}{\color{black}}
\newcommand{\COB}{\color{blue}}

\newcommand{\toth}{\tiny{Toth, Gabor. "The $\DIV \B = 0$ constraint in shock-capturing magnetohydrodynamics codes." J. Comput. Phys. 161.2 (2000): 605-652.}}
\newcommand{\pattison}{\tiny{M. J. Pattison, K. N. Premnath, N. B. Morley, M. Abdou, Fusion Eng. Des. 83 (2008) 557-572.}}
\newcommand{\gujStella}{\tiny{G. Guj, F. Stella, J. Comput. Phys. 298 (1993) 286-298.}}
\newcommand{\bandaru}{\tiny{V. Bandaru, J. Pracht, T. Boeck, J. Schumacher, Theor. Comp. Fluid Dyn. (2015).}}
\newcommand{\guermond}{\tiny{Guermond, J. L., J. Léorat, and C. Nore. "A new Finite Element Method for magneto-dynamical problems: two-dimensional results." European Journal of Mechanics-B/Fluids 22.6 (2003): 555-579.}}
\newcommand{\kawczynski}{\tiny{C. Kawczynski, S. Smolentsev, M. Abdou. "A Lid-Driven Cavity MHD Flow Numerical Study at Moderate Magnetic Reynolds Number using Proper Magnetic Boundary Conditions". In Progress.}}
\newcommand{\roache}{\tiny{P. J. Roache, Annu. Rev. Fluid Mech. 29 (1997) 123-160.}}
\newcommand{\hunt}{\tiny{J. C. R. Hunt, J. Fluid Mech. 21 (1965) 577-590.}}
\newcommand{\shercliff}{\tiny{J. A. Shercliff, Proc. Camb. Phil. Soc. 49 (1953) 136.}}
\newcommand{\turner}{\tiny{Turner, Larry R. "Electromagnetic computations for fusion devices." IEEE Transactions on Magnetics 26.2 (1990): 847-852.}}
\newcommand{\kotulskiA}{\tiny{Kotulski, Joseph D., et al. "Electromagnetic analysis of forces and torques on the baseline and enhanced ITER shield modules due to plasma disruption." IEEE Transactions on Plasma Science 38.4 (2010): 1047.}}
\newcommand{\kotulskiB}{\tiny{Kotulski, Joseph D., Rebecca Coats, and Michael Ulrickson. "The analysis of the electromagnetic loads on selected ITER blanket shield modules due to induced eddy and halo currents." Fusion Engineering (SOFE), 2011 IEEE/NPSS 24th Symposium on. IEEE, 2011.}}
\newcommand{\iskakov}{\tiny{Iskakov, A. B., S. Descombes, and E. Dormy. "An integro-differential formulation for magnetic induction in bounded domains: boundary element-finite volume method." J. Comput. Phys. 197.2 (2004): 540-554.}}
\newcommand{\mikeulrickson}{\tiny{We received numerical data from Dr. Mike Ulrickson, computed from the plasma code DINA.}}
\newcommand{\Leriche}{Leriche, E. \& Gavrilakis, S. Direct numerical simulation of the flow in a lid-driven cubical cavity. Phys. Fluids 12, (2000).}
\newcommand{\Brackbill}{\tiny{Brackbill, Jeremiah U., and Daniel C. Barnes. "The effect of nonzero $\DIV \B = 0$ on the numerical solution of the magnetohydrodynamic equations." Journal of Computational Physics 35.3 (1980): 426-430.}}
\newcommand{\Leuer}{\tiny{Leuer, J. A., and J. C. Wesley. "ITER plasma start-up modeling." Fusion Engineering, 1993., 15th IEEE/NPSS Symposium on. Vol. 2. IEEE, 1993.}}

% \newcommand{\NJ}{New Jersey}
% \newcommand{\UCC}{Union County College}
% \newcommand{\GSR}{Graduate Student Researcher}
% \newcommand{\ME}{Mechanical Engineering}
% \newcommand{\NJSGC}{New Jersey Space Grant Consortium}
% % \newcommand{\NASA}{National Aeronautics and Space Administation}
% \newcommand{\NASA}{NASA}
% \newcommand{\SAS}{School of Arts and Sciences}
% % \newcommand{\UCLA}{University of California Los Angeles}
% \newcommand{\UCLA}{UCLA}
% \newcommand{\IDRE}{Institute for Digital Research and Education}
% \newcommand{\MAE}{Department of Mechanical and Aerospace Engineering}
% \newcommand{\UCLAMAE}{Henri Samueli School Of Engineering And Applied Sciences}
% % \newcommand{\MHD}{Magnetohydrodynamic}
% \newcommand{\MHDs}{Magnetohydrodynamics}
% % \newcommand{\RUTGERS}{Rutgers University, The State University of New Jersey}
% \newcommand{\RUTGERS}{Rutgers University}
% \newcommand{\RU}{Rutgers University}
% \newcommand{\US}{United States}

\newcommand{\height}{0.4}
\newcommand{\radius}{0.1}
\newcommand{\offSet}{12}
\newcommand{\Deltah}{1.6666}
\lstset{language=[90]Fortran,
  basicstyle=\ttfamily,
  keywordstyle=\color{red},
  commentstyle=\color{green},
  morecomment=[l]{!\ }% Comment only with space after !
}
\newif\ifxetexorluatex
\ifxetex
  \renewcommand{\C}{\mathbf{C}}
  \renewcommand{\U}{\mathbf{u}}
  \renewcommand{\G}{\mathbf{g}}
\else
  \newcommand{\C}{\mathbf{C}}
  \newcommand{\U}{\mathbf{u}}
  \newcommand{\G}{\mathbf{g}}
\fi
\renewcommand{\H}{\mathbf{H}}

\newcommand{\Rem}{Re_m}
\newcommand{\Rm}{$Re_m$}
\newcommand{\B}{\mathbf{B}}
\newcommand{\cleanB}{\xi}
\newcommand{\R}{\mathbf{r}}
\newcommand{\T}{\mathbf{T}}
\newcommand{\BBx}{\bar{B_x}}
\newcommand{\BBy}{\bar{B_y}}
\newcommand{\BBz}{\bar{B_z}}
\newcommand{\PB}{\bar{p}}
\newcommand{\UBx}{\bar{u}}
\newcommand{\VBy}{\bar{v}}
\newcommand{\WBz}{\bar{w}}
\newcommand{\E}{\mathbf{E}}
\newcommand{\J}{\mathbf{j}}
\newcommand{\A}{\mathbf{A}}
\newcommand{\N}{\mathbf{n}}
\newcommand{\F}{\mathbf{f}}
\newcommand{\X}{\mathbf{x}}
\newcommand{\Y}{\mathbf{y}}
\newcommand{\zero}{\mathbf{0}}
\newcommand{\ttheta}{\tilde{\theta}}
\newcommand{\M}{{\mu_m}}
\newcommand{\SII}{\SI^{-1}}
\newcommand{\JS}{\frac{\J}{\sigma}}
\newcommand{\MO}{\overline{\M}}
\newcommand{\SO}{\overline{\sigma}}
\newcommand{\JSS}{\frac{\J^2}{\sigma}}

\newcommand{\KE}{\left(\frac{1}{2} \U \DOT \U\right)}
\newcommand{\ME}{\left(\frac{1}{2} \B \DOT \B\right)}
\newcommand{\EPS}{\varepsilon}

\newcommand{\SIG}{\sigma}
\newcommand{\PH}{physical}

\newcommand{\SOVAC}{\SO_{v}}


\newcommand{\NPB}{\num[round-mode=places,round-precision=2]}
\newcommand{\NPU}{\num[round-mode=places,round-precision=5]}
\newcommand{\NPJ}{\num[round-mode=places,round-precision=5]}
\newcommand{\NPS}{\num[scientific-notation=true,round-mode=places,round-precision=2]}
\newcommand{\NPP}{\num[round-mode=places,round-precision=2]}

\newcommand{\MAT}{\left[ \begin{array}}
\newcommand{\EMAT}{\end{array} \right]}
\newcommand{\ARRAY}{\begin{array}}
\newcommand{\EARRAY}{\end{array}}

% \newcommand{\DOT}{\textbullet} % Causes warnings
% \newcommand{\DOT}{\text{\textbullet}}
\newcommand{\DOT}{\ensuremath{\bullet}}
% \newcommand{\textbullet}{}
\newcommand{\CROSS}{\times}
\newcommand{\BUN}{\otimes}
\newcommand{\DEL}{\nabla}
\newcommand{\CURL}{\DEL \CROSS}
\newcommand{\DIV}{\DEL \DOT}
\newcommand{\PD}{\partial}
\newcommand{\MAC}{\mathcal}
\newcommand{\GRAD}{\nabla}
\newcommand{\RA}{\rightarrow}

\newcommand{\EQSIZE}{\footnotesize}

\newcommand{\MC}{\multicolumn}
\newcommand{\MR}{\multirow}
\newcommand{\LL}{\raggedright}
\newcommand{\RR}{\raggedleft}
\newcommand{\CE}{\centering}
\newcommand{\TW}{\textwidth}
\renewcommand{\TH}{\textheight}
\newcommand{\TWP}{em\textwidth}
\newcommand{\qquadmany}{\qquad\qquad\qquad\qquad\qquad\qquad}
\newcommand{\hfillMath}{\hskip \textwidth minus \textwidth}

\newcommand{\tenDots}{..........}
\newcommand{\twentyDots}{\tenDots\tenDots}
\newcommand{\fiftyDots}{\twentyDots\twentyDots\tenDots}
\newcommand{\sixtyDots}{\twentyDots\twentyDots\twentyDots}
\newcommand{\seventyDots}{\fiftyDots\twentyDots}
\newcommand{\hundredDots}{\fiftyDots\fiftyDots}

% \restylefloat{table} % Results in errors in prospectus presentation


\newtoggle{includeImages}
\toggletrue{includeImages}
% \togglefalse{includeImages}

\newcommand{\IG}{\includegraphics}
\newcommand{\MP}{\minipage}
\newcommand{\EMP}{\endminipage}
\newcommand{\FIG}{\figure}
\newcommand{\EFIG}{\endfigure}
\newcommand{\SFIG}{\subfigure}
\newcommand{\ESFIG}{\endsubfigure}

\newcommand{\BLANKFIGURE}{\begin{figure*}[!htp]\centering\IG[width=.3\TW]{\figdir/blank.png}\end{figure*}}
\newcommand{\BLANKGRAPHIC}{\IG[width=.3\TW]{\figdir/blank.png}}
% \renewcommand{\BLANKGRAPHIC}{}
% \renewcommand{\BLANKFIGURE}{}

\newcommand{\FIGTYPE}{_BW}
\renewcommand{\FIGTYPE}{} % uncomment if color
\newcommand{\FIGEXT}{png}
% \renewcommand{\FIGEXT}{eps} % uncomment for .eps (takes forever to build)

\newcommand{\MOONSTITLE}{\title{MHD Object-Oriented Numerical Solver (MOONS)}
\author{C. Kawczynski \\
Department of Mechanical and Aerospace Engineering \\
University of California Los Angeles, USA\\
}}


% % Numbering from: https://www.sharelatex.com/learn-scripts/images/f/fc/Layout-dimensions.png
\setlength{\hoffset}{-.5in}       % 1
\setlength{\voffset}{0in}        % 2
\setlength{\oddsidemargin}{0in}  % 3
\setlength{\topmargin}{-.5in}     % 4
\setlength{\headheight}{0in}     % 5
\setlength{\headsep}{0in}        % 6
\setlength{\textheight}{700pt}   % 7
\setlength{\textwidth}{550pt}    % 8
\setlength{\marginparsep}{0in}   % 9
% \setlength{\marginparwidth}{0in} % 10
\setlength{\footskip}{0in}       % 11 a (see fig)
% \setlength{\marginparpush}{0in}  % 11 b (see fig)
% \setlength{\paperwidth}{0in}     % 11 e (see fig)
% \setlength{\paperheight}{390pt}  % 11 f (see fig)
 % For local use
% % Numbering from: https://www.sharelatex.com/learn-scripts/images/f/fc/Layout-dimensions.png
\setlength{\hoffset}{0in}       % 1
\setlength{\voffset}{0in}        % 2
\setlength{\oddsidemargin}{0in}  % 3
\setlength{\topmargin}{0in}     % 4
\setlength{\headheight}{0in}     % 5
\setlength{\headsep}{0in}        % 6
\setlength{\textheight}{350pt}   % 7
\setlength{\textwidth}{5in}    % 8
\setlength{\marginparsep}{0in}   % 9
\setlength{\marginparwidth}{0in} % 10
\setlength{\footskip}{0in}       % 11 a (see fig)
\setlength{\marginparpush}{0in}  % 11 b (see fig)
\setlength{\hoffset}{0in}        % 11 c (see fig)
\setlength{\voffset}{0in}        % 11 d (see fig)
\setlength{\paperwidth}{0in}     % 11 e (see fig)
\setlength{\paperheight}{390pt}  % 11 f (see fig)
 % For local use
% \input{\rootdir/includes/diagrams/coordinates.tex} % For local use
% \input{\rootdir/includes/margins/margins.tex} % For local use



\begin{document}
\doublespacing
\MOONSTITLE
\maketitle

\section{Outline of CT Method}
A note about notation, $B$ is located at the cell center and $b$ is located at the cell face. Bold faces have been neglected for simplicity. Also, $CC$ means cell center.

Here is the CT process
\subsection{Compute E-field at cell edge}
\begin{equation}
	E_{edge} = \frac{1}{Re_m} \frac{j_{edge}}{\sigma_{edge}} -
	(u_{edge} \times B_{edge})_{edge}
\end{equation}
Where $j_{edge}$ is computed from the 4 surrounding cell centers.

\subsection{Compute the flux of B from the edge curl of the E-field}

\begin{equation}
	\DEL \times E_{edge} \in face
\end{equation}

\subsection{Interpolate Faraday's law from cell faces to cell centers}
To update the magnetic field (from face locations), we use Faraday's law
\begin{equation}
	b^{n+1} = b^{n} - \Delta t
	\left(
	\DEL \times E_{edge}
	\right)_{face}
\end{equation}
Interpolating this we have
\begin{equation}
	B^{n+1} = B^{n} - \Delta t \,
	\text{interp}(
	\left[
	\DEL \times E_{edge}
	\right]_{face}
	)_{face \rightarrow CC}
\end{equation}

This enforces
\begin{equation}
	(\DEL \bullet b)_{CC} = 0
\end{equation}
Maybe this can be represented in a cell centered B-field. This needs to be further investigated.


\section{Index details for uniform grids}

\subsection{Computation of the E-field}
The first step is computed from the electric field

\begin{equation}
	E_{edge} = \frac{1}{Re_m} \frac{j_{edge}}{\sigma_{edge}} -
	(u_{edge} \times B_{edge})_{edge}
\end{equation}

This electric field is located at an edge.

Let the index $i,j$ denote the cell center. Let cell faces be located at half positions appropriately. Looking at the electric field in 2D, the z-component of the E-field would be

\begin{equation}
	E_z = \frac{1}{\sigma_{edge}}
	(\partial_x B_y - \partial_y B_x)_{edge} -
	(u_{edge} \times B_{edge})_{edge}
\end{equation}

Using indexes, in 2D for a uniform grid, current may be written as

\begin{equation}
	j_{z,i+1/2,j+1/2} =
	\frac{4}{\sigma_{i,j} + \sigma_{i+1,j} + \sigma_{i,j+1} + \sigma_{i+1,j+1}}
	\left[
	\partial_x B_{y,i,j+1/2} - \partial_y B_{x,i+1/2,j}
	\right]_{i+1/2,j+1/2}
\end{equation}

The derivative inside the current term must be collocated to subtract, so linearly interpolate them

\begin{equation}
	j_{z,i+1/2,j+1/2} =
	\frac{4}{\sigma_{i+1/2,j+1/2}}
	\left[
	\frac{
	\left(\frac{B_{y,j}+B_{y,j+1}}{2}\right)_{i+1}-\left(\frac{B_{y,j}+B_{y,j+1}}{2}\right)_{i}
	 }{\Delta x}
	-
	\frac{
	\left(\frac{B_{x,i}+B_{x,i+1}}{2}\right)_{j+1}-\left(\frac{B_{x,i}+B_{x,i+1}}{2}\right)_{j}
	}{\Delta y}
	\right]_{i+1/2,j+1/2}
\end{equation}
Where
\begin{equation}
	\sigma_{i+1/2,j+1/2} =
	\frac{\sigma_{i,j} + \sigma_{i+1,j} + \sigma_{i,j+1} + \sigma_{i+1,j+1}}{4}
\end{equation}

Now looking at the $u\times B$ term, we have.

The rest of this documented was deleted because the $u \times B$ term was corrected (and is correct above), and I didn't have time to rewrite the index details for the correct version. It is a straightforward interpolation of the velocity and magnetic fields to the cell edge, and then perform a cross product.


\subsection{Note}
Although the interpolation of the current seems to be the only first order accurate, looking at the stencil for the current, it is clear that 2nd order accuracy is achieved due to symmetry (the current is computed using a symmetric stencil about the cell edge).

\section{Curl of the E-field}
Once the electric field components are computed, take the curl of the electric field (which lives on the cell edge) and update B from Faraday's Law. This result will live on the cell face

\begin{equation}
	\frac{\partial B}{\partial t} = - \DEL \times E
\end{equation}

In 2D, from the above example, updating the x and y components of the magnetic field would look like
\begin{equation}
	b_{x,i+1/2,j}^{n+1} = b_{x,i+1/2,j}^{n} -
	\Delta t
	\frac{(E_{z,j+1/2}-E_{z,j-1/2})_{i+1/2}}{\Delta y}
\end{equation}
\begin{equation}
	b_{y,i,j+1/2}^{n+1} = b_{y,i,j+1/2}^{n} -
	\Delta t
	\frac{(E_{z,i+1/2}-E_{z,i-1/2})_{j+1/2}}{\Delta x}
\end{equation}
Note that the lowercase $b$ indicates that the data lives on the cell face. This equation can be interpolated (for both time steps) to the cell center

\begin{equation}
	B_{x,i,j}^{n+1} = B_{x,i,j}^{n} -
	\Delta t
	\left[
	\frac{\frac{(E_{z,j+1/2}-E_{z,j-1/2})_{i+1/2}}{\Delta y}
	+
	\frac{(E_{z,j+1/2}-E_{z,j-1/2})_{i+1/2}}{\Delta y}}{2}
	\right]
\end{equation}
\begin{equation}
	B_{y,i,j}^{n+1} = B_{y,i,j}^{n} -
	\Delta t
	\left[
	\frac{\frac{(E_{z,i+1/2}-E_{z,i-1/2})_{j+1/2}}{\Delta x}
	+
	\frac{(E_{z,i+1/2}-E_{z,i-1/2})_{j+1/2}}{\Delta x}}{2}
	\right]
\end{equation}

Note that the divergence of the magnetic field must be computed from the magnetic field before interpolating because this method enforces

\begin{equation}
	(\DEL \bullet b)_{i,j} =
	\frac{b_{x,i+1/2}-b_{x,i-1/2}}{\Delta x}
	+
	\frac{b_{y,j+1/2}-b_{y,j-1/2}}{\Delta y}
	=
	0
\end{equation}


\end{document}