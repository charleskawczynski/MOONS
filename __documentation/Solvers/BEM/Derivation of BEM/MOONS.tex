\documentclass[11pt]{article}
\usepackage{graphicx}    % needed for including graphics e.g. EPS, PS
\usepackage{epstopdf}
\usepackage{amsmath}
\usepackage{hyperref}
\usepackage{xspace}
\usepackage{mathtools}
\usepackage{tikz}
\usepackage{epsfig}
\usepackage{float}
%\usepackage{natbib}
\usepackage{subfigure}
\usepackage{setspace}
\usepackage{tabularx,ragged2e,booktabs,caption}

  \newcommand{\B}{\mathbf{B}}
  \renewcommand{\C}{\mathbf{C}}
  \renewcommand{\U}{\mathbf{u}}
  \newcommand{\E}{\mathbf{E}}
  \newcommand{\J}{\mathbf{j}}
  \newcommand{\A}{\mathbf{A}}
  \newcommand{\N}{\mathbf{n}}
  \renewcommand{\G}{\mathbf{g}}
  \newcommand{\F}{\mathbf{f}}
  \newcommand{\X}{\mathbf{x}}
  \newcommand{\Y}{\mathbf{y}}
\renewcommand{\H}{\mathbf{H}}

\newcommand{\DOT}{\text{\textbullet}}
\newcommand{\CROSS}{\times}
\newcommand{\DEL}{\nabla}
\newcommand{\CURL}{\DEL \CROSS}
\newcommand{\DIV}{\DEL \DOT}
\newcommand{\PD}{\partial}
\newcommand{\MAC}{\mathcal}

\newcommand{\MC}{\multicolumn}
\newcommand{\MR}{\multirow}

\newcommand{\PH}{physical}
\newcommand{\M}{{\mu_m}}
\newcommand{\SI}{\sigma}
\newcommand{\SII}{\SI^{-1}}
\newcommand{\JS}{\frac{\J}{\sigma}}
\newcommand{\MO}{\overline{\M}}
\newcommand{\SO}{\overline{\sigma}}
\newcommand{\RA}{\rightarrow}

\newcommand{\JSS}{\frac{\J^2}{\sigma}}

\newcommand{\qquadmany}{\qquad\qquad\qquad\qquad\qquad\qquad}




\setlength{\oddsidemargin}{0.1in}
\setlength{\textwidth}{7.25in}

\setlength{\topmargin}{-1in}     %\topmargin: gap above header
\setlength{\headheight}{0in}     %\headheight: height of header
\setlength{\topskip}{0in}        %\topskip: between header and text
\setlength{\headsep}{0in}        
\setlength{\textheight}{692pt}   %\textheight: height of main text
\setlength{\textwidth}{7.5in}    % \textwidth: width of text
\setlength{\oddsidemargin}{-0.5in}  % \oddsidemargin: odd page left margin
\setlength{\evensidemargin}{0in} %\evensidemargin : even page left margin
\setlength{\parindent}{0.25in}   %\parindent: indentation of paragraphs
\setlength{\parskip}{0pt}        %\parskip: gap between paragraphs
\setlength{\voffset}{0.5in}

% Useful commands:

% \hfill		aligns-right everything right of \hfill

\begin{document}
\doublespacing
\title{Magnetohydrodynamic Object-Oriented Numerical Solver (MOONS)}
\author{C. Kawczynski \\
Department of Mechanical and Aerospace Engineering \\
University of California Los Angeles, USA\\
}
% \maketitle

\section{Boundary Element Method (BEM)}
Here, I follow the work of \cite{Iskakov2004}. Consider boundary $\Gamma$. The tangential component of the electric field on the boundary is
\begin{equation}
  E_{\tau} = -\mathbf{\tau} \DOT (\mathbf{u} \times \B)_{\Gamma} + 
  Re_m^{-1} \mathbf{\tau} \DOT (\DEL \times \B)_{\Gamma}
\end{equation}
If we assume $\mathbf{u}_{\Gamma} \DOT \mathbf{n} = 0$ then this simplifies to
\begin{equation}
  E_{\tau} = -\mathbf{\tau} \DOT (\mathbf{u} \times (B_n \mathbf{n})_{\Gamma}) + 
  Re_m^{-1} \mathbf{\tau} \DOT (\DEL \times \B)_{\Gamma}
\end{equation}
% \begin{equation}
%   (\DEL \times \B)_{\Gamma} = \lim_{r \rightarrow \Gamma} (\DEL \times \B(\mathbf{r}))
% \end{equation}
The first term requires knowledge of $B_n$ only and therefore requires no special difficulty. The second term may be approximated using a 2nd order accurate stencil, but the tangential component of the magnetic field is still unknown. This will be determined from the normal component and from proper boundary conditions.

\section{Vector identities}
\begin{equation}\begin{aligned}
  \CURL (\A \times \B) = \A (\DIV \B) - \B (\DIV \A) + (\B \DOT \DEL)\A - (\A \DOT \DEL)\B \\
  \A \times (\B \times \C) = \B (\A \DOT \C) - \C (\A \DOT \B)
\end{aligned}\end{equation}

\section{Details of first term}
The component of $\E$ tangent to the surface normal $\N$ is
\begin{equation}\begin{aligned}
  \N \times \E = -\N \times (\U \times \B) + Re_m^{-1} \N \times (\SII \DEL \times \B) \\
  \A \times (\B \times \C) = \B (\A \DOT \C) - \C (\A \DOT \B) \\
  \N \times \E = -(\U (\N \DOT \B) - \B (\N \DOT \U)) + Re_m^{-1} \N \times (\SII \DEL \times \B) \\
\end{aligned}\end{equation}

The normal component may be updated from

\begin{equation}\begin{aligned}
  \PD_t (\N \DOT \B) =& - \N \DOT \CURL \E \\
   =& - \N \DOT \CURL (\SII \CURL \B - \U \times \B) \\
   =& \N \DOT \CURL \U \times \B - \N \DOT \CURL (\SII \CURL \B) \\
  & \CURL (\A \times \B) = \A (\DIV \B) - \B (\DIV \A) + (\B \DOT \DEL)\A - (\A \DOT \DEL)\B \\
   =& \N \DOT (\U (\DIV \B) - \B (\DIV \U) + (\B \DOT \DEL)\U - (\U \DOT \DEL)\B) - \N \DOT \CURL (\SII \CURL \B) \\
   =& \N \DOT ((\B \DOT \DEL)\U - (\U \DOT \DEL)\B) - \N \DOT \CURL (\SII \CURL \B) \\
   =& \qquad \qquad \hdots \qquad \qquad \quad  - \N \DOT \CURL (\SII \CURL \B) \\
   =& \qquad \qquad \hdots \qquad \qquad \quad  - \N \DOT \SII \CURL (\CURL \B) \\
   =& \qquad \qquad \hdots \qquad \qquad \quad  - \N \DOT \SII (0 - \DEL^2 \B) \\
   =& (\B \DOT \DEL)(\N \DOT \U) - (\U \DOT \DEL)(\N \DOT \B) + \underbrace{\SII \DEL^2 (\N \DOT \B)}_{\text{source term when $\U=\mathbf{0}$}} \\
\end{aligned}\end{equation}

\section{Integral formulation at the boundary}
The tangential component of the magnetic field depends on the values of its normal component everywhere at the boundary.

Consider matching a potential field as an elliptic problem in $\Omega^c$, the complementary domain of $\Omega$
\begin{equation}
  \B = -\DEL \phi, \qquad \Delta \phi = 0.
\end{equation}
Where $\phi: \Omega^c \rightarrow R$ is the potential function. At infinity, the physical condition for the magnetic field is
\begin{equation}
  \phi \rightarrow O(r^{-2}), \qquad r \rightarrow \infty
\end{equation}
The normal component of B is known, which implies a Neumann BC on the potential:
\begin{equation}
  \frac{\PD \phi}{\PD n}_{\Gamma} = - B_n \qquad (B_n: \Gamma \rightarrow R)
\end{equation}
Expressing this as surface integrals over the fundamental solution $\Delta G = \delta(\X,\Y)$
\begin{equation}
  G(\X,\Y) = \frac{-1}{4 \pi |\X-\Y|}
\end{equation}

Denoting an open ball, $B$, of radius $R$ such that $\overline{\Omega}\in B$, as a consequence of Green's theorem, the magnetic potential at the boundary satisfies
\begin{equation}
  \phi(\X) = 
  -2 \int_{\Gamma} \left( \phi(\Y)
  \frac{\PD G}{\PD n} + B_n(\Y) G \right) ds(\Y)
  +2 \int_{\PD B_R} \left( \phi(\Y)
  \frac{\PD G}{\PD \tilde{n}} - \frac{\PD \phi(\Y}{\PD \tilde{n}} G \right) ds(\Y)
\end{equation}
Where $n$ is the coordinate along the outward normal to $\Omega$ and $\tilde{n}$ is the coordinate along the outward normal to the ball. The second term vanishes when $R\rightarrow \infty$ (keeping the center of the ball fixed). So we may write
\begin{equation} \label{eq:phi}
  \phi(\X) = 
  -2 \int_{\Gamma} \left( \phi(\Y)
  \frac{\PD G}{\PD n} + B_n(\Y) G \right) ds(\Y)
\end{equation}
Consequently, the tangential component of the magnetic field on $\Gamma$ along the unit vector $\tau$ is 
\begin{equation}
  B_{\tau} = - \mathbf{\tau} \DOT \DEL \phi(\X) = 
  2 \mathbf{\tau} \DOT
  \int_{\Gamma}
  \left(
  \phi(\Y) \frac{\PD G(\X,\Y)}{\PD n}
  + B_n(\Y) \DEL_x G(\X,\Y)
  \right) ds(\Y)
\end{equation}
The discrete form of this may be assembeled as 
\begin{equation}
  \frac{1}{2} \Phi = A \Phi + C B_n
\end{equation}
With the resulting solution
\begin{equation}
  \left(\frac{I}{2} - A\right)\Phi = C B_n, \qquad
  \rightarrow
  \Phi = \left(\frac{I}{2} - A\right)^{-1} C B_n
\end{equation}
This can be computed once at the beginning of the sim. Returning to \ref{eq:phi}, and $G(\X_i,\Y) = \frac{-1}{4\pi |\X - \Y|}$ we may write the discrete form as
\begin{equation}
\begin{split}
\phi(\X_i) & = -2 \sum_j \phi_j \int_{S_j} \frac{\PD G(\X_i,\Y)}{\PD n} ds(\Y)
           -2 \sum_j B_{nj} \int_{S_j} G(\X_i,\Y) ds(\Y) \\
       & = \frac{1}{2 \pi} 
       \left[
       \sum_j \phi_j \int_{S_j} \frac{\PD}{\PD n} \frac{1}{|\X_i-\Y|} ds(\Y)
     + \sum_j B_{nj} \int_{S_j} \frac{1}{|\X_i-\Y|} ds(\Y)
       \right] \\
\end{split}
\end{equation}
The term $\frac{\PD}{\PD n} \frac{1}{|\X_i-\Y|} $ may be evaluated to give us
\begin{equation}
\phi(\X_i) = \frac{1}{2 \pi} 
       \left[
       \sum_j \phi_j \int_{S_j} \frac{\Y - \X_i}{| \Y - \X_i|^3} \DOT \N ds(\Y)
     + \sum_j B_{nj} \int_{S_j} \frac{1}{|\X_i-\Y|} ds(\Y)
       \right] \\
\end{equation}

\begin{thebibliography}{1}
\bibitem{Iskakov2004} Iskakov, A. B., Descombes, S. and Dormy, E. An integro-differential formulation for magnetic induction in bounded domains: boundary element–finite volume method. J. Comput. Phys. 197, 540–554 (2004)..
\end{thebibliography}

\end{document}