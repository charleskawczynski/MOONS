\documentclass[11pt]{article}
\newcommand{\VAR}{Success}


\begin{document}
\doublespacing
\MOONSTITLE
\maketitle

\section{Computations of SOR (Uniform coefficient)}
The finite difference equation to solve begins with the Poisson equation


\begin{equation}
	\DEL^2 u = f
\end{equation}

Since the 2nd derivative of a field lives on the same grid as the field itself, the only need to distinguish cell centered vs cell corner derivatives has to do with where the index starts (which depends on the ratio of ghost cells between the two grids).

Both forms use the following convention
\begin{equation}
	\Delta h_{c,i} = h_{c,i+1} - h_{c,i}
\end{equation}
\begin{equation}
	\Delta h_{n,i} = h_{n,i+1} - h_{n,i}
\end{equation}

\subsection{Non-uniform grid for cell centered data}
Our 2nd order accurate spatially discrete Poisson equation is

\begin{equation}
	(L + D + U)_x u +
	(L + D + U)_y u +
	(L + D + U)_z u
	 = f
\end{equation}

For Gauss-Seidel, we iterate using

\begin{equation}
	L_x u^{k} + D_x u^{k+1} + U_x u^{k} +
	L_y u^{k} + D_y u^{k+1} + U_y u^{k} +
	L_z u^{k} + D_z u^{k+1} + U_z u^{k}
	 = f
\end{equation}

Solving we have

\begin{equation}
	D_x u^{k+1} +
	D_y u^{k+1} +
	D_z u^{k+1}
	 = f - (
	L_x u^{k} + U_x u^{k} +
	L_y u^{k} + U_y u^{k} +
	L_z u^{k} + U_z u^{k}
	)
\end{equation}

or

\begin{equation}
	(D_x + D_y + D_z) u^{k+1}
	 = f - (
	L_x u^{k} + U_x u^{k} +
	L_y u^{k} + U_y u^{k} +
	L_z u^{k} + U_z u^{k}
	)
\end{equation}

And finally

\begin{equation}
	u^{k+1}
	 =
	(D_x + D_y + D_z)^{-1}
	\left\{
	f - (
	L_x u^{k} + U_x u^{k} +
	L_y u^{k} + U_y u^{k} +
	L_z u^{k} + U_z u^{k}
	) \right\}
\end{equation}


\end{document}