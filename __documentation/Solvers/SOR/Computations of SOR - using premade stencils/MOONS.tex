\documentclass[11pt]{article}
\usepackage{graphicx}    % needed for including graphics e.g. EPS, PS
\usepackage{epstopdf}
\usepackage{amsmath}
\usepackage{hyperref}
\usepackage{xspace}
\usepackage{mathtools}
\usepackage{tikz}
\usepackage{epsfig}
\usepackage{float}
\usepackage{natbib}
\usepackage{subfigure}
\usepackage{setspace}
\usepackage{tabularx,ragged2e,booktabs,caption}


\setlength{\oddsidemargin}{0.1in}
\setlength{\textwidth}{7.25in}

\setlength{\topmargin}{-1in}     %\topmargin: gap above header
\setlength{\headheight}{0in}     %\headheight: height of header
\setlength{\topskip}{0in}        %\topskip: between header and text
\setlength{\headsep}{0in}        
\setlength{\textheight}{692pt}   %\textheight: height of main text
\setlength{\textwidth}{7.5in}    % \textwidth: width of text
\setlength{\oddsidemargin}{-0.5in}  % \oddsidemargin: odd page left margin
\setlength{\evensidemargin}{0in} %\evensidemargin : even page left margin
\setlength{\parindent}{0.25in}   %\parindent: indentation of paragraphs
\setlength{\parskip}{0pt}        %\parskip: gap between paragraphs
\setlength{\voffset}{0.5in}


% Useful commands:

% \hfill		aligns-right everything right of \hfill

\begin{document}
\doublespacing
\title{Magnetohydrodynamic Object-Oriented Numerical Solver (MOONS)}
\author{C. Kawczynski \\
Department of Mechanical and Aerospace Engineering \\
University of California Los Angeles, USA\\
}
\maketitle

\section{Composite one-sided differencing (uniform properties)}

\begin{equation}
	\frac{\frac{u_{i+1,j,k}-u_{i,j,k}}{\Delta x_{c,i}} - \frac{u_{i,j,k}-u_{i-1,j,k}}{\Delta x_{c,i-1}}}{\Delta x_{n,i}} + 
	\frac{\frac{u_{i,j+1,k}-u_{i,j,k}}{\Delta y_{c,j}} - \frac{u_{i,j,k}-u_{i,j-1,k}}{\Delta y_{c,j-1}}}{\Delta y_{n,j}} +
	\frac{\frac{u_{i,j,k+1}-u_{i,j,k}}{\Delta z_{c,k}} - \frac{u_{i,j,k}-u_{i,j,k-1}}{\Delta z_{c,k-1}}}{\Delta z_{n,k}} = f_{i,j,k}
\end{equation}

We can write this as

\begin{equation}
	\frac{-
	\frac{
	u_{i-1,j,k}+(-u_{i,j,k})
	}{
	\Delta x_{c,i}} + 
	\frac{
	u_{i,j,k}+(-u_{i+1,j,k})
	}{
	\Delta x_{c,i-1}}}{\Delta x_{n,i}} + 
	\frac{-
	\frac{
	u_{i,j-1,k}+(-u_{i,j,k})
	}{
	\Delta y_{c,j}} + 
	\frac{
	u_{i,j,k}+(-u_{i,j+1,k})
	}{
	\Delta y_{c,j-1}}}{\Delta y_{n,j}} +
	\frac{-
	\frac{
	u_{i,j,k-1}+(-u_{i,j,k})
	}{
	\Delta z_{c,k}} + 
	\frac{
	u_{i,j,k}+(-u_{i,j,k+1})
	}{
	\Delta z_{c,k-1}}}{\Delta z_{n,k}} = f_{i,j,k}
\end{equation}

Which is of the form
\begin{equation}
	D_x (D_x u + U_x u) +
	U_x (D_x u + U_x u) +
	D_y (D_y u + U_y u) +
	U_y (D_y u + U_y u) +
	D_z (D_z u + U_z u) +
	U_z (D_z u + U_z u)
	= f
\end{equation}

Treating the appropriate terms implicitly, we have

\begin{multline}
	D_x (D_x u^{ k } + U_x u^{k+1}) +
	U_x (D_x u^{k+1} + U_x u^{ k }) +
	D_y (D_y u^{ k } + U_y u^{k+1}) + \\
	U_y (D_y u^{k+1} + U_y u^{ k }) +
	D_z (D_z u^{ k } + U_z u^{k+1}) +
	U_z (D_z u^{k+1} + U_z u^{ k })
	= f
\end{multline}

Solving for the implicitly treated terms we have

\begin{multline}
	U_x D_x u^{k+1} + D_x U_x u^{k+1} +
	U_y D_y u^{k+1} + D_y U_y u^{k+1} +
	U_z D_z u^{k+1} + D_z U_z u^{k+1}
	=  \\
	f -
	(
	D_x D_x u^k +
	U_x U_x u^k +
	D_y D_y u^k +
	U_y U_y u^k +
	D_z D_z u^k +
	U_z U_z u^k
	)
\end{multline}

Inverting LHS, we have

\begin{equation}
	u^{k+1}
	= 
	A^{-1}
	\left\{
	f -
	(
	D_x D_x u^k +
	U_x U_x u^k +
	D_y D_y u^k +
	U_y U_y u^k +
	D_z D_z u^k +
	U_z U_z u^k
	)
	\right\}
\end{equation}

Where
\begin{equation}
	A
	= 
	(
	U_x D_x + D_x U_x +
	U_y D_y + D_y U_y +
	U_z D_z + D_z U_z
	)
\end{equation}


Comparing this with the old method:

\begin{equation}
	u_{i,j,k} = 
	\frac{ 
	\frac{u_{i+1,j,k}}{\Delta x_{p,i} \Delta x_{d,i}} + \frac{u_{i-1,j,k}}{\Delta x_{p,i-1} \Delta x_{d,i}} + 
	\frac{u_{i,j+1,k}}{\Delta y_{p,j} \Delta y_{d,j}} + \frac{u_{i,j-1,k}}{\Delta y_{p,j-1} \Delta y_{d,j}} +
	\frac{u_{i,j,k+1}}{\Delta z_{p,k} \Delta z_{d,k}} + \frac{u_{i,j,k-1}}{\Delta z_{p,k-1} \Delta z_{d,k}}
	- f_{i,j,k} }{
	\left( 
	\frac{ 1 }{ \Delta x_{d,i} }
	\left[
	\frac{ 1 }{ \Delta x_{p,i} } +
	\frac{ 1 }{ \Delta x_{p,i-1} }
	\right] +
	\frac{ 1 }{ \Delta y_{d,j} }
	\left[
	\frac{ 1 }{ \Delta y_{p,j} } +
	\frac{ 1 }{ \Delta y_{p,j-1} }
	\right] +
	\frac{ 1 }{ \Delta z_{d,k} }
	\left[
	\frac{ 1 }{ \Delta z_{p,k} } +
	\frac{ 1 }{ \Delta z_{p,k-1} }
	\right]
	\right)
	}
\end{equation}

We can equate some things:

\begin{equation}
	D_x U_x +
	U_x D_x
	=
	\frac{1}{\Delta x_{p,i} \Delta x_{d,i}}
	+
	\frac{1}{\Delta x_{p,i-1} \Delta x_{d,i}}
\end{equation}

\begin{equation}
	D_x D_x u +
	U_x U_x u
	=
	\frac{u_{i-1}}{\Delta x_{p,i-1} \Delta x_{d,i}}
	+ 
	\frac{u_{i+1}}{\Delta x_{p,i} \Delta x_{d,i}} 
\end{equation}

\section{Composite one-sided differencing (non-uniform coefficient)}

If the poisson equation has a non-uniform coefficient, e.g.

\begin{equation}
	\nabla \bullet \left( \sigma \nabla u \right) = f
\end{equation}

If an index is missing from the subscript, assume $i,j,k$. The discrete equation is

\begin{equation}
	\frac{\frac{u_{i+1}-u}{\Delta x_{p,i}} \sigma_{i+1/2} - \frac{u-u_{i-1}}{\Delta x_{p,i-1}} \sigma_{i-1/2}}{\Delta x_{d,i}} + 
	\frac{\frac{u_{j+1}-u}{\Delta y_{p,j}} \sigma_{j+1/2} - \frac{u-u_{j-1}}{\Delta y_{p,j-1}} \sigma_{j-1/2}}{\Delta y_{d,j}} +
	\frac{\frac{u_{k+1}-u}{\Delta z_{p,k}} \sigma_{k+1/2} - \frac{u-u_{k-1}}{\Delta z_{p,k-1}} \sigma_{k-1/2}}{\Delta z_{d,k}} = f_{i,j,k}
\end{equation}


Which is of the form
\begin{multline}
	D_x (\sigma_x D_x u + \sigma_x U_x u) +
	U_x (\sigma_x D_x u + \sigma_x U_x u) +
	D_y (\sigma_y D_y u + \sigma_y U_y u) + \\
	U_y (\sigma_y D_y u + \sigma_y U_y u) +
	D_z (\sigma_z D_z u + \sigma_z U_z u) +
	U_z (\sigma_z D_z u + \sigma_z U_z u)
	= f
\end{multline}
Where $\sigma_x = I_x \sigma$ is the interpolated $\sigma$. Treating the appropriate terms implicitly, we have
\begin{multline}
	D_x (\sigma_x D_x u^{ k } + \sigma_x U_x u^{k+1}) +
	U_x (\sigma_x D_x u^{k+1} + \sigma_x U_x u^{ k }) +
	D_y (\sigma_y D_y u^{ k } + \sigma_y U_y u^{k+1}) + \\
	U_y (\sigma_y D_y u^{k+1} + \sigma_y U_y u^{ k }) +
	D_z (\sigma_z D_z u^{ k } + \sigma_z U_z u^{k+1}) +
	U_z (\sigma_z D_z u^{k+1} + \sigma_z U_z u^{ k })
	= f
\end{multline}
Solving for the implicit terms
\begin{multline}
	U_x (\sigma_x D_x u^{k+1}) +
	U_y (\sigma_y D_y u^{k+1}) +
	U_z (\sigma_z D_z u^{k+1}) +
	D_x (\sigma_x U_x u^{k+1}) +
	D_y (\sigma_y U_y u^{k+1}) +
	D_z (\sigma_z U_z u^{k+1}) \\
	=
	f - 
	\left\{
	D_x (\sigma_x D_x u^{ k }) +
	U_x (\sigma_x U_x u^{ k }) +
	D_y (\sigma_y D_y u^{ k }) +
	U_y (\sigma_y U_y u^{ k }) +
	D_z (\sigma_z D_z u^{ k }) +
	U_z (\sigma_z U_z u^{ k })
	\right\}
\end{multline}
Inverting this we have:
\begin{equation}
	\boxed{
	u^{k+1}
	=
	A^{-1}
	\left[
	f - 
	\left\{
	D_x (\sigma_x D_x u^{ k }) +
	U_x (\sigma_x U_x u^{ k }) +
	D_y (\sigma_y D_y u^{ k }) +
	U_y (\sigma_y U_y u^{ k }) +
	D_z (\sigma_z D_z u^{ k }) +
	U_z (\sigma_z U_z u^{ k })
	\right\}
	\right]
	}
\end{equation}
Where
\begin{equation}
	\boxed{
	A = 
	U_x (\sigma_x D_x) +
	U_y (\sigma_y D_y) +
	U_z (\sigma_z D_z) +
	D_x (\sigma_x U_x) +
	D_y (\sigma_y U_y) +
	D_z (\sigma_z U_z)
	}
\end{equation}
Note that $A$ is a 3D scalar field.

\section{3-point stencil (uniform properties)}

\begin{equation}
	\frac{\frac{u_{i+1,j,k}-u_{i,j,k}}{\Delta x_{c,i}} - \frac{u_{i,j,k}-u_{i-1,j,k}}{\Delta x_{c,i-1}}}{\Delta x_{n,i}} + 
	\frac{\frac{u_{i,j+1,k}-u_{i,j,k}}{\Delta y_{c,j}} - \frac{u_{i,j,k}-u_{i,j-1,k}}{\Delta y_{c,j-1}}}{\Delta y_{n,j}} +
	\frac{\frac{u_{i,j,k+1}-u_{i,j,k}}{\Delta z_{c,k}} - \frac{u_{i,j,k}-u_{i,j,k-1}}{\Delta z_{c,k-1}}}{\Delta z_{n,k}} = f_{i,j,k}
\end{equation}

Which can be written as

\begin{equation}
	L_x u + D_x u + U_x u +
	L_y u + D_y u + U_y u +
	L_z u + D_z u + U_z u
	= f
\end{equation}

Treating the appropriate terms implicitly, we have

\begin{equation}
	L_x u^{k} + D_x u^{k+1} + U_x u^{k} +
	L_y u^{k} + D_y u^{k+1} + U_y u^{k} +
	L_z u^{k} + D_z u^{k+1} + U_z u^{k}
	= f
\end{equation}

Solving for the implicitly treated terms we have

\begin{equation}
	u^{k+1}
	= 
	A^{-1}
	\left[
	f - 
	\left\{
	L_x u^{k} + U_x u^{k} +
	L_y u^{k} + U_y u^{k} +
	L_z u^{k} + U_z u^{k}
	\right\}
	\right]
\end{equation}
Where
\begin{equation}
	A = 
	D_x +
	D_y +
	D_z
\end{equation}

\section{Combining 2 one-sided difference operators}
\begin{equation}
	D_x \sigma_x D_x u
\end{equation}

\end{document}