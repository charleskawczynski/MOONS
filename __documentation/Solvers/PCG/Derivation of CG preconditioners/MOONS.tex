\documentclass[11pt]{article}
\usepackage{graphicx}    % needed for including graphics e.g. EPS, PS
\usepackage{epstopdf}
\usepackage{amsmath}
\usepackage{hyperref}
\usepackage{xspace}
\usepackage{mathtools}
\usepackage{tikz}
\usepackage{epsfig}
\usepackage{float}
%\usepackage{natbib}
\usepackage{subfigure}
\usepackage{setspace}
\usepackage{tabularx,ragged2e,booktabs,caption}


\setlength{\oddsidemargin}{0.1in}
\setlength{\textwidth}{7.25in}

\setlength{\topmargin}{-1in}     %\topmargin: gap above header
\setlength{\headheight}{0in}     %\headheight: height of header
\setlength{\topskip}{0in}        %\topskip: between header and text
\setlength{\headsep}{0in}
\setlength{\textheight}{692pt}   %\textheight: height of main text
\setlength{\textwidth}{7.5in}    % \textwidth: width of text
\setlength{\oddsidemargin}{-0.5in}  % \oddsidemargin: odd page left margin
\setlength{\evensidemargin}{0in} %\evensidemargin : even page left margin
\setlength{\parindent}{0.25in}   %\parindent: indentation of paragraphs
\setlength{\parskip}{0pt}        %\parskip: gap between paragraphs
\setlength{\voffset}{0.5in}

% Useful commands:

% \hfill		aligns-right everything right of \hfill

\begin{document}
\doublespacing
\title{Magnetohydrodynamic Object-Oriented Numerical Solver (MOONS)}
\author{C. Kawczynski \\
Department of Mechanical and Aerospace Engineering \\
University of California Los Angeles, USA\\
}
% \maketitle

\section{CG preconditioners}

\subsection{Laplacian (uniform properties)}
Consider the 1-D Laplacian operator

\begin{equation}
  \nabla^2 = D_2(D_1 + U_1) + U_2(D_1 + U_1)
  = D_2D_1 + D_2U_1 + U_2D_1 + U_2U_1
\end{equation}

Where the number subscript indicates cell center or cell corner data. It is clear from the above equation that the diagonal is

\begin{equation}
  diag(\nabla^2) = D_2U_1 + U_2D_1
\end{equation}

The 3D operator will be the sum of the three 1-D operators.

\subsection{Laplacian (variable properties)}
Consider the 1-D Laplacian operator

\begin{equation}
  \nabla (k\nabla) = D_2 [k(D_1 + U_1)] + U_2 [k(D_1 + U_1)]
  = D_2(kD_1) + D_2(kU_1) + U_2(kD_1) + U_2(kU_1)
\end{equation}

Where the number subscript indicates cell center or cell corner data. It is clear from the above equation that the diagonal is

\begin{equation}
  diag(\nabla (k\nabla)) = D_2(kU_1) + U_2(kD_1)
\end{equation}

The 3D operator will be the sum of the three 1-D operators.

\subsection{Curl-curl operator (variable properties)}

Consider the curl-curl operator
\begin{multline}
  \nabla \times (k \nabla \times) \mathbf{B} = \\
  \nabla \times (
  \hat{x} k_x (\partial_y B_z - \partial_z B_y) \\
  - \hat{y} k_y (\partial_x B_z - \partial_z B_x) \\
  + \hat{z} k_z (\partial_x B_y - \partial_y B_x) ) \\  =
   \hat{x} \left\{ \partial_y \left[ k_z (\partial_x B_y - \partial_y B_x) \right] +
    \partial_z \left[ k_y (\partial_x B_z - \partial_z B_x) \right] \right\} \\
  -\hat{y} \left\{ \partial_x \left[ k_z (\partial_x B_y - \partial_y B_x) \right] -
   \partial_z \left[ k_x (\partial_y B_z - \partial_z B_y) \right]  \right\} \\
  +\hat{z} \left\{ -\partial_x \left[ k_y (\partial_x B_z - \partial_z B_x) \right] -
   \partial_y \left[ k_x (\partial_y B_z - \partial_z B_y) \right]  \right\} \\
\end{multline}
Substituting the $\partial$ operator with the discrete derivative operator, $D$, we have
\begin{multline}
  \nabla \times (k \nabla \times) \mathbf{B} = \\
   \hat{x} \left\{ D_y \left[ k_z (D_x B_y - D_y B_x) \right] +
    D_z \left[ k_y (D_x B_z - D_z B_x) \right] \right\} \\
  -\hat{y} \left\{ D_x \left[ k_z (D_x B_y - D_y B_x) \right] -
   D_z \left[ k_x (D_y B_z - D_z B_y) \right]  \right\} \\
  +\hat{z} \left\{ -D_x \left[ k_y (D_x B_z - D_z B_x) \right] -
   D_y \left[ k_x (D_y B_z - D_z B_y) \right]  \right\} \\
\end{multline}
Distributing the terms we have
\begin{multline}
  \nabla \times (k \nabla \times) \mathbf{B} = \\
   \hat{x} \left\{   (D_y k_z D_x B_y - D_y k_z D_y B_x)   +  (D_z k_y D_x B_z - D_z k_y D_z B_x) \right\} \\
  -\hat{y} \left\{   (D_x k_z D_x B_y - D_x k_z D_y B_x)   -  (D_z k_x D_y B_z - D_z k_x D_z B_y) \right\} \\
  +\hat{z} \left\{ - (D_x k_y D_x B_z - D_x k_y D_z B_x)   -  (D_y k_x D_y B_z - D_y k_x D_z B_y) \right\} \\
\end{multline}
Distributing the signs we finally get
\begin{multline}
  \nabla \times (k \nabla \times) \mathbf{B} = \\
  +\hat{x} \left\{ +D_y k_z D_x B_y - D_y k_z D_y B_x +  D_z k_y D_x B_z - D_z k_y D_z B_x \right\} \\
  +\hat{y} \left\{ -D_x k_z D_x B_y + D_x k_z D_y B_x +  D_z k_x D_y B_z - D_z k_x D_z B_y \right\} \\
  +\hat{z} \left\{ -D_x k_y D_x B_z + D_x k_y D_z B_x -  D_y k_x D_y B_z + D_y k_x D_z B_y \right\} \\
\end{multline}
Now the question is: what is the diagonal of this operator? Considering the vector identity, $\nabla \times \nabla \times = \nabla (\nabla \bullet ) - \nabla^2$, it's natural to compare this with the Laplacian diagonal (derived above) which should have some of the same terms for uniform properties. It is obvious that 2 derivatives along the same direction will result in collocated data, which will lead to a diagonal component, these terms have been bolded below
\begin{multline}
  \nabla \times (k \nabla \times) \mathbf{B} = \\
  +\hat{x} \left\{ +D_y k_z D_x B_y - \mathbf{D_y k_z D_y B_x} + D_z k_y D_x B_z - \mathbf{D_z k_y D_z B_x} \right\} \\
  +\hat{y} \left\{ -\mathbf{D_x k_z D_x B_y} + D_x k_z D_y B_x + D_z k_x D_y B_z - \mathbf{D_z k_x D_z B_y} \right\} \\
  +\hat{z} \left\{ -\mathbf{D_x k_y D_x B_z} + D_x k_y D_z B_x - \mathbf{D_y k_x D_y B_z} + D_y k_x D_z B_y \right\} \\
\end{multline}
It is clear that part of the diagonal is $diag(\nabla \times \nabla \times) = -\nabla \bullet (k \nabla)$ except along the component direction and where $k$ is orthogonal to the derivative direction and the component of $\nabla \times \nabla \times$. The other part will not contribute to the diagonal...?

This turns out to be the case!

\end{document}