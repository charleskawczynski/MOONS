\documentclass[landscape]{article}
\usepackage{graphicx}    % needed for including graphics e.g. EPS, PS
\usepackage{epstopdf}
\usepackage{amsmath}
\usepackage{hyperref}
\usepackage{xspace}
\usepackage{mathtools}
\usepackage{tikz}
\usepackage{epsfig}
\usepackage{float}
%\usepackage{natbib}
\usepackage{subfigure}
\usepackage{setspace}
\usepackage{tabularx,ragged2e,booktabs,caption}


% \setlength{\oddsidemargin}{0.1in}
% \setlength{\textwidth}{7.25in}

\setlength{\topmargin}{-1in}     %\topmargin: gap above header
\setlength{\headheight}{0in}     %\headheight: height of header
\setlength{\topskip}{0in}        %\topskip: between header and text
% \setlength{\headsep}{0in}        
\setlength{\textheight}{525pt}   %\textheight: height of main text
\setlength{\textwidth}{10in}    % \textwidth: width of text
\setlength{\oddsidemargin}{-0.5in}  % \oddsidemargin: odd page left margin
\setlength{\evensidemargin}{0in} %\evensidemargin : even page left margin
\setlength{\parindent}{0.25in}   %\parindent: indentation of paragraphs
\setlength{\parskip}{0pt}        %\parskip: gap between paragraphs
% \setlength{\voffset}{0.5in}


% Useful commands:

% \hfill		aligns-right everything right of \hfill

\begin{document}
\doublespacing
\title{Magnetohydrodynamic Object-Oriented Numerical Solver (MOONS)}
\author{C. Kawczynski \\
Department of Mechanical and Aerospace Engineering \\
University of California Los Angeles, USA\\
}
\maketitle

\section{Necessity of symmetrizing to achieve SPD for non-uniform grids}

\subsection{Finite difference scheme construction}

The results in this document are generated from a linear system solver, using Python, in stencilMaker.py. The file is run from stencils.py. The linear system is a set of Taylor expansions about a point in space and is of the form of a Vandermonde matrix:

\begin{equation}
  f_{i+k}
  =
  \sum_{j=0}^n
  \frac{\alpha_{i+k} f_{i}^{(j)}}{(j)!}
  \qquad \qquad
  k = n_{left} \dots n_{right}
\end{equation}

Where the unknowns are the derivatives $f_i^{(j)}$. The input arguments, $n_{left}$ and $n_{right}$, are the number of points to the left and right of $i$ respectively. $f_{i+k}\forall k$ is assumed to be known. The distance between $f_{i}$ and $f_{i+k}$ is
\begin{equation}
  \alpha_{i+k} 
  =
  h_{i+k} - h_{i}
\end{equation}
Since, in MOONS, $\Delta h_1 = h_2 - h_1$, we can relate $\Delta h$ and $\alpha$ as follows

\begin{equation}
  \Delta h_{i} = h_{i+1} - h_{i} = \alpha_{i+1}
  ,\qquad \qquad
  \Delta h_{i} + \Delta h_{i+1} = h_{i+2} - h_{i} = \alpha_{i+2}
\end{equation}

In general, we may write this as

\begin{equation}
  \alpha_{i+k} = h_{i+k} - h_{i}
  ,\qquad \qquad
  \alpha_{i+k} = \sum_{j=0}^{k-1} \Delta h_{i+j}
\end{equation}

Again, to reiterate, the indexing convention is:

\begin{equation}
  \boxed{
  \Delta h_{i}
  =
  h_{i+1} - h_{i}
  }
\end{equation}


\subsection{Cell centered finite difference schemes}

The following finite difference schemes are the result of the above analysis.

\subsubsection{1st Derivative}
\begin{equation} 
f^{{(1)}}_{i} = \left(- \frac{\Delta h_{{i}}}{\Delta h_{{i-1}} \left(\Delta h_{{i-1}} + \Delta h_{{i}}\right)}\right)f_{{i-1}}+ \left(\frac{- \Delta h_{{i-1}} + \Delta h_{{i}}}{\Delta h_{{i-1}} \Delta h_{{i}}}\right)f_{{i}}+ \left(\frac{\Delta h_{{i-1}}}{\Delta h_{{i}} \left(\Delta h_{{i-1}} + \Delta h_{{i}}\right)}\right)f_{{i+1}}
  , \qquad
  i \in (3,s-2)
 \end{equation} 

\subsubsection{2nd Derivative}
\begin{equation} 
f^{{(2)}}_{i} = \left(\frac{2}{\Delta h_{{i-1}} \left(\Delta h_{{i-1}} + \Delta h_{{i}}\right)}\right)f_{{i-1}}+ \left(- \frac{2}{\Delta h_{{i-1}} \Delta h_{{i}}}\right)f_{{i}}+ \left(\frac{2}{\Delta h_{{i}} \left(\Delta h_{{i-1}} + \Delta h_{{i}}\right)}\right)f_{{i+1}}
, \qquad
  i \in (3,s-2)
 \end{equation} 

\subsection{Symmetrizing}

\subsubsection{Laplacian}
To symmetrize the Laplacian, we ask: What can we multiply row $i$ by to ensure that $L_i=U_{i-1}$. Note that the Laplacian in a 2-step process, this is actually
\begin{equation}
   D 
   = \frac{\Delta h_{i-1} + \Delta h_{i-1}}{2} = \Delta h_{d,i}
\end{equation}
Where $\Delta h_{d,i}$ lives on the dual grid. This just turns out to be the cell volume in 3D for CC data.

\subsubsection{1st Derivative}
The same applies to the 1st derivative:
\begin{equation}
   D = - \frac{\Delta h_{i}}{\Delta h_{i-1}} \Delta h_{d,i}
\end{equation}


\subsection{Necessity of symmetrizing}

Clearly, e.g., the Laplacian stencil is a series of 2nd derivatives in different directions, and for symmetry we need

\begin{equation}
  L_{i} = U_{i-1}
\end{equation}
In which case we have
\begin{equation}
  L_{i} = \frac{2}{\Delta h_{{i-1}} \left(\Delta h_{{i-1}} + \Delta h_{{i}}\right)}
\end{equation}
and
\begin{equation}
  U_{i} = \frac{2}{\Delta h_{{i}} \left(\Delta h_{{i-1}} + \Delta h_{{i}}\right)}
  \qquad \rightarrow
  \qquad
  U_{i-1} = \frac{2}{\Delta h_{{i-1}} \left(\Delta h_{{i-2}} + \Delta h_{{i-1}}\right)}
\end{equation}
Clearly, this is not symmetric.

The coefficient to multiply will be

\begin{equation}
  c_i L_{i} = c_{i-1} U_{i-1}
\end{equation}
or
\begin{equation}
  c_i \frac{2}{\Delta h_{{i-1}} \left(\Delta h_{{i-1}} + \Delta h_{{i}}\right)} =
  c_{i-1} \frac{2}{\Delta h_{{i-1}} \left(\Delta h_{{i-2}} + \Delta h_{{i-1}}\right)}
\end{equation}
Which yields
\begin{equation}
  c_i (\Delta h_{{i-2}} + \Delta h_{{i-1}}) = 
  c_{i-1} (\Delta h_{{i-1}} + \Delta h_{{i}}) 
\end{equation}

\section{Matrix diagonal}
\begin{equation}
  D = I + \frac{\Delta t}{Re_m} \nabla \times \sigma^{-1} \nabla \times
\end{equation}

For uniform grids, I think we have
\begin{equation}
  D = I + \frac{\Delta t}{Re_m \sigma \Delta h^2}
\end{equation}




\end{document}