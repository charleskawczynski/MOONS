\documentclass[11pt]{article}
\usepackage{graphicx}    % needed for including graphics e.g. EPS, PS
\usepackage{epstopdf}
\usepackage{amsmath}
\usepackage{hyperref}
\usepackage{xspace}
\usepackage{empheq}
\usepackage{mathtools}
\usepackage{tikz}
\usepackage{epsfig}
\usepackage{float}
\usepackage{natbib}
\usepackage{subfigure}
\usepackage{setspace}
\usepackage{tabularx,ragged2e,booktabs,caption}

\newcommand*\widefbox[1]{\fbox{\hspace{2em}#1\hspace{2em}}}

\setlength{\oddsidemargin}{0.1in}
\setlength{\textwidth}{7.25in}

\setlength{\topmargin}{-1in}     %\topmargin: gap above header
\setlength{\headheight}{0in}     %\headheight: height of header
\setlength{\topskip}{0in}        %\topskip: between header and text
\setlength{\headsep}{0in}        
\setlength{\textheight}{692pt}   %\textheight: height of main text
\setlength{\textwidth}{7.5in}    % \textwidth: width of text
\setlength{\oddsidemargin}{-0.5in}  % \oddsidemargin: odd page left margin
\setlength{\evensidemargin}{0in} %\evensidemargin : even page left margin
\setlength{\parindent}{0.25in}   %\parindent: indentation of paragraphs
\setlength{\parskip}{0pt}        %\parskip: gap between paragraphs
\setlength{\voffset}{0.5in}


% Useful commands:

% \hfill		aligns-right everything right of \hfill

\begin{document}
\doublespacing
\title{Magnetohydrodynamic Object-Oriented Numerical Solver (MOONS)}
\author{C. Kawczynski \\
Department of Mechanical and Aerospace Engineering \\
University of California Los Angeles, USA\\
}
\maketitle

\section{Derivation of Theta Implicit ADI}
Consider the equation
\begin{equation}
	\frac{\partial u}{\partial t}
	=
	\alpha_x \frac{\partial^2 u}{\partial x^2} +
	\alpha_y \frac{\partial^2 u}{\partial y^2} +
	\alpha_z \frac{\partial^2 u}{\partial z^2}
	-
	f
\end{equation}
Splitting this operator into several steps, we have
\begin{equation}
	\alpha_x \Delta_x (u_{n+1}^{*} \theta_x + u_{n} (1- \theta_x)) +
	\alpha_y \Delta_y u_n +
	\alpha_z \Delta_z u_n	-
	f =
	\frac{u_{n+1}^{*} - u_{n}}{\Delta t}
\end{equation}
\begin{equation}
	\alpha_x \Delta_x (u_{n+1}^{*} \theta_x + u_{n} (1- \theta_x)) +
	\alpha_y \Delta_y (u_{n+1}^{**}\theta_y + u_{n} (1- \theta_y)) +
	\alpha_z \Delta_z u_n	-
	f =
	\frac{u_{n+1}^{**} - u_{n}}{\Delta t}
\end{equation}
\begin{equation}
	\alpha_x \Delta_x (u_{n+1}^{*} \theta_x + u_{n} (1- \theta_x)) +
	\alpha_y \Delta_y (u_{n+1}^{**}\theta_y + u_{n} (1- \theta_y)) +
	\alpha_z \Delta_z (u_{n+1}\theta_z + u_{n} (1- \theta_z)) -
	f =
	\frac{u_{n+1} - u_{n}}{\Delta t}
\end{equation}

Where $\theta = 1/2$ refers to the semi-implicit ADI by Douglas, and $\theta = 1$ refers to a fully implicit treatment of the diffusion term, which makes the system singular, so we can only approach 1. After rearranging, we have

\begin{subequations}
\begin{empheq}[box=\widefbox]{align}
	\left(
	I - \theta_x \Delta t \alpha_x \Delta_x
	\right) 
	u_{n+1}^{*}
	=
	\left(
	I + (1-\theta_x) \Delta t \alpha_x \Delta_x +
	\Delta t \alpha_y \Delta_y +
	\Delta t \alpha_z \Delta_z
	\right)
	u_n - \Delta t f \\
	\left(
	I - \theta_y \Delta t \alpha_y \Delta_y
	\right) 
	u_{n+1}^{**}
	=
	u_{n+1}^{*}
	- (1-\theta_y) \Delta t \alpha_y \Delta_y u_n \\
	\left(
	I - \theta_z \Delta t \alpha_z \Delta_z
	\right) 
	u_{n+1}
	=
	u_{n+1}^{**}
	- (1-\theta_z) \Delta t \alpha_z \Delta_z u_n
\end{empheq}
\end{subequations}


\end{document}