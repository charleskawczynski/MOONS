\documentclass[11pt]{article}
\usepackage{graphicx}    % needed for including graphics e.g. EPS, PS
\usepackage{epstopdf}
\usepackage{amsmath}
\usepackage{hyperref}
\usepackage{xspace}
\usepackage{empheq}
\usepackage{mathtools}
\usepackage{tikz}
\usepackage{epsfig}
\usepackage{float}
\usepackage{natbib}
\usepackage{subfigure}
\usepackage{setspace}
\usepackage{tabularx,ragged2e,booktabs,caption}

\newcommand*\widefbox[1]{\fbox{\hspace{2em}#1\hspace{2em}}}

\setlength{\oddsidemargin}{0.1in}
\setlength{\textwidth}{7.25in}

\setlength{\topmargin}{-1in}     %\topmargin: gap above header
\setlength{\headheight}{0in}     %\headheight: height of header
\setlength{\topskip}{0in}        %\topskip: between header and text
\setlength{\headsep}{0in}        
\setlength{\textheight}{692pt}   %\textheight: height of main text
\setlength{\textwidth}{7.5in}    % \textwidth: width of text
\setlength{\oddsidemargin}{-0.5in}  % \oddsidemargin: odd page left margin
\setlength{\evensidemargin}{0in} %\evensidemargin : even page left margin
\setlength{\parindent}{0.25in}   %\parindent: indentation of paragraphs
\setlength{\parskip}{0pt}        %\parskip: gap between paragraphs
\setlength{\voffset}{0.5in}


% Useful commands:

% \hfill		aligns-right everything right of \hfill

\begin{document}
\doublespacing
\title{Magnetohydrodynamic Object-Oriented Numerical Solver (MOONS)}
\author{C. Kawczynski \\
Department of Mechanical and Aerospace Engineering \\
University of California Los Angeles, USA\\
}
\maketitle

\section{Derivation of 3D Douglas ADI}
Consider the equation
\begin{equation}
	\frac{\partial u}{\partial t}
	=
	\alpha_x \frac{\partial^2 u}{\partial x^2} +
	\alpha_y \frac{\partial^2 u}{\partial y^2} +
	\alpha_z \frac{\partial^2 u}{\partial z^2}
	-
	f
\end{equation}
Splitting this operator into several steps, we have
\begin{equation}
	\frac{1}{2} \alpha_x \Delta_x (u_{n+1}^{*} + u_{n}) +
	\alpha_y \Delta_y u +
	\alpha_z \Delta_z u	-
	f =
	\frac{u_{n+1}^{*} - u_{n}}{\Delta t}
\end{equation}

\begin{equation}
	\frac{1}{2} \alpha_x \Delta_x (u_{n+1}^{*} + u_{n}) +
	\frac{1}{2} \alpha_y \Delta_y (u_{n+1}^{**} + u_{n}) +
	\alpha_z \Delta_z u	-
	f =
	\frac{u_{n+1}^{**} - u_{n}}{\Delta t}
\end{equation}

\begin{equation}
	\frac{1}{2} \alpha_x \Delta_x (u_{n+1}^{*} + u_{n}) +
	\frac{1}{2} \alpha_y \Delta_y (u_{n+1}^{**} + u_{n}) +
	\frac{1}{2} \alpha_z \Delta_z (u_{n+1} + u_{n}) -
	f =
	\frac{u_{n+1} - u_{n}}{\Delta t}
\end{equation}

After rearranging, we have

\begin{subequations}
\begin{empheq}[box=\widefbox]{align}
	\left(
	I - \frac{\Delta t}{2} \alpha_x \Delta_x
	\right) 
	u_{n+1}^{*}
	=
	\left(
	I + \frac{\Delta t}{2} \alpha_x \Delta_x +
	\Delta t \alpha_y \Delta_y +
	\Delta t \alpha_z \Delta_z
	\right)
	u_n - \Delta t f \\
	\left(
	I - \frac{\Delta t}{2} \alpha_y \Delta_y
	\right) 
	u_{n+1}^{**}
	=
	u_{n+1}^{*}
	- \frac{\Delta t}{2} \alpha_y \Delta_y u_n \\
	\left(
	I - \frac{\Delta t}{2} \alpha_z \Delta_z
	\right) 
	u_{n+1}
	=
	u_{n+1}^{**}
	- \frac{\Delta t}{2} \alpha_z \Delta_z u_n
\end{empheq}
\end{subequations}

\section{Amplification factor}

For simplicity, let

\begin{equation}
	D_{\theta} = \frac{\Delta t}{2} \alpha_{\theta} \Delta_{\theta}, \qquad \theta = x,y,z
\end{equation}

Now, note that the error to the discrete equation satisfies the same governing equation without the source term, namely

\begin{equation}
	(I-D_x)e_{n+1}^* = (I+D_x+2D_y+2D_z)e_n
\end{equation}
\begin{equation}
	(I-D_y)e_{n+1}^{**} = e_{n+1}^* - D_ye_n
\end{equation}
\begin{equation}
	(I-D_z)e_{n+1} = e_{n+1}^{**} - D_z e_n
\end{equation}

Therefore
\begin{equation}
	e_{n+1}^{*} = 
	\frac{(I+D_x+2D_y+2D_z)}{(I-D_x)}e_n
\end{equation}

\begin{equation}
	e_{n+1}^{**} = 
	\frac{
	\left[
	\frac{(I+D_x+2D_y+2D_z)}{(I-D_x)}-D_y
	\right]
	}{(I-D_y)}e_n
\end{equation}

\begin{equation}
	e_{n+1} = 
	\frac{
	\left[
	\frac{(I+D_x+2D_y+2D_z)}{(I-D_x)(I-D_y)}
	-
	\frac{D_y}{(I-D_y)}
	-
	D_z
	\right]
	}{(I-D_z)}e_n
\end{equation}

\begin{equation}
	e_{n+1} = 
	\left[
	\frac{(I+D_x+2D_y+2D_z)}{(I-D_x)(I-D_y)(I-D_z)}
	-
	\frac{D_y}{(I-D_y)(I-D_z)}
	-
	\frac{D_z}{(I-D_z)}
	\right]
	e_n
\end{equation}

The amplification factor is defined as

\begin{equation}
	G = \frac{e_{n+1}}{e_{n}}
\end{equation}
Finding the least common denominator, we have
\begin{equation}
	G = 
	\frac{I+D_x+2D_y+2D_z - (I-D_x)D_y - (I-D_x)(I-D_y)D_z}{(I-D_x)(I-D_y)(I-D_z)}
\end{equation}
Only looking at the numerator, we have
\begin{equation}
	G_{num} = I+D_x+2D_y+2D_z-D_y+D_x D_y - (I-D_y-D_x+D_xD_y)D_z
\end{equation}
\begin{equation}
	= I+D_x+2D_y+2D_z-D_y+D_x D_y - D_z +D_yD_z +D_xD_z - D_xD_yD_z
\end{equation}
\begin{equation}
	= I+(D_x+D_y+D_z)+(D_xD_y+D_yD_z+D_xD_z)-(D_xD_yD_z)
\end{equation}

\begin{equation}
	\boxed{
	G = 
	\frac{I+(D_x+D_y+D_z)+(D_xD_y+D_yD_z+D_xD_z)-(D_xD_yD_z)}{(I-D_x)(I-D_y)(I-D_z)}
	}
\end{equation}

Note that this is a 3rd order perturbation of 

\begin{equation}
	G = 
	\frac{I+(D_x+D_y+D_z)+(D_xD_y+D_yD_z+D_xD_z)+(D_xD_yD_z)}{(I-D_x)(I-D_y)(I-D_z)}
\end{equation}
Which may be rewritten as
\begin{equation}
	G = 
	\frac{(I+D_x)}{(I-D_x)}
	\frac{(I+D_y)}{(I-D_y)}
	\frac{(I+D_z)}{(I-D_z)}
\end{equation}
Which is the product of the three 1D amplification factors. This justifies using a timestep derived from analysis for a 1D aplification factor.


\end{document}