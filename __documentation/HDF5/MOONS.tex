\documentclass[11pt]{article}
\usepackage{graphicx}    % needed for including graphics e.g. EPS, PS
\usepackage{epstopdf}
\usepackage{amsmath}
\usepackage{hyperref}
\usepackage{xspace}
\usepackage{mathtools}
\usepackage{tikz}
\usepackage{epsfig}
\usepackage{float}
\usepackage{natbib}
\usepackage{subfigure}
\usepackage{setspace}
\usepackage{tabularx,ragged2e,booktabs,caption}


\setlength{\oddsidemargin}{0.1in}
\setlength{\textwidth}{7.25in}

\setlength{\topmargin}{-1in}     %\topmargin: gap above header
\setlength{\headheight}{0in}     %\headheight: height of header
\setlength{\topskip}{0in}        %\topskip: between header and text
\setlength{\headsep}{0in}        
\setlength{\textheight}{692pt}   %\textheight: height of main text
\setlength{\textwidth}{7.5in}    % \textwidth: width of text
\setlength{\oddsidemargin}{-0.5in}  % \oddsidemargin: odd page left margin
\setlength{\evensidemargin}{0in} %\evensidemargin : even page left margin
\setlength{\parindent}{0.25in}   %\parindent: indentation of paragraphs
\setlength{\parskip}{0pt}        %\parskip: gap between paragraphs
\setlength{\voffset}{0.5in}


% Useful commands:

% \hfill		aligns-right everything right of \hfill

\begin{document}
\doublespacing
\title{Magnetohydrodynamic Object-Oriented Numerical Solver (MOONS)}
\author{C. Kawczynski \\
Department of Mechanical and Aerospace Engineering \\
University of California Los Angeles, USA\\
}
\maketitle

\section{HDF5 - Setting up from scratch on Windows 7 with MinGW}
Note that all paths in this document is written with a forward slash "/" instead of a backward slash "\textbackslash" because a backward slash has a special meaning in LaTeX. The paths may be taken from the .tex document exactly, which is contained in comments (not shown in the PDF).

\section{Outline}
Here is an outline of what needs to be done, in a short list form

\section{Installing CMake}
CMake must be installed from

http://www.cmake.org/download/

In the table titled "Platform". Download and run the executable. Currently, cmake-3.2.1-win32-x86.exe is the latest version.

NOTE: The path may be too long to add, so chose "Do not add CMake to the system PATH"

Now, the CMake GUI should be available from the start menu.

\section{Installing HDF5}

HDF5 must be installed from

http://www.hdfgroup.org/HDF5/release/obtain5.html

In the table titled "Platform". Download both the hdf5 installer and Utilities.

Unzip the installer folder (hdf5-1.8.14) and run the executable inside.

\section{Setting environment variables (if path was too long)}
Navigate to 

computer $\rightarrow$ properties $\rightarrow$ advanced system settings $\rightarrow$ Environment Variables

And append to the PATH under "User variables for USERNAME"

C:/Program Files/HDF\_Group/HDF5/1.8.14/cmake/hdf5

And

C:/Program Files (x86)/CMake/bin

Also, set

HDF5\_DIR = C:/Users/Charlie/Desktop/MOONS - hdf5

\section{Building with CMake}

Now that HDF5 is installed, navigate to the folder "C:/Program Files/HDF\_Group " and copy the files

HDF5Examples-0.1.1-Source.zip

and 

HDF518\_Examples.CMAKE

To a new project directory. Do not unzip the zipped file. Open a command prompt in this directory and type

ctest -S HDF518\_Examples.cmake,HDF5Examples-0.1.1-Source -C Release -V -O test.log

\section{Get CMake through HDF5 website}

For some reason, using the command above doesn't seem to work well. Download the 

hdf5-CMakeWindows.zip from the top of the table in the "Build Instructions" section of 

http://www.hdfgroup.org/HDF5/release/cmakebuild.html

Unzip this file to find a folder named CMake.

Navigate to the CMake GUI

- Where is the source code: CMake/hdf5-1.8.14 folder

- Where to build the binaries: project folder containing HDF5Examples-0.1.1-Source.zip and HDF518\_Examples.CMAKE

Also, often the directory for CMAKE\_INSTALL\_PREFIX is incorrect, and must be adjusted for windows 64-bit

Press Configure. This should run smoothly through with only a couple warnings.

Check: BUILD\_SHARED\_LIBS and HDF5\_BUILD\_FORTRAN

Configure

Configure

Generate

Open the project/fortran directory and run gmake. This will take a few minutes. Once complete, .mod files and .lib files will appear in the bin project/directory.


\section{Additional notes}



\end{document}