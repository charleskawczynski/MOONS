\documentclass[11pt]{article}
\usepackage{graphicx}    % needed for including graphics e.g. EPS, PS
\usepackage{epstopdf}
\usepackage{amsmath}
\usepackage{hyperref}
\usepackage{xspace}
\usepackage{mathtools}
\usepackage{tikz}
\usepackage{epsfig}
\usepackage{float}
\usepackage{natbib}
\usepackage{subfigure}
\usepackage{setspace}
\usepackage{tabularx,ragged2e,booktabs,caption}


\setlength{\oddsidemargin}{0.1in}
\setlength{\textwidth}{7.25in}

\setlength{\topmargin}{-1in}     %\topmargin: gap above header
\setlength{\headheight}{0in}     %\headheight: height of header
\setlength{\topskip}{0in}        %\topskip: between header and text
\setlength{\headsep}{0in}        
\setlength{\textheight}{692pt}   %\textheight: height of main text
\setlength{\textwidth}{7.5in}    % \textwidth: width of text
\setlength{\oddsidemargin}{-0.5in}  % \oddsidemargin: odd page left margin
\setlength{\evensidemargin}{0in} %\evensidemargin : even page left margin
\setlength{\parindent}{0.25in}   %\parindent: indentation of paragraphs
\setlength{\parskip}{0pt}        %\parskip: gap between paragraphs
\setlength{\voffset}{0.5in}


% Useful commands:

% \hfill		aligns-right everything right of \hfill

\begin{document}
\doublespacing
\title{Magnetohydrodynamic Object-Oriented Numerical Simulation (MOONS)}
\author{C. Kawczynski \\
Department of Mechanical and Aerospace Engineering \\
University of California Los Angeles, USA\\
}
\maketitle

\section{Stretching function for non-uniform grids}
MOONS implements several stretching functions. These stretching factors were adopted from \cite{pletcher2012computational}.

\subsection{Roberts Transformation 1}
\begin{equation}
y = h \frac{(\beta + 2 \alpha) \left[ \frac{\beta+1}{\beta-1}^{(\bar{y}-\alpha)/(1-\alpha)} \right] - \beta + 2 \alpha}{(2\alpha+1)(1+\left[ \frac{\beta+1}{\beta-1}^{(\bar{y}-\alpha)/(1-\alpha)} \right])}
\end{equation}

Where $h$ is the length of the domain. Note that

\begin{equation}
	\alpha = 0 \qquad \rightarrow \text{mesh will be refined near $y=h$ only}
\end{equation}
\begin{equation}
	\alpha = 1/2 \qquad \rightarrow \text{mesh will be refined near $y=h$ and $y=0$}
\end{equation}
\begin{equation}
	\bar{y} = \text{uniformly spaced grid}
\end{equation}

If there exists a boundary layer of thickness $\delta$, then an adequate stretching parameter, $\beta$, may be chosen as

\begin{equation}
	\beta = \left( 1 - \frac{\delta}{h} \right)^{-1/2}
\end{equation}


\subsection{Roberts Transformation 2}

\subsection{Roberts Transformation 3}


\bibliography{MOONS}
\bibliographystyle{plain}


\end{document}


