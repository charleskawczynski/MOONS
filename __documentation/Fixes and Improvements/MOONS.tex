\documentclass[11pt]{article}
\usepackage{graphicx}    % needed for including graphics e.g. EPS, PS
\usepackage{epstopdf}
\usepackage{amsmath}
\usepackage{hyperref}
\usepackage{xspace}
\usepackage{mathtools}
\usepackage{tikz}
\usepackage{epsfig}
\usepackage{float}
\usepackage{natbib}
\usepackage{subfigure}
\usepackage{setspace}
\usepackage{tabularx,ragged2e,booktabs,caption}


\setlength{\oddsidemargin}{0.1in}
\setlength{\textwidth}{7.25in}

\setlength{\topmargin}{-1in}     %\topmargin: gap above header
\setlength{\headheight}{0in}     %\headheight: height of header
\setlength{\topskip}{0in}        %\topskip: between header and text
\setlength{\headsep}{0in}        
\setlength{\textheight}{692pt}   %\textheight: height of main text
\setlength{\textwidth}{7.5in}    % \textwidth: width of text
\setlength{\oddsidemargin}{-0.5in}  % \oddsidemargin: odd page left margin
\setlength{\evensidemargin}{0in} %\evensidemargin : even page left margin
\setlength{\parindent}{0.25in}   %\parindent: indentation of paragraphs
\setlength{\parskip}{0pt}        %\parskip: gap between paragraphs
\setlength{\voffset}{0.5in}


% Useful commands:

% \hfill		aligns-right everything right of \hfill

\begin{document}
\doublespacing
\title{Magnetohydrodynamic Object-Oriented Numerical Simulation (MOONS)}
\author{C. Kawczynski \\
Department of Mechanical and Aerospace Engineering \\
University of California Los Angeles, USA\\
}
\maketitle

\section{Document Rules}
In order to keep this document organized
\begin{itemize}
\item Refer to this document for 1 hour each morning to asses priorities
\item Keep fewer than 5 items in Goals and Requirements / Today / Tomorrow
\item Keep fewer than 10 sections total in this document
\end{itemize}

\section{Goals / Requirements}

- Investigate validity of pseudo-vacuum BCs compared with true vacuum BCs

- Compare methods for enforcing divB and trade-offs
--Plot(Rem vs Runtime vs div(B))

- Mimicking plasma disruption

- Benchmarking (convergence rates / Hydro flows / Hunt and Shercliff flow)


\section{Today}

- Sweep through Re/Ha/Rem while testing convergence / Compute KE and KB equations and plot norms of terms vs time

- First step is descriptive - phases (linear, non-linear, decay)

- Look at KE equation and Bfield energy equation / read literature

- Read more about benchmarking

- Develop routine for Richardson Extrapolation

- Look in literature for interface conditions for current and BCs for bfield

- Figure out how to view time in tecplot as simulation runs


\section{Tomorrow}
- diff/adv terms on wall coincident boundaries

- Develop routine for Richardson Extrapolation to estimate rate of convergence

- Rigorously test gridGen and get rid of old one.


\section{Future Plans}

- Make a prepare directory routine in MOONS

- Try refinement studies to make sure order of accuracy is correct

- Compute residuals for momentum/induction equations allow for transient exporting using pre-processor directives

- Make Poisson solvers (all) initialized outside of the call routine.

- Look into benchmarking these solvers by looking at convergence rates

- Introduce staggered B so it can be cleaned

- Test cleaning procedure and compare with CT method

- Consider making setXMinType(1,2) for dirichlet / neumann and determine wall coincident / incoincident internally.

- Make a uniform grid option for boundary layers (maybe make a function for this)

- Implement Newtons method for estimating betaw. Check if betai/betaw
are above 1000, if so just use uniform grid.

- Profile to find latest bottle necks

- Look into how expensive shape() is.

- Calculate cw in induction solver!

\section{Futuristic Plans}

- Try changing smoother to Jacobi and check convergence rate (including when coarsest level is \textit{very} coarse)

- Check Non-uniform N/CC data grids for D/N BCs

- Try Jack's Experiment

- Fix Time module: export and import time so that the average can be appropriately
printed to the screen.

- Consider making applyFaceBCs, applyEdgeBCs, applyCornerBCs and interpolate appropriately. This seems like the most general and care-free approach. At least implement the capability, and offer for using applyFaceBCs in a non-arbitrary order.

- Implement an upwind scheme for the non-uniform grid.

- Make pressure term 2nd order accurate in time (trap)

- Make advection term 2nd order accurate in time (AB2)

- Test ADI for LDC case

- Right now, there are many things that will (likely) break when MOONS tries to restart a simulation. Restarting may require manual overrides on certain things, so generalizing restart for many modules is difficult.

- Consider different time marching method (RK4 - eldredge)

- Consider re-formulating using conservative higher order finite difference methods by Moin

- Consider using pre-processor directives instead of if statements, this needs to be organized on a large level.

- Get 3D douglas ADI working for fixed time step

- Get 3D douglas ADI working for non-uniform grids

- Get 3D douglas ADI working for variable $\alpha$

- Look into Kolmogorov scale for LDC and try DNS

- Make a BMC for a purely hydrodynamic turbulent flow (Re=2,000 and 3,200 compare with literature)

- Consider making an obstacle module that allows for more complex flows
e.g. flow over a sphere

- Consider making an obstacle module that is passed into the solver.
This could assign the velocity to satisfy any desired BCs and internal
velocity (likely u=0 inside). More tests could be done this way.

- implement: 1) chimney flow 2) impinging jet flow

- test the cylinder driven flow case =)

- Finish developing solverSettings estimateRemaining() subroutine

- Solve energy equation!

- Allow for compressibility!

- Use higher order spatial operators! (stencils.f90 / interpOps.f90 / applyBCs.f90)

- Use hgiher order time marching methods! (RK4 maybe?)

- Update export: make a switch to control what to export:
      U-field: (face, cc, node)\_interior (face, cc, node)\_total domain
      B-field: (face, cc, node)\_interior (face, cc, node)\_total domain
      J-field: (face, cc, node)\_interior (face, cc, node)\_total domain

\section{10 minute tasks}
- remove 'my' in all of MOONS.

- Convert real(cp),dimension(:,:,:) to scalar field for parallelization

- rename interpOps.f90 to ops\_interp.f90.

- rename scalarField and vectorField to field\_scalar and field\_vector

- Update matlab script to plot KU and KB


\section{Analysis and Literature}
- Consider trying to assume separable with space and time, B induced = Bstar exp(-omega t). Exponents may cancel. Velocity will not be trivial though. But perhaps at early times motion is due to diffusion only.

- Paper by Priede - Linear stability on Hunt flow ("not easy")

- Work on analysis for simple PD

- Read thesis sent by Naveen

\section{Unsorted}

- The zeroWallCoincidentPoints in momentum solver decreased div(U) by several orders of magnitude. After changes below, try removing this routine call and see if it still affects div(U).

- Change induction equation to be located on the cell face. This allows a natural CT method implementation, as well as allowing for cleaning of B for Gauss-Seidel Solver.

- Re-run several benchmark tests.

- div(B) cleaning for face data

- Try making properties of mom/ind/nrg private!

- Remove unecessary sigma fields

- Test Poisson cases for face data

- Test inductionSolver: Gauss-Seidel / cleaning procedure

- Finish documentation: non-uniform grid stencils/outline of results: develop a python script to solve the system symbollically.

- Look into Paul Roberts (UCLA)

- Fix the explicit Neumann BCs in apply BCs. It looks like only 1st or 2nd order accuracy is capable for du/dn = 0

- Right now there are two instances of restartU and restartB, this inconsistency must to be resolved.

- Re-run Hunt/Shercliff flows with uniform grid in BL and with p = 0 at outlet

- Consider using ADI for uniform grids, this may be the best way to look at validity of pseudo vacuum BCs

- Maybe ADI is not correct (even though it seems to be) since it DOES NOT work (diverges) for large $\Delta t$.

- Maybe look at older versions of ADI

- Update documentation for stencils (include ghost node)

- Maybe include documentation on Nin,Nice,Nici?

- Look into paper by Pattsson? Morely co-authored ~2008/2009

- Try comparing 1/Ha with cw for HIMAG vs MOONS and compare continuity of tangent(j)/sigma across interface

- IMPORTANT: del.f90 has no safe way to assign the output when pad is present. This is VERY dangerous and needs to be addressed immediately!

- For Poisson solver, check if mean(f) or mean(f\_interior) must = 0

- Member of committee for Naveen was in Geo-physics, Jonathon "Arnaud"? Look online about him and maybe we can collaborate.

\end{document}