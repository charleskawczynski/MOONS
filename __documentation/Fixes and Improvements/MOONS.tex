\documentclass[11pt]{article}
\usepackage{graphicx}    % needed for including graphics e.g. EPS, PS
\usepackage{epstopdf}
\usepackage{amsmath}
\usepackage{hyperref}
\usepackage{xspace}
\usepackage{mathtools}
\usepackage{tikz}
\usepackage{epsfig}
\usepackage{float}
\usepackage{natbib}
\usepackage{subfigure}
\usepackage{setspace}
\usepackage{tabularx,ragged2e,booktabs,caption}


\setlength{\oddsidemargin}{0.1in}
\setlength{\textwidth}{7.25in}

\setlength{\topmargin}{-1in}     %\topmargin: gap above header
\setlength{\headheight}{0in}     %\headheight: height of header
\setlength{\topskip}{0in}        %\topskip: between header and text
\setlength{\headsep}{0in}        
\setlength{\textheight}{692pt}   %\textheight: height of main text
\setlength{\textwidth}{7.5in}    % \textwidth: width of text
\setlength{\oddsidemargin}{-0.5in}  % \oddsidemargin: odd page left margin
\setlength{\evensidemargin}{0in} %\evensidemargin : even page left margin
\setlength{\parindent}{0.25in}   %\parindent: indentation of paragraphs
\setlength{\parskip}{0pt}        %\parskip: gap between paragraphs
\setlength{\voffset}{0.5in}


% Useful commands:

% \hfill		aligns-right everything right of \hfill

\begin{document}
\doublespacing
\title{Magnetohydrodynamic Object-Oriented Numerical Simulation (MOONS)}
\author{C. Kawczynski \\
Department of Mechanical and Aerospace Engineering \\
University of California Los Angeles, USA\\
}
\maketitle

\section{Now}

- Start / Develop document for periodicDuct1BfieldComponent

- Look at KE equation and Bfield energy equation / read literature

- Read more about benchmarking

- Develop routine for Richardson Extrapolation

- Look in literature for interface conditions for current and BCs for bfield

- Figure out how to view time in tecplot as simulation runs

\section{Next}

- Big concerns include 
	- diff/adv terms on wall coincident boundaries
	- benchmarking unsteady flow problems
	- Asymmetry from BMC\_1003 with movie

- Attempts to fix asymmetry
	- look into (interp.f90 / stencils.f90 / applyBCs.f90 / embedExtract.f90)
	- Try again using fine grid / uniform grid

- Consider different time marching method (RK4 maybe)

- Consider trying to assume separable with space and time, B induced = Bstar exp(-omega t). Exponents may cancel. Velocity will not be trivial though. But perhaps at early times motion is due to diffusion only.

- Paper by Priede - Linear stability on Hunt flow ("not easy")

- First step is descriptive - phases (linear, non-linear, decay)

- Plot movie / Identify regimes / sweep through Rem / Compute KE and KB equations and plot norms of terms vs time

- Look at diffusion term on boundaries

- Update matlab script to plot KU and KB

- Work on analysis for simple PD

- Make a module for embed extract

- Develop routine for Richardson Extrapolation to estimate rate of convergence

- Run some simple PD sims

- Prep document for $B^0 = (e^{-t},0,0)$ sim

- Try to develop routines for staggering B

- Start simulations for $B^0 = (e^{-t},0,0)$ sim

- Compare interp(u\_CCxB\_CC)\_E with (interp(u\_F)\_E x interp(B\_CC)\_E)\_E for BMC\_102. Note that the source term in gauss-seidel may need to be carefully defined.


\section{Today}

- Consider making setXMinType(1,2) for dirichlet / neumann and determine wall coincident / incoincident internally.

- Merge with master, commit current version

- In order to fully stagger B, there is a need to compute $(u_{face}\times B_{face})_{edge}$ for the CT method and
$\frac{\partial}{\partial x_j} (u_j B_i-u_iB_j)_{face}$ for Gauss Seidel on the face. This is not easy, but certainly not impossible. Note that j is a dummy index, so interpolating to the center and taking the dervative to the face will not work.

- Rigorously test gridGen and get rid of old one.

- Rename myError to norms.

- remove my in interpolations.

- rename interpOps.f90 to ops\_interp.f90.

- rename scalarField and vectorField to field\_scalar and field\_vector

- Convert real(cp),dimension(:,:,:) to scalar field for parallelization

- Plot 1D results for plasma disruption cases for inflection points

- The zeroWallCoincidentPoints in momentum solver decreased div(U) by several orders of magnitude. After changes below, try removing this routine call and see if it still affects div(U).

- Make sure diffusion term is being computed correctly (should be equivalent to one-sided difference formula). 

- E = j/sig - uxB, u must be interpolated before multiplication.

- Change induction equation to be located on the cell face. This allows a natural CT method implementation, as well as allowing for cleaning of B for Gauss-Seidel Solver.

- Re-run several benchmark tests. The uxB interpolation being done before multiplication might fix this issue of jxB not balancing pressure well (since div(U) now seems to be a problem for these plasma disruption flows)

- Send Sergey results so far with caveat about 1) boundary source terms 2) uxB interp 3) induction-> face instead (BCs will be applied on the cell face instead of cell center), this means a second interpolation to compute J will no longer be necessary which means the result may be more accurate.

- Try Jack's Experiment

- div(B) cleaning for face data

- Try making properties of mom/ind/nrg private!

- Remove unecessary sigma fields

- Test Poisson cases for face data

- Test inductionSolver: Gauss-Seidel / cleaning procedure

- Finish documentation: non-uniform grid stencils/outline of results: develop a python script to solve the system symbollically.

- Look into Paul Roberts (UCLA)

- Consider re-writing advect donor using interpolation and div

- Try to see if B\_face can be used to compute J at the start of the induction solver.

- Fix the explicit Neumann BCs in apply BCs. It looks like only 1st or 2nd order accuracy is capable for du/dn = 0

- Make a summary of test cases performed so far
-- U-field / B-field configuration 
-- analytic solutions (if they exist)
-- comparison results / discussion

- Organize folders

\section{This File}
This file contains all of the things that must be fixed / adjusted and improved in MOONS. It is organized into pre-processing, Solution procedure, and post-processing sections to help manage organization of basically a glorified list.

\section{Goals and Requirements}

- Investigate validity of pseudo-vacuum BCs compared with true vacuum BCs

- Compare methods for enforcing divB and trade-offs
--Plot(Rem vs Runtime vs div(B))

- Mimicking plasma disruption

- Benchmarking (convergence rates / Hydro flows / Hunt and Shercliff flow)

\section{To Do and Notes}
- Right now there are two instances of restartU and restartB, this inconsistency must to be resolved.

- Re-run Hunt/Shercliff flows with uniform grid in BL and with p = 0 at outlet

- Consider using ADI for uniform grids, this may be the best way to look at validity of pseudo vacuum BCs

- Maybe ADI is not correct (even though it seems to be) since it DOES NOT work (diverges) for large $\Delta t$.

- Maybe look at older versions of ADI

- Update documentation for stencils (include ghost node)

- Maybe include documentation on Nin,Nice,Nici?

- Look into paper by Pattsson? Morely co-authored ~2008/2009

- Try comparing 1/Ha with cw for HIMAG vs MOONS and compare continuity of tangent(j)/sigma across interface

- IMPORTANT: del.f90 has no safe way to assign the output when pad is present. This is VERY dangerous and needs to be addressed immediately!

- For Poisson solver, check if mean(f) or mean(f\_interior) must = 0

- Member of committee for Naveen was in Geo-physics, Jonathon "Arnaud"? Look online about him and maybe we can collaborate.

- Continue digitizing books

- Read thesis sent by Naveen

- Clean office - get paper towels

- Start writing thesis

- Get armorall wipes for car

\section{Issues}

- The residuals for the PPE are large, but don't seem to be reflected in the solution. Compute and export the residuals and check where the residuals are large for both D and N BCs.

\section{Code verification}

- Try refinement studies to make sure order of accuracy is correct

- Compute residuals for momentum/induction equations allow for transient exporting using pre-processor directives

\section{Pre-processor Directives}
- Consider using pre-processor directives instead of if statements, this needs to be organized on a large level.

\section{Restart capability}
Right now, there are many things that will (likely) break when MOONS tries to restart a simulation. Restarting may require manual overrides on certain things, so generalizing restart for many modules is difficult.

- Fix Time module: export and import time so that the average can be appropriately
printed to the screen.



\section{Mesh Generation}

- Compare new and old grid generation (quantitatively)

- Make a uniform grid option for boundary layers (maybe make a function for this)

- Implement Newtons method for estimating betaw. Check if betai/betaw
are above 1000, if so just use uniform grid.


\section{BMCs}

- Re-run BMC against Guj and Stella with hmin = -1, hmax = 1 and adjusted Re like Sergey
  did. This will reduce the number of geometries.

\section{Performance}
- Profile to find latest bottle necks

- Parallelize SOR on larger level

- Look into a zeroGhost and zeroGhostAndBoundary or something to avoid using assignments like f = 0 (which operates on the whole domain)

- Look into how expensive shape() is.


\section{Stencils.f90}
- Consider re-formulating using conservative higher order finite difference methods by Moin

\section{Boundary Conditions}
- Highly consider making applyFaceBCs, applyEdgeBCs, applyCornerBCs and interpolate appropriately. This seems like the most general and care-free approach. At least implement the capability, and offer for using applyFaceBCs in a non-arbitrary order.

\section{Momentum Solver}

- Implement an upwind scheme for the non-uniform grid.

- Make pressure term 2nd order accurate in time (trap)

- Make advection term 2nd order accurate in time (AB2)

- Test ADI for LDC case


\section{Induction Solver}

- Also try the 'consistent' fringing field and run caseB2

- Calculate cw in induction solver!

\section{CT Method}

- Check if maxval(jxe),maxval(jye),maxval(jze) are zero with CT method (should be
  close to zero (numerically) if equivalent to when mixed term is ignored) for non-uniform grids


\subsection{Cleaning procedure}
- Introduce staggered B so it can be cleaned

- Test cleaning procedure and compare with CT method

\section{Poisson Solvers}
- Make Poisson solvers (all) initialized outside of the call routine.

- Look into benchmarking these solvers by looking at convergence rates

\subsection{3D Douglas ADI}
- Get working for fixed time step

- Get working for non-uniform grids

- Get working for variable $\alpha$

\subsection{Multi-grid}
- Try changing smoother to Jacobi and check convergence rate (including when coarsest level is \textit{very} coarse)

- Check Non-uniform N/CC data grids for D/N BCs

\section{Exporting data}
- Make a prepare directory routine in MOONS

- HDF5 - get working

- Make interfaces for binary file output

- Update export: make a switch to control what to export:
      U-field: (face, cc, node)\_interior (face, cc, node)\_total domain
      B-field: (face, cc, node)\_interior (face, cc, node)\_total domain
      J-field: (face, cc, node)\_interior (face, cc, node)\_total domain

- Make a dynamic 'exportSolutionNow' parameter that is read
with the same frequency as the killSwitch



\section{Experimental / fun / Low priority}
- Look into Kolmogorov scale for LDC and try DNS

- Make a BMC for a purely hydrodynamic turbulent flow (Re=2,000 and 3,200 compare with literature)

- Consider making an obstacle module that allows for more complex flows
e.g. flow over a sphere

- Consider making an obstacle module that is passed into the solver.
This could assign the velocity to satisfy any desired BCs and internal
velocity (likely u=0 inside). More tests could be done this way.

- implement: 1) chimney flow 2) impinging jet flow

- test the cylinder driven flow case =)

- Finish developing solverSettings estimateRemaining() subroutine

- Solve energy equation!

- Allow for compressibility!

- Use higher order spatial operators! (stencils.f90 / interpOps.f90 / applyBCs.f90)

- Use hgiher order time marching methods! (RK4 maybe?)


\end{document}