\documentclass[11pt]{article}
\usepackage{graphicx}    % needed for including graphics e.g. EPS, PS
\usepackage{epstopdf}
\usepackage{amsmath}
\usepackage{hyperref}
\usepackage{xspace}
\usepackage{mathtools}
\usepackage{tikz}
\usepackage{epsfig}
\usepackage{float}
\usepackage{natbib}
\usepackage{subfigure}
\usepackage{setspace}
\usepackage{tabularx,ragged2e,booktabs,caption}


\setlength{\oddsidemargin}{0.1in}
\setlength{\textwidth}{7.25in}

\setlength{\topmargin}{-1in}     %\topmargin: gap above header
\setlength{\headheight}{0in}     %\headheight: height of header
\setlength{\topskip}{0in}        %\topskip: between header and text
\setlength{\headsep}{0in}        
\setlength{\textheight}{692pt}   %\textheight: height of main text
\setlength{\textwidth}{7.5in}    % \textwidth: width of text
\setlength{\oddsidemargin}{-0.5in}  % \oddsidemargin: odd page left margin
\setlength{\evensidemargin}{0in} %\evensidemargin : even page left margin
\setlength{\parindent}{0.25in}   %\parindent: indentation of paragraphs
\setlength{\parskip}{0pt}        %\parskip: gap between paragraphs
\setlength{\voffset}{0.5in}


% Useful commands:

% \hfill		aligns-right everything right of \hfill

\begin{document}
\doublespacing
\title{Magnetohydrodynamic Object-Oriented Numerical Solver (MOONS)}
\author{C. Kawczynski \\
Department of Mechanical and Aerospace Engineering \\
University of California Los Angeles, USA\\
}
\maketitle

\section{Document Rules}
In order to keep this document organized
\begin{itemize}
\item Refer to this document for 1 hour each morning to asses priorities
\item Keep fewer than 5 items in Goals and Requirements / Today / Tomorrow
\item Keep fewer than 10 sections total in this document
\end{itemize}

\section{Goals / Requirements}

\begin{itemize}
\item Investigate validity of pseudo-vacuum BCs compared with true vacuum BCs
	\begin{itemize}
	\item Read papers on coupling fluid and vacuum domains
	\item Read papers / start writing
	\end{itemize}
\item Finite $Re_m$ Shercliff/Hunt flows
\item Mimicking plasma disruption
\end{itemize}


\section{Today}

- Read Moreau book about energy transfer, write up some documentation for new implementation

- Figure out compilation error for debug mode. This is becoming a more pressing issue.

- There are some gaps in the mesh generation and grid stretch-parameter matching. This needs to be generalized in a simple and easy to use way.

- betaRobertsRight(hmin,hmax,L,dh) - is this dh on the left side or right side? I think betaRoberts routines need a dh1 or dhend input, because when connecting, betaRobertsRight makes stretching occur on right-side, but needs dh on right side, which is a problem for con\_APP\_Roberts\_R.

- Organize scanned notes from gmail to folders.

- Baby steps towards BEM.

- Complex geometry - do some more testing.

- Develop organization of prospectus document

- Plot KB in vacuum in BC sim, see if that differs more / less then KBi\_f.

- Run prelim cases

- Work on paper.

- Need to re-develop restart capability... This will require some thought.

- Move the makeDir to the initialization routines (mom,ind,nrg)

- Try to incorporate anti-symmetric BCs again...

- Try re-incorporating better parallelization for collocated / staggered VF / TF fields

- Consider MG with Jacobi method, need to restrict $\sigma$...

- Try cleaning procedure with $\mathbf{B}_{tangent}=0$ and $\frac{\partial \phi}{\partial n} = \nabla \bullet \mathbf{B}^*$ and see if cleaning works. This might be equivalent to Pseudo vacuum. Also, look into paper / source that discussed this.


- Eventually, the lap\_centered routine should be replaced by a allocation-less one. This will be complicated though because this may require changing the matrix-free operators interface. A better, more general solution may be needed.

- Eventually, checks of face/node/CC/edge data could be made in routines like advect\_B and advect\_U, which could help prevent searching for what is a incorrectly located variable into a subroutine that results in a interpolation or del failure.

- Re-consider HDF5

\section{Tomorrow}

- Documentation for benchmark cases (methodology, format, execution)

- Decide on format of input file (what parameters will be set in input file and what will be coded)

- Look to improve performance of apply\_edges

- Consider new organization of solver folder

- Continue fixing file names and folder names for better organization

- Consider/try making solvers (momentum/induction) functional instead of OO.

- Read plasma papers

- First step of SK flow is descriptive - phases (linear, non-linear, decay)

- Read literature for interface conditions BCs for bfield

\section{Future Plans}

- Make a coordinates / grid import and corresponding corrections to the readme routines in the IO files.

- Try performing refinement on 2D flow

- Look at KE equation and Bfield energy equation

- Figure out how to view time in tecplot as simulation runs

- Naming convention for all ops should be (Real,SF,VF) = (real(cp),scalarField,vectorField)

- Convert matlab to python.

- Make a python routine to compute the number of lines in MOONS

- Make a prepare directory routine in MOONS

- Make Poisson solvers (all) initialized outside of the call routine and look into benchmarking these solvers by looking at convergence rates

- Introduce staggered B so it can be cleaned and test cleaning procedure and compare with CT method

- Consider making setXMinType(1,2) for dirichlet / neumann and determine wall coincident / incoincident internally.

- Make a uniform grid option for boundary layers (maybe make a function for this)

- Implement Newtons method for estimating betaw. Check if betai/betaw
are above 1000, if so just use uniform grid.

- Profile to find latest bottle necks

- Look into how expensive shape() is.

- Calculate cw in induction solver!

- Try out pre-processor directive '-ftree -parallelize -loops=n' = split loops into n-threads

- Look into different forms of $\mathbf{j}\times \mathbf{B}$, for example 
decomposing into $\mathbf{j}\times \mathbf{B} = (\mathbf{B}\bullet\nabla) \left(\frac{\mathbf{B}}{\mu}\right) - \nabla \left( \frac{\mathbf{B}^2}{2\mu} \right)$ where the second term is the magnetic pressure

\section{Futuristic Plans}

Dimensionless parameters to add:

Elsasser number = ration of lorentz to coriolis forces = A = sigma B\^2 / (2 rho Omega)

Ra = t\_k t\_nu / t\_f\^2

Where t\_f is the free fall time = sqrt(H/(alpha g dT))

Q = Chandrasekhar number = ratio of magneto-viscous diffusion time scales and the square of the alfven speed crossing time
 = t\_eta t\_nu / t\_A\^2

Eckmann number = ratio of rotation time to diffusion time
 = t\_Omega / t\_nu = nu / (2 Omega H\^2)


Convective Reynolds number = Re\_c = t\_nu / t\_f

- Try changing smoother to Jacobi and check convergence rate (including when coarsest level is \textit{very} coarse)

- Check Non-uniform N/CC data grids for D/N BCs

- Try Jack's Experiment

- Fix Time module: export and import time so that the average can be appropriately
printed to the screen.

- Consider making applyFaceBCs, applyEdgeBCs, applyCornerBCs and interpolate appropriately. This seems like the most general and care-free approach. At least implement the capability, and offer for using applyFaceBCs in a non-arbitrary order.

- Implement an upwind scheme for the non-uniform grid.

- Make pressure term 2nd order accurate in time (trap)

- Make advection term 2nd order accurate in time (AB2)

- Test ADI for LDC case

- Right now, there are many things that will (likely) break when MOONS tries to restart a simulation. Restarting may require manual overrides on certain things, so generalizing restart for many modules is difficult.

- Consider different time marching method (RK4 - eldredge)

- Consider re-formulating using conservative higher order finite difference methods by Moin

- Consider using pre-processor directives instead of if statements, this needs to be organized on a large level.

- Get 3D douglas ADI working for fixed time step

- Get 3D douglas ADI working for non-uniform grids

- Get 3D douglas ADI working for variable $\alpha$

- Look into Kolmogorov scale for LDC and try DNS

- Make a BMC for a purely hydrodynamic turbulent flow (Re=2,000 and 3,200 compare with literature)

- Consider making an obstacle module that allows for more complex flows
e.g. flow over a sphere

- Consider making an obstacle module that is passed into the solver.
This could assign the velocity to satisfy any desired BCs and internal
velocity (likely u=0 inside). More tests could be done this way.

- implement: 1) chimney flow 2) impinging jet flow

- test the cylinder driven flow case

- Finish developing solverSettings estimateRemaining() subroutine

- Allow for compressibility!

- Use higher order spatial operators! (stencils.f90 / interpOps.f90 / applyBCs.f90)

- Use hgiher order time marching methods! (RK4 maybe?)

- Update export: make a switch to control what to export:
      U-field: (face, cc, node)\_interior (face, cc, node)\_total domain
      B-field: (face, cc, node)\_interior (face, cc, node)\_total domain
      J-field: (face, cc, node)\_interior (face, cc, node)\_total domain

\section{10 minute tasks}
- Update matlab script to plot KU and KB


\section{Analysis and Literature}
- Paper by Priede - Linear stability on Hunt flow ("not easy")

- Work on analysis for simple PD

- Read thesis sent by Naveen

\section{Unsorted}

- CG

- Stencils

- KE/ME equation terms

- Re-run several benchmark tests.

- Try making properties of mom/ind/nrg private!

- Remove unecessary sigma fields

- Finish documentation: non-uniform grid stencils/outline of results: develop a python script to solve the system symbollically.

- Look into Paul Roberts (UCLA)

- Fix the explicit Neumann BCs in apply BCs. It looks like only 1st or 2nd order accuracy is capable for du/dn = 0

- Right now there are two instances of restartU and restartB, this inconsistency must to be resolved.

- Re-run Hunt/Shercliff flows with uniform grid in BL and with p = 0 at outlet

- Update documentation for stencils (include ghost node)

- Maybe include documentation on Nin,Nice,Nici?

- Try comparing 1/Ha with cw for HIMAG vs MOONS and compare continuity of tangent(j)/sigma across interface

- IMPORTANT: del.f90 has no safe way to assign the output when pad is present. This is VERY dangerous and needs to be addressed immediately!

- For Poisson solver, check if mean(f) or mean(f\_interior) must = 0

- Member of committee for Naveen was in Geo-physics, Jonathon "Arnaud"? Look online about him and maybe we can collaborate.


\end{document}