\documentclass[11pt]{article}
\usepackage{graphicx}    % needed for including graphics e.g. EPS, PS
\usepackage{epstopdf}
\usepackage{amsmath}
\usepackage{hyperref}
\usepackage{xspace}
\usepackage{mathtools}
\usepackage{tikz}
\usepackage{epsfig}
\usepackage{float}
\usepackage{natbib}
\usepackage{subfigure}
\usepackage{setspace}
\usepackage{tabularx,ragged2e,booktabs,caption}
\usepackage{bbding}


\setlength{\oddsidemargin}{0.1in}
\setlength{\textwidth}{7.25in}

\setlength{\topmargin}{-1in}     %\topmargin: gap above header
\setlength{\headheight}{0in}     %\headheight: height of header
\setlength{\topskip}{0in}        %\topskip: between header and text
\setlength{\headsep}{0in}        
\setlength{\textheight}{692pt}   %\textheight: height of main text
\setlength{\textwidth}{7.5in}    % \textwidth: width of text
\setlength{\oddsidemargin}{-0.5in}  % \oddsidemargin: odd page left margin
\setlength{\evensidemargin}{0in} %\evensidemargin : even page left margin
\setlength{\parindent}{0.25in}   %\parindent: indentation of paragraphs
\setlength{\parskip}{0pt}        %\parskip: gap between paragraphs
\setlength{\voffset}{0.5in}


% Useful commands:

% \hfill		aligns-right everything right of \hfill

\begin{document}
\doublespacing
\title{Magnetohydrodynamic Object-Oriented Numerical Solver (MOONS)}
\author{C. Kawczynski \\
Department of Mechanical and Aerospace Engineering \\
University of California Los Angeles, USA\\
}
\maketitle

\section{Long Term To Do}

\begin{itemize}
\item Write prospectus, and refer to http://fire.pppl.gov/fpa\_annual\_meet.html
\begin{itemize}
\item Code development and benchmarking \Checkmark
\item Investigate validity of pseudo-vacuum BCs compared with real vacuum BCs
\item Finite $Re_m$ Shercliff / Hunt flows, analytic solutions / perturbation analysis for secondary flow
\item SK flow, analytic solutions / perturbation analysis for secondary flow
\item LM MHD flows in the presence of plasma disruptions
\end{itemize}
\end{itemize}

\newpage

\section{Papers / Writing}

\subsection{BC paper}
\begin{itemize}
\setlength\itemsep{-1em}
\item Include all BCs formally, point out that "Psuedo-vacuum" is the insulating case of thin-walled BCs
\item Meet with Sergey to show energy budget terms before post-processing all sims.
\item Read Moreau book about energy transfer, write up some documentation for new implementation
\item Plot KB in vacuum in BC sim, see if that differs more / less then KBi\_f.
\item Look into different forms of $\mathbf{j}\times \mathbf{B}$, e.g. $\mathbf{j}\times \mathbf{B} = (\mathbf{B}\bullet\nabla) \left(\frac{\mathbf{B}}{\mu}\right) - \nabla \left( \frac{\mathbf{B}^2}{2\mu} \right)$
\item Write a Python script to compare PV / RV BC sims
\item Means of comparison
\begin{itemize}
\setlength\itemsep{-1em}
\item KE equation and ME equation
\item Energy budget
\item Direct primitives (1-D)
\item qualitative plots (stream-traces)
\item Energy content - location of maximum energy
\end{itemize}
\item focus on difference between PV and RV
\item setup post-processing for J energy (don't need sigma since it's uniform through fluid-solid)
\item Finalize energy budget in code
\item restart  mesh? to make energy budget easier?
\item restart solution to make easier
\item brainstorm what to plot
\item way to detect $\delta_{Ha}$?
\item Use thin wall BC for driven cavity
\item 2nd order forward diff may be necessary to get higher accuracy since the BC is partly Neumann.
\item Remove color: symbols. filled/empty = RV/PV. Arrows for components. (left,up,right) = (z,y,x)
\item consider making python output include $<$sub$>$\_x,y,z$<$\textbackslash sub$>$ so that legends in tecplot have subscripts
\item Start new sims that bridge $Re_m$
\end{itemize}

\subsection{Finite Rem Shercliff / Hunt Flow}
\begin{itemize}
\setlength\itemsep{-1em}
\item Read papers
\item Paper by Priede - Linear stability on Hunt flow ("not easy")
\item Perform systematic numerical experiments to show if secondary flow is physical or not. Parameters to vary:
\begin{itemize}
\setlength\itemsep{-1em}
\item Magnetic Reynolds number (controls the magnitude of secondary flow)
\item Mesh density (to show results are not numerical)
\item Hartmann number (This is a factor in the source term for the secondary flow equation)
\end{itemize}
\begin{itemize}
\setlength\itemsep{-1em}
\item 3 sims: $Re = 100, Re_m = 1, Ha = 10, N_{cells} = (1,64,64), (1,96,96), (1,128,128)$
\item 3 sims: $Re = 100, Re_m = 100, Ha = 10, N_{cells} = (1,64,64), (1,96,96), (1,128,128)$
\end{itemize}
\end{itemize}

\subsection{Plasma Disruption - LM MHD flows in its presence}
\begin{itemize}
\setlength\itemsep{-1em}
\item Read papers
\end{itemize}

\subsection{SK flow paper}
PRL: A new two-dimensional fully coupled MHD model problem with closing currents for finite magnetic Reynolds number: description, flow regimes and stability
\begin{itemize}
\setlength\itemsep{-1em}
\item First step of SK flow is descriptive - phases (linear, non-linear, decay)
\end{itemize}

\subsection{Prospectus}
\begin{itemize}
\setlength\itemsep{-1em}
\item Start writing introduction
\item PPT
\item Develop organization of prospectus document
\end{itemize}

\section{Projects}

\subsection{BEM}
\begin{itemize}
\setlength\itemsep{-1em}
\item Make coordinates capable of handling a zero-cell, two ghost point mesh \Checkmark
\item Make ops capable of handling a zero-cell mesh
\item stitching is needed so that diffusion can traverse between different grids.
\item need to look into methods to solve equation
\item need data structures to store solution
\end{itemize}

\subsection{HPC scalability}
\begin{itemize}
\setlength\itemsep{-1em}
\item HD = Hoffman2 / DOE cluster
\item Test MOONS on HD
\item Add option for printing or exporting, maybe introduce a .log file or something, similar to HIMAG.
\item Try some MPI statements on HD
\end{itemize}

\subsection{Mesh}
\begin{itemize}
\setlength\itemsep{-1em}
\item Gaps in mesh generation + grid stretch-parameter match. Needs to be generalized, simple + easy to use.
\item Make a uniform grid option for boundary layers (maybe make a function for this)
\item betaRobertsRight(hmin,hmax,L,dh) - is this dh on the left side or right side? I think betaRoberts routines need a dh1 or dhend input, because when connecting, betaRobertsRight makes stretching occur on right-side, but needs dh on right side, which is a problem for con\_APP\_Roberts\_R.
\end{itemize}

\subsection{SF / VF / TF}
\begin{itemize}
\setlength\itemsep{-1em}
\item Make TF composed of 9 SF so that it can be composed of different sizes.
\item Try re-incorporating better parallelization for collocated / staggered VF / TF fields
\end{itemize}

\subsection{IO}
\begin{itemize}
\setlength\itemsep{-1em}
\item export\_raw\_processed works almost perfectly, but there are several design flaws. 1) remove global vars (from sim params) 2) check for when the correct field is exported 3) remove "external" and look for different options. SOME OF THESE HAVE BEEN CORRECTED, WITH THE EXCEPTION OF THE TRANSIENT ROUTINES.
\item Fix IO\_tools.f90. Pick better names and separate into IO tools and conversion (int2str, etc.)
\item Re-consider HDF5
\item Figure out how to view time in tecplot as animation runs
\item FIX THE NAME OF THE PROBES. THEY ARE TOTALLY INCONSISTENT.
\end{itemize}

\subsection{Exception handling}
\begin{itemize}
\setlength\itemsep{-1em}
\item Eventually, checks of face/node/CC/edge data could be made in routines like advect\_B and advect\_U, which could help prevent searching for what is a incorrectly located variable into a subroutine that results in a interpolation or del failure.
\item Update export: make a switch to control what to export. Exporting interior domain may help for post-processing
\begin{itemize}
\setlength\itemsep{-1em}
\item U-field: (face, cc, node)\_interior (face, cc, node)\_total domain
\item B-field: (face, cc, node)\_interior (face, cc, node)\_total domain
\item J-field: (face, cc, node)\_interior (face, cc, node)\_total domain
\end{itemize}
\item Try making more properties private, and more functions pure
\item IMPORTANT: del.f90 has no safe way to assign the output when pad is present. This is VERY dangerous and needs to be addressed immediately!
\end{itemize}

\subsection{Improving Speed}
\begin{itemize}
\setlength\itemsep{-1em}
\item Try to incorporate anti-symmetric BCs again...
\item Remove allocation in lap\_centered. May require changing matrix-free operator interface. Brainstorm.
\item Look to improve performance of apply\_edges
\item Consider replacing applyBCs with assign using embedExtract
\item Profile to find latest bottle necks
\item Try out pre-processor directive '-ftree -parallelize -loops=n' = split loops into n-threads
\item Consider using pre-processor directives instead of if statements, this needs to be organized on a large level.
\end{itemize}

\subsection{Makefile}
\begin{itemize}
\setlength\itemsep{-1em}
\item Need to clean up directory. Many unused and old files exist in folders and knowing which files are grabbed by the compiler is not clear. Maybe consider making an "old" folder for backed-up old files.
\end{itemize}

\subsection{Complex Geometry}
\begin{itemize}
\setlength\itemsep{-1em}
\item more testing
\item Consider making an obstacle module that allows for more complex flows
e.g. flow over a sphere
\item Consider making an obstacle module that is passed into the solver.
This could assign the velocity to satisfy any desired BCs and internal
velocity (likely u=0 inside). More tests could be done this way.
\end{itemize}

\subsection{Documentation / Passing down MOONS}
\begin{itemize}
\setlength\itemsep{-1em}
\item Continue fixing file names and folder names for better organization
\item Clean up / organize documentation.
\item Documentation for benchmark cases (methodology, format, execution)
\item Naming convention for all ops should be (Real,SF,VF) = (real(cp),scalarField,vectorField)
\item Convert matlab to python.
\item Document benchmark tests.
\end{itemize}

\subsection{Benchmarks}

\begin{itemize}
\setlength\itemsep{-1em}
\item Make benchmarks and post-processing automatic. Consider using makefile definitions that override defaults
\begin{enumerate}
\setlength\itemsep{-1em}
\item 3D LDC Hydro flow - grid refinement
\item 3D LDC Hydro flow, $Re = 400, Re = 1000$ (Guj and Stella)
\item 2D LDC Hydro flow, $Re = $ (Ghia)
\item 3D LDC MHD flow, $Re = 100, Ha = 45$ at low $Re_m$ (Pattison)
\item 2D Channel MHD flow, $Re = 200, Ha = 25.82, Re_m = 1$ (Bandaru)
\item Benchmarks to consider adding:
\begin{itemize}
\setlength\itemsep{-1em}
\item 3D LDC hydro flow - turbulent (Re=2,000 and 3,200 compare with literature)
\item 2D kinematic MHD flow (Parker or Weiss)
\item 3D MHD Duct (Kenjeres)
\item 3D MHD Isotropic Turbulence decay (Knaepen)
\item 3D Duct MHD flow - DNS (Kinet, Knaepen, Molokov)
\item 2D LDC Hydro flow - grid refinement
\end{itemize}
\end{enumerate}
\end{itemize}


\subsection{Restart Capability}
\begin{itemize}
\setlength\itemsep{-1em}
\item Need to re-develop restart capability... This will require some thought.
\item This is complex. What should be restarted and what should be re-configured?
\item Can every module have a "natural" restart capability? YES! \Checkmark But they require more thought for solvers.
\item Should source terms in momentum solver be fixed? This way, restart can be used with a parameter.
\item Should the momentum\_solver be picked with a compiler flag?
\item momentum import + override inputs? or inputs imported so MHD solver parameters stay the same?
\item Decide on format of input file (what parameters will be set in input file and what will be coded)
\item Right now, there are many things that will (likely) break when MOONS tries to restart a simulation. Restarting may require manual overrides on certain things, so generalizing restart for many modules is difficult.
\item rethink how to use printPhysicalMinMax in momentum,induction,energy display routines
\end{itemize}

\section{Ideas}

\subsection{Numerical Methods}
\begin{itemize}
\setlength\itemsep{-1em}
\item Re-test MG
\item Consider MG with Jacobi method, need to restrict $\sigma$...
\item Try changing smoother to Jacobi and check convergence rate (including when coarsest level is \textit{very} coarse)
\end{itemize}

\subsection{New Features}
\begin{itemize}
\setlength\itemsep{-1em}
\item Implement an upwind scheme for the non-uniform grid.
\item Make pressure term 2nd order accurate in time (trap)
\item Make advection term 2nd order accurate in time (AB2)
\item Consider different time marching method (RK4 - eldredge)
\item Consider re-formulating using conservative higher order finite difference methods by Moin
\item Allow for compressibility!
\item Use higher order spatial operators! (stencils.f90 / interpOps.f90 / applyBCs.f90)
\item Use hgiher order time marching methods! (RK4 maybe?)
\item Consider making applyFaceBCs, applyEdgeBCs, applyCornerBCs and interpolate appropriately. This seems like the most general and care-free approach. At least implement the capability, and offer for using applyFaceBCs in a non-arbitrary order.
\item Consider passing MFP into PCG solver, this way, different coefficients can be used with the same PCG object.
\end{itemize}

\subsection{Dimensionless Groups}
\begin{itemize}
\setlength\itemsep{-1em}
\item Elsasser number = ration of lorentz to coriolis forces = $A = sigma B^2 / (2 rho \Omega)$
$Ra = t_k t_{\nu} / t_f^2$ Where $t_f$ is the free fall time $ = \sqrt{ H/(\alpha g \Delta T)} $
\item Q = Chandrasekhar number = ratio of magneto-viscous diffusion time scales and the square of the alfven speed crossing time $ = t_{\eta} t_{\nu} / t_A^2$
\item Eckmann number = ratio of rotation time to diffusion time $ = t_{\Omega} / t_{\nu} = nu / (2 \Omega H^2)$
\item Convective Reynolds number $= Re_c = t_nu / t_f$
\end{itemize}


\subsection{Miscellaneous}

- Try cleaning procedure with $\mathbf{B}_{tangent}=0$ and $\frac{\partial \phi}{\partial n} = \nabla \bullet \mathbf{B}^*$ and see if cleaning works. This might be equivalent to Pseudo vacuum. Also, look into paper / source that discussed this.

- Make a python routine to compute the number of lines in MOONS

- Try Jack's Experiment

- Finish developing solverSettings estimateRemaining() subroutine

- Look into Paul Roberts (UCLA)

- Try comparing 1/Ha with cw for HIMAG vs MOONS and compare continuity of tangent(j)/sigma across interface

- Member of committee for Naveen was in Geo-physics, Jonathon "Arnaud"? Look online about him and maybe we can collaborate.

\newpage
\section{Non-work related}
\begin{itemize}
\setlength\itemsep{-1em}
\item delete emails and organize dropbox
\item look into optical Nand gates
\item get veggie / fruit recipes
\item get tablet for refrigerator?
\item email google glass (July 17th) - Matlab $\rightarrow$ c++, paid please and computer provided?
\item prep / post tutoring flyers
\item make birthday wishlist
\item convert to charleskawczynski ?
\item organize photos and make slideshow
\item delete duplicate photos
\item sign up for 596 for the fall
\item return Christa's sketchers (if possible)
\item look up transient solution on workstation, delete results
\end{itemize}

\end{document}