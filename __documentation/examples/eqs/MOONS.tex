\documentclass[11pt]{article}
\usepackage{amsmath}
\usepackage{tabularx, multirow, booktabs}
\usepackage{graphicx}
\usepackage{caption}
\usepackage{subcaption}
\usepackage{float}
\usepackage{xspace}
\usepackage{mathtools}
% \usepackage{epsfig}
\usepackage{natbib}
\usepackage{setspace}
\usepackage{ragged2e}
\usepackage{etoolbox}
\usepackage{geometry}
% \usepackage{floatrow}
\usepackage{subcaption}

\renewcommand{\tabularxcolumn}[1]{>{\small}m{#1}}
% \def\tabularxcolumn#1{m{#1}}
\newenvironment{tightcenter}
{\par\centering}
{\par}
\begin{document}

% \begin{tabularx}{0.45\textwidth}{>{\centering}m{1in} >{\centering}m{1in} >{\centering}m{1in}}
%   \begin{equation}\label{eq:divU}
%     \nabla \bullet \mathbf{u} = 0 , \notag
%     \addtocounter{equation}{1}
%   \end{equation}  &
%   \begin{equation}\label{eq:divB}
%     \nabla \bullet \mathbf{B} = 0 .    \notag
%     \addtocounter{equation}{1}
%   \end{equation} &
%     (\ref{eq:divU},\ref{eq:divB})
% \end{tabularx}


\begin{tabularx}{0.5\textwidth}{>{\tightcenter}m{1in} >{\tightcenter}m{1in} >{\tightcenter}m{1in}}
  \hline
  \begin{equation}\label{eq:divU}
    \nabla \bullet \mathbf{u} = 0 , \notag
    \addtocounter{equation}{1}
  \end{equation}
  &
  \begin{equation}\label{eq:divB}
    \nabla \bullet \mathbf{B} = 0 .    \notag
    \addtocounter{equation}{1}
  \end{equation} &
    (3,4) % (\ref{eq:divU},\ref{eq:divB})
\end{tabularx}

% \begin{tabularx}{0.8\textwidth}{XXX}
%   \begin{equation}\label{eq:divU}
%     \nabla \bullet \mathbf{u} = 0 , \notag
%     \addtocounter{equation}{1}
%   \end{equation}  &
%   \begin{equation}\label{eq:divB}
%     \nabla \bullet \mathbf{B} = 0 .    \notag
%     \addtocounter{equation}{1}
%   \end{equation} &
%     (\ref{eq:divU},\ref{eq:divB})
% \end{tabularx}


% \begin{tabular}{ccc}
%   \begin{equation}\label{eq:divU}
%     \nabla \myCdot \mathbf{u} = 0 , \notag
%     \addtocounter{equation}{1}
%   \end{equation}  &
%   \begin{equation}\label{eq:divB}
%     \nabla \myCdot \mathbf{B} = 0 .    \notag
%     \addtocounter{equation}{1}
%   \end{equation} &
%     (\ref{eq:divU},\ref{eq:divB}) \\
% \end{tabular}


\end{document}