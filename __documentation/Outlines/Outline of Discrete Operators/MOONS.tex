\documentclass[11pt]{article}
\usepackage{graphicx}    % needed for including graphics e.g. EPS, PS
\usepackage{epstopdf}
\usepackage{amsmath}
\usepackage{hyperref}
\usepackage{xspace}
\usepackage{mathtools}
\usepackage{tikz}
\usepackage{epsfig}
\usepackage{float}
\usepackage{natbib}
\usepackage{subfigure}
\usepackage{setspace}
\usepackage{tabularx,ragged2e,booktabs,caption}


\setlength{\oddsidemargin}{0.1in}
\setlength{\textwidth}{7.25in}

\setlength{\topmargin}{-1in}     %\topmargin: gap above header
\setlength{\headheight}{0in}     %\headheight: height of header
\setlength{\topskip}{0in}        %\topskip: between header and text
\setlength{\headsep}{0in}        
\setlength{\textheight}{692pt}   %\textheight: height of main text
\setlength{\textwidth}{7.5in}    % \textwidth: width of text
\setlength{\oddsidemargin}{-0.5in}  % \oddsidemargin: odd page left margin
\setlength{\evensidemargin}{0in} %\evensidemargin : even page left margin
\setlength{\parindent}{0.25in}   %\parindent: indentation of paragraphs
\setlength{\parskip}{0pt}        %\parskip: gap between paragraphs
\setlength{\voffset}{0.5in}


% Useful commands:

% \hfill		aligns-right everything right of \hfill

\begin{document}
\doublespacing
\title{Magnetohydrodynamic Object-Oriented Numerical Simulation (MOONS)}
\author{C. Kawczynski \\
Department of Mechanical and Aerospace Engineering \\
University of California Los Angeles, USA\\
}
\maketitle

\section{Outline of Discrete Operators}

There are several discrete operators in MOONS. This document discusses a general outline of data movement from its original location to its final location after an operator has been applied.

In this document, N,CC,F,E represent Node (Cell Corner), Cell Center, Face, and Edge data respectively.

\section{Divergence (D)}

\begin{equation}
	\text{N} \rightarrow \text{N} 
	\qquad / \qquad 
	\text{CC} \rightarrow \text{CC}
	\qquad / \qquad 
	\text{F} \rightarrow \text{CC}
	\qquad / \qquad 
	\text{E} \rightarrow \text{N}
\end{equation}

\section{Laplacian (L)}

\begin{equation}
	\text{N} \rightarrow \text{N} 
	\qquad / \qquad 
	\text{CC} \rightarrow \text{CC}
	\qquad / \qquad 
	\text{F} \rightarrow \text{F}
	\qquad / \qquad 
	\text{E} \rightarrow \text{E}
\end{equation}

\section{Gradient (G)}

\begin{equation}
	\text{N} \rightarrow \text{N} 
	\qquad / \qquad 
	\text{CC} \rightarrow \text{CC}
\end{equation}

\begin{equation}
	\text{F} \rightarrow \text{CC}
	\qquad / \qquad 
	\text{CC} \rightarrow \text{F}
	\qquad / \qquad 
	\text{E} \rightarrow \text{N}
	\qquad / \qquad 
	\text{N} \rightarrow \text{E}
\end{equation}

\section{Curl (C)}

\begin{equation}
	\text{N} \rightarrow \text{N} 
	\qquad / \qquad 
	\text{CC} \rightarrow \text{CC}
\end{equation}

\begin{equation}
	\text{F} \rightarrow \text{E}
	\qquad / \qquad 
	\text{E} \rightarrow \text{F}
\end{equation}



\section{Notes}

Note that there is a symmetry between the primary grid and dual grid operators.

Also, node that all of these operators should be capable of operating on collocated data with a result that lives on the original grid, which means that N$\rightarrow$N and CC$\rightarrow$CC are natural candidates for all operators.


\end{document}