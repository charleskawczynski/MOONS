\documentclass[landscape,11pt]{article}
\newcommand{\PSCHAIN}{..}
\edef\PSCHAIN{\PSCHAIN/LATEX_INCLUDES}

\newcommand{\rootdir}{\PSCHAIN}
\newcommand{\VAR}{Success}



\begin{document}
\doublespacing
\MOONSTITLE
\maketitle

\section{Consideration of boundary conditions}
In contrast to Jacobi / Gauss-Seidel method, Conjugate Gradient method seems to require, in addition to symmetry, built in boundary conditions into matrix $A$. This means that a simple matrix-free approach cannot simply reach to ghost points that are updated at each iteration.

The following document explains how the stencils must be adjusted to achieve a consistent and symmetric system of equations by specific examples, but these may be applied to derivatives in general.

\newpage
\section{Cell centered data}
Consider the Laplace stencil,

\begin{equation}
   \left(\frac{\partial^2 u}{\partial x^2}\right)_{1} =
   \frac{u_g - 2 u_1 + u_{2}}{\Delta x^2} = f_1
\end{equation}
Where $u_g,u_1,u_2$ are the ghost, first interior and second interior cells. This Laplacian operator may be written in matrix form as

\[ A = \frac{1}{\Delta x^2} \left[\begin{array}{ccccccccc}
1  & 1     &           &           &           &           &           &         &  0 \\
1  & -2    & 1         &           &           &           &           &         &    \\
   & 1     & -2        & 1         &           &           &           &         &    \\
   &       &           & \ddots    & \ddots    & \ddots    &           &         &    \\
   &       &           &           &           & 1         & -2        & 1       &    \\
   &       &           &           &           &           &  1        & -2      &  1 \\
0  &       &           &           &           &           &           & 1       &  1 \\
\end{array} \right]
\]

Simple linear interpolation will not provide a consistent system to solve using CG. So let's substitute $u_g$ and adjust the stencil accordingly.
\subsection{Dirichlet}
Consider Dirichlet BCs, $u_g$ may be computed observing that the boundary value is the average of the neighboring two cell center values.
\begin{equation}
    u_b = \frac{u_g + u_1}{2}
    \rightarrow
    u_g = 2u_b - u_1
\end{equation}
So our Laplacian stencil becomes:
\begin{equation}
   \left(\frac{\partial^2 u}{\partial x^2}\right)_{1} =
   \frac{(2 u_b - u_1) - 2 u_1 + u_{2}}{\Delta x^2} = f_1
\end{equation}
This boundary point must be moved to the RHS to maintain a consistent matrix-vector multiplication. Therefore our equation changes:
\begin{equation}
   \frac{u_g - 2 u_1 + u_{2}}{\Delta x^2} = f_1
   \rightarrow
   \frac{- 3 u_1 + u_{2}}{\Delta x^2} = f_1 - \frac{2 u_b}{\Delta x^2}
\end{equation}
Correspondingly, the matrix $A$ changes:
\[ A = \frac{1}{\Delta x^2} \left[\begin{array}{ccccccccc}
1  & 1     &           &           &           &           &           &         &  0 \\
1  & -2    & 1         &           &           &           &           &         &    \\
   & 1     & -2        & 1         &           &           &           &         &    \\
   &       &           & \ddots    & \ddots    & \ddots    &           &         &    \\
   &       &           &           &           & 1         & -2        & 1       &    \\
   &       &           &           &           &           &  1        & -2      &  1 \\
0  &       &           &           &           &           &           & 1       &  1 \\
\end{array} \right]
\rightarrow
A = \frac{1}{\Delta x^2} \left[\begin{array}{ccccccccc}
0  & 0     & 0         &           &           &           &           &         &  0 \\
0  & -3    & 1         &           &           &           &           &         &    \\
0  & 1     & -2        & 1         &           &           &           &         &    \\
   &       &           & \ddots    & \ddots    & \ddots    &           &         &    \\
   &       &           &           &           & 1         & -2        & 1       &  0 \\
   &       &           &           &           &           &  1        & -3      &  0 \\
0  &       &           &           &           &           &  0        & 0       &  0 \\
\end{array} \right]
\]
Note that the first and last equations are identities ($0=0$) which are reserved for the ghost points.

\subsection{Neumann}
Consider Neumann BCs, $u_g$ may be computed by writing the equation for the first derivative on the boundary
\begin{equation}
  \left(\frac{\partial u}{\partial x}\right)_{boundary}
  = \frac{u_g - u_1}{\Delta x} = n \theta
  \rightarrow
  u_g = u_1 + n \theta \Delta x, \qquad \text{n = 1 at xmax,ymax,zmin and n = -1 at xmin,ymin,zmin}
\end{equation}
So our Laplacian stencil becomes:
\begin{equation}
   \left(\frac{\partial^2 u}{\partial x^2}\right)_{1} =
   \frac{(u_1 + n \theta \Delta x) - 2 u_1 + u_{2}}{\Delta x^2} = f_1
\end{equation}
This boundary information must be moved to the RHS to maintain a consistent matrix-vector multiplication. Therefore our equation changes:
\begin{equation}
   \frac{u_g - 2 u_1 + u_{2}}{\Delta x^2} = f_1
   \rightarrow
   \frac{- u_1 + u_{2}}{\Delta x^2} = f_1 - \frac{n\theta \Delta x}{\Delta x^2}
\end{equation}
Correspondingly, the matrix $A$ changes:
\[ A = \frac{1}{\Delta x^2} \left[\begin{array}{ccccccccc}
1  & 1     &           &           &           &           &           &         &  0 \\
1  & -2    & 1         &           &           &           &           &         &    \\
   & 1     & -2        & 1         &           &           &           &         &    \\
   &       &           & \ddots    & \ddots    & \ddots    &           &         &    \\
   &       &           &           &           & 1         & -2        & 1       &    \\
   &       &           &           &           &           &  1        & -2      &  1 \\
0  &       &           &           &           &           &           & 1       &  1 \\
\end{array} \right]
\rightarrow
A = \frac{1}{\Delta x^2} \left[\begin{array}{ccccccccc}
0  & 0     & 0         &           &           &           &           &         &  0 \\
0  & -1    & 1         &           &           &           &           &         &    \\
0  & 1     & -2        & 1         &           &           &           &         &    \\
   &       &           & \ddots    & \ddots    & \ddots    &           &         &    \\
   &       &           &           &           & 1         & -2        & 1       &  0 \\
   &       &           &           &           &           &  1        & -1      &  0 \\
0  &       &           &           &           &           &  0        & 0       &  0 \\
\end{array} \right]
\]
Note that the first and last equations are identities ($0=0$) which are reserved for the ghost points.

ALSO: In this case, $A$ is singular, so the nullspace must be removed. This can be addressed in several ways. In MOONS, the nullspace is removed by subtracting the weighted mean of the RHS. Notice that the matrix $A$ is already nicely symmetric, so CG may be applied.

\subsection{Periodic}
Consider periodic BCs, $u_g$ may is known on the ghost point since it wraps around to $u_i$ on the opposing side.
\begin{equation}
  \left(\frac{\partial u}{\partial x}\right)_{boundary}
  = \frac{u_g - u_i}{\Delta x} = \frac{{u_i}_{opp} - u_i}{\Delta x}
  \rightarrow
  u_g = {u_i}_{opp}
\end{equation}
So our Laplacian stencil becomes:
\begin{equation}
   \left(\frac{\partial^2 u}{\partial x^2}\right)_{1} =
   \frac{- 2 u_1 + u_{2}}{\Delta x^2} = f_1 - \frac{{u_i}_{opp}}{\Delta x^2}
\end{equation}
This boundary information must be moved to the RHS to maintain a consistent matrix-vector multiplication. However, we must still see the matrix $A$:
\[ A = \frac{1}{\Delta x^2} \left[\begin{array}{ccccccccc}
1  & 1     &           &           &           &           &           & 0       &  0 \\
1  & -2    & 1         &           &           &           &           & 0       &  0 \\
   & 1     & -2        & 1         &           &           &           &         &    \\
   &       &           & \ddots    & \ddots    & \ddots    &           &         &    \\
   &       &           &           &           & 1         & -2        & 1       &    \\
0  & 0     &           &           &           &           &  1        & -2      &  1 \\
0  & 0     &           &           &           &           &           & 1       &  1 \\
\end{array} \right]
\rightarrow
A = \frac{1}{\Delta x^2} \left[\begin{array}{ccccccccc}
0  & 0     & 0         &           &           &           &           & 0       &  0 \\
0  & -2    & 1         &           &           &           &           & 0       &  0 \\
0  & 1     & -2        & 1         &           &           &           &         &    \\
   &       &           & \ddots    & \ddots    & \ddots    &           &         &    \\
   &       &           &           &           & 1         & -2        & 1       &  0 \\
0  & 0     &           &           &           &           &  1        & -2      &  0 \\
0  & 0     &           &           &           &           &  0        & 0       &  0 \\
\end{array} \right]
\]

Although the matrix does not appear circulant, the periodic part is moved to the RHS.

\subsection{Robin}
The using FDM, for CC data, we have
\begin{equation}\begin{aligned}
  c \PD_n u + u = \theta \\
  c \hat{n} \frac{u_g - u_i}{\Delta h} + .5 (u_g+u_i) = \theta \\
  2 c \hat{n} (u_g - u_i) + \Delta h (u_g+u_i) = 2 \theta \Delta h \\
  u_g (2 c \hat{n} + \Delta h) + u_i (\Delta h - 2c \hat{n}) = 2 \theta \Delta h \\
  u_g (2 c \hat{n} + \Delta h) = 2 \theta \Delta h - u_i (\Delta h - 2c \hat{n}) \\
  u_g = \frac{2 \theta \Delta h - u_i (\Delta h - 2c \hat{n})}{2 c \hat{n} + \Delta h} \\
  u_g = \frac{2 \theta \Delta h}{2 c \hat{n} + \Delta h} + u_i \frac{2c \hat{n} - \Delta h}{2 c \hat{n} + \Delta h} \\
  u_g = \theta \frac{2 \Delta h}{2 c \hat{n} + \Delta h} + u_i \frac{2c \hat{n} - \Delta h}{2 c \hat{n} + \Delta h} \\
  u_g = \Theta + u_i K, \qquad K = \frac{2c \hat{n} - \Delta h}{2 c \hat{n} + \Delta h} \\
\end{aligned}\end{equation}
Note that
\begin{equation}\begin{aligned}
  K = -1 , \qquad c<<1 \\
  K = 1  , \qquad c>>1 \\
\end{aligned}\end{equation}

So our Laplacian stencil becomes:
\begin{equation}
   \left(\frac{\partial^2 u}{\partial x^2}\right)_{1} =
   \frac{(K u_1) - 2 u_1 + u_{2}}{\Delta x^2} = f_1
\end{equation}
This boundary information must be moved to the RHS to maintain a consistent matrix-vector multiplication. Since there is nothing to move, we have:
\begin{equation}
   \frac{u_g - 2 u_1 + u_{2}}{\Delta x^2} = f_1
   \rightarrow
   \frac{u_1(K-2) + u_{2}}{\Delta x^2} = f_1
\end{equation}
Correspondingly, the matrix $A$ changes:
\[ A = \frac{1}{\Delta x^2} \left[\begin{array}{ccccccccc}
1  & 1     &           &           &           &           &           &         &  0 \\
1  & -2+K  & 1         &           &           &           &           &         &    \\
   & 1     & -2        & 1         &           &           &           &         &    \\
   &       &           & \ddots    & \ddots    & \ddots    &           &         &    \\
   &       &           &           &           & 1         & -2        & 1       &    \\
   &       &           &           &           &           &  1        & -2      &  1 \\
0  &       &           &           &           &           &           & 1       &  1 \\
\end{array} \right]
\rightarrow
A = \frac{1}{\Delta x^2} \left[\begin{array}{ccccccccc}
0  & 0     & 0         &           &           &           &           &         &  0 \\
0  & -2+K  & 1         &           &           &           &           &         &    \\
0  & 1     & -2        & 1         &           &           &           &         &    \\
   &       &           & \ddots    & \ddots    & \ddots    &           &         &    \\
   &       &           &           &           & 1         & -2        & 1       &  0 \\
   &       &           &           &           &           &  1        & -1      &  0 \\
0  &       &           &           &           &           &  0        & 0       &  0 \\
\end{array} \right]
\]
Note that the first and last equations are identities ($0=0$) which are reserved for the ghost points.

ALSO: In this case, $A$ may be near singular, but the nullspace cannot be removed since it will never be exactly singular. Notice that the matrix $A$ is already nicely symmetric, so CG may be applied.

\subsection{Robin (Neumann type)}
Consider Robin BCs, $u_g$ may be computed by writing the equation for the first derivative plus $u$ on the boundary:
\begin{equation}\begin{aligned}
  c \PD_n u + u = \theta \\
  c \hat{n} \frac{u_g - u_i}{\Delta h} + .5 (u_g+u_i) = \theta \\
  2 c \hat{n} (u_g - u_i) + \Delta h (u_g+u_i) = 2 \theta \Delta h \\
  u_g (2 c \hat{n} + \Delta h) + u_i (\Delta h - 2c \hat{n}) = 2 \theta \Delta h \\
  u_g (2 c \hat{n} + \Delta h) = 2 \theta \Delta h - u_i (\Delta h - 2c \hat{n}) \\
  u_g = \frac{2 \theta \Delta h - u_i (\Delta h - 2c \hat{n})}{2 c \hat{n} + \Delta h} \\
  u_g = \frac{2 \theta \Delta h}{2 c \hat{n} + \Delta h} + u_i \frac{2c \hat{n} - \Delta h}{2 c \hat{n} + \Delta h} \\
  u_g = \theta \frac{2 \Delta h}{2 c \hat{n} + \Delta h} + u_i \frac{2c \hat{n} - \Delta h}{2 c \hat{n} + \Delta h} = \Theta + u_i K \\
\end{aligned}\end{equation}
So our Laplacian stencil becomes:
\begin{equation}
   \left(\frac{\partial^2 u}{\partial x^2}\right)_{1} =
   \frac{(K u_1 + \Theta) - 2 u_1 + u_{2}}{\Delta x^2} = f_1
\end{equation}
This boundary information must be moved to the RHS to maintain a consistent matrix-vector multiplication. Since there is nothing to move, we have:
\begin{equation}
   \frac{u_g - 2 u_1 + u_{2}}{\Delta x^2} = f_1
   \rightarrow
   \frac{u_1(K-2) + u_{2}}{\Delta x^2} = f_1 - \frac{\Theta}{\Delta x^2}
\end{equation}
Correspondingly, the matrix $A$ changes:
\[ A = \frac{1}{\Delta x^2} \left[\begin{array}{ccccccccc}
1  & 1     &           &           &           &           &           &         &  0 \\
1  & -2+K  & 1         &           &           &           &           &         &    \\
   & 1     & -2        & 1         &           &           &           &         &    \\
   &       &           & \ddots    & \ddots    & \ddots    &           &         &    \\
   &       &           &           &           & 1         & -2        & 1       &    \\
   &       &           &           &           &           &  1        & -2      &  1 \\
0  &       &           &           &           &           &           & 1       &  1 \\
\end{array} \right]
\rightarrow
A = \frac{1}{\Delta x^2} \left[\begin{array}{ccccccccc}
0  & 0     & 0         &           &           &           &           &         &  0 \\
0  & -2+K  & 1         &           &           &           &           &         &    \\
0  & 1     & -2        & 1         &           &           &           &         &    \\
   &       &           & \ddots    & \ddots    & \ddots    &           &         &    \\
   &       &           &           &           & 1         & -2        & 1       &  0 \\
   &       &           &           &           &           &  1        & -1      &  0 \\
0  &       &           &           &           &           &  0        & 0       &  0 \\
\end{array} \right]
\]
Note that the first and last equations are identities ($0=0$) which are reserved for the ghost points.

ALSO: In this case, $A$ may be near singular, but the nullspace cannot be removed since it will never be exactly singular. Notice that the matrix $A$ is already nicely symmetric, so CG may be applied.

\newpage
\section{Node data}
Consider the Laplace stencil,
\begin{equation}
   \left(\frac{\partial^2 u}{\partial x^2}\right)_{boundary} =
   \frac{u_g - 2 u_b + u_i}{\Delta x^2} = f_b
\end{equation}
Where $u_g,u_b,u_i$ are the ghost, boundary and first interior node points. This Laplacian operator may be written in matrix form as
\[ A = \frac{1}{\Delta x^2} \left[\begin{array}{ccccccccc}
0  & 0     & 0         &           &           &           &           &         &  0 \\
1  & -2    & 1         &           &           &           &           &         &    \\
0  & 1     & -2        & 1         &           &           &           &         &    \\
   &       &           & \ddots    & \ddots    & \ddots    &           &         &    \\
   &       &           &           &           & 1         & -2        & 1       &  0 \\
   &       &           &           &           &           &  1        & -2      &  1 \\
0  &       &           &           &           &           &  0        & 0       &  0 \\
\end{array} \right]
\]
Simple linear interpolation will not provide a consistent system to solve using CG. So let's substitute $u_g$ and adjust the stencil accordingly.

\subsection{Dirichlet}
Consider Dirichlet BCs, $u_b$ IS KNOWN, AND THE SOLVE IS ONLY FROM $u_i$ ONWARD. Therefore, the only equation we must consider is at location $i$:
\begin{equation}
   \left(\frac{\partial^2 u}{\partial x^2}\right)_{i} =
   \frac{u_b - 2 u_i + u_{i+1}}{\Delta x^2} = f_i
\end{equation}
If we were to write an equation for the boundary point, we could write
\begin{equation}
   \left(\frac{\partial^2 u}{\partial x^2}\right)_{b} =
   \frac{u_g - 2 u_b + u_{i}}{\Delta x^2} = f_b
   \rightarrow
   u_g  = 2 u_b - u_{i} - \Delta x^2 f_b
\end{equation}
If we insist $f_b=0$ then we have a simplified version:
\begin{equation}
   u_g  = 2 u_b - u_{i}
\end{equation}
So the equation changes to
\begin{equation}
   \left(\frac{\partial^2 u}{\partial x^2}\right)_{b} =
   0 = 0
\end{equation}
And we may remove it from our system, BUT the matrix is no longer symmetric since interior stencils are used on the interior domain where the equations are solved. Let's move the last boundary point in equation
\begin{equation}
   \left(\frac{\partial^2 u}{\partial x^2}\right)_{i} =
   \frac{u_b - 2 u_i + u_{i+1}}{\Delta x^2} = f_i
   \rightarrow
   \frac{- 2 u_i + u_{i+1}}{\Delta x^2} = f_i - \frac{u_b}{\Delta x^2}
\end{equation}

This means that the ghost point SHOULD NOT ENTER THE COMPUTATIONS AT ALL. Correspondingly, the matrix $A$ is:
\[ A = \frac{1}{\Delta x^2} \left[\begin{array}{ccccccccc}
0  & 0     & 0         &           &           &           &           &         &  0 \\
1  & -2    & 1         &           &           &           &           &         &    \\
0  & 1     & -2        & 1         &           &           &           &         &    \\
   &       &           & \ddots    & \ddots    & \ddots    &           &         &    \\
   &       &           &           &           & 1         & -2        & 1       &  0 \\
   &       &           &           &           &           &  1        & -2      &  1 \\
0  &       &           &           &           &           &  0        & 0       &  0 \\
\end{array} \right]
\rightarrow
A = \frac{1}{\Delta x^2} \left[\begin{array}{ccccccccc}
0  & 0     & 0         &           &           &           &           &         &  0 \\
0  & 0     & 0         &           &           &           &           &         &    \\
0  & \textcolor{red}{0}     & -2        & 1         &           &           &           &         &    \\
0  & 0     & 1         & -2        &           &           &           &         &    \\
   &       &           & \ddots    & \ddots    & \ddots    &           &         &    \\
   &       &           &           &           & -2        & 1         & 0       &  0 \\
   &       &           &           &           & 1         & -2        & \textcolor{red}{0}       &  0 \\
0  &       &           &           &           &           &  0        & 0       &  0 \\
0  &       &           &           &           &           &  0        & 0       &  0 \\
\end{array} \right]
\]
This is nice because the equation has effectively become smaller. NOTE: the first two 0's in the above equation refer to the ghost and boundary equations.
\subsection{Neumann}
Consider Neumann BCs, $u_g$ may be computed by writing the equation for the first derivative on the boundary
\begin{equation}
  \left(\frac{\partial u}{\partial x}\right)_{boundary}
  = \frac{u_g - u_i}{2\Delta x} = n \theta
  \rightarrow
  u_g = u_i + 2 n\theta \Delta x, \qquad \text{n = 1 at xmax,ymax,zmin and n = -1 at xmin,ymin,zmin}
\end{equation}
So our Laplacian stencil becomes:
\begin{equation}
   \left(\frac{\partial^2 u}{\partial x^2}\right)_{b} =
   \frac{(u_i + 2 n \theta \Delta x) - 2 u_b + u_i}{\Delta x^2} = f_b
\end{equation}
This boundary information must be moved to the RHS to maintain a consistent matrix-vector multiplication. Therefore our equation changes:
\begin{equation}
   \frac{u_g - 2 u_b + u_i}{\Delta x^2} = f_b
   \rightarrow
   \frac{- 2u_b + 2u_i}{\Delta x^2} = f_b - \frac{2 n\theta \Delta x}{\Delta x^2}
\end{equation}
Correspondingly, the matrix $A$ changes:
\[ A = \frac{1}{\Delta x^2} \left[\begin{array}{ccccccccc}
0  & 0     & 0         &           &           &           &           &         &  0 \\
1  & -2    & 1         &           &           &           &           &         &    \\
0  & 1     & -2        & 1         &           &           &           &         &    \\
   &       &           & \ddots    & \ddots    & \ddots    &           &         &    \\
   &       &           &           &           & 1         & -2        & 1       &  0 \\
   &       &           &           &           &           &  1        & -2      &  1 \\
0  &       &           &           &           &           &  0        & 0       &  0 \\
\end{array} \right]
\rightarrow
A = \frac{1}{\Delta x^2} \left[\begin{array}{ccccccccc}
0  & 0     & 0         &           &           &           &           &         &  0 \\
0  & -2    & 2         &           &           &           &           &         &    \\
0  & 1     & -2        & 1         &           &           &           &         &    \\
   &       &           & \ddots    & \ddots    & \ddots    &           &         &    \\
   &       &           &           &           & 1         & -2        & 1       &  0 \\
   &       &           &           &           &           &  2        & -2      &  0 \\
0  &       &           &           &           &           &  0        & 0       &  0 \\
\end{array} \right]
\]
Two things to note here. One is that this matrix is singular (it has a non-zero nullspace). This can easily be confirmed by multiplying by $\mathbf{v} = (1,1,\dots,1,1)^T$. The second thing to note is that it is no longer symmetric. We can easily symmetrize this however by dividing the first and last equation by 2:
\[A = \frac{1}{\Delta x^2} \left[\begin{array}{ccccccccc}
0  & 0     & 0         &           &           &           &           &         &  0 \\
0  & -1    & 1         &           &           &           &           &         &    \\
0  & 1     & -2        & 1         &           &           &           &         &    \\
   &       &           & \ddots    & \ddots    & \ddots    &           &         &    \\
   &       &           &           &           & 1         & -2        & 1       &  0 \\
   &       &           &           &           &           &  1        & -1      &  0 \\
0  &       &           &           &           &           &  0        & 0       &  0 \\
\end{array} \right]
\]
Which means the equation is
\begin{equation}
   \frac{- u_b + u_i}{\Delta x^2} = \frac{f_b}{2}  - \frac{n\theta \Delta x}{\Delta x^2}
\end{equation}
Note that the mean must be subtracted from the RHS AFTER dividing by the factor of 2. To apply this division correctly with applying BCs, we may use the following definition for $u_g$ to achieve solving the same equation:
\begin{equation}
  \boxed{
  u_g = u_i, \qquad
  f_{b} = f_{b} - \frac{2n\theta \Delta x}{\Delta x^2}, \qquad
  (Au)_{b} = \frac{1}{2} (Au)_{b}
  }
\end{equation}
This way, we ultimately have:
\begin{equation}
   \frac{u_g - 2 u_b + u_i}{\Delta x^2} = f_b
   \qquad \rightarrow \qquad
   \frac{- u_b + u_i}{\Delta x^2} = \frac{f_b}{2} - \frac{n \theta \Delta x}{\Delta x^2}
\end{equation}


\subsection{Periodic}
Consider periodic BCs, $u_g$ is the first interior node on the opposite side of the domain, which is known. Forcing on the wall-coincident nodes must match up naturally. The derivatives are computed using interior stencils on the boundary as:
\begin{equation}
   \left(\frac{\partial^2 u}{\partial x^2}\right)_{b} =
   \frac{u_g - 2 u_b + u_{i}}{\Delta x^2} = f_b
\end{equation}
We also know that
\begin{equation}
  u_g = u_{i,opp}
\end{equation}
Moving the opposite part to the RHS, we have
\begin{equation}
   \left(\frac{\partial^2 u}{\partial x^2}\right)_{b} =
   \frac{- 2 u_b + u_{i}}{\Delta x^2} = f_b - \frac{- u_g}{\Delta x^2}
\end{equation}

Since the ghost node is part of the unknowns on the opposite side of the domain interior, the matrix $A$ becomes (a circulant one):
\[ A = \frac{1}{\Delta x^2}
\left[\begin{array}{ccccccccc}
0  & 0     & 0         &           &           &           &           &         &  0 \\
1  & -2    & 1         &           &           &           &           &         &    \\
0  & 1     & -2        & 1         &           &           &           &         &    \\
   &       &           & \ddots    & \ddots    & \ddots    &           &         &    \\
   &       &           &           &           & 1         & -2        & 1       &  0 \\
   &       &           &           &           &           &  1        & -2      &  1 \\
0  &       &           &           &           &           &  0        & 0       &  0 \\
\end{array} \right]
\rightarrow
\left[\begin{array}{ccccccccc}
0  & 0     & 0         &           &           &           &  0        & 0       &  0 \\
0  & -2    & 1         &           &           &           &  1        & 0       &  0 \\
0  & 1     & -2        & 1         &           &           &           &         &    \\
   &       &           & \ddots    & \ddots    & \ddots    &           &         &    \\
   &       &           &           &           & 1         & -2        & 1       &  0 \\
0  & 0     & 1         &           &           &           &  1        & -2      &  0 \\
0  & 0     & 0         &           &           &           &  0        & 0       &  0 \\
\end{array} \right]
\]

This matrix is symmetric and suitable for PCG so long both front and back BCs are periodic. If one is periodic and the other is, e.g. Neumann, CG cannot be used. NOTE: the first two 0's in the above equation refer to the ghost and boundary equations.

\subsection{Robin}
Consider Robin BCs, $u_g$ may be computed by writing the equation for the first derivative plus $u$ on the boundary:
\begin{equation}\begin{aligned}
  c \PD_n u + u = \theta \\
  c \hat{n} \frac{u_g - u_1}{2 \Delta h} + u_b = \theta \\
  c \hat{n} (u_g - u_i) + 2 \Delta h u_b = 2 \Delta h \theta \\
  c \hat{n} (u_g - u_i) = 2 \Delta h \theta - 2 \Delta h u_b \\
  u_g - u_i = \frac{2 \Delta h \theta - 2 \Delta h u_b}{c \hat{n}} \\
  u_g = u_i + \frac{2 \Delta h \theta - 2 \Delta h u_b}{c \hat{n}} \\
  u_g = u_i - u_b \frac{2 \Delta h}{c \hat{n}} + \theta \frac{2 \Delta h}{c \hat{n}}\\
\end{aligned}\end{equation}

\newpage

\section{Summary of final equations and matrices}
\subsection{Node}
\subsubsection{Dirichlet}
\subsubsection{Neumann}
\subsection{Cell center}
\subsubsection{Dirichlet}
\subsubsection{Neumann}

\newpage
\section{Removing the nullspace}
A matlab example...


\end{document}