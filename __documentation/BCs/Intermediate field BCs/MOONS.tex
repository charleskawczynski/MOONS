\documentclass[landscape]{article}
\usepackage{graphicx}    % needed for including graphics e.g. EPS, PS
\usepackage{epstopdf}
\usepackage{amsmath}
\usepackage{hyperref}
\usepackage{xspace}
\usepackage{mathtools}
\usepackage{tikz}
\usepackage{epsfig}
\usepackage{float}
\usepackage{xcolor}
%\usepackage{natbib}
\usepackage{subfigure}
\usepackage{setspace}
\usepackage{tabularx,ragged2e,booktabs,caption}


% \setlength{\oddsidemargin}{0.1in}
% \setlength{\textwidth}{7.25in}
\newcommand{\DEL}{\nabla}
\newcommand{\PD}{\partial}

\setlength{\topmargin}{-1in}     %\topmargin: gap above header
\setlength{\headheight}{0in}     %\headheight: height of header
\setlength{\topskip}{0in}        %\topskip: between header and text
% \setlength{\headsep}{0in}
\setlength{\textheight}{525pt}   %\textheight: height of main text
\setlength{\textwidth}{10in}    % \textwidth: width of text
\setlength{\oddsidemargin}{-0.5in}  % \oddsidemargin: odd page left margin
\setlength{\evensidemargin}{0in} %\evensidemargin : even page left margin
\setlength{\parindent}{0.25in}   %\parindent: indentation of paragraphs
\setlength{\parskip}{0pt}        %\parskip: gap between paragraphs
% \setlength{\voffset}{0.5in}


% Useful commands:

% \hfill		aligns-right everything right of \hfill

\begin{document}
\doublespacing
\title{Magnetohydrodynamic Object-Oriented Numerical Solver (MOONS)}
\author{C. Kawczynski \\
Department of Mechanical and Aerospace Engineering \\
University of California Los Angeles, USA\\
}
% \maketitle

\section{Intermediate field BCs}
Intermediate field BCs, as described in \cite{Kim} are derived below:

\section{Dirichlet}
The intermediate field prescribed BCs are:
\begin{equation}\begin{aligned}
  u_i^* = u_i^{n+1} + \Delta t \PD_i \phi^n \\
\end{aligned}\end{equation}
\section{Neumann}
The Neumann conditions are derived as follows
\begin{equation}\begin{aligned}
  u_i^* = u_i^{n+1} + \Delta t \PD_i \phi^n \\
  \PD_n u_i^* = \PD_n u_i^{n+1} + \Delta t \PD_n \PD_i \phi^n \\
\end{aligned}\end{equation}
\section{Periodic}
The Periodic conditions are the same as dirichlet
\begin{equation}\begin{aligned}
  u_i^* = u_i^{n+1} + \Delta t \PD_i \phi^n \\
\end{aligned}\end{equation}
\section{Robin}
The Robin conditions are derived starting from Neumann BCs
\begin{equation}\begin{aligned}
  \PD_n u_i^* = \PD_n u_i^{n+1} + \Delta t \PD_n \PD_i \phi^n \\
  c \PD_n u_i^* = c \PD_n u_i^{n+1} + c \Delta t \PD_n \PD_i \phi^n \\
  c \PD_n u_i^* + u_i^{n+1} = c \PD_n u_i^{n+1} + u_i^{n+1} + c \Delta t \PD_n \PD_i \phi^n \\
  c \PD_n u_i^* = - u_i^{n+1} + c \Delta t \PD_n \PD_i \phi^n \\
  \PD_n u_i^* = -\frac{1}{c} u_i^{n+1} + \Delta t \PD_n \PD_i \phi^n \\
\end{aligned}\end{equation}
\section{Summary}
\begin{equation}\begin{aligned}
  u_i^* = u_i^{n+1} + \Delta t \PD_i \phi^n, \qquad & \text{Dirichlet} \\
  u_i^* = u_i^{n+1} + \Delta t \PD_i \phi^n, \qquad & \text{Periodic} \\
  \PD_n u_i^* = \PD_n u_i^{n+1} + \Delta t \PD_n \PD_i \phi^n, \qquad & \text{Neumann} \\
  \PD_n u_i^* = -\frac{1}{c} u_i^{n+1} + \Delta t \PD_n \PD_i \phi^n, \qquad & \text{Robin} \\
\end{aligned}\end{equation}

\begin{thebibliography}{1}
\bibitem{Kim} Kim, John, and Parviz Moin. "Application of a fractional-step method to incompressible Navier-Stokes equations." Journal of computational physics 59.2 (1985): 308-323.
\end{thebibliography}

\end{document}