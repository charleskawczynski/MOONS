\documentclass[landscape]{article}
\newcommand{\VAR}{Success}


\begin{document}
\doublespacing
\MOONSTITLE
\maketitle

\section{Intermediate field BCs}
Intermediate field BCs, as described in \cite{Kim1984} are derived below:

\section{Dirichlet}
The intermediate field prescribed BCs are:
\begin{equation}\begin{aligned}
  u_i^* = u_i^{n+1} + \Delta t \PD_i \phi^n \\
\end{aligned}\end{equation}
\section{Neumann}
The Neumann conditions are derived as follows
\begin{equation}\begin{aligned}
  u_i^* = u_i^{n+1} + \Delta t \PD_i \phi^n \\
  \PD_n u_i^* = \PD_n u_i^{n+1} + \Delta t \PD_n \PD_i \phi^n \\
\end{aligned}\end{equation}
\section{Periodic}
The Periodic conditions are the same as dirichlet
\begin{equation}\begin{aligned}
  u_i^* = u_i^{n+1} + \Delta t \PD_i \phi^n \\
\end{aligned}\end{equation}

\section{Robin}
The Robin conditions are derived as follows
\begin{equation}\begin{aligned}
  c \PD_n u_i^{n+1} + u_i^{n+1} = \theta \\
  c \PD_n (u_i^* - \Delta t \PD_i \phi^n) + u_i^* - \Delta t \PD_i \phi^n = \theta \\
  c \PD_n u_i^* - c \PD_n \Delta t \PD_i \phi^n + u_i^* - \Delta t \PD_i \phi^n = \theta \\
  c \PD_n u_i^* + u_i^* = \theta + c \Delta t \PD_n \PD_i \phi^n + \Delta t \PD_i \phi^n \\
  c \PD_n u_i^* + u_i^* = \theta + \Delta t (c \PD_n \PD_i \phi^n + \PD_i \phi^n) = g \\
  c \PD_n u_i^* + u_i^* = g \\
\end{aligned}\end{equation}

\section{Summary}
\begin{equation}\begin{aligned}
  u_i^* = u_i^{n+1} + \Delta t \PD_i \phi^n, \qquad & \text{Dirichlet} \\
  u_i^* = u_i^{n+1} + \Delta t \PD_i \phi^n, `'\qquad & \text{Periodic} \\
  \PD_n u_i^* = \PD_n u_i^{n+1} + \Delta t \PD_n \PD_i \phi^n, \qquad & \text{Neumann} \\
  c \PD_n u_i^* + u_i^* = g, \qquad & \text{Robin} \\
\end{aligned}\end{equation}

\input{\rootdir/includes/include_bib.tex}




\end{document}