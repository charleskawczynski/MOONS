\documentclass[11pt]{article}
% \documentclass[3p,twocolumn,10pt]{elsarticle}
\usepackage{graphicx}    % needed for including graphics e.g. EPS, PS
\usepackage{epstopdf}
\usepackage{amsmath}
\usepackage{amssymb}
\usepackage{hyperref}
\usepackage{xspace}
\usepackage{mathtools}
\usepackage{tikz}
\usepackage{epsfig}
\usepackage{float}
%\usepackage{natbib}
\usepackage{subfigure}
\usepackage{setspace}
\usepackage{tabularx,ragged2e,booktabs,caption}

\usepackage{xcolor}
\usepackage{xspace}
\usepackage{longtable}
\usepackage{mathtools}
\usepackage{natbib}
\usepackage{setspace}
\usepackage{ragged2e}
\usepackage{etoolbox}
\usepackage{geometry}

\newcommand{\PATHTOUSEFULCOMMANDS}{../../../../useful_commands.tex}

\input{\PATHTOUSEFULCOMMANDS}

% \setlength{\textfloatsep}{0.1cm}
\newcommand{\volume}{\mathop{\ooalign{\hfil$V$\hfil\cr\kern0.08em--\hfil\cr}}\nolimits}
\newcommand{\figWidth}{0.45\textwidth}
\newcommand{\figSuffix}{_png.png}
\newcommand{\pushright}[1]{\ifmeasuring@#1\else\omit\hfill$\displaystyle#1$\fi\ignorespaces}
\newcommand{\hfillMath}{\hskip \textwidth minus \textwidth}
% \renewcommand{\arraystretch}{0.0} % Removes vertical spaces before/after tabularx


\setlength{\oddsidemargin}{0.1in}
\setlength{\textwidth}{7.25in}

\setlength{\topmargin}{-1in}     %\topmargin: gap above header
\setlength{\headheight}{0in}     %\headheight: height of header
\setlength{\topskip}{0in}        %\topskip: between header and text
\setlength{\headsep}{0in}
\setlength{\textheight}{692pt}   %\textheight: height of main text
\setlength{\textwidth}{7.5in}    % \textwidth: width of text
\setlength{\oddsidemargin}{-0.5in}  % \oddsidemargin: odd page left margin
\setlength{\evensidemargin}{0in} %\evensidemargin : even page left margin
\setlength{\parindent}{0.25in}   %\parindent: indentation of paragraphs
\setlength{\parskip}{0pt}        %\parskip: gap between paragraphs
\setlength{\voffset}{0.5in}

% Useful commands:

% \hfill    aligns-right everything right of \hfill

\begin{document}
\doublespacing
\title{Magnetohydrodynamic Object-Oriented Numerical Solver (MOONS)}
\author{C. Kawczynski \\
Department of Mechanical and Aerospace Engineering \\
University of California Los Angeles, USA\\
}
\maketitle

\section{Robin BCs}
The Robin BCs are
\begin{equation}\begin{aligned}
  c \PD_n u + u = \theta \\
\end{aligned}\end{equation}
\subsection{CC data O2 accurate}
The using FDM, for CC data, we have
\begin{equation}\begin{aligned}
  c \PD_n u + u = \theta \\
  c \hat{n} \frac{u_g - u_i}{\Delta h} + .5 (u_g+u_i) = \theta \\
  2 c \hat{n} (u_g - u_i) + \Delta h (u_g+u_i) = 2 \theta \Delta h \\
  u_g (2 c \hat{n} + \Delta h) + u_i (\Delta h - 2c \hat{n}) = 2 \theta \Delta h \\
  u_g (2 c \hat{n} + \Delta h) = 2 \theta \Delta h - u_i (\Delta h - 2c \hat{n}) \\
  u_g = \frac{2 \theta \Delta h - u_i (\Delta h - 2c \hat{n})}{2 c \hat{n} + \Delta h} \\
  u_g = \frac{2 \theta \Delta h}{2 c \hat{n} + \Delta h} + u_i \frac{2c \hat{n} - \Delta h}{2 c \hat{n} + \Delta h} \\
  u_g = \theta \frac{2 \Delta h}{2 c \hat{n} + \Delta h} + u_i \frac{2c \hat{n} - \Delta h}{2 c \hat{n} + \Delta h} \\
  u_g = \Theta + u_i K, \qquad K = \frac{2c \hat{n} - \Delta h}{2 c \hat{n} + \Delta h} \\
\end{aligned}\end{equation}


\subsection{Matlab Code demonstration}
Here is a matlab code that plots $K$ for an arbitrary $\Delta h$ and a range of $c$:

clear all; clc; close all; \\
dh = 3e-3; \\
n = [-1 1]; \\
a = 0; \\
b = 1; \\
L = b-a; \\
c = linspace(a,b,1000); \\
K1=(2*c*n(1)-dh)./(2*c*n(1)+dh); \\
K2=(2*c*n(2)-dh)./(2*c*n(2)+dh); \\
for j=1:10/L \\
  i=find(K1==max(abs(K1))); c(i) = []; \\
  K1=(2*c*n(1)-dh)./(2*c*n(1)+dh); \\
  K2=(2*c*n(2)-dh)./(2*c*n(2)+dh); \\
end \\
plot(c,K1,'-r',c,K2,'-b') \\
legend('nhat=-1','nhat=1') \\


\end{document}