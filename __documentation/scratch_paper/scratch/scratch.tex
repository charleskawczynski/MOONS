\documentclass[11pt]{article}
\newcommand{\VAR}{Success}


\begin{document}
\doublespacing
\MOONSTITLE
\maketitle

\section{Hartmann Layer thickness}
The ratio of length and time scales are
\begin{equation}\begin{aligned}
	\frac{\delta}{L} = Re^{-1/2} \\
	\frac{\delta_{Ha}}{L_{||}} = \frac{1}{Ha} \\
	\frac{\delta_m}{L} = Re_m^{-1/2} \\
	% \frac{t_{\nu}}{t_u} = \frac{L^2}{\nu} \frac{U}{L} = Re \\
	% \sqrt{\frac{t_{j}^{-1}}{t_{\nu}^{-1}}} = \sqrt{\frac{(\rho/(\sigma B^2))^{-1}}{(L_{||}^2/\nu)^{-1}}} = Ha \\
	% \frac{t_{\nu_m}}{t_u} = \frac{L^2}{(\sigma \mu_m)^{-1}} \frac{U}{L} = Re_m
\end{aligned}\end{equation}

\section{Comparing 2 at a time}
\begin{equation}\begin{aligned}
	\delta   <<  \delta_{Ha}, & \qquad \RA \qquad & Ha   << Re^{1/2}  , \qquad & \RA \qquad & N   << 1    \\
	\delta   >>  \delta_{Ha}, & \qquad \RA \qquad &                            & \RA \qquad & N   >> 1    \\
	\delta   <<  \delta_m   , & \qquad \RA \qquad & Re_m << Re        , \qquad & \RA \qquad & Pr_m<< 1    \\
	\delta   >>  \delta_m   , & \qquad \RA \qquad &                            & \RA \qquad & Pr_m>> 1    \\
	\delta_m <<  \delta_{Ha}, & \qquad \RA \qquad & Ha   << Re_m^{1/2}, \qquad & \RA \qquad & N   << Pr_m \\
	\delta_m >>  \delta_{Ha}, & \qquad \RA \qquad &                            & \RA \qquad & N   >> Pr_m \\
\end{aligned}\end{equation}


\section{Comparing all 3}

From this, we can see there are 1 in 5 possible values of $\delta$ for each boundary layer, for each

1) 2 large gaps (3 permutes in order + 3 permutes out of order):
\begin{itemize}
\setlength\itemsep{-1em}
	\item $\delta      <<    \delta_{Ha}   <<   \delta_m      $
	\item $\delta_m    <<    \delta        <<   \delta_{Ha}   $
	\item $\delta_{Ha} <<    \delta_m      <<   \delta        $
	\item $\delta_{Ha} <<    \delta        <<   \delta_m      $
	\item $\delta_m    <<    \delta_{Ha}   <<   \delta        $
	\item $\delta      <<    \delta_m      <<   \delta_{Ha}   $
\end{itemize}

2) 1 large gap:
\begin{itemize}
\setlength\itemsep{-1em}
	\item $\delta      =     \delta_{Ha}   <<   \delta_m      $
	\item $\delta_m    =     \delta        <<   \delta_{Ha}   $
	\item $\delta_{Ha} =     \delta_m      <<   \delta        $
	\item $\delta_{Ha} <<    \delta        =    \delta_m      $
	\item $\delta_m    <<    \delta_{Ha}   =    \delta        $
	\item $\delta      <<    \delta_m      =    \delta_{Ha}   $
\end{itemize}

3) No gap:
\begin{itemize}
\setlength\itemsep{-1em}
	\item $\delta      =     \delta_{Ha}   =    \delta_m      $
\end{itemize}

\subsection{Low Rem}
The magnetic boundary layer is
\begin{equation}\begin{aligned}
	\delta_m \sim L
\end{aligned}\end{equation}
The Hartmann layer at low $Re_m$ is estimated by equating the viscous and Lorentz force scales:
\begin{equation}\begin{aligned}
	\rho \nu \frac{U}{\delta^2} = J_c B
\end{aligned}\end{equation}
In the low $Re_m$ limit, we use $[J_c] = \sigma U B$, and therefore have
\begin{equation}\begin{aligned}
	\rho \nu \frac{U}{\delta^2} =& \sigma U B^2 \\
	\delta^2 =& \rho \nu \frac{U}{\sigma U B^2}
	         = & \frac{\rho \nu}{\sigma B^2} \\
	\frac{\delta^2}{L^2} =& \frac{1}{Ha^2} \\
	\frac{\delta}{L} =& \frac{1}{Ha}
\end{aligned}\end{equation}

\subsection{Finite Rem}
The magnetic boundary layer is
\begin{equation}\begin{aligned}
	\frac{\delta_m}{L} = Re_m^{-1/2} \\
\end{aligned}\end{equation}
At finite $Re_m$, the only difference is the appropriateness of the scale chosen for $J_c$. The general result is
\begin{equation}\begin{aligned}
	\rho \nu \frac{U}{\delta^2} =& J_c B \\
	\delta^2 =& \frac{\rho \nu U}{J_c B} \\
	\frac{\delta}{L} =& \frac{1}{Ha} \frac{\sigma U B}{J_c}
\end{aligned}\end{equation}
Where $J_c$ may be specified.

\section{What to transfer after B-field reconstruction}
\begin{itemize}
\item $B_{np}$
\item $B_{f_x},B_{f_y},B_{f_z}$ (for post-processing)
\item $\DEL \DOT \B$
\item $E_{\B}(\Omega,SS)$, $E_{\B}(\Omega_f,SS)$, $E_{\B}(\Omega_c,SS)$
\end{itemize}

\section{2D parabolic duct flow solution}
\begin{equation}\begin{aligned}
	\frac{1}{Re}\PD_{zz} u = - \PD_x p \\
	\PD_{zz} u = - Re \PD_x p \\
	\PD_{z} u = - Re \PD_x p z + c_1 \\
	u = - Re \PD_x p z^2/2 + c_1 z + c_2\\
	u(-1) = - Re \PD_x p /2 - c_1 + c_2 = 0\\
	u(1)  = - Re \PD_x p /2 + c_1 + c_2 = 0\\
	c_1 = 0 \\
	c_2 = Re \PD_x p /2 \\
	u = - Re \PD_x p z^2/2 + Re \PD_x p /2 \\
	u = \frac{Re \PD_x p}{2} \left(1-z^2 \right) \\
	u_{max} = \frac{Re \PD_x p}{2} \\
\end{aligned}\end{equation}

\end{document}