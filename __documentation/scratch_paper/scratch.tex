\documentclass[11pt]{article}
\usepackage{graphicx}    % needed for including graphics e.g. EPS, PS
\usepackage{epstopdf}
\usepackage{amsmath}
\usepackage{hyperref}
\usepackage{xspace}
\usepackage{mathtools}
\usepackage{tikz}
\usepackage{epsfig}
\usepackage{float}
%\usepackage{natbib}
\usepackage{subfigure}
\usepackage{setspace}
\usepackage{tabularx,ragged2e,booktabs,caption}
\usepackage{esint}


\newcommand{\G}{\mathbf{g}}
\newcommand{\U}{\mathbf{u}}
\newcommand{\C}{\mathbf{C}}
\newcommand{\B}{\mathbf{B}}
\newcommand{\E}{\mathbf{E}}
\newcommand{\J}{\mathbf{j}}
\newcommand{\A}{\mathbf{A}}
\newcommand{\N}{\mathbf{n}}
\newcommand{\F}{\mathbf{f}}
\newcommand{\X}{\mathbf{x}}
\newcommand{\Y}{\mathbf{y}}

\newcommand{\DOT}{\text{\textbullet}}
\newcommand{\CROSS}{\times}
\newcommand{\DEL}{\nabla}
\newcommand{\CURL}{\DEL \CROSS}
\newcommand{\DIV}{\DEL \DOT}
\newcommand{\PD}{\partial}
\newcommand{\MAC}{\mathcal}

\newcommand{\M}{{\mu_m}}
\newcommand{\SI}{\sigma}
\newcommand{\SII}{\SI^{-1}}
\newcommand{\JS}{\frac{\J}{\sigma}}
\newcommand{\MO}{\overline{\mu}_m}
\newcommand{\SO}{\overline{\sigma}}
\newcommand{\RA}{\rightarrow}

\renewcommand{\H}{\frac{\B}{\MO}}

\newcommand{\JSS}{\frac{\J^2}{\sigma}}
\newcommand{\HD}{\frac{\B}{\M}}

\setlength{\oddsidemargin}{0.1in}
\setlength{\textwidth}{7.25in}

\setlength{\topmargin}{-1in}     %\topmargin: gap above header
\setlength{\headheight}{0in}     %\headheight: height of header
\setlength{\topskip}{0in}        %\topskip: between header and text
\setlength{\headsep}{0in}        
\setlength{\textheight}{692pt}   %\textheight: height of main text
\setlength{\textwidth}{7.5in}    % \textwidth: width of text
\setlength{\oddsidemargin}{-0.5in}  % \oddsidemargin: odd page left margin
\setlength{\evensidemargin}{0in} %\evensidemargin : even page left margin
\setlength{\parindent}{0.25in}   %\parindent: indentation of paragraphs
\setlength{\parskip}{0pt}        %\parskip: gap between paragraphs
\setlength{\voffset}{0.5in}


% Useful commands:

% \hfill		aligns-right everything right of \hfill

\begin{document}
\doublespacing
\title{Magnetohydrodynamic Object-Oriented Numerical Solver (MOONS)}
\author{C. Kawczynski \\
Department of Mechanical and Aerospace Engineering \\
University of California Los Angeles, USA\\
}
\maketitle

\section{Hartmann Layer thickness}
The ratio of length and time scales are
\begin{equation}\begin{aligned}
	\frac{\delta}{L} = Re^{-1/2} \\
	\frac{\delta_{Ha}}{L_{||}} = \frac{1}{Ha} \\
	\frac{\delta_m}{L} = Re_m^{-1/2} \\
	% \frac{t_{\nu}}{t_u} = \frac{L^2}{\nu} \frac{U}{L} = Re \\
	% \sqrt{\frac{t_{j}^{-1}}{t_{\nu}^{-1}}} = \sqrt{\frac{(\rho/(\sigma B^2))^{-1}}{(L_{||}^2/\nu)^{-1}}} = Ha \\
	% \frac{t_{\nu_m}}{t_u} = \frac{L^2}{(\sigma \mu_m)^{-1}} \frac{U}{L} = Re_m
\end{aligned}\end{equation}

\section{Comparing 2 at a time}
\begin{equation}\begin{aligned}
	\delta   <<  \delta_{Ha}, & \qquad \RA \qquad & Ha   << Re^{1/2}  , \qquad & \RA \qquad & N   << 1    \\
	\delta   >>  \delta_{Ha}, & \qquad \RA \qquad &                            & \RA \qquad & N   >> 1    \\
	\delta   <<  \delta_m   , & \qquad \RA \qquad & Re_m << Re        , \qquad & \RA \qquad & Pr_m<< 1    \\
	\delta   >>  \delta_m   , & \qquad \RA \qquad &                            & \RA \qquad & Pr_m>> 1    \\
	\delta_m <<  \delta_{Ha}, & \qquad \RA \qquad & Ha   << Re_m^{1/2}, \qquad & \RA \qquad & N   << Pr_m \\
	\delta_m >>  \delta_{Ha}, & \qquad \RA \qquad &                            & \RA \qquad & N   >> Pr_m \\
\end{aligned}\end{equation}


\section{Comparing all 3}

From this, we can see there are 1 in 5 possible values of $\delta$ for each boundary layer, for each 

1) 2 large gaps (3 permutes in order + 3 permutes out of order):
\begin{itemize}
\setlength\itemsep{-1em}
	\item $\delta      <<    \delta_{Ha}   <<   \delta_m      $
	\item $\delta_m    <<    \delta        <<   \delta_{Ha}   $
	\item $\delta_{Ha} <<    \delta_m      <<   \delta        $
	\item $\delta_{Ha} <<    \delta        <<   \delta_m      $
	\item $\delta_m    <<    \delta_{Ha}   <<   \delta        $
	\item $\delta      <<    \delta_m      <<   \delta_{Ha}   $
\end{itemize}

2) 1 large gap:
\begin{itemize}
\setlength\itemsep{-1em}
	\item $\delta      =     \delta_{Ha}   <<   \delta_m      $
	\item $\delta_m    =     \delta        <<   \delta_{Ha}   $
	\item $\delta_{Ha} =     \delta_m      <<   \delta        $
	\item $\delta_{Ha} <<    \delta        =    \delta_m      $
	\item $\delta_m    <<    \delta_{Ha}   =    \delta        $
	\item $\delta      <<    \delta_m      =    \delta_{Ha}   $
\end{itemize}

3) No gap:
\begin{itemize}
\setlength\itemsep{-1em}
	\item $\delta      =     \delta_{Ha}   =    \delta_m      $
\end{itemize}

\subsection{Low Rem}
The magnetic boundary layer is 
\begin{equation}\begin{aligned}
	\delta_m \sim L
\end{aligned}\end{equation}
The Hartmann layer at low $Re_m$ is estimated by equating the viscous and Lorentz force scales:
\begin{equation}\begin{aligned}
	\rho \nu \frac{U}{\delta^2} = J_c B
\end{aligned}\end{equation}
In the low $Re_m$ limit, we use $[J_c] = \sigma U B$, and therefore have
\begin{equation}\begin{aligned}
	\rho \nu \frac{U}{\delta^2} =& \sigma U B^2 \\
	\delta^2 =& \rho \nu \frac{U}{\sigma U B^2} 
	         = & \frac{\rho \nu}{\sigma B^2} \\
	\frac{\delta^2}{L^2} =& \frac{1}{Ha^2} \\
	\frac{\delta}{L} =& \frac{1}{Ha}
\end{aligned}\end{equation}

\subsection{Finite Rem}
The magnetic boundary layer is 
\begin{equation}\begin{aligned}
	\frac{\delta_m}{L} = Re_m^{-1/2} \\
\end{aligned}\end{equation}
At finite $Re_m$, the only difference is the appropriateness of the scale chosen for $J_c$. The general result is
\begin{equation}\begin{aligned}
	\rho \nu \frac{U}{\delta^2} =& J_c B \\
	\delta^2 =& \frac{\rho \nu U}{J_c B} \\
	\frac{\delta}{L} =& \frac{1}{Ha} \frac{\sigma U B}{J_c}
\end{aligned}\end{equation}
Where $J_c$ may be specified.

\section{What to transfer after B-field reconstruction}
\begin{itemize}
\item $B_{np}$
\item $B_{f_x},B_{f_y},B_{f_z}$ (for post-processing)
\item $\DEL \DOT \B$
\item $E_{\B}(\Omega,SS)$, $E_{\B}(\Omega_f,SS)$, $E_{\B}(\Omega_c,SS)$
\end{itemize}


\end{document}