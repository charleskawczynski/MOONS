\documentclass[11pt]{article}
\usepackage{graphicx}    % needed for including graphics e.g. EPS, PS
\usepackage{epstopdf}
\usepackage{amsmath}
\usepackage{hyperref}
\usepackage{xspace}
\usepackage{mathtools}
\usepackage{tikz}
\usepackage{epsfig}
\usepackage{float}
%\usepackage{natbib}
\usepackage{subfigure}
\usepackage{setspace}
\usepackage{tabularx,ragged2e,booktabs,caption}

\newcommand{\B}{\mathbf{B}}
\renewcommand{\H}{\mathbf{H}}
\newcommand{\C}{\mathbf{C}}
\newcommand{\U}{\mathbf{u}}
\newcommand{\SI}{\sigma}
\newcommand{\M}{\mu}
\newcommand{\curl}{\nabla \times}

\newcommand{\A}{\mathbf{A}}
\newcommand{\PD}{\partial}
\newcommand{\MC}{\mathcal}
\newcommand{\BM}{\frac{\B}{\mu}}
\newcommand{\J}{\mathbf{j}}
\newcommand{\E}{\mathbf{E}}
\newcommand{\N}{\mathbf{n}}
\newcommand{\JS}{\frac{\J}{\sigma}}
\newcommand{\JoS}{\frac{\J^1}{\sigma}}
\newcommand{\JSS}{\frac{\J^2}{\sigma}}
\newcommand{\SII}{\sigma^{-1}}
\newcommand{\MO}{\overline{\mu}}
\newcommand{\SO}{\overline{\sigma}}
\newcommand{\DOT}{\text{\textbullet}}


\setlength{\oddsidemargin}{0.1in}
\setlength{\textwidth}{7.25in}

\setlength{\topmargin}{-1in}     %\topmargin: gap above header
\setlength{\headheight}{0in}     %\headheight: height of header
\setlength{\topskip}{0in}        %\topskip: between header and text
\setlength{\headsep}{0in}        
\setlength{\textheight}{692pt}   %\textheight: height of main text
\setlength{\textwidth}{7.5in}    % \textwidth: width of text
\setlength{\oddsidemargin}{-0.5in}  % \oddsidemargin: odd page left margin
\setlength{\evensidemargin}{0in} %\evensidemargin : even page left margin
\setlength{\parindent}{0.25in}   %\parindent: indentation of paragraphs
\setlength{\parskip}{0pt}        %\parskip: gap between paragraphs
\setlength{\voffset}{0.5in}


% Useful commands:

% \hfill		aligns-right everything right of \hfill

\begin{document}
\doublespacing
\title{Magnetohydrodynamic Object-Oriented Numerical Solver (MOONS)}
\author{C. Kawczynski \\
Department of Mechanical and Aerospace Engineering \\
University of California Los Angeles, USA\\
}
\maketitle

\section{Derivation of the Dimensionless thermal energy equation}
The dimensional energy equation is

\begin{equation}
\rho C_p \left[ \frac{\PD T}{\PD t} + \frac{\PD (u_j T)}{\PD x_j} \right] = 
\frac{\PD}{\PD x_j} \left( k \frac{\PD T}{\PD x_j} \right) + \frac{1}{\sigma} j^2 + \Phi + \dot{Q}
\end{equation}

Where $\frac{1}{\sigma} j^2$ is the joule dissipation, $Q$ is the volumetric heat generation and

\begin{equation}
\Phi = \mu \left( \frac{\PD u_j}{\PD x_i} - \frac{\PD u_i}{\PD x_j} \right) \frac{\PD u_i}{\PD x_j}
\end{equation}
is the viscous dissipation.

\subsection{Non-dimensionalizing}
Using
\begin{equation}\begin{aligned}
u_j^* = u_j/U_c \qquad t^* = t/t_c \qquad 
t_c = L_c/U_c  \qquad x_j^* = x_j/L_c \qquad 
T^* = \frac{T - T_c}{\Delta T_c} \\
\rho = \rho_c \qquad \mu = \mu_c \qquad
C_p = C_{p,c} \qquad k = k_c \qquad
j_k^* = j_k/ \sigma_c U_c B_c \qquad
\dot{Q}^* = \frac{\dot{Q}}{\rho_c C_{p,c} \Delta T_c / t_c}
\end{aligned}\end{equation}

Yields

\begin{multline}
	\rho_c C_{p,c}
	\left[
	\frac{\Delta T_c}{t_c} \frac{\PD T^*}{\PD t^*} + \frac{\Delta T_c U_c}{L_c} \frac{\PD (u_j^* T^*)}{\PD x_j^*}
	\right]
	= 
	\frac{k_c \Delta T_c}{L_c^2}
	\frac{\PD}{\PD x_j^*}
	\left( k^* \frac{\PD T^*}{\PD x_j^*} \right) +
	\frac{(\sigma_c U_c B_c)^2}{\sigma_c}
	\frac{1}{\sigma^*} {j^*}^2 + \\
	\frac{U_c^2}{L_c^2}\mu_c
	\left( \frac{\PD u_j^*}{\PD x_i^*} - \frac{\PD u_i^*}{\PD x_j^*} \right)
	\frac{\PD u_i^*}{\PD x_j^*} +
	\frac{\rho_c C_{p,c} \Delta T_c}{t_c}
	\dot{Q}^*
\end{multline}

Dividing by $\rho_c C_{p,c} \frac{\Delta T_c U_c}{L_c}$ yields

\begin{multline}
	\frac{\PD T^*}{\PD t^*} +
	\frac{\PD (u_j^* T^*)}{\PD x_j^*}
	= 
	\alpha_c 
	\frac{L_c}{U_c \Delta T_c}
	\frac{\Delta T_c}{L_c^2}
	\frac{\PD}{\PD x_j^*}
	\left( k^* \frac{\PD T^*}{\PD x_j^*} \right) + 
	\frac{L_c}{U_c \Delta T_c}
	\frac{\sigma_c}{\rho_c C_{p,c}}
	\frac{U_c^2 B_c^2}{\sigma^*} {j^*}^2 + \\
	\frac{L_c}{U_c \Delta T_c}
	\frac{1}{\rho_c C_{p,c}}
	\frac{U_c^2}{L_c^2}\mu_c
	\left( \frac{\PD u_j^*}{\PD x_i^*} - \frac{\PD u_i^*}{\PD x_j^*} \right)
	\frac{\PD u_i^*}{\PD x_j^*} +
	\frac{L_c}{U_c \Delta T_c}
	\frac{1}{\rho_c C_{p,c}}
	\frac{\rho_c C_{p,c} \Delta T_c}{t_c}
	\dot{Q}^*
\end{multline}

Where $\alpha_c = \frac{k_c}{\rho_c C_{p,c}}$

\subsection{Coefficient of diffusion term}

The coefficient of the diffusion term, we have

\begin{equation}
\alpha_c  \frac{L_c}{U_c \Delta T_c} \frac{\Delta T_c}{L_c^2} =
\frac{\alpha_c}{\nu_c} \frac{\nu_c}{U_c L_c} = \frac{1}{Pr Re}
\end{equation}

Where $Pr$ and $Re$ are the Prandtl and Reynolds number.

\subsection{Coefficient of Joule heating term}

The coefficient of the Joule heating term, we have

\begin{equation}
\frac{L_c}{U_c \Delta T_c} \frac{\sigma_c}{\rho_c C_{p,c}} U_c^2 B_c^2 =
\frac{L_c}{\Delta T_c} \frac{\sigma_c}{\rho_c C_{p,c}} U_c B_c^2 =
\frac{U_c^2}{C_{p,c} \Delta T_c} \frac{\sigma_c L_c}{\rho_c U_c} B_c^2 =
Ec \frac{B_c^2}{\mu_c \rho_c U_c^2} \mu_c \sigma_c L_c U_c =
Ec Al Re_m = Ec \frac{Ha^2}{Re}
\end{equation}

Where $Ec,Ha,Re$ are the Eckert, Hartmann and Reynolds numbers respectively.

\subsection{Coefficient of the viscous dissipation term}
The coefficient of the viscous dissipation term, we have

\begin{equation}
\frac{L_c}{U_c \Delta T_c} \frac{1}{\rho_c C_{p,c}} \frac{U_c^2}{L_c^2}\mu_c =
\frac{U_c^2}{C_{p,c} \Delta T_c} \frac{L_c}{\rho_c U_c} \frac{\mu_c}{L_c^2} =
Ec \frac{\mu_c}{\rho_c U_c L_c} =
\frac{Ec}{Re}
\end{equation}

\subsection{Coefficient of the heat generation term}
The coefficient of the heat generation term, we have

\begin{equation}
	\frac{L_c}{U_c \Delta T_c} \frac{1}{\rho_c C_{p,c}} \frac{\rho_c C_{p,c} \Delta T_c}{t_c} =
	\frac{L_c}{U_c} \frac{1}{t_c} = 1
\end{equation}

Note that in order to have no coefficients in front of heat generation terms, they must be normalized by the enthalpy per unit convective time.

\begin{equation}
\dot{Q}^* = \frac{\dot{Q}}{\rho_c C_{p,c} \Delta T_c / t_c}
\end{equation}

It is readily confirmed that the units are correct here since 

\begin{equation}
Force \times distance = work \qquad \qquad
\frac{force \times distance}{time} = power
\end{equation}

\begin{equation}
\frac{force \times distance}{volume \times time} = \frac{power}{volume} \qquad \qquad
\frac{pressure}{time} = \frac{power}{volume}
\end{equation}

Since the viscous dissipation and generation terms have units power/volume, the inverse of this is necessary to result in a dimensionless equation. As derived from above, the coefficients of these terms have units of time over pressure.

Putting this all together, we have
\begin{equation}
	\frac{\PD T^*}{\PD t^*} + \frac{\PD (u_j^* T^*)}{\PD x_j^*}
	= 
	\frac{1}{Re Pr}
	\frac{\PD}{\PD x_j^*}
	\left( k^* \frac{\PD T^*}{\PD x_j^*} \right) + 
	Ec \frac{Ha^2}{Re}
	\frac{{j^*}^2}{\sigma^*} +
	\frac{Ec}{Re}
	\left( \frac{\PD u_j^*}{\PD x_i^*} - \frac{\PD u_i^*}{\PD x_j^*} \right)
	\frac{\PD u_i^*}{\PD x_j^*} +
	\dot{Q}^*
\end{equation}

Removing the asterisks for simplicity, we have

\begin{equation}
	\frac{\PD T}{\PD t} +
	\frac{\PD (u_j T)}{\PD x_j}
	= 
	\frac{1}{Re Pr}
	\frac{\PD}{\PD x_j}
	\left( k \frac{\PD T}{\PD x_j} \right) + 
	Ec \frac{Ha^2}{Re}
	\frac{{j}^2}{\sigma} +
	\frac{Ec}{Re}
	\left( \frac{\PD u_j}{\PD x_i} - \frac{\PD u_i}{\PD x_j} \right)
	\frac{\PD u_i}{\PD x_j} +
	\dot{Q}
\end{equation}

\section{Implementing Periodic temperature BCs}
Implementing periodic temperature BCs is delicate because the net amount of heat must remain zero to avoid unbounded momentum forces. To achieve this, we may decompose temperature into mean and fluctuating, and implement periodic BCs on the fluctuating temperature.



\end{document}