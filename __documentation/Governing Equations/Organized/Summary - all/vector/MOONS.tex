\documentclass[11pt]{article}
\usepackage{graphicx}    % needed for including graphics e.g. EPS, PS
\usepackage{epstopdf}
\usepackage{amsmath}
\usepackage{hyperref}
\usepackage{xspace}
\usepackage{mathtools}
\usepackage{tikz}
\usepackage{epsfig}
\usepackage{float}
%\usepackage{natbib}
\usepackage{subfigure}
\usepackage{setspace}
\usepackage{tabularx,ragged2e,booktabs,caption}

\newcommand{\PATHTOUSEFULCOMMANDS}{../../useful_commands.tex}

\input{\PATHTOUSEFULCOMMANDS}

\setlength{\oddsidemargin}{0.1in}
\setlength{\textwidth}{7.25in}

\setlength{\topmargin}{-1in}     %\topmargin: gap above header
\setlength{\headheight}{0in}     %\headheight: height of header
\setlength{\topskip}{0in}        %\topskip: between header and text
\setlength{\headsep}{0in}
\setlength{\textheight}{692pt}   %\textheight: height of main text
\setlength{\textwidth}{7.5in}    % \textwidth: width of text
\setlength{\oddsidemargin}{-0.5in}  % \oddsidemargin: odd page left margin
\setlength{\evensidemargin}{0in} %\evensidemargin : even page left margin
\setlength{\parindent}{0.25in}   %\parindent: indentation of paragraphs
\setlength{\parskip}{0pt}        %\parskip: gap between paragraphs
\setlength{\voffset}{0.5in}


% Useful commands:

% \hfill		aligns-right everything right of \hfill

\begin{document}
\doublespacing
\title{Magnetohydrodynamic Object-Oriented Numerical Solver (MOONS)}
\author{C. Kawczynski \\
Department of Mechanical and Aerospace Engineering \\
University of California Los Angeles, USA\\}
% \maketitle

\section{Thermal Energy}
\Large
\begin{equation}
	\frac{\partial \theta}{\partial t}
	+ \DEL \DOT (\U \theta)
	=
	\frac{1}{Pe} \DEL \DOT (k \DEL \theta)
	+ Ec N \frac{{\J}^2}{\sigma}
	- \frac{Ec}{Re} \DEL^2 E_K
	+ \frac{Ec}{Re} (\DEL \U) \DOT (\DEL \U)
	+ \dot{Q}
\end{equation}
\section{Momentum}
\begin{equation}
	\frac{\partial \U}{\partial t}
	+ \DEL \DOT (\U^T \U)
	=
	- \DEL p
	+ \frac{1}{Re} \DEL^2 \U
	+ \frac{1}{Fr^2} \G
	+ \frac{Gr}{Re^2} \theta \G
	+ N \J \CROSS \B
\end{equation}
\section{Magnetic Field}
\begin{equation}
	\frac{\partial \B}{\partial t}
	+ \frac{1}{Re_m} \CURL \left(\frac{1}{\SO} \CURL \B \right)
	= \CURL (\U \CROSS \B)
\end{equation}
\section{Kinetic energy equation}
\begin{multline}
	\frac{\PD E_K}{\PD t}
	+ (\U \DOT \DEL) E_K
	=
	- \DEL \DOT (\U p)
	+ \frac{1}{Re} \DEL^2 E_K
	- \frac{1}{Re} (\DEL \U) \DOT (\DEL \U)
	- Al (\U \DOT \DEL) E_M
	+ Al \U \DOT (\B \DOT \DEL \H) \\
	+ \frac{Gr}{Re^2} \U \DOT (\rho \G \beta \theta)
	+ \U \DOT (\rho \G)
	+ \frac{L_c f_c}{\rho_c U_c^2} \U \DOT \F
\end{multline}
\section{Magnetic energy equation}
\begin{equation}
	\frac{\PD E_m}{\PD t} =
	- Re_m \frac{{\J}^2}{\SI}
	+ \DEL \DOT \left[\left(\U \CROSS \B - \frac{\J}{\SI} \right) \times \H \right]
	+ (\U \DOT \DEL) E_m
	- \U \DOT \left[(\B \DOT \DEL) \H \right]
\end{equation}
\section{Dimensionless groups, definitions and chosen scales}
\begin{equation}
	Re = \frac{U_c L_c}{\nu_c} \qquad
	Ha = B_c L_c \sqrt{\frac{\sigma_c}{\mu_c}} \qquad
	Re_m = \mu_c \sigma_c U_c L_c \qquad
	Pr = \frac{\nu_c}{\alpha_c} \qquad
	Al = \frac{B_c^2}{\mu_c \rho_c U_c^2} \qquad
\end{equation}
\begin{equation}
	Gr = \frac{g_c \beta_c \theta_c L_c^3}{\nu_c^2} \qquad
	Ec = \frac{U_c^2}{C_{p,c} \theta_c} \qquad
	Fr = \frac{U_c}{\sqrt{g_c L_c}} \qquad
	N = \frac{Ha^2}{Re} \qquad
	Pr = \frac{Pe}{Re}=\frac{Ra}{Gr}
\end{equation}
\begin{equation}
	Pr_{m} = \frac{Re_m}{Re} \qquad
	[t_c] = L_c/U_c \qquad
	[p] = \rho_c U_c^2 \qquad
	[j] = \sigma_c U_c B_c \qquad
	[\dot{Q}] = \rho_c C_{p,c} \theta_c / t_c
\end{equation}

\end{document}
