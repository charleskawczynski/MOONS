\documentclass[11pt]{article}
\usepackage{graphicx}    % needed for including graphics e.g. EPS, PS
\usepackage{epstopdf}
\usepackage{amsmath}
\usepackage{hyperref}
\usepackage{xspace}
\usepackage{mathtools}
\usepackage{tikz}
\usepackage{epsfig}
\usepackage{float}
%\usepackage{natbib}
\usepackage{subfigure}
\usepackage{setspace}
\usepackage{tabularx,ragged2e,booktabs,caption}
\usepackage{esint}

\newcommand{\PATHTOUSEFULCOMMANDS}{../../../../useful_commands.tex}

\input{\PATHTOUSEFULCOMMANDS}

\setlength{\oddsidemargin}{0.1in}
\setlength{\textwidth}{7.25in}

\setlength{\topmargin}{-1in}     %\topmargin: gap above header
\setlength{\headheight}{0in}     %\headheight: height of header
\setlength{\topskip}{0in}        %\topskip: between header and text
\setlength{\headsep}{0in}        
\setlength{\textheight}{692pt}   %\textheight: height of main text
\setlength{\textwidth}{7.5in}    % \textwidth: width of text
\setlength{\oddsidemargin}{-0.5in}  % \oddsidemargin: odd page left margin
\setlength{\evensidemargin}{0in} %\evensidemargin : even page left margin
\setlength{\parindent}{0.25in}   %\parindent: indentation of paragraphs
\setlength{\parskip}{0pt}        %\parskip: gap between paragraphs
\setlength{\voffset}{0.5in}


% Useful commands:

% \hfill		aligns-right everything right of \hfill

\begin{document}
\doublespacing
\title{Magnetohydrodynamic Object-Oriented Numerical Solver (MOONS)}
\author{C. Kawczynski \\
Department of Mechanical and Aerospace Engineering \\
University of California Los Angeles, USA\\
}
\maketitle

\section{Dimensional kinetic energy equation}
The dimensional momentum equation:
\begin{equation}
	\rho \left( \frac{\PD \U}{\PD t} + (\U \bullet \DEL) \U \right) = 
	-\DEL p + \mu \DEL^2 \U + \J \times \B + \rho \G + \rho \G \beta \Delta T + \F
\end{equation}
Dotting this with the velocity yields
\begin{equation}
	\U \DOT \left\{ \rho \left( \frac{\PD \U}{\PD t} + (\U \bullet \DEL) \U \right) = 
	-\DEL p + \mu \DEL^2 \U + \J \times \B + \rho \G + \rho \G \beta \Delta T + \F \right\}
\end{equation}
First, let
\begin{equation}
	E_K = \frac{1}{2} \rho \U \DOT \U
\end{equation}
Many of these derivations are performed in index notation for convenience and clarity.
\subsection{Unsteady term}
\begin{equation}
	\frac{\PD u_i u_i}{\PD t} =
	2 u_i \frac{\PD u_i}{\PD t}
	\rightarrow
	u_i \rho \frac{\PD u_i}{\PD t} =
	\frac{\PD}{\PD t} \left( \frac{1}{2} \rho u_i u_i \right) =
	\frac{\PD E_K}{\PD t}
\end{equation}

\subsection{Convection term}
\begin{equation}
	u_j \rho \frac{\PD u_i u_i}{\PD x_j} =
	2 \rho u_j u_i \frac{\PD u_i}{\PD x_j}
	\rightarrow
	\rho u_i u_j \frac{\PD u_i}{\PD x_j} =
	\frac{1}{2} \rho u_j \frac{\PD u_i u_i}{\PD x_j}
	= u_j \frac{\PD E_K}{\PD x_j}
\end{equation}
\subsection{Diffusion term}
\begin{equation}\begin{aligned}
	u_i \PD_{jj} u_i     = & \PD_j (u_i \PD_j u_i ) - (\PD_j u_i)^2  \\
	                       & u_i \PD_j u_i = \frac{1}{2} \PD_j (u_i u_i), \qquad \PD_j u_j = 0 \\
	                     = & \frac{1}{2} \PD_{jj} (u_i u_i) - (\PD_j u_i)^2 \\
	\mu u_i \PD_{jj} u_i = & \frac{\rho \mu}{\rho} \frac{1}{2} \PD_{jj} (u_i u_i) - \frac{\rho \mu}{\rho} (\PD_j u_i)^2 \\
	                     = & \nu \PD_{jj} E_K - \mu (\PD_j u_i)^2 \\
\end{aligned}\end{equation}

\subsection{Pressure term}
\begin{equation}
	\frac{\PD p u_i}{\PD x_i}
	= u_i \frac{\PD p}{\PD x_i}
	+ p \frac{\PD u_i}{\PD x_i}
	= u_i \frac{\PD p}{\PD x_i}
	\rightarrow
	u_i \frac{\PD p}{\PD x_i} =
	\frac{\PD (u_i p)}{\PD x_i}
\end{equation}

\subsection{Lorentz term}
Using vector notation (which is simpler here), we may compute this term as is, or:
\begin{equation}
	\U \DOT (\J \times \B) = 
	\U \DOT \left[\B \DOT \DEL \H -\DEL \left( \frac{1}{2} \B \DOT \H \right)\right] = 
	- (\U \DOT \DEL) E_m
	+ \U \DOT (\B \DOT \DEL \H)
	,\qquad
	E_m = \frac{1}{2} \B \DOT \H
\end{equation}

\subsection{Gravity term}
\begin{equation}
	\U \DOT (\rho \G)
\end{equation}
\subsection{Buoyancy term}
\begin{equation}
	\U \DOT (\rho \G \beta \Delta T)
\end{equation}
\subsection{General term}
\begin{equation}
	\U \DOT \F
\end{equation}

\section{Putting it all together}
\begin{equation}
	\boxed{
	\frac{\PD E_K}{\PD t} + 
	(\U \DOT \DEL) E_K = 
	- \DEL \DOT (\U p)
	+ \nu \DEL^2 E_K
	- \mu (\DEL \U) \DOT (\DEL \U)
	- (\U \DOT \DEL) E_m
	+ \U \DOT (\B \DOT \DEL \H)
	+ \U \DOT (\rho \G)
	+ \U \DOT (\rho \G \beta \Delta T)
	+ \U \DOT \F
	}
\end{equation}
Where
\begin{equation}
	E_K = \frac{1}{2} \rho \U \DOT \U, \qquad
	E_m = \frac{1}{2} \B \DOT \H
\end{equation}

\end{document}