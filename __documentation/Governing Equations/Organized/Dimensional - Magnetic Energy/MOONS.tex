\documentclass[11pt]{article}
\usepackage{graphicx}    % needed for including graphics e.g. EPS, PS
\usepackage{epstopdf}
\usepackage{amsmath}
\usepackage{hyperref}
\usepackage{xspace}
\usepackage{mathtools}
\usepackage{tikz}
\usepackage{epsfig}
\usepackage{float}
%\usepackage{natbib}
\usepackage{subfigure}
\usepackage{setspace}
\usepackage{esint}
\usepackage{tabularx,ragged2e,booktabs,caption}

\newcommand{\PATHTOUSEFULCOMMANDS}{../../../../useful_commands.tex}

\input{\PATHTOUSEFULCOMMANDS}

\setlength{\oddsidemargin}{0.1in}
\setlength{\textwidth}{7.25in}

\setlength{\topmargin}{-1in}     %\topmargin: gap above header
\setlength{\headheight}{0in}     %\headheight: height of header
\setlength{\topskip}{0in}        %\topskip: between header and text
\setlength{\headsep}{0in}        
\setlength{\textheight}{692pt}   %\textheight: height of main text
\setlength{\textwidth}{7.5in}    % \textwidth: width of text
\setlength{\oddsidemargin}{-0.5in}  % \oddsidemargin: odd page left margin
\setlength{\evensidemargin}{0in} %\evensidemargin : even page left margin
\setlength{\parindent}{0.25in}   %\parindent: indentation of paragraphs
\setlength{\parskip}{0pt}        %\parskip: gap between paragraphs
\setlength{\voffset}{0.5in}


% Useful commands:

% \hfill		aligns-right everything right of \hfill

\begin{document}
\doublespacing
\title{Magnetohydrodynamic Object-Oriented Numerical Solver (MOONS)}
\author{C. Kawczynski \\
Department of Mechanical and Aerospace Engineering \\
University of California Los Angeles, USA\\
}
\maketitle

\section{Dimensional magnetic energy equation}
The dimensional induction equation:
\begin{equation}
	\frac{\PD \B}{\PD t} + \CURL \left( \sigma^{-1}\CURL \H \right) 
	= \CURL (\U \times \B)
\end{equation}
Multiplying by $\B / \M$, we have
\begin{equation}
	\H \DOT
	\left(
	\frac{\PD \B}{\PD t} +
	\DEL \times (\SII \DEL \times \H) =
	\DEL \times (\U \times \B)
	\right), \qquad \H = \frac{\B}{\M}
\end{equation}
First, let
\begin{equation}
	E_M = \frac{\B^2}{2\M}
\end{equation}

\subsection{Transient term}
\begin{equation}
	\frac{\PD}{\PD t} \frac{\B^2}{\M} = 2 \H \DOT \frac{\PD \B}{\PD t} 
	\rightarrow
	\H \DOT \frac{\PD \B}{\PD t} =
	\frac{\PD E_M}{\PD t}
\end{equation}
\subsection{Diffusion term}
\begin{equation}\begin{aligned}
	\H \DOT \DEL \times \left( \SII \DEL \times \H \right) = \H \DOT \DEL \times \left( \JS \right) & \\
	\DEL \DOT (\A \times \B) = \B \DOT \DEL \times \A - \A \DOT \DEL \times \B &, \qquad \text{vector identity} \\
	\DEL \DOT \left(\JS \times \H \right) = 
	\H \DOT \DEL \times \JS - \JS \DOT \DEL \times \H = 
	\H \DOT \DEL \times \JS - \JSS, & \qquad \text{ we may write} \\
	\H \DOT \DEL \times \JS = 
	\left\{ \DEL \DOT \left(\JS \times \H \right) + \JSS \right\}
\end{aligned}\end{equation}

\section{Convection term}
We may apply the vector identity from above to the advection term
\begin{equation}\begin{aligned}
	\DEL \DOT (\A \times \B) = \B \DOT \DEL \times \A - \A \DOT \DEL \times \B                                                                                                                      , & \qquad \text{vector identity} \\
	\DEL \DOT \left(\H \times \A \right) = \A \DOT \DEL \times \H - \H \DOT \DEL \times \A, \qquad \A = \U \times \B                                                                             , & \qquad \text{using vector identity} \\
	\H \DOT \DEL \times \A = \A \DOT \DEL \times \H - \DEL \DOT \left(\H \times \A \right), \qquad \A = \U \times \B                                                                             , & \\
	\H \DOT \DEL \times (\U \times \B) = (\U \times \B) \DOT \J - \DEL \DOT \left(\H \times (\U \times \B) \right)                                                                                  , & \qquad \text{subs $\A$ \& Ampere's law} \\
	\A \DOT (\B \times \C) = \B \DOT (\C \times \A) = \C \DOT (\A \times \B) = - \C \DOT (\B \times \A)                                                                                                   , & \qquad \text{vector identity} & \\
	\H \DOT \DEL \times (\U \times \B) = -\U \DOT (\J \times \B) - \DEL \DOT \left(\H \times (\U \times \B) \right)                                                                                 , & \\
	\H \DOT \DEL \times (\U \times \B) = - \U \DOT (\J \times \B) - \DEL \DOT \left(\H \times \left(\JS - \E \right) \right)                                                                        , & \qquad \text{Apply Ohm's law} \\
	\H \DOT \CURL (\U \times \B) = - \U \DOT (\J \times \B) + \DEL \DOT \left(\JS \times \H \right) - \DEL \DOT \left(\E \times \H \right)                                                         , & \\
	\H \DOT \CURL (\U \times \B) = (\U \DOT \DEL) E_M - \U \DOT \left[(\B \DOT \DEL) \left( \H \right) \right] + \DEL \DOT \left(\JS \times \H \right) - \DEL \DOT \left(\E \times \H \right) , & \qquad \text{$\B$ pressure \& shear}
\end{aligned}\end{equation}
which used $\J \times \B = (\B \DOT \DEL) \left(\frac{\B}{\M}\right) - \DEL \left( \frac{\B^2}{2\M} \right) = (\B \DOT \DEL) \left(\frac{\B}{\M}\right) - \DEL E_M$

\section{Putting it all together}
Putting it all together, we have
\begin{equation}
	\frac{\PD E_M}{\PD t} + 
	\left\{ \DEL \DOT \left(\JS \times \H \right) + \JSS \right\} =
	(\U \DOT \DEL) E_M
	- \U \DOT \left[(\B \DOT \DEL) \left( \H \right) \right] 
	+ \DEL \DOT \left(\JS \times \H \right) 
	- \DEL \DOT \left(\E \times \H \right)
\end{equation}
Canceling terms yields two forms
\begin{equation}\boxed{\begin{aligned}
	\frac{\PD E_M}{\PD t} = &
	- \JSS
	- \DEL \DOT \left(\E \times \H \right)
	+ (\U \DOT \DEL) E_M
	- \U \DOT \left[(\B \DOT \DEL) \left( \H \right) \right]  \\
	\frac{\PD E_M}{\PD t} = &
	- \JSS
	- \DEL \DOT \left(\E \times \H \right)
	- \U \DOT (\J \times \B)
\end{aligned}}\end{equation}

\end{document}