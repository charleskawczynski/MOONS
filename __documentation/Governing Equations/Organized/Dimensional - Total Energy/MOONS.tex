\documentclass[11pt]{article}
\usepackage{graphicx}    % needed for including graphics e.g. EPS, PS
\usepackage{epstopdf}
\usepackage{amsmath}
\usepackage{hyperref}
\usepackage{xspace}
\usepackage{mathtools}
\usepackage{tikz}
\usepackage{epsfig}
\usepackage{float}
%\usepackage{natbib}
\usepackage{subfigure}
\usepackage{setspace}
\usepackage{tabularx,ragged2e,booktabs,caption}

\newcommand{\PATHTOUSEFULCOMMANDS}{../../../../useful_commands.tex}

\input{\PATHTOUSEFULCOMMANDS}

\setlength{\oddsidemargin}{0.1in}
\setlength{\textwidth}{7.25in}

\setlength{\topmargin}{-1in}     %\topmargin: gap above header
\setlength{\headheight}{0in}     %\headheight: height of header
\setlength{\topskip}{0in}        %\topskip: between header and text
\setlength{\headsep}{0in}        
\setlength{\textheight}{692pt}   %\textheight: height of main text
\setlength{\textwidth}{7.5in}    % \textwidth: width of text
\setlength{\oddsidemargin}{-0.5in}  % \oddsidemargin: odd page left margin
\setlength{\evensidemargin}{0in} %\evensidemargin : even page left margin
\setlength{\parindent}{0.25in}   %\parindent: indentation of paragraphs
\setlength{\parskip}{0pt}        %\parskip: gap between paragraphs
\setlength{\voffset}{0.5in}


% Useful commands:

% \hfill		aligns-right everything right of \hfill

\begin{document}
\doublespacing
\title{Magnetohydrodynamic Object-Oriented Numerical Solver (MOONS)}
\author{C. Kawczynski \\
Department of Mechanical and Aerospace Engineering \\
University of California Los Angeles, USA\\
}
\maketitle

\section{Dimensional total energy equation}
Adding the thermal, kinetic and magnetic energy equations
\begin{equation}
	\rho C_p \left[ \frac{\PD T}{\PD t} 
	+ (\U \DOT \DEL) T \right] 
	= \DEL \DOT (k \DEL T) 
	+ \frac{1}{\sigma} \J^2 
	- \nu \DEL^2 E_K
	+ \mu (\DEL \U) \DOT (\DEL \U)
\end{equation}
\begin{equation}
	\frac{\PD E_K}{\PD t} + 
	(\U \DOT \DEL) E_K = 
	- \DEL \DOT (\U p)
	+ \nu \DEL^2 E_K
	- \mu (\DEL \U) \DOT (\DEL \U)
	- (\U \DOT \DEL) E_M
	+ \U \DOT \left[\B \DOT \DEL \left( \H \right) \right] 
\end{equation}
\begin{equation}
	\frac{\PD E_M}{\PD t} =
	- \JSS
	- \DEL \DOT \left(\E \times \H \right)
	+ (\U \DOT \DEL) E_M
	- \U \DOT \left[(\B \DOT \DEL) \left( \H \right) \right]
\end{equation}
Magnetic + Kinetic
\begin{equation}
	\frac{\PD E_K}{\PD t} 
	+ \frac{\PD E_M}{\PD t}
	+ (\U \DOT \DEL) E_K = 
	- \JSS
	- \DEL \DOT \left(\E \times \H \right)
	- \DEL \DOT (\U p)
	+ \nu \DEL^2 E_K
	- \mu (\DEL \U) \DOT (\DEL \U)
\end{equation}
Assuming viscous losses are negligible, and since flow is isothermal, the Joule heating will balance kinetic energy diffusion, so we'll have
\begin{equation}\begin{aligned}
	\frac{\PD (E_K + E_M)}{\PD t} 
	+ (\U \DOT \DEL) E_K = 
	- \DEL \DOT \left(\E \times \H \right)
	- \DEL \DOT (\U p) \\
	\JSS = \nu \DEL^2 E_K
\end{aligned}\end{equation}
At steady state, we have
\begin{equation}\begin{aligned}
	(\U \DOT \DEL) E_K = 
	- \DEL \DOT \left(\E \times \H \right)
	- \DEL \DOT (\U p) \\
	\JSS = \nu \DEL^2 E_K
\end{aligned}\end{equation}
If we consider the magnetic energy equation in the vacuum domain, there is not motion, and no currents, so we have
\begin{equation}
	\frac{\PD E_M}{\PD t} =	- \DEL \DOT \left(\E \times \H \right)
\end{equation}
which means, at steady state, the RHS is zero! Therefore we have
\begin{equation}\begin{aligned}
	(\U \DOT \DEL) E_K = - \DEL \DOT (\U p) \\
	\JSS = \nu \DEL^2 E_K
\end{aligned}\end{equation}
Let's scale this by
\begin{equation}\begin{aligned}
	[\U] = U, \qquad
	[\DEL] = 1/L, \qquad
	[p] = \frac{1}{2} \rho U^2, \qquad
	[\J] = \sigma_c U B
\end{aligned}\end{equation}
To get
\begin{equation}\begin{aligned}
	\frac{\rho U^3}{2L} (\U \DOT \DEL) ( \U \DOT \U ) = 
	- \frac{\rho U^3}{2L} \DEL \DOT (\U p) \\
	(\U \DOT \DEL) ( \U \DOT \U ) = 
	- \DEL \DOT (\U p), \qquad
\end{aligned}\end{equation}
and
\begin{equation}\begin{aligned}
	\frac{\sigma_c^2 U^2 B^2}{\sigma_c} \JSS = 
	\frac{U^2}{L^2} \mu \DEL^2 \left( \frac{1}{2} \U \DOT \U \right) \\
	\frac{\sigma_c B^2 L^2}{\mu} \JSS = 
	\DEL^2 \left( \frac{1}{2} \U \DOT \U \right) \\
	Ha^2 \JSS = 
	\DEL^2 \left( \frac{1}{2} \U \DOT \U \right)
\end{aligned}\end{equation}

% According to the paper I was reading about how Hartmann number is equal to Reynolds times magnetic Reynolds, we have
% \begin{equation}\begin{aligned}
% 	Re Re_m \JSS = \DEL^2 \left( \frac{1}{2} \U \DOT \U \right) \\
% 	\frac{\JSS}{\DEL^2 \left( \frac{1}{2} \U \DOT \U \right)} = \frac{(Re_m)^{-1}}{Re}
% \end{aligned}\end{equation}

\end{document}