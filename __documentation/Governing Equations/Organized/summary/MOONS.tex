\documentclass[11pt]{article}
\usepackage{graphicx}    % needed for including graphics e.g. EPS, PS
\usepackage{epstopdf}
\usepackage{amsmath}
\usepackage{hyperref}
\usepackage{xspace}
\usepackage{mathtools}
\usepackage{tikz}
\usepackage{epsfig}
\usepackage{float}
%\usepackage{natbib}
\usepackage{subfigure}
\usepackage{setspace}
\usepackage{tabularx,ragged2e,booktabs,caption}

\newcommand{\PATHTOUSEFULCOMMANDS}{../../../../useful_commands.tex}

\input{\PATHTOUSEFULCOMMANDS}

\newcommand{\OB}{\overbrace{\rightarrow}^{\left(\frac{t_c}{\rho_c {C_p}_c \theta_c}\right)}}
\newcommand{\Cp}{{C_p}_c}

\setlength{\oddsidemargin}{0.1in}
\setlength{\textwidth}{7.25in}

\setlength{\topmargin}{-1in}     %\topmargin: gap above header
\setlength{\headheight}{0in}     %\headheight: height of header
\setlength{\topskip}{0in}        %\topskip: between header and text
\setlength{\headsep}{0in}
\setlength{\textheight}{692pt}   %\textheight: height of main text
\setlength{\textwidth}{7.5in}    % \textwidth: width of text
\setlength{\oddsidemargin}{-0.5in}  % \oddsidemargin: odd page left margin
\setlength{\evensidemargin}{0in} %\evensidemargin : even page left margin
\setlength{\parindent}{0.25in}   %\parindent: indentation of paragraphs
\setlength{\parskip}{0pt}        %\parskip: gap between paragraphs
\setlength{\voffset}{0.5in}


% Useful commands:

% \hfill		aligns-right everything right of \hfill

\begin{document}
\doublespacing
\title{Magnetohydrodynamic Object-Oriented Numerical Solver (MOONS)}
\author{C. Kawczynski \\
Department of Mechanical and Aerospace Engineering \\
University of California Los Angeles, USA\\
}
\maketitle

Using scales
\begin{equation}\begin{aligned}
	t_c = L_c/ U_c \qquad
	J_c = U_c \sigma_c B_c \qquad
	g_c = U_c^2/L_c \qquad
	p_c = \rho_c U_c^2 \qquad
	\alpha_c = \frac{k_c}{\rho_c {C_p}_c}
\end{aligned}\end{equation}

\section{Thermal energy equation}
\begin{equation}
	\frac{\PD \theta}{\PD t}
	+ (\U \DOT \DEL) \theta
	= \frac{1}{Pe} \DEL \DOT (k \DEL \theta)
	+ \frac{Ec}{Re} ( \DEL {\U}^T + \DEL \U ) \DEL \U
	+ Ec \left( \frac{L_c}{U_c} \frac{\dot{Q}_c}{\rho_c U_c^2} \right) \dot{Q}
\end{equation}
\section{Kinetic energy equation}
\begin{multline}
	\frac{\PD E_K}{\PD t}
	+ (\U \DOT \DEL) E_K
	=
	- \DEL \DOT (\U p)
	+ \frac{1}{Re} \DEL^2 E_K
	- \frac{1}{Re} (\DEL \U) \DOT (\DEL \U)
	- Al (\U \DOT \DEL) E_M
	+ Al \U \DOT (\B \DOT \DEL \H) \\
	+ \U \DOT (\rho \G)
	+ \frac{Gr}{Re^2} \U \DOT (\rho \G \beta \theta)
	+ \frac{L_c f_c}{\rho_c U_c^2} \U \DOT \F
\end{multline}

\section{Magnetic energy equation}
\begin{equation}
	\frac{\PD E_m}{\PD t} =
	- Re_m \frac{{\J^*}^2}{\SI^*}
	+ \DEL \DOT \left[\left(\U \CROSS \B - \frac{\J}{\SI} \right) \times \H \right]
	+ (\U \DOT \DEL) E_m
	- \U \DOT \left[(\B \DOT \DEL) \H \right]
\end{equation}

NOTE: Equations must be multiplied by appropriate factors to compare energies ($E_m = \frac{1}{2} \B \DOT \H$).

\end{document}