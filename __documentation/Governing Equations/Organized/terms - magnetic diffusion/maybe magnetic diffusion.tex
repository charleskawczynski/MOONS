\documentclass[11pt]{article}
\usepackage{graphicx}    % needed for including graphics e.g. EPS, PS
\usepackage{epstopdf}
\usepackage{amsmath}
\usepackage{hyperref}
\usepackage{xspace}
\usepackage{mathtools}
\usepackage{tikz}
\usepackage{epsfig}
\usepackage{float}
%\usepackage{natbib}
\usepackage{subfigure}
\usepackage{setspace}
\usepackage{esint}
\usepackage{tabularx,ragged2e,booktabs,caption}

\newcommand{\PATHTOUSEFULCOMMANDS}{../../../../useful_commands.tex}

\input{\PATHTOUSEFULCOMMANDS}

\setlength{\oddsidemargin}{0.1in}
\setlength{\textwidth}{7.25in}

\setlength{\topmargin}{-1in}     %\topmargin: gap above header
\setlength{\headheight}{0in}     %\headheight: height of header
\setlength{\topskip}{0in}        %\topskip: between header and text
\setlength{\headsep}{0in}        
\setlength{\textheight}{692pt}   %\textheight: height of main text
\setlength{\textwidth}{7.5in}    % \textwidth: width of text
\setlength{\oddsidemargin}{-0.5in}  % \oddsidemargin: odd page left margin
\setlength{\evensidemargin}{0in} %\evensidemargin : even page left margin
\setlength{\parindent}{0.25in}   %\parindent: indentation of paragraphs
\setlength{\parskip}{0pt}        %\parskip: gap between paragraphs
\setlength{\voffset}{0.5in}


% Useful commands:

% \hfill		aligns-right everything right of \hfill

\begin{document}
\doublespacing
\title{Magnetohydrodynamic Object-Oriented Numerical Solver (MOONS)}
\author{C. Kawczynski \\
Department of Mechanical and Aerospace Engineering \\
University of California Los Angeles, USA\\
}
\maketitle

\section{Dimensional magnetic energy equation}

\section{Magnetic energy diffusion?}
\begin{align}
	\frac{\PD}{\PD x_j} \frac{\PD}{\PD x_j} \frac{\B^2}{2 \mu}
	& =
	\frac{\PD}{\PD x_j} 
	\left[
	\frac{\B}{2\mu} \frac{\PD}{\PD x_j} \B +  
	\frac{\B}{2} \frac{\PD}{\PD x_j} \H
	\right] \\
	& =
	\frac{\B}{2\mu} \frac{\PD^2 \B}{\PD x_jx_j} + 
	\frac{\PD}{\PD x_j} \H \frac{\PD}{\PD x_j} \frac{\B}{2} + 
	\frac{\B}{2} \frac{\PD^2 }{\PD x_jx_j} \H + 
	\frac{\PD}{\PD x_j} \frac{\B}{2} \frac{\PD}{\PD x_j} \H \\
	& =
	\frac{\B}{\mu} \frac{\PD^2 \B}{\PD x_jx_j} + 
	\frac{\PD}{\PD x_j} \H \frac{\PD}{\PD x_j} \B \qquad \text{for uniform $\mu$} \\
	& = \H \DEL^2 \B + \mu \left(\DEL \left( \H \right) \right)^2
\end{align}

\begin{equation}\begin{aligned}
	\DEL^2 E_m = \H \DOT \DEL^2 \B + \mu \left( \DEL \H \right)^2 \\
	\mu \DEL^2 E_m = \mu \H \DOT \DEL^2 \B + \left( \mu \DEL \H \right)^2 \\
\end{aligned}\end{equation}

\begin{align}
	\frac{\PD}{\PD x_j} \frac{\PD}{\PD x_j} \frac{B_i^2}{2 \mu}
	& =
	\frac{\PD}{\PD x_j} 
	\left[
	\frac{B_i}{2\mu} \frac{\PD}{\PD x_j} B_i +  
	\frac{B_i}{2} \frac{\PD}{\PD x_j} \frac{B_i}{\mu}
	\right] \\
	& =
	\frac{B_i}{2\mu} \frac{\PD^2 B_i}{\PD x_jx_j} + 
	\frac{\PD}{\PD x_j} \frac{B_i}{\mu} \frac{\PD}{\PD x_j} \frac{B_i}{2} + 
	\frac{B_i}{2} \frac{\PD^2 }{\PD x_jx_j} \frac{B_i}{\mu} + 
	\frac{\PD}{\PD x_j} \frac{B_i}{2} \frac{\PD}{\PD x_j} \frac{B_i}{\mu} \\
	& =
	\frac{B_i}{\mu} \frac{\PD^2 B_i}{\PD x_jx_j} + 
	\frac{\PD}{\PD x_j} \frac{B_i}{\mu} \frac{\PD}{\PD x_j} B_i \qquad \text{for uniform $\mu$} \\
	& = \frac{B_i}{\mu} \DEL^2 B_i + \mu \left(\DEL \left( \frac{B_i}{\mu} \right) \right)^2
\end{align}

Assuming uniform $\mu$, we have
\begin{align}
	\PD_{jj} B_i^2 = 
	& =	2 \PD_j (B_i \PD_j B_i) \\
	& =	2 (B_i \PD_{jj} B_i + (\PD_j B_i)^2) \\
\end{align}
In vector form, we may write this as
\begin{equation}
	\frac{1}{2\mu} \DEL^2 (\B \DOT \B) = \frac{1}{\mu} (\B \DOT \DEL^2 \B + \DEL \B \DOT \DEL \B)
\end{equation}

Or
\begin{equation}\begin{aligned}
	\U \DOT \left[(\B \DOT \DEL) \left( \H \right) \right] \\
	B_j \PD_j B_i \\
\end{aligned}\end{equation}

\end{document}