\documentclass[11pt]{article}
\usepackage{graphicx}    % needed for including graphics e.g. EPS, PS
\usepackage{epstopdf}
\usepackage{amsmath}
\usepackage{hyperref}
\usepackage{xspace}
\usepackage{mathtools}
\usepackage{tikz}
\usepackage{epsfig}
\usepackage{float}
%\usepackage{natbib}
\usepackage{subfigure}
\usepackage{setspace}
\usepackage{tabularx,ragged2e,booktabs,caption}

\newcommand{\PATHTOUSEFULCOMMANDS}{../../../../useful_commands.tex}

\input{\PATHTOUSEFULCOMMANDS}

\setlength{\oddsidemargin}{0.1in}
\setlength{\textwidth}{7.25in}

\setlength{\topmargin}{-1in}     %\topmargin: gap above header
\setlength{\headheight}{0in}     %\headheight: height of header
\setlength{\topskip}{0in}        %\topskip: between header and text
\setlength{\headsep}{0in}
\setlength{\textheight}{692pt}   %\textheight: height of main text
\setlength{\textwidth}{7.5in}    % \textwidth: width of text
\setlength{\oddsidemargin}{-0.5in}  % \oddsidemargin: odd page left margin
\setlength{\evensidemargin}{0in} %\evensidemargin : even page left margin
\setlength{\parindent}{0.25in}   %\parindent: indentation of paragraphs
\setlength{\parskip}{0pt}        %\parskip: gap between paragraphs
\setlength{\voffset}{0.5in}


% Useful commands:

% \hfill		aligns-right everything right of \hfill

\begin{document}
\doublespacing
\title{Magnetohydrodynamic Object-Oriented Numerical Solver (MOONS)}
\author{C. Kawczynski \\
Department of Mechanical and Aerospace Engineering \\
University of California Los Angeles, USA\\
}
\maketitle

\section{Definition}
Using index notation, the dimensional viscous dissipation term, $\Phi$, is (as defined in Professor Lavine's notes)
\begin{equation}
\Phi=\mu \left( \frac{\PD u_j}{\PD x_i}+\frac{\PD u_i}{\PD x_j} \right) \frac{\PD u_i}{\PD x_j}, \qquad i=\text{ $i$th component of momentum equation}
\end{equation}
This definition may be for the thermal energy equation (which may have a different sign).

This term, and its sign, is most easily obtained in the kinetic energy equation by dotting momentum with velocity:
\begin{equation}
	\boxed{
	u_i \mu \frac{\PD^2 u_i}{\PD x_j^2} =
	\mu \frac{\PD^2}{\PD x_j^2} \frac{1}{2} u_i u_i -
	\mu \frac{\PD u_i}{\PD x_j}\frac{\PD u_i}{\PD x_j}
	}
\end{equation}
As outlined in Sergey's suggested form.

Take special note that writing in terms of $KE$, which includes density, will take the form
\begin{equation}
	\boxed{
	u_i \mu \frac{\PD^2 u_i}{\PD x_j^2} =
	\nu \frac{\PD^2 KE}{\PD x_j^2} -
	\mu \frac{\PD u_i}{\PD x_j}\frac{\PD u_i}{\PD x_j}
	}
\end{equation}

\section{Forms of viscous dissipation term}
The viscous dissipation term, $\Phi$, may take several forms:

\subsection{Professor Lavine's class notes}
\begin{equation}
\Phi=\mu \left( \frac{\PD u_j}{\PD x_i}+\frac{\PD u_i}{\PD x_j} \right) \frac{\PD u_i}{\PD x_j}
\end{equation}

\subsection{Professor Kim's Turbulence class (and text) in kinetic energy equation}
First, let
\begin{equation} \label{eq:strainRate}
	S_{ij} = \frac{1}{2}
	\left(
	\frac{\PD u_i}{\PD x_j} +
	\frac{\PD u_j}{\PD x_i}
	\right)
\end{equation}
And note that
\begin{equation} \label{eq:strainIdentity}
	2 S_{ij} S_{ij} =
	\frac{1}{2} \left( \frac{\PD u_i}{\PD x_j} + \frac{\PD u_j}{\PD x_i} \right)
	\left( \frac{\PD u_i}{\PD x_j} + \frac{\PD u_j}{\PD x_i} \right)
	= \frac{1}{2} \left( \frac{\PD u_i}{\PD x_j} \frac{\PD u_i}{\PD x_j}
	+ 2 \frac{\PD u_i}{\PD x_j} \frac{\PD u_j}{\PD x_i}
	+ \frac{\PD u_j}{\PD x_i} \frac{\PD u_j}{\PD x_i} \right)
	= \frac{\PD u_i}{\PD x_j} \frac{\PD u_i}{\PD x_j}
	+ \frac{\PD u_i}{\PD x_j} \frac{\PD u_j}{\PD x_i}
\end{equation}
since no free indexes exist and indexes may be swapped. In addition, we may write
\begin{equation}
	\frac{\PD}{\PD x_j} u_i \frac{\PD u_i}{\PD x_j} =
	\frac{\PD u_i}{\PD x_j}\frac{\PD u_i}{\PD x_j} +
	u_i \frac{\PD^2 u_i}{\PD x_j^2}
\end{equation}
Therefore
\begin{equation} \label{eq:secondIdentity}
	u_i \frac{\PD^2 u_i}{\PD x_j^2} =
	\frac{\PD}{\PD x_j} \left[ u_i \frac{\PD u_i}{\PD x_j} \right] -
	\frac{\PD u_i}{\PD x_j}\frac{\PD u_i}{\PD x_j}
\end{equation}
Adding and subtracting $\frac{\PD}{\PD x_j} u_i \frac{\PD u_j}{\PD x_i}$ on the RHS, we have
\begin{equation}
	u_i \frac{\PD^2 u_i}{\PD x_j^2} =
	\frac{\PD}{\PD x_j} \left[ u_i \frac{\PD u_i}{\PD x_j} + u_i \frac{\PD u_j}{\PD x_i} \right] -
	\frac{\PD}{\PD x_j} \left[ u_i \frac{\PD u_j}{\PD x_i} \right] -
	\frac{\PD u_i}{\PD x_j}\frac{\PD u_i}{\PD x_j}
\end{equation}
Using \ref{eq:strainRate}, and expanding $\frac{\PD}{\PD x_j} u_i \frac{\PD u_j}{\PD x_i}$, we have
\begin{equation}
	u_i \frac{\PD^2 u_i}{\PD x_j^2} =
	\frac{\PD}{\PD x_j} \left[ u_i 2 S_{ij} \right] -
	\left(
	\underbrace{u_i \frac{\PD^2 u_j}{\PD x_j x_i}}_{=0 \text{ (swap ij, $\PD_i u_i =0$)}}
	+
	\frac{\PD u_i}{\PD x_j} \frac{\PD u_j}{\PD x_i}
	\right) -
	\frac{\PD u_i}{\PD x_j}\frac{\PD u_i}{\PD x_j}
\end{equation}
Using \ref{eq:strainIdentity}, we have
\begin{equation}
	u_i \frac{\PD^2 u_i}{\PD x_j^2} =
	\frac{\PD}{\PD x_j} \left[ 2 u_i S_{ij} \right] -
	2 S_{ij} S_{ij}
\end{equation}
Swapping the index of the first term on the RHS, and using $S_{ij}=S_{ji}$, we have
\begin{equation}
	u_i \frac{\PD^2 u_i}{\PD x_j^2} =
	\frac{\PD}{\PD x_i} \left[ 2 u_j S_{ij} \right] -
	2 S_{ij} S_{ij}
\end{equation}
Finally, we have
\begin{equation}
	\boxed{
	u_i \mu \frac{\PD^2 u_i}{\PD x_j^2} =
	\frac{\PD}{\PD x_i} \left[ 2 \mu u_j S_{ij} \right] -
	2 \mu S_{ij} S_{ij}
	}
\end{equation}

\subsection{Sergey's suggested form (in kinetic energy equation)}
Starting from
\begin{equation}
	u_i \frac{\PD^2 u_i}{\PD x_j^2} =
	\frac{\PD}{\PD x_j} \left[ u_i \frac{\PD u_i}{\PD x_j} \right] -
	\frac{\PD u_i}{\PD x_j}\frac{\PD u_i}{\PD x_j}
\end{equation}
and using
\begin{equation}
	u_i \frac{\PD u_i}{\PD x_j} =
	\frac{1}{2} \frac{\PD u_i u_i}{\PD x_j}
\end{equation}
with $\PD_j u_j = 0$, we have
\begin{equation}
	u_i \frac{\PD^2 u_i}{\PD x_j^2} =
	\frac{1}{\rho} \frac{\PD^2 E_K}{\PD x_j^2} -
	\frac{\PD u_i}{\PD x_j}\frac{\PD u_i}{\PD x_j}
\end{equation}
Therefore
\begin{equation}
	\boxed{
	u_i \mu \frac{\PD^2 u_i}{\PD x_j^2} =
	\nu \frac{\PD^2 E_K}{\PD x_j^2} -
	\mu \frac{\PD u_i}{\PD x_j}\frac{\PD u_i}{\PD x_j}
	}
\end{equation}

\section{Summary}
Sergey's definition
\begin{equation}
	\boxed{
	u_i \mu \frac{\PD^2 u_i}{\PD x_j^2} =
	\nu \frac{\PD^2 E_K}{\PD x_j^2} -
	\mu \frac{\PD u_i}{\PD x_j}\frac{\PD u_i}{\PD x_j}
	}
\end{equation}
John Kim and Lavine definition
\begin{equation}
	\boxed{
	u_i \mu \frac{\PD^2 u_i}{\PD x_j^2} =
	\frac{\PD}{\PD x_i} \left[ 2 \mu u_j S_{ij} \right] -
	2 \mu S_{ij} S_{ij}
	}
\end{equation}


\end{document}