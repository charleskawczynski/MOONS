\documentclass[11pt]{article}
\usepackage{graphicx}    % needed for including graphics e.g. EPS, PS
\usepackage{epstopdf}
\usepackage{amsmath}
\usepackage{hyperref}
\usepackage{xspace}
\usepackage{mathtools}
\usepackage{tikz}
\usepackage{epsfig}
\usepackage{float}
%\usepackage{natbib}
\usepackage{subfigure}
\usepackage{setspace}
\usepackage{tabularx,ragged2e,booktabs,caption}


\setlength{\oddsidemargin}{0.1in}
\setlength{\textwidth}{7.25in}

\setlength{\topmargin}{-1in}     %\topmargin: gap above header
\setlength{\headheight}{0in}     %\headheight: height of header
\setlength{\topskip}{0in}        %\topskip: between header and text
\setlength{\headsep}{0in}
\setlength{\textheight}{692pt}   %\textheight: height of main text
\setlength{\textwidth}{7.5in}    % \textwidth: width of text
\setlength{\oddsidemargin}{-0.5in}  % \oddsidemargin: odd page left margin
\setlength{\evensidemargin}{0in} %\evensidemargin : even page left margin
\setlength{\parindent}{0.25in}   %\parindent: indentation of paragraphs
\setlength{\parskip}{0pt}        %\parskip: gap between paragraphs
\setlength{\voffset}{0.5in}


% Useful commands:

% \hfill		aligns-right everything right of \hfill

\begin{document}
\doublespacing
\title{Magnetohydrodynamic Object-Oriented Numerical Solver (MOONS)}
\author{C. Kawczynski \\
Department of Mechanical and Aerospace Engineering \\
University of California Los Angeles, USA\\
}
\maketitle

\section{Energy}

\subsection{Kinetic}
Consider an infinitesimal measure of kinetic energy $KE$
\begin{equation}
	d(KE) = \frac{1}{2} \mathbf{u}^2 dm
\end{equation}
Where $m=\rho V$ is the mass, $V$ is the volume, $u$ is velocity and $\rho$ is density.
\begin{equation}
	d(KE) = \frac{1}{2} \rho \mathbf{u}^2 dV
\end{equation}
Integrating this we have
\begin{equation}
	KE = \frac{1}{2} \int_V \rho \mathbf{u}^2 dV
\end{equation}
For uniform density we have
\begin{equation}
	KE = \frac{1}{2} \rho \int_V \mathbf{u}^2 dV
\end{equation}
This is the dimensional form of the kinetic energy. Using the following non-dimensionalization
\begin{equation}
	u^* = \frac{u}{U},
	\qquad
	\rho^* = \frac{\rho}{\rho_c},
	\qquad
	V^* = \frac{V}{L^3}
\end{equation}
We have
\begin{equation}
	\boxed{
	KE = \rho_c U^2 L^3 \left( \frac{1}{2} \rho^* \int_V {\mathbf{u}^*}^2 dV^* \right)
	}
\end{equation}
\subsection{Magnetic}
Consider an infinitesimal measure of magnetic energy $ME$
\begin{equation}
	d(ME) = \frac{1}{2} \frac{\mathbf{B}^2}{\mu} dV
\end{equation}
Where $V$ is the volume and $\mu$ is magnetic permeability.
\begin{equation}
	d(ME) = \frac{1}{2} \frac{\mathbf{B}^2}{\mu} dV
\end{equation}
The total magnetic energy is
\begin{equation}
	ME = \frac{1}{2} \int_V \frac{\mathbf{B}^2}{\mu} dV
\end{equation}
For uniform $\mu$, we have
\begin{equation}
	ME = \frac{1}{2} \mu^{-1} \int_V \mathbf{B}^2 dV
\end{equation}
This is the dimensional form of the magnetic energy. Using the following non-dimensionalization
\begin{equation}
	B^* = \frac{B}{B_c},
	\qquad
	\mu^* = \frac{\mu}{\mu_c},
	\qquad
	V^* = \frac{V}{L^3}
\end{equation}
We have
\begin{equation}
	\boxed{
	ME = \mu_c^{-1} B_c^2 L^3 \left( \frac{1}{2} {\mu^*}^{-1} \int_V {\mathbf{B}^*}^2 dV^* \right)
	}
\end{equation}

\end{document}