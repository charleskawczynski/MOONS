\documentclass[11pt]{article}
\usepackage{graphicx}    % needed for including graphics e.g. EPS, PS
\usepackage{epstopdf}
\usepackage{amsmath}
\usepackage{hyperref}
\usepackage{xspace}
\usepackage{mathtools}
\usepackage{tikz}
\usepackage{epsfig}
\usepackage{float}
%\usepackage{natbib}
\usepackage{subfigure}
\usepackage{setspace}
\usepackage{tabularx,ragged2e,booktabs,caption}


\setlength{\oddsidemargin}{0.1in}
\setlength{\textwidth}{7.25in}

\setlength{\topmargin}{-1in}     %\topmargin: gap above header
\setlength{\headheight}{0in}     %\headheight: height of header
\setlength{\topskip}{0in}        %\topskip: between header and text
\setlength{\headsep}{0in}        
\setlength{\textheight}{692pt}   %\textheight: height of main text
\setlength{\textwidth}{7.5in}    % \textwidth: width of text
\setlength{\oddsidemargin}{-0.5in}  % \oddsidemargin: odd page left margin
\setlength{\evensidemargin}{0in} %\evensidemargin : even page left margin
\setlength{\parindent}{0.25in}   %\parindent: indentation of paragraphs
\setlength{\parskip}{0pt}        %\parskip: gap between paragraphs
\setlength{\voffset}{0.5in}


% Useful commands:

% \hfill		aligns-right everything right of \hfill

\begin{document}
\doublespacing
\title{Magnetohydrodynamic Object-Oriented Numerical Solver (MOONS)}
\author{C. Kawczynski \\
Department of Mechanical and Aerospace Engineering \\
University of California Los Angeles, USA\\
}
\maketitle

\section{Derivation of the Kinetic Energy Equation}
The dimensionless momentum equation is

\begin{equation}
	\frac{\partial u_i}{\partial t} + 
	\frac{\partial (u_i u_j)}{\partial x_j}
	= 
	- \frac{\partial p}{\partial x_i}
	+ \frac{1}{Re}
	\frac{\partial^2 u_i}{\partial x_j^2}
	+ \frac{Ha^2}{Re}
	(j \times B)_i
\end{equation}

Or
\begin{equation}
	\frac{\partial u_i}{\partial t} + 
	u_j\frac{\partial u_i}{\partial x_j}
	= 
	- \frac{\partial p}{\partial x_i}
	+ \frac{1}{Re}
	\frac{\partial^2 u_i}{\partial x_j^2}
	+ \frac{Ha^2}{Re}
	(j \times B)_i
\end{equation}


We must dot this with the velocity field $u_i$, which yields

\begin{equation}
	\boxed{
	u_i
	\frac{\partial u_i}{\partial t} + 
	u_i
	u_j\frac{\partial u_i}{\partial x_j}
	= 
	- u_i
	\frac{\partial p}{\partial x_i}
	+ \frac{1}{Re}
	u_i
	\frac{\partial^2 u_i}{\partial x_j^2}
	+ \frac{Ha^2}{Re}
	u_i
	(j \times B)_i
	}
\end{equation}

This equation works perfectly fine to compute each of the terms of the kinetic energy. In Professor Kim's class on turbulence, we went a step further.

Let $K = \frac{1}{2} u_i u_i$

\subsection{Unsteady term}
\begin{equation}
	\frac{\partial u_i u_i}{\partial t}
	=
	2 u_i \frac{\partial u_i}{\partial t}
\end{equation}

Therefore

\begin{equation}
	u_i \frac{\partial u_i}{\partial t}
	=
	\frac{1}{2} \frac{\partial u_i u_i}{\partial t}
	=
	\frac{\partial K}{\partial t}
\end{equation}

\subsection{Advection term}
\begin{equation}
	\frac{\partial u_i u_i}{\partial x_j}
	=
	2 u_i \frac{\partial u_i}{\partial x_j}
\end{equation}

Therefore
\begin{equation}
	u_j u_i \frac{\partial u_i}{\partial x_j}
	=
	\frac{1}{2} u_j \frac{\partial u_i u_i}{\partial x_j}
	=
	u_j \frac{\partial K}{\partial x_j}
\end{equation}


\subsection{Diffusion Term}

\begin{equation}
	\frac{\partial^2 K}{\partial x_j^2}
	=
	\frac{\partial}{\partial x_j}
	\frac{\partial K}{\partial x_j}
	=
	\frac{\partial}{\partial x_j}
	u_i
	\frac{\partial u_i}{\partial x_j}
	=
	\left( \frac{\partial u_i}{\partial x_j} \right)^2
	+
	u_i
	\frac{\partial^2 u_i}{\partial x_j^2}
\end{equation}

Therefore

\begin{equation}
	\frac{1}{Re}
	\left\{
	u_i
	\frac{\partial^2 u_i}{\partial x_j^2}
	\right\}
	=
	\frac{1}{Re}
	\left\{
	\frac{\partial^2 K}{\partial x_j^2}
	-
	\left( \frac{\partial u_i}{\partial x_j} \right)^2
	\right\}
\end{equation}


\subsection{Pressure term}
\begin{equation}
	\frac{\partial p u_i}{\partial x_i}
	=
	u_i \frac{\partial p}{\partial x_i}
	+
	p \frac{\partial u_i}{\partial x_i}
	=
	u_i \frac{\partial p}{\partial x_i}
\end{equation}

Therefore

\begin{equation}
	u_i
	\frac{\partial p}{\partial x_i}
	=
	\frac{\partial (u_i p)}{\partial x_i}
\end{equation}

\subsection{Lorentz Force term}

\subsection{Putting this all together}
We have

\begin{equation}
	\frac{\partial K}{\partial t}
	+
	u_j \frac{\partial K}{\partial x_j}
	= 
	-
	\frac{\partial (u_i p)}{\partial x_i}
	+
	\frac{1}{Re}
	\left\{
	\frac{\partial^2 K}{\partial x_j^2}
	-
	\left( \frac{\partial u_i}{\partial x_j} \right)^2
	\right\}
	+
	\frac{Ha^2}{Re}
	u_i
	(j \times B)_i
\end{equation}

or

\begin{equation}
	\boxed{
	\frac{\partial K}{\partial t}
	+
	\frac{\partial (u_j K)}{\partial x_j}
	= 
	-
	\frac{\partial (u_i p)}{\partial x_i}
	+
	\frac{1}{Re}
	\left\{
	\frac{\partial^2 K}{\partial x_j^2}
	-
	\left( \frac{\partial u_i}{\partial x_j} \right)^2
	\right\}
	+
	\frac{Ha^2}{Re}
	u_i
	(j \times B)_i
	}
\end{equation}

Note that sometimes the pressure term is written as $\frac{\partial (u_i p \delta_{ij})}{\partial x_j}$ and is grouped with the viscous terms.

\section{Derivation of the Kinetic Energy Equation - using Prof Kim's notes}
The dimensionless momentum equation is

% \begin{equation}
% 	\frac{\partial u_i}{\partial t} + 
% 	\frac{\partial (u_i u_j)}{\partial x_j}
% 	= 
% 	- \frac{\partial p}{\partial x_i}
% 	+ \frac{1}{Re}
% 	\frac{\partial^2 u_i}{\partial x_j^2}
% 	+ \frac{1}{Fr^2}
% 	g_i
% 	+ \frac{Gr}{Re^2}
% 	T g_i
% 	+ \frac{Ha^2}{Re}
% 	(j \times B)_i
% \end{equation}

\begin{equation}
	\frac{\partial u_i}{\partial t} + 
	\frac{\partial (u_i u_j)}{\partial x_j}
	= 
	- \frac{\partial p}{\partial x_i}
	+ \frac{1}{Re}
	\frac{\partial^2 u_i}{\partial x_j^2}
	+ \frac{Ha^2}{Re}
	(j \times B)_i
\end{equation}

Or
\begin{equation}
	\frac{\partial u_i}{\partial t} + 
	u_j\frac{\partial u_i}{\partial x_j}
	= 
	- \frac{\partial p}{\partial x_i}
	+ \frac{1}{Re}
	\frac{\partial^2 u_i}{\partial x_j^2}
	+ \frac{Ha^2}{Re}
	(j \times B)_i
\end{equation}


We must dot this with the velocity field $u_i$, which yields

\begin{equation}
	\boxed{
	u_i
	\frac{\partial u_i}{\partial t} + 
	u_i
	u_j\frac{\partial u_i}{\partial x_j}
	= 
	- u_i
	\frac{\partial p}{\partial x_i}
	+ \frac{1}{Re}
	u_i
	\frac{\partial^2 u_i}{\partial x_j^2}
	+ \frac{Ha^2}{Re}
	u_i
	(j \times B)_i
	}
\end{equation}

This equation works perfectly fine to compute each of the terms of the kinetic energy. In Professor Kim's class on turbulence, we went a step further.

Let $K = \frac{1}{2} u_i u_i$

\subsection{Unsteady term}
\begin{equation}
	\frac{\partial u_i u_i}{\partial t}
	=
	2 u_i \frac{\partial u_i}{\partial t}
\end{equation}

Therefore

\begin{equation}
	u_i \frac{\partial u_i}{\partial t}
	=
	\frac{1}{2} \frac{\partial u_i u_i}{\partial t}
	=
	\frac{\partial K}{\partial t}
\end{equation}

\subsection{Advection term}
\begin{equation}
	\frac{\partial u_i u_i}{\partial x_j}
	=
	2 u_i \frac{\partial u_i}{\partial x_j}
\end{equation}

Therefore
\begin{equation}
	u_j u_i \frac{\partial u_i}{\partial x_j}
	=
	\frac{1}{2} u_j \frac{\partial u_i u_i}{\partial x_j}
	=
	u_j \frac{\partial K}{\partial x_j}
\end{equation}


\subsection{Diffusion and Pressure Terms}
Let
\begin{equation}
	S_{ij} 
	=
	\frac{1}{2}
	\left(
	\frac{\partial u_i}{\partial x_j}
	+
	\frac{\partial u_j}{\partial x_i}
	\right)
	\qquad \qquad
	\Omega_{ij} 
	=
	\frac{1}{2}
	\left(
	\frac{\partial u_i}{\partial x_j}
	-
	\frac{\partial u_j}{\partial x_i}
	\right)
\end{equation}
And
\begin{equation}
	T_{ij}
	=
	-p \delta_{ij}
	+
	2 Re^{-1}
	S_{ij}
\end{equation}

Note that the total stress tensor, $T_{ij}$, and the mean strain-rate tensor, $S_{ij}$ are symmetric

\begin{equation}
	\frac{\partial u_i}{\partial x_j}
	= S_{ij} + \Omega_{ij}
\end{equation}

We may then write the diffusion and pressure terms as

\begin{equation}
	-\frac{\partial p}{\partial x_i}
	+ \frac{1}{Re}
	\frac{\partial^2 u_i}{\partial x_j^2}
	=
	\frac{\partial}{\partial x_j} T_{ij}
\end{equation}

Now, multiplying this by $u_i$, we have

\begin{equation}
	u_i \frac{\partial}{\partial x_j} T_{ij}
	=
	\frac{\partial}{\partial x_j} (T_{ij} u_i)
	-
	T_{ij} \frac{\partial u_i}{\partial x_j}
	=
	\frac{\partial}{\partial x_j} (T_{ij} u_i)
	- T_{ij} S_{ij}
\end{equation}

The last equality is due to the fact that $T_{ij} \Omega_{ij} = 0$.

Also note that

\begin{equation}
	T_{ij} S_{ij}
	=
	-p \delta_{ij} S_{ij}
	+ 2 Re^{-1} S_{ij} S_{ij}
	=
	-p S_{ii}
	+ 2 Re^{-1} S_{ij} S_{ij}
	=
	2 Re^{-1} S_{ij} S_{ij}
\end{equation}

Finally, we have

\begin{equation}
	-u_i \frac{\partial p}{\partial x_i}
	+ \frac{1}{Re}u_i
	\frac{\partial^2 u_i}{\partial x_j^2}
	=
	u_i \frac{\partial}{\partial x_j} T_{ij}
	=
	\frac{\partial}{\partial x_j} (T_{ij} u_i)
	-
	T_{ij} \frac{\partial u_i}{\partial x_j}
	=
	\frac{\partial}{\partial x_j} (T_{ij} u_i)
	- T_{ij} S_{ij}
	=
	\frac{\partial}{\partial x_j} (T_{ij} u_i)
	-
	2 Re^{-1} S_{ij} S_{ij}
\end{equation}


\subsection{Lorentz Force term}

\subsection{Putting this all together}
We have

\begin{equation}
	\frac{\partial K}{\partial t}
	+
	u_j \frac{\partial K}{\partial x_j}
	= 
	\frac{\partial}{\partial x_j} (T_{ij} u_i)
	-
	2 Re^{-1} S_{ij} S_{ij}
	+
	\frac{Ha^2}{Re}
	u_i
	(j \times B)_i
\end{equation}
\begin{equation}
	\boxed{
	\frac{\partial K}{\partial t}
	+
	u_j \frac{\partial K}{\partial x_j}
	= 
	\frac{\partial}{\partial x_j} 
	(
	-p \delta_{ij}
	u_i
	+
	2 Re^{-1}
	S_{ij}
	u_i
	)
	-
	2 Re^{-1} S_{ij} S_{ij}
	+
	\frac{Ha^2}{Re}
	u_i
	(j \times B)_i
	}
\end{equation}

or

\begin{equation}
	\boxed{
	\frac{\partial K}{\partial t}
	+
	\frac{\partial u_j K}{\partial x_j}
	= 
	\frac{\partial}{\partial x_j} 
	(
	-p \delta_{ij}
	u_i
	+
	2 Re^{-1}
	S_{ij}
	u_i
	)
	-
	2 Re^{-1} S_{ij} S_{ij}
	+
	\frac{Ha^2}{Re}
	u_i
	(j \times B)_i
	}
\end{equation}


\end{document}