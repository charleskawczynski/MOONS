\documentclass[11pt]{article}
\usepackage{graphicx}    % needed for including graphics e.g. EPS, PS
\usepackage{epstopdf}
\usepackage{amsmath}
\usepackage{hyperref}
\usepackage{xspace}
\usepackage{mathtools}
\usepackage{tikz}
\usepackage{epsfig}
\usepackage{float}
%\usepackage{natbib}
\usepackage{subfigure}
\usepackage{setspace}
\usepackage{tabularx,ragged2e,booktabs,caption}
\usepackage{esint}


\newcommand{\A}{\mathbf{A}}
\newcommand{\B}{\mathbf{B}}
\newcommand{\C}{\mathbf{C}}
\newcommand{\PD}{\partial}
\newcommand{\BM}{\frac{\mathbf{B}}{\mu}}
\newcommand{\J}{\mathbf{j}}
\newcommand{\E}{\mathbf{E}}
\newcommand{\N}{\mathbf{n}}
\newcommand{\JS}{\frac{\mathbf{j}}{\sigma}}
\newcommand{\JSS}{\frac{\mathbf{j}^2}{\sigma}}
\newcommand{\U}{\mathbf{u}}
\newcommand{\SII}{\sigma^{-1}}
\newcommand{\SI}{\sigma}
\newcommand{\M}{\mu}
\newcommand{\MO}{\overline{\mu}}
\newcommand{\SO}{\overline{\sigma}}
\newcommand{\DOT}{\text{\textbullet}}

\setlength{\oddsidemargin}{0.1in}
\setlength{\textwidth}{7.25in}

\setlength{\topmargin}{-1in}     %\topmargin: gap above header
\setlength{\headheight}{0in}     %\headheight: height of header
\setlength{\topskip}{0in}        %\topskip: between header and text
\setlength{\headsep}{0in}        
\setlength{\textheight}{692pt}   %\textheight: height of main text
\setlength{\textwidth}{7.5in}    % \textwidth: width of text
\setlength{\oddsidemargin}{-0.5in}  % \oddsidemargin: odd page left margin
\setlength{\evensidemargin}{0in} %\evensidemargin : even page left margin
\setlength{\parindent}{0.25in}   %\parindent: indentation of paragraphs
\setlength{\parskip}{0pt}        %\parskip: gap between paragraphs
\setlength{\voffset}{0.5in}


% Useful commands:

% \hfill		aligns-right everything right of \hfill

\begin{document}
\doublespacing
\title{Magnetohydrodynamic Object-Oriented Numerical Solver (MOONS)}
\author{C. Kawczynski \\
Department of Mechanical and Aerospace Engineering \\
University of California Los Angeles, USA\\
}
\maketitle

\section{Derivation of the Kinetic Energy Equation}
The dimensional momentum equation is

\begin{equation}
	\rho \left( 
	\frac{\PD u_i}{\PD t} + 
	u_j\frac{\PD u_i}{\PD x_j}
	\right)
	= 
	- \frac{\PD p}{\PD x_i}
	+ \frac{\PD}{\PD x_j} \left( \mu \frac{\PD u_i}{\PD x_j} \right)
	+ (j \times B)_i
\end{equation}

Dotting this with the velocity yields

\begin{equation}
	\boxed{
	u_i
	\left(
	\rho \left( 
	\frac{\PD u_i}{\PD t} + 
	u_j\frac{\PD u_i}{\PD x_j}
	\right)
	= 
	- \frac{\PD p}{\PD x_i}
	+ \frac{\PD}{\PD x_j} \left( \mu \frac{\PD u_i}{\PD x_j} \right)
	+ (j \times B)_i
	\right)
	}
\end{equation}

Let's analyze each term, but first, let's define
\begin{equation}
	E_K = \frac{1}{2} \rho u_i u_i
\end{equation}

\subsection{Unsteady term}
\begin{equation}
	\frac{\PD u_i u_i}{\PD t} =
	2 u_i \frac{\PD u_i}{\PD t}
\end{equation}
Therefore
\begin{equation}
	\boxed{
	u_i \rho \frac{\PD u_i}{\PD t} =
	\frac{\PD}{\PD t} \left( \frac{1}{2} \rho u_i u_i \right) =
	\frac{\PD E_K}{\PD t}
	}
\end{equation}

\subsection{Convection term}
\begin{equation}
	u_j \rho \frac{\PD u_i u_i}{\PD x_j} =
	2 \rho u_j u_i \frac{\PD u_i}{\PD x_j}
\end{equation}
Therefore
\begin{equation}
	\boxed{
	\rho u_i u_j \frac{\PD u_i}{\PD x_j} =
	\frac{1}{2} \rho u_j \frac{\PD u_i u_i}{\PD x_j}
	= u_j \frac{\PD E_K}{\PD x_j}
	}
\end{equation}

\subsection{Diffusion Term}
First, let
\begin{equation} \label{eq:strainRate}
	S_{ij} = \frac{1}{2}
	\left(
	\frac{\PD u_i}{\PD x_j} + 
	\frac{\PD u_j}{\PD x_i}
	\right)
\end{equation}
And note that
\begin{equation} \label{eq:strainIdentity}
	2 S_{ij} S_{ij} =
	\frac{1}{2} \left( \frac{\PD u_i}{\PD x_j} + \frac{\PD u_j}{\PD x_i} \right)
	\left( \frac{\PD u_i}{\PD x_j} + \frac{\PD u_j}{\PD x_i} \right)
	= \frac{1}{2} \left( \frac{\PD u_i}{\PD x_j} \frac{\PD u_i}{\PD x_j}
	+ 2 \frac{\PD u_i}{\PD x_j} \frac{\PD u_j}{\PD x_i} 
	+ \frac{\PD u_j}{\PD x_i} \frac{\PD u_j}{\PD x_i} \right)
	= \frac{\PD u_i}{\PD x_j} \frac{\PD u_i}{\PD x_j}
	+ \frac{\PD u_i}{\PD x_j} \frac{\PD u_j}{\PD x_i}
\end{equation}
since no free indexes exist and indexes may be swapped. In addition, we may write
\begin{equation}
	\frac{\PD}{\PD x_j} u_i \frac{\PD u_i}{\PD x_j} =
	\frac{\PD u_i}{\PD x_j}\frac{\PD u_i}{\PD x_j} + 
	u_i \frac{\PD^2 u_i}{\PD x_j^2}
\end{equation}
Therefore
\begin{equation}
	u_i \frac{\PD^2 u_i}{\PD x_j^2} = 
	\frac{\PD}{\PD x_j} \left[ u_i \frac{\PD u_i}{\PD x_j} \right] - 
	\frac{\PD u_i}{\PD x_j}\frac{\PD u_i}{\PD x_j}
\end{equation}
Adding and subtracting $\frac{\PD}{\PD x_j} u_i \frac{\PD u_j}{\PD x_i}$ on the RHS, we have
\begin{equation}
	u_i \frac{\PD^2 u_i}{\PD x_j^2} = 
	\frac{\PD}{\PD x_j} \left[ u_i \frac{\PD u_i}{\PD x_j} + u_i \frac{\PD u_j}{\PD x_i} \right] - 
	\frac{\PD}{\PD x_j} \left[ u_i \frac{\PD u_j}{\PD x_i} \right] - 
	\frac{\PD u_i}{\PD x_j}\frac{\PD u_i}{\PD x_j}
\end{equation}
Using \ref{eq:strainRate}, and expanding $\frac{\PD}{\PD x_j} u_i \frac{\PD u_j}{\PD x_i}$, we have
\begin{equation}
	u_i \frac{\PD^2 u_i}{\PD x_j^2} = 
	\frac{\PD}{\PD x_j} \left[ u_i 2 S_{ij} \right] - 
	\left( 
	\underbrace{u_i \frac{\PD^2 u_j}{\PD x_j x_i}}_{=0 \text{ (swap ij, $\PD_i u_i =0$)}}
	+
	\frac{\PD u_i}{\PD x_j} \frac{\PD u_j}{\PD x_i}
	\right) - 
	\frac{\PD u_i}{\PD x_j}\frac{\PD u_i}{\PD x_j}
\end{equation}
Using \ref{eq:strainIdentity}, we have
\begin{equation}
	u_i \frac{\PD^2 u_i}{\PD x_j^2} = 
	\frac{\PD}{\PD x_j} \left[ 2 u_i S_{ij} \right] - 
	2 S_{ij} S_{ij}
\end{equation}
Swapping the index of the first term on the RHS, and using $S_{ij}=S_{ji}$, we have
\begin{equation}
	u_i \frac{\PD^2 u_i}{\PD x_j^2} = 
	\frac{\PD}{\PD x_i} \left[ 2 u_j S_{ij} \right] - 
	2 S_{ij} S_{ij}
\end{equation}
Finally, we have
\begin{equation}
	\boxed{
	u_i \mu \frac{\PD^2 u_i}{\PD x_j^2} = 
	\frac{\PD}{\PD x_i} \left[ 2 \mu u_j S_{ij} \right] - 
	2 \mu S_{ij} S_{ij}
	}
\end{equation}


\subsection{Pressure term}
\begin{equation}
	\frac{\PD p u_i}{\PD x_i}
	=
	u_i \frac{\PD p}{\PD x_i}
	+
	p \frac{\PD u_i}{\PD x_i}
	=
	u_i \frac{\PD p}{\PD x_i}
\end{equation}

Therefore

\begin{equation}
	\boxed{
	u_i \frac{\PD p}{\PD x_i} =
	\frac{\PD (u_i p)}{\PD x_i}
	}
\end{equation}

\subsection{Lorentz Force term}
We will simply leave this as
\begin{equation}
	\boxed{
	u_i (j \times B)_i = u_i (j \times B)_i
	}
\end{equation}

\section{Kinetic Energy Equation}
Putting it all together, we have

\begin{equation}
	\frac{\PD E_K}{\PD t} + 
	u_j \frac{\PD E_K}{\PD x_j} = 
	- \frac{\PD (u_i p)}{\PD x_i} + 
	\frac{\PD}{\PD x_i} \left[ 2 \mu u_j S_{ij} \right] - 
	2 \mu S_{ij} S_{ij}
\end{equation}
Combining the compressive viscous stress with pressure, we have
\begin{equation}
	\frac{\PD E_K}{\PD t} + 
	\underbrace{u_j \frac{\PD E_K}{\PD x_j}}_{\U \DOT \nabla E_K} +
	\underbrace{\frac{\PD}{\PD x_i} \left( u_i p - 2 \mu u_j S_{ij} \right)}_{\nabla \DOT \mathbf{T}} = 
	- \underbrace{2 \mu S_{ij} S_{ij}}_{\varepsilon}
\end{equation}
Note that there is no production term, $P$, since turbulent fluctuations were not accounted for.

\section{Moreau}
Writing the stress terms as
\begin{equation}
	\frac{\PD \sigma_{ji}}{\PD x_j} = 
	\frac{\PD}{\PD x_j} (\sigma_{ji}u_i) - 
	\sigma_{ji} \frac{\PD u_i}{\PD x_j}
\end{equation}
Applying divergence theorem, the first term becomes
\begin{equation}
	\mathcal P_S = \oiint \mathbf{T} \DOT \U dA = 
	\text{power of  the external surface forces}
\end{equation}
Where $\mathbf{T}$ is the stress appliedfrom the exterior to the interior of the domain. The interpretation of the integral isthe power of the external surface forces. This includes power of frictional, pressure, and gravitational forces.

The second term may be written as
\begin{equation}
	\sigma_{ji} \frac{\PD u_i}{\PD x_j} = 
	\left( -p \delta_{ij} + 2\mu e_{ij}\right) \left[ e_{ij} + \tilde{e}_{ij} \right]
\end{equation}
Where
\begin{equation}
	e_{ij} = \frac{1}{2} 
	\left( 
	\frac{\PD u_i}{\PD x_j} + \frac{\PD u_j}{\PD x_i}
	\right), 
	\qquad
	\tilde{e}_{ij} = \frac{1}{2}
	\left( 
	\frac{\PD u_i}{\PD x_j} - \frac{\PD u_j}{\PD x_i}
	\right)
\end{equation}
Note the difference comparedto Moreau,compressibility have been removed here. The corresponding integral over the domain yields

\begin{equation}
	\mathcal P_i = \int 2 \mu e_{ij} e_{ij} dV = 
	\text{power of the internal forces}
\end{equation}
Note that compression power,
\begin{equation}
	\mathcal P_C = \int p \frac{d}{dt} \left( \frac{1}{\rho} \right) dm = 
	\text{compression power},
\end{equation}
Has been neglected, and we finally have
\begin{equation}
	\frac{d E_K}{dt} = 
	\mathcal P_S + 
	\underbrace{\mathcal P_C}_{=0} - 
	\mathcal D_v +
	\mathcal P_{em}
\end{equation}
Where 
\begin{equation}
	\mathcal D_v = \int 2 \mu e_{ij} e_{ij} dV = 
	\text{power dissipated into heat by viscosity}
\end{equation}

\subsection{Viscous dissipation term}
\begin{equation}
	2 \mu e_{ij} e_{ij} = \frac{1}{2} \mu 
	\left( 
	\frac{\PD u_i}{\PD x_j} + \frac{\PD u_j}{\PD x_i}
	\right)
	\left( 
	\frac{\PD u_i}{\PD x_j} + \frac{\PD u_j}{\PD x_i}
	\right)
\end{equation}
\begin{equation}
	2 \mu e_{ij} e_{ij} = \frac{1}{2} \mu 
	\left[
	\left( \frac{\PD u_i}{\PD x_j} \right)^2 + 
	2 \frac{\PD u_j}{\PD x_i} \frac{\PD u_i}{\PD x_j} +
	\left( \frac{\PD u_j}{\PD x_i} \right)^2
	\right]
	??????????
\end{equation}


\subsection{Putting this all together}
We have

\begin{equation}
	\frac{\PD E_K}{\PD t}
	+
	u_j \frac{\PD E_K}{\PD x_j}
	= 
	-
	\frac{\PD (u_i p)}{\PD x_i}
	+
	\frac{1}{Re}
	\left\{
	\frac{\PD^2 E_K}{\PD x_j^2}
	-
	\left( \frac{\PD u_i}{\PD x_j} \right)^2
	\right\}
	+
	\frac{Ha^2}{Re}
	u_i
	(j \times B)_i
\end{equation}

or

\begin{equation}
	\boxed{
	\frac{\PD E_K}{\PD t}
	+
	\frac{\PD (u_j E_K)}{\PD x_j}
	= 
	-
	\frac{\PD (u_i p)}{\PD x_i}
	+
	\frac{1}{Re}
	\left\{
	\frac{\PD^2 E_K}{\PD x_j^2}
	-
	\left( \frac{\PD u_i}{\PD x_j} \right)^2
	\right\}
	+
	\frac{Ha^2}{Re}
	u_i
	(j \times B)_i
	}
\end{equation}

Note that sometimes the pressure term is written as $\frac{\PD (u_i p \delta_{ij})}{\PD x_j}$ and is grouped with the viscous terms.

\section{Derivation of the Kinetic Energy Equation - using Prof Kim's notes}
The dimensionless momentum equation is

% \begin{equation}
% 	\frac{\PD u_i}{\PD t} + 
% 	\frac{\PD (u_i u_j)}{\PD x_j}
% 	= 
% 	- \frac{\PD p}{\PD x_i}
% 	+ \frac{1}{Re}
% 	\frac{\PD^2 u_i}{\PD x_j^2}
% 	+ \frac{1}{Fr^2}
% 	g_i
% 	+ \frac{Gr}{Re^2}
% 	T g_i
% 	+ \frac{Ha^2}{Re}
% 	(j \times B)_i
% \end{equation}

\begin{equation}
	\frac{\PD u_i}{\PD t} + 
	\frac{\PD (u_i u_j)}{\PD x_j}
	= 
	- \frac{\PD p}{\PD x_i}
	+ \frac{1}{Re}
	\frac{\PD^2 u_i}{\PD x_j^2}
	+ \frac{Ha^2}{Re}
	(j \times B)_i
\end{equation}

Or
\begin{equation}
	\frac{\PD u_i}{\PD t} + 
	u_j\frac{\PD u_i}{\PD x_j}
	= 
	- \frac{\PD p}{\PD x_i}
	+ \frac{1}{Re}
	\frac{\PD^2 u_i}{\PD x_j^2}
	+ \frac{Ha^2}{Re}
	(j \times B)_i
\end{equation}


We must dot this with the velocity field $u_i$, which yields

\begin{equation}
	\boxed{
	u_i
	\frac{\PD u_i}{\PD t} + 
	u_i
	u_j\frac{\PD u_i}{\PD x_j}
	= 
	- u_i
	\frac{\PD p}{\PD x_i}
	+ \frac{1}{Re}
	u_i
	\frac{\PD^2 u_i}{\PD x_j^2}
	+ \frac{Ha^2}{Re}
	u_i
	(j \times B)_i
	}
\end{equation}

This equation works perfectly fine to compute each of the terms of the kinetic energy. In Professor Kim's class on turbulence, we went a step further.

Let $E_K = \frac{1}{2} u_i u_i$

\subsection{Unsteady term}
\begin{equation}
	\frac{\PD u_i u_i}{\PD t}
	=
	2 u_i \frac{\PD u_i}{\PD t}
\end{equation}

Therefore

\begin{equation}
	u_i \frac{\PD u_i}{\PD t}
	=
	\frac{1}{2} \frac{\PD u_i u_i}{\PD t}
	=
	\frac{\PD E_K}{\PD t}
\end{equation}

\subsection{Advection term}
\begin{equation}
	\frac{\PD u_i u_i}{\PD x_j}
	=
	2 u_i \frac{\PD u_i}{\PD x_j}
\end{equation}

Therefore
\begin{equation}
	u_j u_i \frac{\PD u_i}{\PD x_j}
	=
	\frac{1}{2} u_j \frac{\PD u_i u_i}{\PD x_j}
	=
	u_j \frac{\PD E_K}{\PD x_j}
\end{equation}


\subsection{Diffusion and Pressure Terms}
Let
\begin{equation}
	S_{ij} 
	=
	\frac{1}{2}
	\left(
	\frac{\PD u_i}{\PD x_j}
	+
	\frac{\PD u_j}{\PD x_i}
	\right)
	\qquad \qquad
	\Omega_{ij} 
	=
	\frac{1}{2}
	\left(
	\frac{\PD u_i}{\PD x_j}
	-
	\frac{\PD u_j}{\PD x_i}
	\right)
\end{equation}
And
\begin{equation}
	T_{ij}
	=
	-p \delta_{ij}
	+
	2 Re^{-1}
	S_{ij}
\end{equation}

Note that the total stress tensor, $T_{ij}$, and the mean strain-rate tensor, $S_{ij}$ are symmetric

\begin{equation}
	\frac{\PD u_i}{\PD x_j}
	= S_{ij} + \Omega_{ij}
\end{equation}

We may then write the diffusion and pressure terms as

\begin{equation}
	-\frac{\PD p}{\PD x_i}
	+ \frac{1}{Re}
	\frac{\PD^2 u_i}{\PD x_j^2}
	=
	\frac{\PD}{\PD x_j} T_{ij}
\end{equation}

Now, multiplying this by $u_i$, we have

\begin{equation}
	u_i \frac{\PD}{\PD x_j} T_{ij}
	=
	\frac{\PD}{\PD x_j} (T_{ij} u_i)
	-
	T_{ij} \frac{\PD u_i}{\PD x_j}
	=
	\frac{\PD}{\PD x_j} (T_{ij} u_i)
	- T_{ij} S_{ij}
\end{equation}

The last equality is due to the fact that $T_{ij} \Omega_{ij} = 0$.

Also note that

\begin{equation}
	T_{ij} S_{ij}
	=
	-p \delta_{ij} S_{ij}
	+ 2 Re^{-1} S_{ij} S_{ij}
	=
	-p S_{ii}
	+ 2 Re^{-1} S_{ij} S_{ij}
	=
	2 Re^{-1} S_{ij} S_{ij}
\end{equation}

Finally, we have

\begin{equation}
	-u_i \frac{\PD p}{\PD x_i}
	+ \frac{1}{Re}u_i
	\frac{\PD^2 u_i}{\PD x_j^2}
	=
	u_i \frac{\PD}{\PD x_j} T_{ij}
	=
	\frac{\PD}{\PD x_j} (T_{ij} u_i)
	-
	T_{ij} \frac{\PD u_i}{\PD x_j}
	=
	\frac{\PD}{\PD x_j} (T_{ij} u_i)
	- T_{ij} S_{ij}
	=
	\frac{\PD}{\PD x_j} (T_{ij} u_i)
	-
	2 Re^{-1} S_{ij} S_{ij}
\end{equation}


\subsection{Lorentz Force term}

\subsection{Putting this all together}
We have

\begin{equation}
	\frac{\PD E_K}{\PD t}
	+
	u_j \frac{\PD E_K}{\PD x_j}
	= 
	\frac{\PD}{\PD x_j} (T_{ij} u_i)
	-
	2 Re^{-1} S_{ij} S_{ij}
	+
	\frac{Ha^2}{Re}
	u_i
	(j \times B)_i
\end{equation}
\begin{equation}
	\boxed{
	\frac{\PD E_K}{\PD t}
	+
	u_j \frac{\PD E_K}{\PD x_j}
	= 
	\frac{\PD}{\PD x_j} 
	(
	-p \delta_{ij}
	u_i
	+
	2 Re^{-1}
	S_{ij}
	u_i
	)
	-
	2 Re^{-1} S_{ij} S_{ij}
	+
	\frac{Ha^2}{Re}
	u_i
	(j \times B)_i
	}
\end{equation}

or

\begin{equation}
	\boxed{
	\frac{\PD E_K}{\PD t}
	+
	\frac{\PD u_j E_K}{\PD x_j}
	= 
	\frac{\PD}{\PD x_j} 
	(
	-p \delta_{ij}
	u_i
	+
	2 Re^{-1}
	S_{ij}
	u_i
	)
	-
	2 Re^{-1} S_{ij} S_{ij}
	+
	\frac{Ha^2}{Re}
	u_i
	(j \times B)_i
	}
\end{equation}


\end{document}