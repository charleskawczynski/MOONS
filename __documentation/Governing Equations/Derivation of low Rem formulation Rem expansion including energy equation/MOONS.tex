\documentclass[11pt]{article}
\usepackage{graphicx}    % needed for including graphics e.g. EPS, PS
\usepackage{epstopdf}
\usepackage{amsmath}
\usepackage{hyperref}
\usepackage{xspace}
\usepackage{mathtools}
\usepackage{tikz}
\usepackage{epsfig}
\usepackage{float}
%\usepackage{natbib}
\usepackage{subfigure}
\usepackage{setspace}
\usepackage{tabularx,ragged2e,booktabs,caption}
\usepackage{esint}


\newcommand{\A}{\mathbf{A}}
\newcommand{\B}{\mathbf{B}}
\newcommand{\MC}{\mathcal}
\renewcommand{\C}{\mathbf{C}}
\newcommand{\PD}{\partial}
\newcommand{\BM}{\frac{\mathbf{B}}{\mu}}
\newcommand{\J}{\mathbf{j}}
\newcommand{\E}{\mathbf{E}}
\newcommand{\ep}{\epsilon}
\newcommand{\N}{\mathbf{n}}
\newcommand{\JS}{\frac{\mathbf{j}}{\sigma}}
\newcommand{\JSS}{\frac{\mathbf{j}^2}{\sigma}}
\renewcommand{\U}{\mathbf{u}}
\newcommand{\SII}{\sigma^{-1}}
\newcommand{\SI}{\sigma}
\newcommand{\M}{\mu}
\newcommand{\MO}{\overline{\mu}}
\newcommand{\SO}{\overline{\sigma}}
\newcommand{\DOT}{\text{\textbullet}}

\setlength{\oddsidemargin}{0.1in}
\setlength{\textwidth}{7.25in}

\setlength{\topmargin}{-1in}     %\topmargin: gap above header
\setlength{\headheight}{0in}     %\headheight: height of header
\setlength{\topskip}{0in}        %\topskip: between header and text
\setlength{\headsep}{0in}
\setlength{\textheight}{692pt}   %\textheight: height of main text
\setlength{\textwidth}{7.5in}    % \textwidth: width of text
\setlength{\oddsidemargin}{-0.5in}  % \oddsidemargin: odd page left margin
\setlength{\evensidemargin}{0in} %\evensidemargin : even page left margin
\setlength{\parindent}{0.25in}   %\parindent: indentation of paragraphs
\setlength{\parskip}{0pt}        %\parskip: gap between paragraphs
\setlength{\voffset}{0.5in}


% Useful commands:

% \hfill		aligns-right everything right of \hfill

\begin{document}
\doublespacing
\title{Magnetohydrodynamic Object-Oriented Numerical Solver (MOONS)}
\author{C. Kawczynski \\
Department of Mechanical and Aerospace Engineering \\
University of California Los Angeles, USA\\
}
\maketitle

\section{Low Rem formulation}
This document demonstrates what happens to the momentum, induction and magnetic energy equations in the low $Re_m$ limit, and we try to demonstrate that the pseudo-vacuum BCs are exactly equivalent to real-vacuum BCs at this limit.

\section{Induction, momentum and magnetic energy equations}
The induction, Lorentz force and magnetic energy equations, when decomposed into induced and applied magnetic fields, are as follows.

Using dimensional scales $x_k^* = x_k/L_c, \nabla^* = L_c \nabla,t^* = t/(L_c/U_c),u_k^* = u_k/U_c,\sigma^* = \sigma/\sigma_c,\mu^* = \mu/\mu_c$, the induction equation is written with $\B$ expanded in terms of $Re_m$
\begin{equation}
	\B = \B^0 + \B^{ind} Re_m + \text{H.O.T.}
\end{equation}
\begin{equation}
  \frac{\PD (\B^0 + Re_m \B^{ind})}{\PD t} +
  \nabla \times \left( \SII \nabla \times \frac{\B^{ind}}{\mu} \right)
  = \nabla \times \U \times \left( \B^0 + Re_m \B^{ind} \right)
\end{equation}
Note that the strength of the induced magnetic field is $\B^{ind} Re_m$, and not simply $\B^{ind}$, which has no physical meaning but its curl defines the electrical currents. Using the same scaling, the Lorentz force in the momentum equation is
\begin{equation}
	 \frac{Ha^2}{Re} \J \times \B =
	 \frac{Ha^2}{Re} \left( \nabla \times \frac{\B^{ind}}{\mu} \right) \times {\left( \B^0 + Re_m \B^{ind} \right)}
\end{equation}
\section{Energy equation}
\subsection{non-dimensionalization}
The dimensional magnetic energy equation is
\begin{equation}
	\frac{\PD}{\PD t} \frac{\B^2}{2\mu}
	=
	- \nabla \DOT \left(\E \times \BM \right)
	- \U \DOT \left[\B \DOT \nabla \left( \BM \right) \right]
	+ (\U \DOT \nabla) \frac{\B^2}{2\mu}
	- \JSS
\end{equation}
Writing $\E$ in terms of $\B,\J$ and $\U$, we have
\begin{equation}
	\frac{\PD}{\PD t} \frac{\B^2}{2\mu}
	=
	- \nabla \DOT \left( \left(\JS - \U \times \B \right) \times \BM \right)
	- \U \DOT \left[\B \DOT \nabla \left( \BM \right) \right]
	+ (\U \DOT \nabla) \frac{\B^2}{2\mu}
	- \JSS
\end{equation}
Expanding $\B$ in terms of $Re_m$, and noting that $\nabla \times \frac{\B}{\mu} = Re_m \nabla \times \frac{\B^{ind}}{\mu}$ since $\nabla \times \frac{\B^0}{\mu} = 0$, we have
\begin{multline}
	\underbrace{\frac{\PD}{\PD t} \frac{(\B^0 + Re_m \B^{ind})^2}{2\mu}}_{1}
	=
	- \nabla \DOT \left( \underbrace{ \left(\JS - \U \times (\B^0 + Re_m \B^{ind}) \right) \times \frac{(\B^0 + Re_m \B^{ind})}{\mu} }_{2} \right) \\
	- \underbrace{\U \DOT \left[(\B^0 + Re_m \B^{ind}) \DOT \nabla \left( \frac{(\B^0 + Re_m \B^{ind})}{\mu} \right) \right]}_{3}
	+ \underbrace{(\U \DOT \nabla) \frac{(\B^0 + Re_m \B^{ind})^2}{2\mu}}_{4}
	- \underbrace{\JSS}_{5}
\end{multline}
\subsubsection{1}
\begin{align}
	\frac{\PD}{\PD t} \frac{(\B^0 + Re_m \B^{ind})^2}{2\mu} \notag
	& = \frac{\PD}{\PD t} \frac{({\B^0}^2 + 2 Re_m \B^0 \B^{ind} + Re_m^2 {\B^{ind}}^2)}{2\mu} \notag \\
	& = \frac{\PD}{\PD t} \frac{{\B^0}^2}{2 \mu} + Re_m \frac{\PD}{\PD t} \frac{(\B^0 \DOT \B^{ind})}{\mu} + Re_m^2 \frac{\PD}{\PD t} \frac{{\B^{ind}}^2}{2 \mu} \notag \\
\end{align}

\subsubsection{2}
\begin{align}
	2 \notag
	& = \left(\JS - \U \times (\B^0 + Re_m \B^{ind}) \right) \times \frac{(\B^0 + Re_m \B^{ind})}{\mu} \notag \\
	& = \JS \times \frac{(\B^0 + Re_m \B^{ind})}{\mu}
	- \left[ \U \times (\B^0 + Re_m \B^{ind}) \right] \times \frac{(\B^0 + Re_m \B^{ind})}{\mu}	\notag \\
	& = \JS \times \frac{\B^0}{\mu} + Re_m \JS \times \frac{\B^{ind}}{\mu}
	- \left[ \U \times (\B^0 + Re_m \B^{ind}) \right] \times \frac{(\B^0 + Re_m \B^{ind})}{\mu}	\notag \\
	& = \JS \times \frac{\B^0}{\mu} + Re_m \JS \times \frac{\B^{ind}}{\mu}
	- \left[ \U \times \B^0 + Re_m \U \times \B^{ind} \right] \times \frac{(\B^0 + Re_m \B^{ind})}{\mu}	\notag \\
	& = \JS \times \frac{\B^0}{\mu} + Re_m \JS \times \frac{\B^{ind}}{\mu}
	- (\U \times \B^0) \times \frac{(\B^0 + Re_m \B^{ind})}{\mu}
	- Re_m (\U \times \B^{ind}) \times \frac{(\B^0 + Re_m \B^{ind})}{\mu} \notag \\
\end{align}
\begin{multline}
	2 = \JS \times \frac{\B^0}{\mu} + Re_m \JS \times \frac{\B^{ind}}{\mu}
	- (\U \times \B^0) \times \frac{\B^0}{\mu}
	- Re_m (\U \times \B^0) \times \frac{\B^{ind}}{\mu} \\
	- Re_m (\U \times \B^{ind}) \times \frac{\B^0}{\mu}
	- Re_m^2 (\U \times \B^{ind}) \times \frac{\B^{ind}}{\mu} \notag \\
\end{multline}
Finally, we have
\begin{multline}
	\E \times \frac{\B}{\mu} =
	\left( \JS \times \frac{\B^0}{\mu}
	- (\U \times \B^0) \times \frac{\B^0}{\mu}
	\right) \\
	+ Re_m
	\left(
	\JS \times \frac{\B^{ind}}{\mu}
	- (\U \times \B^0) \times \frac{\B^{ind}}{\mu}
	- (\U \times \B^{ind}) \times \frac{\B^0}{\mu}
	\right)
	+Re_m^2 \left(
	- (\U \times \B^{ind}) \times \frac{\B^{ind}}{\mu}
	\right)
\end{multline}

\end{document}