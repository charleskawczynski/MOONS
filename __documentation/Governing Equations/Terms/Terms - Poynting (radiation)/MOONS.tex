\documentclass[11pt]{article}
\newcommand{\PSCHAIN}{..}
\edef\PSCHAIN{\PSCHAIN/LATEX_INCLUDES}

\newcommand{\rootdir}{\PSCHAIN}
\newcommand{\bibdir}{\rootdir/includes/bib}

\usepackage{graphicx}
\usepackage{adjustbox}
% \usepackage{epstopdf} % Does not work with pdflatex
\usepackage{amsmath}
\usepackage{amssymb}
\usepackage{hyperref}
\usepackage{xspace}
\usepackage{mathtools}
\usepackage{tikz}
\usepackage{epsfig}
\usepackage{float}
\usepackage{subfigure}
\usepackage{setspace}
\usepackage{tabularx}
\usepackage{multirow}
\usepackage{ragged2e}
\usepackage{booktabs}
\usepackage{caption}
% \usepackage{subcaption}
\usepackage{xcolor}
\usepackage{longtable}
% \usepackage{natbib} % Sometimes causes errors
\usepackage{etoolbox}
\usepackage{geometry}
\usepackage{esint}
\usepackage{ifxetex}
\usepackage{ifluatex}
% \usepackage{siunitx}
\usepackage{numprint}
\usepackage{empheq}
\usepackage{slashbox}


 % For local use
\newcommand{\figWidth}{0.45\textwidth}
%\newcommand{\figW}{6.5in}
%\newcommand{\figH}{5.5in}
\newcommand{\figW}{5.in}
\newcommand{\figH}{4.in}
%\newcommand{\figW}{6.825in}
%\newcommand{\figH}{5.775in}
\newcommand{\ffigW}{3.1in}
\newcommand{\ffigH}{3.1in}

\newcommand{\captionWidth}{0.45}
\newcommand{\subFigPad}{0.52\textwidth} % good for subcaption
\newcommand{\TWOFIGCOLUMN}{0.52\textwidth} % good for subcaption
\newcommand{\TWOFIGCOLUMNCAPTION}{0.52\textwidth} % good for subcaption
% \newcommand{\subFigPad}{0.45} % good for subfig
\newcommand{\miniPageSize}{.1\textwidth}

\newcommand{\subFigSize}{16em} % good for subcaption
\newcommand{\figSize}{0.45\textwidth}


% \renewcommand{\contcaption}{%
%   \expandafter\addtocounter\expandafter{\@captype alt}{\m@ne}% Step alias cntr back
%   \expandafter\refstepcounter\expandafter{\@captype alt}% Make reference
%   \@contcaption\@captype}

% \definecolor{darkGreen}{rgb}{0.0, 0.5, 0.0}
\newcommand{\COR}{\color{red}}
\newcommand{\COG}{\color{darkGreen}}
\newcommand{\COM}{\color{magenta}}
\newcommand{\COK}{\color{black}}
\newcommand{\COB}{\color{blue}}

\newcommand{\toth}{\tiny{Toth, Gabor. "The $\DIV \B = 0$ constraint in shock-capturing magnetohydrodynamics codes." J. Comput. Phys. 161.2 (2000): 605-652.}}
\newcommand{\pattison}{\tiny{M. J. Pattison, K. N. Premnath, N. B. Morley, M. Abdou, Fusion Eng. Des. 83 (2008) 557-572.}}
\newcommand{\gujStella}{\tiny{G. Guj, F. Stella, J. Comput. Phys. 298 (1993) 286-298.}}
\newcommand{\bandaru}{\tiny{V. Bandaru, J. Pracht, T. Boeck, J. Schumacher, Theor. Comp. Fluid Dyn. (2015).}}
\newcommand{\guermond}{\tiny{Guermond, J. L., J. Léorat, and C. Nore. "A new Finite Element Method for magneto-dynamical problems: two-dimensional results." European Journal of Mechanics-B/Fluids 22.6 (2003): 555-579.}}
\newcommand{\kawczynski}{\tiny{C. Kawczynski, S. Smolentsev, M. Abdou. "A Lid-Driven Cavity MHD Flow Numerical Study at Moderate Magnetic Reynolds Number using Proper Magnetic Boundary Conditions". In Progress.}}
\newcommand{\roache}{\tiny{P. J. Roache, Annu. Rev. Fluid Mech. 29 (1997) 123-160.}}
\newcommand{\hunt}{\tiny{J. C. R. Hunt, J. Fluid Mech. 21 (1965) 577-590.}}
\newcommand{\shercliff}{\tiny{J. A. Shercliff, Proc. Camb. Phil. Soc. 49 (1953) 136.}}
\newcommand{\turner}{\tiny{Turner, Larry R. "Electromagnetic computations for fusion devices." IEEE Transactions on Magnetics 26.2 (1990): 847-852.}}
\newcommand{\kotulskiA}{\tiny{Kotulski, Joseph D., et al. "Electromagnetic analysis of forces and torques on the baseline and enhanced ITER shield modules due to plasma disruption." IEEE Transactions on Plasma Science 38.4 (2010): 1047.}}
\newcommand{\kotulskiB}{\tiny{Kotulski, Joseph D., Rebecca Coats, and Michael Ulrickson. "The analysis of the electromagnetic loads on selected ITER blanket shield modules due to induced eddy and halo currents." Fusion Engineering (SOFE), 2011 IEEE/NPSS 24th Symposium on. IEEE, 2011.}}
\newcommand{\iskakov}{\tiny{Iskakov, A. B., S. Descombes, and E. Dormy. "An integro-differential formulation for magnetic induction in bounded domains: boundary element-finite volume method." J. Comput. Phys. 197.2 (2004): 540-554.}}
\newcommand{\mikeulrickson}{\tiny{We received numerical data from Dr. Mike Ulrickson, computed from the plasma code DINA.}}
\newcommand{\Leriche}{Leriche, E. \& Gavrilakis, S. Direct numerical simulation of the flow in a lid-driven cubical cavity. Phys. Fluids 12, (2000).}
\newcommand{\Brackbill}{\tiny{Brackbill, Jeremiah U., and Daniel C. Barnes. "The effect of nonzero $\DIV \B = 0$ on the numerical solution of the magnetohydrodynamic equations." Journal of Computational Physics 35.3 (1980): 426-430.}}
\newcommand{\Leuer}{\tiny{Leuer, J. A., and J. C. Wesley. "ITER plasma start-up modeling." Fusion Engineering, 1993., 15th IEEE/NPSS Symposium on. Vol. 2. IEEE, 1993.}}

% \newcommand{\NJ}{New Jersey}
% \newcommand{\UCC}{Union County College}
% \newcommand{\GSR}{Graduate Student Researcher}
% \newcommand{\ME}{Mechanical Engineering}
% \newcommand{\NJSGC}{New Jersey Space Grant Consortium}
% % \newcommand{\NASA}{National Aeronautics and Space Administation}
% \newcommand{\NASA}{NASA}
% \newcommand{\SAS}{School of Arts and Sciences}
% % \newcommand{\UCLA}{University of California Los Angeles}
% \newcommand{\UCLA}{UCLA}
% \newcommand{\IDRE}{Institute for Digital Research and Education}
% \newcommand{\MAE}{Department of Mechanical and Aerospace Engineering}
% \newcommand{\UCLAMAE}{Henri Samueli School Of Engineering And Applied Sciences}
% % \newcommand{\MHD}{Magnetohydrodynamic}
% \newcommand{\MHDs}{Magnetohydrodynamics}
% % \newcommand{\RUTGERS}{Rutgers University, The State University of New Jersey}
% \newcommand{\RUTGERS}{Rutgers University}
% \newcommand{\RU}{Rutgers University}
% \newcommand{\US}{United States}

\newcommand{\height}{0.4}
\newcommand{\radius}{0.1}
\newcommand{\offSet}{12}
\newcommand{\Deltah}{1.6666}
\lstset{language=[90]Fortran,
  basicstyle=\ttfamily,
  keywordstyle=\color{red},
  commentstyle=\color{green},
  morecomment=[l]{!\ }% Comment only with space after !
}
\newif\ifxetexorluatex
\ifxetex
  \renewcommand{\C}{\mathbf{C}}
  \renewcommand{\U}{\mathbf{u}}
  \renewcommand{\G}{\mathbf{g}}
\else
  \newcommand{\C}{\mathbf{C}}
  \newcommand{\U}{\mathbf{u}}
  \newcommand{\G}{\mathbf{g}}
\fi
\renewcommand{\H}{\mathbf{H}}

\newcommand{\Rem}{Re_m}
\newcommand{\Rm}{$Re_m$}
\newcommand{\B}{\mathbf{B}}
\newcommand{\cleanB}{\xi}
\newcommand{\R}{\mathbf{r}}
\newcommand{\T}{\mathbf{T}}
\newcommand{\BBx}{\bar{B_x}}
\newcommand{\BBy}{\bar{B_y}}
\newcommand{\BBz}{\bar{B_z}}
\newcommand{\PB}{\bar{p}}
\newcommand{\UBx}{\bar{u}}
\newcommand{\VBy}{\bar{v}}
\newcommand{\WBz}{\bar{w}}
\newcommand{\E}{\mathbf{E}}
\newcommand{\J}{\mathbf{j}}
\newcommand{\A}{\mathbf{A}}
\newcommand{\N}{\mathbf{n}}
\newcommand{\F}{\mathbf{f}}
\newcommand{\X}{\mathbf{x}}
\newcommand{\Y}{\mathbf{y}}
\newcommand{\zero}{\mathbf{0}}
\newcommand{\ttheta}{\tilde{\theta}}
\newcommand{\M}{{\mu_m}}
\newcommand{\SII}{\SI^{-1}}
\newcommand{\JS}{\frac{\J}{\sigma}}
\newcommand{\MO}{\overline{\M}}
\newcommand{\SO}{\overline{\sigma}}
\newcommand{\JSS}{\frac{\J^2}{\sigma}}

\newcommand{\KE}{\left(\frac{1}{2} \U \DOT \U\right)}
\newcommand{\ME}{\left(\frac{1}{2} \B \DOT \B\right)}
\newcommand{\EPS}{\varepsilon}

\newcommand{\SIG}{\sigma}
\newcommand{\PH}{physical}

\newcommand{\SOVAC}{\SO_{v}}


\newcommand{\NPB}{\num[round-mode=places,round-precision=2]}
\newcommand{\NPU}{\num[round-mode=places,round-precision=5]}
\newcommand{\NPJ}{\num[round-mode=places,round-precision=5]}
\newcommand{\NPS}{\num[scientific-notation=true,round-mode=places,round-precision=2]}
\newcommand{\NPP}{\num[round-mode=places,round-precision=2]}

\newcommand{\MAT}{\left[ \begin{array}}
\newcommand{\EMAT}{\end{array} \right]}
\newcommand{\ARRAY}{\begin{array}}
\newcommand{\EARRAY}{\end{array}}

% \newcommand{\DOT}{\textbullet} % Causes warnings
% \newcommand{\DOT}{\text{\textbullet}}
\newcommand{\DOT}{\ensuremath{\bullet}}
% \newcommand{\textbullet}{}
\newcommand{\CROSS}{\times}
\newcommand{\BUN}{\otimes}
\newcommand{\DEL}{\nabla}
\newcommand{\CURL}{\DEL \CROSS}
\newcommand{\DIV}{\DEL \DOT}
\newcommand{\PD}{\partial}
\newcommand{\MAC}{\mathcal}
\newcommand{\GRAD}{\nabla}
\newcommand{\RA}{\rightarrow}

\newcommand{\EQSIZE}{\footnotesize}

\newcommand{\MC}{\multicolumn}
\newcommand{\MR}{\multirow}
\newcommand{\LL}{\raggedright}
\newcommand{\RR}{\raggedleft}
\newcommand{\CE}{\centering}
\newcommand{\TW}{\textwidth}
\renewcommand{\TH}{\textheight}
\newcommand{\TWP}{em\textwidth}
\newcommand{\qquadmany}{\qquad\qquad\qquad\qquad\qquad\qquad}
\newcommand{\hfillMath}{\hskip \textwidth minus \textwidth}

\newcommand{\tenDots}{..........}
\newcommand{\twentyDots}{\tenDots\tenDots}
\newcommand{\fiftyDots}{\twentyDots\twentyDots\tenDots}
\newcommand{\sixtyDots}{\twentyDots\twentyDots\twentyDots}
\newcommand{\seventyDots}{\fiftyDots\twentyDots}
\newcommand{\hundredDots}{\fiftyDots\fiftyDots}

% \restylefloat{table} % Results in errors in prospectus presentation


\newtoggle{includeImages}
\toggletrue{includeImages}
% \togglefalse{includeImages}

\newcommand{\IG}{\includegraphics}
\newcommand{\MP}{\minipage}
\newcommand{\EMP}{\endminipage}
\newcommand{\FIG}{\figure}
\newcommand{\EFIG}{\endfigure}
\newcommand{\SFIG}{\subfigure}
\newcommand{\ESFIG}{\endsubfigure}

\newcommand{\BLANKFIGURE}{\begin{figure*}[!htp]\centering\IG[width=.3\TW]{\figdir/blank.png}\end{figure*}}
\newcommand{\BLANKGRAPHIC}{\IG[width=.3\TW]{\figdir/blank.png}}
% \renewcommand{\BLANKGRAPHIC}{}
% \renewcommand{\BLANKFIGURE}{}

\newcommand{\FIGTYPE}{_BW}
\renewcommand{\FIGTYPE}{} % uncomment if color
\newcommand{\FIGEXT}{png}
% \renewcommand{\FIGEXT}{eps} % uncomment for .eps (takes forever to build)

\newcommand{\MOONSTITLE}{\title{MHD Object-Oriented Numerical Solver (MOONS)}
\author{C. Kawczynski \\
Department of Mechanical and Aerospace Engineering \\
University of California Los Angeles, USA\\
}}


% % Numbering from: https://www.sharelatex.com/learn-scripts/images/f/fc/Layout-dimensions.png
\setlength{\hoffset}{-.5in}       % 1
\setlength{\voffset}{0in}        % 2
\setlength{\oddsidemargin}{0in}  % 3
\setlength{\topmargin}{-.5in}     % 4
\setlength{\headheight}{0in}     % 5
\setlength{\headsep}{0in}        % 6
\setlength{\textheight}{700pt}   % 7
\setlength{\textwidth}{550pt}    % 8
\setlength{\marginparsep}{0in}   % 9
% \setlength{\marginparwidth}{0in} % 10
\setlength{\footskip}{0in}       % 11 a (see fig)
% \setlength{\marginparpush}{0in}  % 11 b (see fig)
% \setlength{\paperwidth}{0in}     % 11 e (see fig)
% \setlength{\paperheight}{390pt}  % 11 f (see fig)
 % For local use
% % Numbering from: https://www.sharelatex.com/learn-scripts/images/f/fc/Layout-dimensions.png
\setlength{\hoffset}{0in}       % 1
\setlength{\voffset}{0in}        % 2
\setlength{\oddsidemargin}{0in}  % 3
\setlength{\topmargin}{0in}     % 4
\setlength{\headheight}{0in}     % 5
\setlength{\headsep}{0in}        % 6
\setlength{\textheight}{350pt}   % 7
\setlength{\textwidth}{5in}    % 8
\setlength{\marginparsep}{0in}   % 9
\setlength{\marginparwidth}{0in} % 10
\setlength{\footskip}{0in}       % 11 a (see fig)
\setlength{\marginparpush}{0in}  % 11 b (see fig)
\setlength{\hoffset}{0in}        % 11 c (see fig)
\setlength{\voffset}{0in}        % 11 d (see fig)
\setlength{\paperwidth}{0in}     % 11 e (see fig)
\setlength{\paperheight}{390pt}  % 11 f (see fig)
 % For local use
% \input{\rootdir/includes/diagrams/coordinates.tex} % For local use
% \input{\rootdir/includes/margins/margins.tex} % For local use




\begin{document}
\doublespacing
\MOONSTITLE
\maketitle

\section{Analysis of Poynting (radiation)}

\begin{equation}
	\boxed{
	\underbrace{\frac{\PD E_m}{\PD t} }_
	{\substack{\text{Time rate of}\\\text{change of $E_M$}}}
	=
	- \underbrace{\DIV \left(\E \CROSS \H \right)}_
	{\substack{\text{Power of}\\\text{Poynting term $\mathcal P_{P}$}}}
	- \underbrace{\U \DOT (\J \CROSS \B)}_
	{\substack{\text{Power of}\\\text{EM forces $\mathcal P_{em}$}}}
	- \underbrace{\JSS}_
	{\substack{\text{Joule}\\\text{dissipation}\\\text{$D_J$}}}
	}
\end{equation}
\subsection{Term representation}
Let's write this as an integral equation. Let
\begin{equation}
	E_m = \frac{\B^2}{2 \MO}
\end{equation}
\begin{equation}
	D_J = \int \JSS dV = \text{Joule dissipation of heat}
\end{equation}
\begin{equation}
	\mathcal P_{em} = \int \U \DOT (\J \CROSS \B) dV = \text{Power of electromagnetic forces}
\end{equation}
\begin{equation}
	\Phi_P = \oiint \left( \E \CROSS \H \right) \DOT \N dA = \text{Active and reactive power}
\end{equation}

Giving us
\begin{equation}
	\boxed{
	\frac{\PD E_m}{\PD t}
	=
	- \Phi_P
	- \mathcal P_{em}
	- D_J
	}
\end{equation}
Additionally note
\begin{equation}
	D_J + \mathcal P_{em} = \text{Total power supplied to charges}
\end{equation}

\subsection{Notes}
"Part of this power, used up to drive the flow of charge against the ohmic resistance of the material, corresponds to the Joule effect heating. The rest is transformed into mechanical power by the collective pressure of the electrons on the atoms and globally on the matter." [P. 22 Magnetohydrodynamics, Moreau]

"As a general integral of the field equations, the validity of Poynting's theorem is unimpeachable. Its physical interpretation, however, is open to some criticism" [P. 133 Electromagnetic Theory, Stratton].

"If, finally, all material bodies in the field are absolutely rigid, thereby excluding possible transformations of electromagnetic energy into elastic energy of a stressed medium, the balance can be maintained only by a flow of electromagnetic energy across the surface bounding $V$. This, according to Poynting, is the significance of the surface integral" [P. 132 Electromagnetic Theory, Stratton].

\section{Flux term}
The integrand of the second term in \ref{eq:nrg} is called Poynting's vector. The electric field can be decomposed into a scalar and vector potential
\begin{equation}
	\E = -\DEL \varphi - \frac{\PD \A}{\PD t}, \qquad \B = \CURL \A
\end{equation}
Allowing us to write $\Phi_P$ as
\begin{equation}
	\Phi_P = \oiint \left( \E \CROSS \H \right) \DOT \N dA =
	\underbrace{- \oiint \left( \DEL \varphi \CROSS \H \right) \DOT \N dA}_{\Phi_1} +
	\underbrace{- \oiint \left( \frac{\PD \A}{\PD t} \CROSS \H \right) \DOT \N dA}_{\Phi_2}
\end{equation}
The vector identity
\begin{equation}
	\CURL (\psi \A) = \psi (\CURL \A) + (\DEL \psi) \CROSS \A
	, \qquad \rightarrow \qquad
	(\DEL \psi) \CROSS \A = \CURL (\psi \A) - \psi (\CURL \A)
\end{equation}
and Ampere's law allows use to write the first term on the RHS as
\begin{equation}
	\Phi_1 = - \underbrace{\oiint \CURL \left(\varphi \H \right) \DOT \N dA}_{= 0}
	+ \oiint \varphi \J \DOT \N dA
\end{equation}
Since the flux of a curl over a closed surface is identically zero (since there are no boundaries), this reduces to
\begin{equation}
	\Phi_1 = \oiint \varphi \J \DOT \N dA
\end{equation}
Therefore we have
\begin{equation}
	\boxed{
	\Phi_P = \oiint \left( \E \CROSS \H \right) \DOT \N dA = \Phi_1 + \Phi_2 =
	\underbrace{\oiint (\varphi \J) \DOT \N dA}_{\Phi_1} +
	\underbrace{- \oiint \left( \frac{\PD \A}{\PD t} \CROSS \H \right) \DOT \N dA}_{\Phi_2}
	}
\end{equation}

This may decomposed into "active power" and "reactive power", which is briefly discussed by Moreau.

\section{Final representation}
Finally, we have
\begin{equation}
	\boxed{
	\underbrace{\frac{\PD E_m}{\PD t} }_
	{\substack{\text{Time rate of}\\\text{change of $E_M$}}}
	=
	- \underbrace{\DIV \left(\varphi \J \right) }_
	{\substack{\text{Power of}\\\text{Poynting 1 term $\mathcal P_{P_1}$}}}
	+ \underbrace{\DIV \left( \frac{\PD \A}{\PD t} \CROSS \H \right)}_
	{\substack{\text{Power of}\\\text{Poynting 2 term $\mathcal P_{P_2}$}}}
	- \underbrace{\U \DOT (\J \CROSS \B)}_
	{\substack{\text{Power of}\\\text{EM forces $\mathcal P_{em}$}}}
	- \underbrace{\JSS}_
	{\substack{\text{Joule}\\\text{dissipation}\\\text{$D_J$}}}
	}
\end{equation}
Writing these as integrals, we have
\begin{equation}
	\boxed{
	\underbrace{\int \frac{\PD E_m}{\PD t} dV}_
	{\substack{\text{Time rate of}\\\text{change of $E_M$}}}
	=
	- \underbrace{\oiint \left(\varphi \J \right) \DOT \N dA }_
	{\substack{\text{Power of}\\\text{Poynting 1 term $\mathcal P_{P_1}$}}}
	+ \underbrace{\oiint \left( \frac{\PD \A}{\PD t} \CROSS \H \right)\DOT \N dA }_
	{\substack{\text{Power of}\\\text{Poynting 2 term $\mathcal P_{P_2}$}}}
	- \underbrace{\int \U \DOT (\J \CROSS \B) dV}_
	{\substack{\text{Power of}\\\text{EM forces $\mathcal P_{em}$}}}
	- \underbrace{\int \JSS dV}_
	{\substack{\text{Joule}\\\text{dissipation}\\\text{$D_J$}}}
	}
\end{equation}
Where
\begin{equation}
	\B = \CURL \A, \E = - \DEL \varphi - \frac{\PD \A}{\PD t}
\end{equation}
\section{Decomposition}
\subsection{Applied in conductor and vacuum}
Trivially $0 = 0$, since $\B^0$ is constant and uniform.
\subsection{Vacuum}
\begin{equation}
	\underbrace{\int \frac{\PD }{\PD t} \frac{{\B^{ind}}^2}{2 \mu} dV}_
	{\substack{\text{Time rate of}\\\text{change of $E_M$}}}
	=
	\underbrace{\oiint \left( \frac{\PD \A^{ind}}{\PD t} \CROSS \frac{\B^{0} + \B^{ind}}{\mu} \right)\DOT \N dA }_
	{\substack{\text{Power of second}\\\text{Poynting term $\mathcal P_{P_2}$}}}
\end{equation}
\subsection{Conductor}
\begin{equation}
	\underbrace{\int \frac{\PD }{\PD t} \frac{{\B^{ind}}^2}{2 \mu} dV}_
	{\substack{\text{Time rate of}\\\text{change of $E_M$}}}
	=
	\underbrace{\oiint \left( \frac{\PD \A^{ind}}{\PD t} \CROSS \frac{\B^{0} + \B^{ind}}{\mu} \right)\DOT \N dA }_
	{\substack{\text{Power of second}\\\text{Poynting term $\mathcal P_{P_2}$}}}
	- \underbrace{\int \U \DOT (\J \CROSS \left[ \B^0 + \B^{ind} \right]) dV}_
	{\substack{\text{Power of}\\\text{EM forces $\mathcal P_{em}$}}}
	- \underbrace{\int \JSS dV}_
	{\substack{\text{Joule}\\\text{dissipation}\\\text{$D_J$}}}
\end{equation}

\begin{equation}
	\B = 0, \in \Gamma_v
\end{equation}
\section{Non-dimensionalization}
Consider non-dimensionalization $t=t^*/(L/U), \MO = \mu^*/\mu, \SO = \sigma^*/\sigma, \DEL = L \DEL^*, \U = \U^*/U, \B = \B^*/B, \E = \E^* / (B/ \sigma \mu L), \J = \J^* / (B/\mu L)$, where the asterisk implies local values. Then we have
\begin{equation}
	\frac{UB^2}{L \mu}
	\frac{\PD }{\PD t} \frac{\B^2}{2 \MO}
	=
	- \frac{B^2}{\sigma \mu^2 L^2} \DIV \left(\E \CROSS \frac{\B}{\MO} \right)
	- \frac{U B^2}{\mu L} \U \DOT (\J \CROSS \B)
	- \frac{B^2}{\sigma \mu^2 L^2} \JSS
\end{equation}
Dividing by the transient term coefficient gives
\begin{equation}
	\frac{\PD }{\PD t} \frac{\B^2}{2 \MO}
	=
	- \frac{1}{\mu \sigma U L} \DIV \left(\E \CROSS \frac{\B}{\MO} \right)
	- \U \DOT (\J \CROSS \B)
	- \frac{1}{\mu \sigma U L} \frac{\J^2}{\SO}
\end{equation}
Which yields
\begin{equation} \label{eq:nrg}
	\boxed{
	\frac{\PD }{\PD t} \frac{\B^2}{2 \MO}
	=
	- \frac{1}{Re_m} \DIV \left(\E \CROSS \frac{\B}{\MO} \right)
	- \U \DOT (\J \CROSS \B)
	- \frac{1}{Re_m} \frac{\J^2}{\SO}
	}
\end{equation}
If $\E$ is scaled by $UB$ (Ohm's law) instead of $\frac{B}{\mu L}$ (Ampere's law), then we get
\begin{equation}
	\boxed{
	\frac{\PD }{\PD t} \frac{\B^2}{2 \MO}
	=
	- \DIV \left(\E \CROSS \frac{\B}{\MO} \right)
	- \U \DOT (\J \CROSS \B)
	- \frac{1}{Re_m} \frac{\J^2}{\SO}
	}
\end{equation}

\end{document}