\documentclass[11pt]{article}
\newcommand{\VAR}{Success}


\begin{document}
\doublespacing
\MOONSTITLE
\maketitle

\section{Divergence form of magnetic field convection}

Using the vector identity
\begin{equation}
	\CURL (A\CROSS B) =
	A(\DIV B)
	- B(\DIV A)
	+(B \DOT \DEL)A - (A \DOT \DEL)B
\end{equation}
And noting that (index notation helps here)
\begin{equation}
	\PD_j (u_i B_j) =
	u_i \underbrace{\PD_j B_j}_\text{= 0} + B_j \PD_j u_i
	= B_j \PD_j u_i
\end{equation}
\begin{equation}
	\PD_j (u_j B_i) =
	u_j \PD_j B_i + B_i \underbrace{\PD_j u_j}_\text{=0}
	= u_j \PD_j B_i
\end{equation}
We can write our advective term as
\begin{equation}
	\CURL (\U \CROSS B)
	= \U(\DIV B)
	- B(\DIV \U)
	+ (B \DOT \DEL)\U
	- (\U \DOT \DEL)B
	=
	\PD_j (u_i B_j) - \PD_j (u_j B_i)
	=
	-\PD_j (u_j B_i - u_i B_j)
\end{equation}

\end{document}