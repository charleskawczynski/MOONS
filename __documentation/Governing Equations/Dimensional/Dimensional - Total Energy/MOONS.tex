\documentclass[11pt]{article}
\newcommand{\VAR}{Success}


\begin{document}
\doublespacing
\MOONSTITLE
\maketitle

\section{Dimensional total energy equation}
Adding the thermal, kinetic and magnetic energy equations
\begin{equation}
	\rho C_p \left[ \frac{\PD T}{\PD t}
	+ (\U \DOT \DEL) T \right]
	= \DIV (k \DEL T)
	+ \frac{1}{\sigma} \J^2
	- \nu \DEL^2 E_K
	+ \mu (\DEL \U) \DOT (\DEL \U)
\end{equation}
\begin{equation}
	\frac{\PD E_K}{\PD t} +
	(\U \DOT \DEL) E_K =
	- \DIV (\U p)
	+ \nu \DEL^2 E_K
	- \mu (\DEL \U) \DOT (\DEL \U)
	- (\U \DOT \DEL) E_M
	+ \U \DOT \left[\B \DOT \DEL \left( \H \right) \right]
\end{equation}
\begin{equation}
	\frac{\PD E_M}{\PD t} =
	- \JSS
	- \DIV \left(\E \CROSS \H \right)
	+ (\U \DOT \DEL) E_M
	- \U \DOT \left[(\B \DOT \DEL) \left( \H \right) \right]
\end{equation}
Magnetic + Kinetic
\begin{equation}
	\frac{\PD E_K}{\PD t}
	+ \frac{\PD E_M}{\PD t}
	+ (\U \DOT \DEL) E_K =
	- \JSS
	- \DIV \left(\E \CROSS \H \right)
	- \DIV (\U p)
	+ \nu \DEL^2 E_K
	- \mu (\DEL \U) \DOT (\DEL \U)
\end{equation}
Assuming viscous losses are negligible, and since flow is isothermal, the Joule heating will balance kinetic energy diffusion, so we'll have
\begin{equation}\begin{aligned}
	\frac{\PD (E_K + E_M)}{\PD t}
	+ (\U \DOT \DEL) E_K =
	- \DIV \left(\E \CROSS \H \right)
	- \DIV (\U p) \\
	\JSS = \nu \DEL^2 E_K
\end{aligned}\end{equation}
At steady state, we have
\begin{equation}\begin{aligned}
	(\U \DOT \DEL) E_K =
	- \DIV \left(\E \CROSS \H \right)
	- \DIV (\U p) \\
	\JSS = \nu \DEL^2 E_K
\end{aligned}\end{equation}
If we consider the magnetic energy equation in the vacuum domain, there is not motion, and no currents, so we have
\begin{equation}
	\frac{\PD E_M}{\PD t} =	- \DIV \left(\E \CROSS \H \right)
\end{equation}
which means, at steady state, the RHS is zero! Therefore we have
\begin{equation}\begin{aligned}
	(\U \DOT \DEL) E_K = - \DIV (\U p) \\
	\JSS = \nu \DEL^2 E_K
\end{aligned}\end{equation}
Let's scale this by
\begin{equation}\begin{aligned}
	[\U] = U, \qquad
	[\DEL] = 1/L, \qquad
	[p] = \frac{1}{2} \rho U^2, \qquad
	[\J] = \sigma_c U B
\end{aligned}\end{equation}
To get
\begin{equation}\begin{aligned}
	\frac{\rho U^3}{2L} (\U \DOT \DEL) ( \U \DOT \U ) =
	- \frac{\rho U^3}{2L} \DIV (\U p) \\
	(\U \DOT \DEL) ( \U \DOT \U ) =
	- \DIV (\U p), \qquad
\end{aligned}\end{equation}
and
\begin{equation}\begin{aligned}
	\frac{\sigma_c^2 U^2 B^2}{\sigma_c} \JSS =
	\frac{U^2}{L^2} \mu \DEL^2 \left( \frac{1}{2} \U \DOT \U \right) \\
	\frac{\sigma_c B^2 L^2}{\mu} \JSS =
	\DEL^2 \left( \frac{1}{2} \U \DOT \U \right) \\
	Ha^2 \JSS =
	\DEL^2 \left( \frac{1}{2} \U \DOT \U \right)
\end{aligned}\end{equation}

% According to the paper I was reading about how Hartmann number is equal to Reynolds times magnetic Reynolds, we have
% \begin{equation}\begin{aligned}
% 	Re Re_m \JSS = \DEL^2 \left( \frac{1}{2} \U \DOT \U \right) \\
% 	\frac{\JSS}{\DEL^2 \left( \frac{1}{2} \U \DOT \U \right)} = \frac{(Re_m)^{-1}}{Re}
% \end{aligned}\end{equation}

\end{document}