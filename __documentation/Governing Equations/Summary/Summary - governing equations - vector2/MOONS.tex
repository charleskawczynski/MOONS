\documentclass[landscape,11pt]{article}
\newcommand{\PSCHAIN}{..}
\edef\PSCHAIN{\PSCHAIN/LATEX_INCLUDES}

\newcommand{\rootdir}{\PSCHAIN}
\newcommand{\VAR}{Success}



\begin{document}
\doublespacing
\MOONSTITLE
% \maketitle

\section{Thermal Energy}
\Large
\begin{equation}
	\frac{\PD \theta}{\PD t}
	+ \DIV (\U \theta)
	=
	\frac{1}{Pe} \DIV (k \DEL \theta)
	+ Ec N \frac{{\J}^2}{\sigma}
	- \frac{Ec}{Re} \DEL^2 E_K
	+ \frac{Ec}{Re} (\DEL \U) \DOT (\DEL \U)
	+ \dot{Q}
\end{equation}
\section{Momentum}
\begin{equation}
	\frac{\PD \U}{\PD t}
	+ \DIV (\U^T \U)
	=
	- \DEL p
	+ \frac{1}{Re} \DEL^2 \U
	+ \frac{1}{Fr^2} \G
	+ \frac{Gr}{Re^2} \theta \G
	+ N \J \CROSS \B
\end{equation}
\section{Magnetic Field}
\begin{equation}
	\frac{\PD \B}{\PD t}
	+ \frac{1}{Re_m} \CURL \left(\frac{1}{\SO} \CURL \B \right)
	= \CURL (\U \CROSS \B)
\end{equation}
\section{Kinetic energy equation}
\begin{multline}
	\frac{\PD E_K}{\PD t}
	+ (\U \DOT \DEL) E_K
	=
	- \DIV (\U p)
	+ \frac{1}{Re} \DEL^2 E_K
	- \frac{1}{Re} (\DEL \U) \DOT (\DEL \U)
	- Al (\U \DOT \DEL) E_M
	+ Al \U \DOT (\B \DOT \DEL \H) \\
	+ \frac{Gr}{Re^2} \U \DOT (\rho \G \beta \theta)
	+ \U \DOT (\rho \G)
	+ \frac{L_c f_c}{\rho_c U_c^2} \U \DOT \F
\end{multline}
\section{Magnetic energy equation}
\begin{equation}
	\frac{\PD E_m}{\PD t} =
	- Re_m \frac{{\J}^2}{\SI}
	+ \DIV \left[\left(\U \CROSS \B - \frac{\J}{\SI} \right) \CROSS \H \right]
	+ (\U \DOT \DEL) E_m
	- \U \DOT \left[(\B \DOT \DEL) \H \right]
\end{equation}
\section{Dimensionless groups, definitions and chosen scales}
\begin{equation}
	Re = \frac{U_c L_c}{\nu_c} \qquad
	Ha = B_c L_c \sqrt{\frac{\sigma_c}{\mu_c}} \qquad
	Re_m = \mu_c \sigma_c U_c L_c \qquad
	Pr = \frac{\nu_c}{\alpha_c} \qquad
	Al = \frac{B_c^2}{\mu_c \rho_c U_c^2} \qquad
\end{equation}
\begin{equation}
	Gr = \frac{g_c \beta_c \theta_c L_c^3}{\nu_c^2} \qquad
	Ec = \frac{U_c^2}{C_{p,c} \theta_c} \qquad
	Fr = \frac{U_c}{\sqrt{g_c L_c}} \qquad
	N = \frac{Ha^2}{Re} \qquad
	Pr = \frac{Pe}{Re}=\frac{Ra}{Gr}
\end{equation}
\begin{equation}
	Pr_{m} = \frac{Re_m}{Re} \qquad
	[t_c] = L_c/U_c \qquad
	[p] = \rho_c U_c^2 \qquad
	[j] = \sigma_c U_c B_c \qquad
	[\dot{Q}] = \rho_c C_{p,c} \theta_c / t_c
\end{equation}

\end{document}