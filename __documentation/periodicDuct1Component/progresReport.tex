\documentclass[11pt]{article}
\usepackage{graphicx}    % needed for including graphics e.g. EPS, PS
\usepackage{epstopdf}
\usepackage{amsmath}
\usepackage{hyperref}
\usepackage{xspace}
\usepackage{mathtools}
\usepackage{tikz}
\usepackage{epsfig}
\usepackage{float}
\usepackage{natbib}
\usepackage{subfigure}
\usepackage{setspace}
\usepackage{tabularx,ragged2e,booktabs,caption}
\usepackage{etoolbox}

\newcommand{\figS}{22.1em}
\newcommand{\BfigS}{50em}
\newcommand{\ffigS}{15.1em}
\newcommand{\figH}{\figS}
\newcommand{\figW}{\figS}
\newcommand{\BfigH}{\BfigS}
\newcommand{\BfigW}{\BfigS}
\newcommand{\ffigH}{\ffigS}
\newcommand{\ffigW}{\ffigS}


\setlength{\oddsidemargin}{0.1in}
\setlength{\textwidth}{7.25in}

\setlength{\topmargin}{-1in}     %\topmargin: gap above header
\setlength{\headheight}{0in}     %\headheight: height of header
\setlength{\topskip}{0in}        %\topskip: between header and text
\setlength{\headsep}{0in}        
\setlength{\textheight}{692pt}   %\textheight: height of main text
\setlength{\textwidth}{7.5in}    % \textwidth: width of text
\setlength{\oddsidemargin}{-0.5in}  % \oddsidemargin: odd page left margin
\setlength{\evensidemargin}{0in} %\evensidemargin : even page left margin
\setlength{\parindent}{0.25in}   %\parindent: indentation of paragraphs
\setlength{\parskip}{0pt}        %\parskip: gap between paragraphs
\setlength{\voffset}{0.5in}

\newtoggle{plotUstream}
\toggletrue{plotUstream}
\togglefalse{plotUstream}

% Useful commands:

% \hfill		aligns-right everything right of \hfill

\begin{document}
\doublespacing
\title{Results of Dynamic MHD Periodic Duct for finite
 \texorpdfstring{$Re_m$} \\ with 1 Magnetic field Component}
\author{S. Smolentsev, C. Kawczynski \\
Department of Mechanical and Aerospace Engineering \\
University of California Los Angeles, USA\\
}
\maketitle


\section{Case D: Fully Coupled 1 Component Magnetic field in a Periodic Duct}

Here, we made the following assumptions:

\begin{itemize}
\item Uniform $\sigma$, uniform $\mu$
\item A uniform applied magnetic field of the form $B^0 = (B_x^0,B_y^0,B_z^0) = (e^{-t},0,0)$
\item We considered the domain enclosure of size $-0.5 < x < 0.5 \qquad -1.1 < y < 1.1 \qquad -1.1 < z < 1.1$
\item The number of cells in the fluid and walls were $N_{fluid} = (1,64,64)$ \qquad $N_{wall} = (0,8,8)$
\item Or $N_{fluid} = (1,200,200)$ \qquad $N_{wall} = (0,10,10)$
\end{itemize}

For this problem, we solved the momentum and induction equations. For BCs we used

\begin{equation}
	\pmb{u},
	\pmb{B},
	p = \text{periodic}
	\qquad
	x_{min},x_{max}
	\qquad \qquad
\end{equation}

\begin{equation}
	\pmb{u} = \pmb{u}_{walls} = 0
	\qquad
	\frac{\partial p}{\partial n} = 0
	\qquad
	\frac{\partial B_{n}}{\partial n} = 0
	\qquad
	B_{tangent} = 0
	\qquad
	y_{min},y_{max},
	z_{min},z_{max}
\end{equation}

\subsection{Momentum equation}

First note that there is no pressure force along x, and since the flow is periodic in x, the only way for flow to be driven in x requires a non-zero $j\times B$ force, which doesn't exist since the applied magnetic field induces currents in planes of x, resulting in a $j\times B$ only in planes of x. In addition, let's assume that the applied magnetic field does not induce any currents ($\nabla \times B^0=0$). Therefore, using the $\frac{\partial ()}{\partial x} = 0$, and $u = 0$, the momentum equations become

\begin{equation}
	u=0
\end{equation}
\begin{equation}
	\frac{\partial v}{\partial t} 
	+ \frac{\partial (v v)}{\partial y}
	+ \frac{\partial (v w)}{\partial z}
	= 
	- \frac{\partial p}{\partial y}
	+ \frac{1}{Re}
	\left(
	\frac{\partial^2 v}{\partial y^2}
	+\frac{\partial^2 v}{\partial z^2}
	\right)
	+ \frac{Ha^2}{Re}
	(j \times (B^{tot}))_y
\end{equation}
\begin{equation}
	\frac{\partial w}{\partial t} 
	+ \frac{\partial (w v)}{\partial y}
	+ \frac{\partial (w w)}{\partial z}
	= 
	- \frac{\partial p}{\partial z}
	+ \frac{1}{Re}
	\left(
	\frac{\partial^2 w}{\partial y^2}
	+\frac{\partial^2 w}{\partial z^2}
	\right)
	+ \frac{Ha^2}{Re}
	(j \times (B^{tot}))_z
\end{equation}

Where

\begin{equation}
	B^{tot} = B^0 + B
\end{equation}

\subsection{Induction equation}

The induction equation in its general form is

\begin{equation}
	\frac{\partial B_i}{\partial t} 
	=
	- \frac{\partial B_i^0}{\partial t} 
	- \frac{\partial}{\partial x_j} (u_j B_i^{tot} - u_i B_j^{tot}) 
	+
	\frac{1}{Re_m}
	\frac{\partial^2 B_i}{\partial x_j^2}
\end{equation}

First note that there is only a source term for the induced magnetic field for the x-component. This means that, since the initial induced magnetic field is zero, the y and z components of the induced magnetic field are always zero. Using this fact, $\frac{\partial}{\partial x} ()=0$, the uniform $\sigma$, and uniform $\mu$, the induction equations become


\begin{equation}
	\frac{\partial B_x}{\partial t} 
	=
	- \frac{\partial B_x^0}{\partial t}
	- \frac{\partial}{\partial y} (v B_x^{tot})
	- \frac{\partial}{\partial z} (w B_x^{tot})
	+ \frac{1}{Re_m}
	\left(
	\frac{\partial^2 B_x}{\partial y^2}
	+
	\frac{\partial^2 B_x}{\partial z^2}
	\right)
\end{equation}
\begin{equation}
	B_y = B_z = 0
\end{equation}

\subsection{Lorentz force}

Looking back at the momentum equation, the $j\times B$ force does not act in x, and we only have one component of the magnetic field, so we have

\begin{equation}
	(j \times B^{tot})_x = 0
	\qquad \qquad
	(j \times B^{tot})_y = - B_x^{tot} \frac{\partial B_x}{\partial y} 
	\qquad \qquad
	(j \times B^{tot})_z = - B_x^{tot} \frac{\partial B_x}{\partial z} 
\end{equation}

\subsection{Final Form of Governing Equations}

\subsubsection{Momentum}
\begin{equation}
	u=0
\end{equation}
\begin{equation}
	\frac{\partial v}{\partial t} 
	+ \frac{\partial (v v)}{\partial y}
	+ \frac{\partial (v w)}{\partial z}
	= 
	- \frac{\partial p}{\partial y}
	+ \frac{1}{Re}
	\left(
	\frac{\partial^2 v}{\partial y^2}
	+\frac{\partial^2 v}{\partial z^2}
	\right)
	- \frac{Ha^2}{Re}
	B_x^{tot} \frac{\partial B_x}{\partial y} 
\end{equation}
\begin{equation}
	\frac{\partial w}{\partial t} 
	+ \frac{\partial (w v)}{\partial y}
	+ \frac{\partial (w w)}{\partial z}
	= 
	- \frac{\partial p}{\partial z}
	+ \frac{1}{Re}
	\left(
	\frac{\partial^2 w}{\partial y^2}
	+\frac{\partial^2 w}{\partial z^2}
	\right)
	-\frac{Ha^2}{Re}
	B_x^{tot} \frac{\partial B_x}{\partial z}
\end{equation}

\subsubsection{Induction}
\begin{equation}
	\frac{\partial B_x}{\partial t} 
	=
	- \frac{\partial B_x^0}{\partial t}
	- \frac{\partial}{\partial y} (v B_x^{tot})
	- \frac{\partial}{\partial z} (w B_x^{tot})
	+ \frac{1}{Re_m}
	\left(
	\frac{\partial^2 B_x}{\partial y^2}
	+
	\frac{\partial^2 B_x}{\partial z^2}
	\right)
\end{equation}
\begin{equation}
	B_y = B_z = 0
\end{equation}

\section{Energy}


\subsection{Kinetic}

\subsection{Magnetic}

% \section{Given Ha=10}
% \subsection{Given Rm=1}

% \subsubsection{Re = 100}
% \subsubsection{Re = 500}
% \subsubsection{Re = 1000}
% \subsubsection{Re = 10000}
% \subsection{Given Rm=10}
% \subsubsection{Re = 100}
% \subsubsection{Re = 500}
% \subsubsection{Re = 1000}
% \subsubsection{Re = 10000}
% \subsection{Given Rm=100}
% \subsubsection{Re = 100}
% \subsubsection{Re = 500}
% \subsubsection{Re = 10000}
% \section{Given Ha=100}
% \subsection{Given Rm=1}
% \subsubsection{Re = 100}
% \subsubsection{Re = 500}
% \subsubsection{Re = 1000}
% \subsubsection{Re = 10000}
% \subsection{Given Rm=10}
% \subsubsection{Re = 100}
% \subsubsection{Re = 500}
% \subsubsection{Re = 1000}
% \subsubsection{Re = 10000}
% \subsection{Given Rm=100}

\bibliographystyle{unsrt}
\bibliography{MHD,Math,Interface,fluids,CFD,handpicked}


\end{document}