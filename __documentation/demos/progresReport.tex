\documentclass[11pt]{article}
\usepackage{graphicx}    % needed for including graphics e.g. EPS, PS
\usepackage{epstopdf}
\usepackage{amsmath}
\usepackage{hyperref}
\usepackage{xspace}
\usepackage{mathtools}
\usepackage{tikz}
\usepackage{epsfig}
\usepackage{float}
\usepackage{natbib}
\usepackage{subfigure}
\usepackage{setspace}
\usepackage{tabularx,ragged2e,booktabs,caption}

\newcommand{\figS}{22.1em}
\newcommand{\ffigS}{15.1em}
\newcommand{\figH}{\figS}
\newcommand{\figW}{\figS}
\newcommand{\ffigH}{\ffigS}
\newcommand{\ffigW}{\ffigS}


\setlength{\oddsidemargin}{0.1in}
\setlength{\textwidth}{7.25in}

\setlength{\topmargin}{-1in}     %\topmargin: gap above header
\setlength{\headheight}{0in}     %\headheight: height of header
\setlength{\topskip}{0in}        %\topskip: between header and text
\setlength{\headsep}{0in}        
\setlength{\textheight}{692pt}   %\textheight: height of main text
\setlength{\textwidth}{7.5in}    % \textwidth: width of text
\setlength{\oddsidemargin}{-0.5in}  % \oddsidemargin: odd page left margin
\setlength{\evensidemargin}{0in} %\evensidemargin : even page left margin
\setlength{\parindent}{0.25in}   %\parindent: indentation of paragraphs
\setlength{\parskip}{0pt}        %\parskip: gap between paragraphs
\setlength{\voffset}{0.5in}


% Useful commands:

% \hfill		aligns-right everything right of \hfill

\begin{document}
\doublespacing
\title{Demonstration cases for ISFNT Paper}
\author{S. Smolentsev, C. Kawczynski \\
Department of Mechanical and Aerospace Engineering \\
University of California Los Angeles, USA\\
}
\maketitle

\section{Geometry}

Domain range
\begin{equation}
	-1.1 < x < 1.1 \qquad -1.1 < y < 1.1 \qquad -1.1 < z < 1.1
\end{equation}

\begin{figure}[H]
 \centering
 \subfigure[]{
  \includegraphics[width=\ffigW,height=\ffigH]{geometry.eps}
  }
   \caption[Optional ]{}
\end{figure}

\section{Governing Equations And Scaling}

\subsection{Momentum}
Non-dimensionalizing the momentum equation with

\begin{equation}
	\rho = \rho_c \qquad
	\mu = \mu_c \qquad
	t_c = L_c/U_c \qquad
	u_j^* = \frac{u_j}{U_c} \qquad
	t^* = \frac{t}{t_c} \qquad
	x_j^* = \frac{x_j}{L_c} \qquad
	p^* = \frac{p}{\rho_c U_c^2} \qquad
\end{equation}

Yields the non-dimensional momentum equation. In the most general case considered, the ith component of the non-dimensional momentum equation is
\begin{equation}
	\frac{\partial u_i}{\partial t} + 
	\frac{\partial (u_i u_j)}{\partial x_j}
	= 
	- \frac{\partial p}{\partial x_i}
	+ \frac{1}{Re}
	\frac{\partial^2 u_i}{\partial x_j^2}
	+ \frac{Ha^2}{Re}
	(j \times B)_i
\end{equation}

Where

\begin{equation}
	Re = \frac{U_c L_c}{\nu_c}
	\qquad \qquad
	Ha = B_c L_c \sqrt{\frac{\sigma_c}{\mu_c}}
\end{equation}

\subsection{Induction}
Non-dimensionalizing the induction equation with the same parameters used to non-dimensionalize the momentum equation, plus

\begin{equation}
	B_k^* = \frac{B_k}{B_c}  \qquad
	j_k^* = \frac{j_k}{\sigma_c U_c B_c} \qquad
	E_k^* = \frac{E_k}{U_c B_c} \qquad
	\sigma^* = \frac{\sigma}{\sigma_c} \qquad
	\mu^* = \frac{\mu}{\mu_c} \qquad
\end{equation}

Yields the non-dimensional induction equation. In the most general case considered, the ith component of the non-dimensional induction equation is

\begin{equation}
	\frac{\partial B_i}{\partial t} 
	+ \frac{\partial}{\partial x_j} (u_j B_i - u_i B_j) 
	+ \frac{1}{Re_m}
	\frac{\partial}{\partial x_j} 
	\left\{ \frac{1}{\sigma} 
	\left[ 
	\frac{\partial}{\partial x_i} 
	\left( \frac{B_j}{\mu} \right) - 
	\frac{\partial}{\partial x_j} 
	\left( \frac{B_i}{\mu} \right)
	\right]
	\right\} = 0
\end{equation}

Where

\begin{equation}
	Re_m = \mu_c \sigma_c U_c L_c \qquad
\end{equation}

If we were to non-dimensionalize $B$ with

\begin{equation}
	B_k^* = \frac{B_k}{Re_m B^0}
\end{equation}

Where $B_k$ and $B^0$ are the induced and applied magnetic fields, we get

\begin{equation}
	\frac{\partial B_i}{\partial t} 
	+ \frac{\partial}{\partial x_j} (u_j B_i - u_i B_j) 
	+
	\frac{\partial}{\partial x_j} 
	\left\{ \frac{1}{\sigma} 
	\left[ 
	\frac{\partial}{\partial x_i} 
	\left( \frac{B_j}{\mu} \right) - 
	\frac{\partial}{\partial x_j} 
	\left( \frac{B_i}{\mu} \right)
	\right]
	\right\} = 0
\end{equation}


In these comparisons, $\mu$ was assumed to be uniform in space and constant in time.


\section{Boundary Conditions}

\subsection{Velocity field}
The boundary conditions for the velocity are 

\subsubsection{Dirichlet}

\begin{equation}
	\pmb{u} = \pmb{u}_{wall}
	\qquad \qquad
	\pmb{u} = \pmb{u}_{inlet}
	\qquad \qquad
	\pmb{u} = \pmb{u}_{outlet}
\end{equation}

\subsubsection{Neumann}
\begin{equation}
	\frac{\partial \pmb{u}}{\partial \pmb{n}} = 0
\end{equation}

\subsection{Magnetic field}
\subsubsection{Pseudo-vacuum}
\begin{equation}
	\frac{\partial \pmb{B}_{n}}{\partial \pmb{n}} = 0
	\qquad \qquad
	\pmb{B}_{tangent} = 0
\end{equation}

\subsubsection{Periodic along x}
\begin{equation}
	\frac{\partial \pmb{B}_{n}}{\partial \pmb{n}} = 0
	\qquad
	\pmb{B}_{tangent} = 0
	\qquad
	y_{min},y_{max}
	z_{min},z_{max}
\end{equation}

\begin{equation}
	\frac{\partial \pmb{B}}{\partial x} = 0
	\qquad
	x_{min},x_{max}
\end{equation}


\bibliographystyle{unsrt}
\bibliography{MHD,Math,Interface,fluids,CFD,handpicked}


\end{document}