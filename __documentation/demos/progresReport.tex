\documentclass[11pt]{article}
\usepackage{graphicx}    % needed for including graphics e.g. EPS, PS
\usepackage{epstopdf}
\usepackage{amsmath}
\usepackage{hyperref}
\usepackage{xspace}
\usepackage{mathtools}
\usepackage{tikz}
\usepackage{epsfig}
\usepackage{float}
\usepackage{natbib}
\usepackage{subfigure}
\usepackage{setspace}
\usepackage{tabularx,ragged2e,booktabs,caption}

\newcommand{\figS}{22.1em}
\newcommand{\ffigS}{15.1em}
\newcommand{\figH}{\figS}
\newcommand{\figW}{\figS}
\newcommand{\ffigH}{\ffigS}
\newcommand{\ffigW}{\ffigS}


\setlength{\oddsidemargin}{0.1in}
\setlength{\textwidth}{7.25in}

\setlength{\topmargin}{-1in}     %\topmargin: gap above header
\setlength{\headheight}{0in}     %\headheight: height of header
\setlength{\topskip}{0in}        %\topskip: between header and text
\setlength{\headsep}{0in}        
\setlength{\textheight}{692pt}   %\textheight: height of main text
\setlength{\textwidth}{7.5in}    % \textwidth: width of text
\setlength{\oddsidemargin}{-0.5in}  % \oddsidemargin: odd page left margin
\setlength{\evensidemargin}{0in} %\evensidemargin : even page left margin
\setlength{\parindent}{0.25in}   %\parindent: indentation of paragraphs
\setlength{\parskip}{0pt}        %\parskip: gap between paragraphs
\setlength{\voffset}{0.5in}


% Useful commands:

% \hfill		aligns-right everything right of \hfill

\begin{document}
\doublespacing
\title{Demonstration cases for ISFNT Paper}
\author{S. Smolentsev, C. Kawczynski \\
Department of Mechanical and Aerospace Engineering \\
University of California Los Angeles, USA\\
}
\maketitle

\section{Geometry}

Domain range (both fluid and magnetic)
\begin{equation}
	-1 < x < 1 \qquad -1 < y < 1 \qquad -1 < z < 1
\end{equation}

\begin{figure}[H]
 \centering
 \subfigure[]{
  \includegraphics[width=\ffigW,height=\ffigH]{geometry.eps}
  }
   \caption[Optional ]{}
\end{figure}

\section{Governing Equations And Scaling}

\subsection{Momentum}
The momentum equation is solved with Explicit Euler

\begin{equation}
	\label{eq:mom}
	\frac{\partial \pmb{u}}{\partial t} + 
	\nabla \bullet (\pmb{u} \pmb{u})
	= 
	- \nabla p
	+ \frac{1}{Re}
	\nabla^2 \pmb{u}
	\\
	+ \frac{Ha^2}{Re}
	(\nabla \times \pmb{B}^{ind}) \times (\pmb{B}^0+Re_m\pmb{B}^{ind})
\end{equation}
\begin{equation}
	\nabla \bullet \pmb{u} = 0
\end{equation}

Where

\noindent
\begin{subequations}
\begin{align}
  Re = \frac{U_c L_c}{\nu_c} \label{eq:Re}
  \\
  Ha = B_c L_c \sqrt{\frac{\sigma_c}{\mu_c}} \label{eq:Ha}
  \\
  Re_m = \mu_c \sigma_c U_c L_c \label{eq:Rem}
\end{align}
\end{subequations}

\subsection{Induction}
Let
\begin{equation}
  \label{eq:ind}
  G(\pmb{B})
  =
  \frac{\partial \pmb{B}}{\partial t} 
  -
  \nabla \times (\pmb{u} \times \pmb{B})
  + \\
  \frac{1}{Re_m}
  \nabla \times
  \left(
  \frac{1}{\sigma}
  \nabla \times
  \frac{\pmb{B}}{\mu}
  \right)
  =
  0
\end{equation}
The transient induced magnetic field term is solved for and updated using explicit Euler.
\begin{equation}
  G(\pmb{B}^0) + G(\pmb{B}^{ind}) Re_m = 0
\end{equation}

\section{Boundary Conditions}

\subsection{Velocity field}
Lid driven cavity BCs

\begin{equation}
	\pmb{u} = 
	\begin{cases}
      \hat{i}, & \text{if}\ y=y_{max} \\
      \pmb{0}, & \text{otherwise}
    \end{cases}
    \qquad \qquad
    \frac{\partial p}{\partial n} = 0
\end{equation}

% \begin{equation}
% 	\pmb{u} = \pmb{u}_{wall}
% 	\qquad \qquad
% 	\frac{\partial p}{\partial n} = 0
% \end{equation}

\subsection{Magnetic field}
Pseudo vacuum BCs
\begin{equation}
	\frac{\partial \pmb{B}_{n}}{\partial \pmb{n}} = 0
	\qquad \qquad
	\pmb{B}_{tangent} = 0
\end{equation}

\section{Material properties if walls are present}
If the walls are present, what should the material properties of lid be?

\begin{itemize}
\item $\sigma_{lid} = \sigma_{fluid}$
\item $\sigma_{lid} = \sigma_{wall}$
\end{itemize}

\section{Simulation Parameters}

\begin{itemize}
\item $N_{fluid} = (67^3)$ (number of cells in fluid domain)
\item $N_{PPE} = 5$ (number of iterations in Gauss Seidel for pressure poisson equation)
\item $dt = 10^{-3}$ (momentum time step)
\item $ds = dt$ (induction time step)
\item $\beta = 1.1$ (grid stretching parameter)
\end{itemize}

\section{Parametric Sweep Parameters}

Parameters for all cases will be the same as that for Guj and Stella as well as
\begin{itemize}
\item $Re = 1000$ (Reynolds number)
\item $Ha = 20$ (Hartmann number)
\item $N = 0.4$ (Interaction number)
\item $Re_m = (0,1,10,100,1000) = \text{Case} (1,2,3,4)$ (Magnetic Reynolds number)
\item $B^0 = (B^0_x,B^0_y,B^0_z) = \text{Case} (A,B,C)$ (Applied magnetic field)
\end{itemize}


\end{document}